\chapter{Trigonometria}
\section{Teorema di Carnot}\index{Carnot!Teorema}
\begin{thm}[Teorema di Carnot]\label{thm:TeoremadiCarnot}
	In un triangolo, il quadrato della lunghezza di un lato è uguale alla somma dei quadrati delle lunghezze dei rimanenti diminuito dal doppio del prodotto delle lunghezze di questi lati per il coseno dell'angolo fra essi compreso. 
	\begin{align*}
	a^2=&b^2+c^2-2bc\cos\alpha\\
	b^2=&a^2+c^2-2ac\cos\beta\\
	c^2=&a^2+b^2-2ab\cos\gamma\\
	\end{align*}
\end{thm}
\begin{figure}
	\centering
	\includestandalone{geometria/triangolopitagorico1}
	\caption{Teorema di Carnot, angolo acuto}
	\label{fig:triangolopitagorico1}
\end{figure}
\begin{proof}
	Consideriamo il triangolo~\vref{fig:triangolopitagorico1} cioè supponiamo che l'angolo $\beta$ sia un angolo acuto. Prendiamo il triangolo rettangolo $ADC$ allora:
	\begin{align*}
	\intertext{Per il teorema di Pitagora}
\widebar{AC}^2=&\widebar{AD}^2+\widebar{DC}^2\\
\widebar{AC}=&b\\
\widebar{AD}=&c\sin\beta\\
\widebar{BD}=&c\cos\beta\\
\widebar{DC}=&a-c\cos\beta\\
\intertext{Otteniamo}
b^2=&c^2\sin^2\beta+\left(a-c\cos\beta\right)^2\\
=&c^2\sin^2\beta+a^2+c^2\cos^2\beta-2ac\cos\beta\\
=&c^2\left(\sin^2\beta+\cos^2\beta\right)+a^2-2ac\cos\beta\\
=&c^2+a^2-2ac\cos\beta\\
	\end{align*}
	Consideriamo il triangolo come in~\vref{fig:triangolopitagorico2} cioè supponiamo che l'angolo $\beta$ sia un angolo ottuso. Prendiamo il triangolo rettangolo $ADC$ allora:
		\begin{align*}
	\intertext{Per il teorema di Pitagora}
	\widebar{AC}^2=&\widebar{AD}^2+\widebar{DC}^2\\
	\widebar{AC}=&b\\
	\widebar{AD}=&c\sin\left(\ang{180}-\beta\right)\\
	=&c\sin\beta\\
	\widebar{BD}=&c\cos\left(\ang{180}-\beta\right)\\
	=&-c\cos\beta\\
	\widebar{DC}=&a+c\cos\left(\ang{180}-\beta\right)\\
	=&a-c\cos\beta\\
	\intertext{Otteniamo}
	b^2=&c^2\sin^2\beta+\left(a-c\cos\beta\right)^2\\
	=&c^2\sin^2\beta+a^2+c^2\cos^2\beta-2ac\cos\beta\\
	=&c^2\left(\sin^2\beta+\cos^2\beta\right)+a^2-2ac\cos\beta\\
	=&c^2+a^2-2ac\cos\beta\\
	\end{align*}
	Da cui la tesi.
\end{proof}
\begin{cor}[Teorema di Carnot]\label{cor:TeoremaCarnot1}
	In un triangolo avremo che
	\begin{align*}
	\cos\alpha=&\dfrac{b^2+c^2-a^2}{2bc}\\
	\cos\beta=&\dfrac{a^2+c^2-b^2}{2ac}\\
	\cos\gamma=&\dfrac{a^2+b^2-c^2}{2bc}\\
	\end{align*}
\end{cor}
\begin{cor}[Teorema di Carnot]
	In un triangolo un angolo $\alpha\in[0,\ang{180}]$ è acuto, retto o ottuso se $b^2+c^2-a^2$ è positivo, nullo o negativo.
\end{cor}
\begin{proof}
	Dal~\cref{cor:TeoremaCarnot1} \begin{align*}
	\cos\alpha=&\dfrac{b^2+c^2-a^2}{2bc}\\
	\intertext{Il segno della frazione è il segno del numeratore, quindi}
	\intertext{Se:}
	b^2+c^2-a^2>&0\\
	\intertext{Il coseno è positivo, quindi l'angolo è acuto}
	\intertext{Se:}
	b^2+c^2-a^2=&0\\
	\intertext{Il coseno è nullo, quindi l'angolo è retto}
		\intertext{Se:}
	b^2+c^2-a^2<&0\\
	\intertext{Il coseno è negativo, quindi l'angolo è ottuso}
	\end{align*}
	Da cui la tesi.
\end{proof}
\begin{figure}
	\centering
	\includestandalone{geometria/triangolopitagorico2}
	\caption{Teorema di Carnot, angolo ottuso}
	\label{fig:triangolopitagorico2}
\end{figure}