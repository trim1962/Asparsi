\chapter{Contro esempi}
\section{Radicali}
\subsection{Somma}
\begin{cexmp}[Somma radicali]
\begin{align*}
\sqrt[n]{a}+\sqrt[n]{b}\neq&\sqrt[n]{a+b}\\
\sqrt{36}+\sqrt{64}\neq&\sqrt{100}\\
\sqrt{36}+\sqrt{64}=&6+8=14\\
\sqrt{100}=10\\
\end{align*}
Quindi la somma di radicali non il radicale della somma.
\end{cexmp}
\subsection{Prodotto}
\begin{cexmp}[Esistenza]
	Se $\sqrt[n]{a}$ e $\sqrt[n]{b}$ esistono allora esiste $\sqrt[n]{ab}$ Il viceversa non sempre è vero infatti mentre esiste $\sqrt{(-4)(-2)}$ non esistono   $\sqrt{-4}$ $\sqrt{-2}$ non esistono.
\end{cexmp}
\section{Goniometria}
\subsection{Coseno}
\begin{cexmp}[Non periodico]
	$f(x)=cosx^2$ non è periodica
\end{cexmp}\index{Coseno}
\begin{proof}
	Supponiamo che sia periodica di periodo $T$ allora
	\begin{align*}
	\cos(x+T)^2=&\cos x^2\\
	x^2+2xT+T^2\pm x^2=&2k\pi&z\in&\Z\\
	\end{align*}
	Assurdo, a sinistra abbiamo un'espressione che dipende dalla variabile continua $x$ a destra l'espressione può assumere solo valori interi.
\end{proof}