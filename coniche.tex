% !TeX root = Asparsi.tex
% !BIB TS-program = biber
% !TeX encoding = UTF-8
% !TeX spellcheck = it_IT
\chapter{Coniche}
\section{Parabola}
\begin{defn}[Definizione parabola]
	Definisco parabola l'equazione $y=ax^2+bx+c$ $a\neq0$
\end{defn}\index{Parabola}
\subsection{Proprietà}
\begin{thm}[Complemento al quadrato]\label{thm:Parabola_complemento}
	La parabola $y=ax^2+bx+c$ $a\neq0$ è equivalente a \begin{equation*}
		y+\dfrac{\Delta}{4a}=a\left(x+\dfrac{b}{2a}\right)^2\quad\Delta=b^2-4ac
	\end{equation*}\label{equa:Parabola_scomposizione}
\end{thm}\index{Parabola!complemento!quadrato}
\begin{proof}
	\begin{align*}
		y=&ax^2+bx+c\\
		y-c=&ax^2+bx\\
		y+\dfrac{b^2}{4a}-c=&ax^2+bx+\dfrac{b^2}{4a}\\
		y+\dfrac{b^2-4ac}{4a}=&a\dfrac{4a^2x^2+4abx+b^2}{4a^2}\\
		y+\dfrac{\Delta}{4a}=&a\left(\dfrac{2ax+b}{2a}\right)^2\\
		y+\dfrac{\Delta}{4a}=&a\left(x+\dfrac{b}{2a}\right)^2
	\end{align*}
	Da cui la tesi.
\end{proof}
\subsection{Concavità}
\begin{lem}[Ordine]
	Se $m>0$ $\forall n$ $m-n>-n$ se $m<0$ $\forall n$ $m-n<-n$
\end{lem}
\begin{thm}[Concavità]\label{thm:concavitaparabola}
	Data una parabola~\cite{Zwirner1988} $y=ax^2+bx+c$ $a\neq0$ allora se $a>0$ 
	la parabola ha la concavità rivolta verso l'alto e il vertice è il punto di 
	minima ordinata. Se $a<0$ la parabola ha la concavità rivolta verso il 
	basso e il vertice è il punto di massima ordinata
\end{thm}
\begin{proof}
	Dal~\cref{thm:Parabola_complemento} abbiamo
	\begin{align*}
		y=&a\left(x+\dfrac{b}{2a}\right)^2-\dfrac{b^2-4ac}{4a}& 
		x\neq&\dfrac{b}{2a}\\
		\intertext{Il vertice della parabola ha ordinata}
		y_v=&-\dfrac{b^2-4ac}{4a}
		\intertext{Se $a>0$ }
		a\left(x+\dfrac{b}{2a}\right)^2>&0\\
		\intertext{Quindi}
		a\left(x+\dfrac{b}{2a}\right)^2-\dfrac{b^2-4ac}{4a}>&-\dfrac{b^2-4ac}{4a}\\
		\intertext{Analogamente}
		\intertext{Se $a<0$ }
		a\left(x+\dfrac{b}{2a}\right)^2<&0\\
		\intertext{Quindi}
		a\left(x+\dfrac{b}{2a}\right)^2-\dfrac{b^2-4ac}{4a}<&-\dfrac{b^2-4ac}{4a}\\
	\end{align*}
	Da cui la tesi
\end{proof}
\subsection{Tangente formula di sdoppiamento}
\begin{thm}[Formula di sdoppiamento]\label{thm:Formulasdoppiamento_parab}
	Data una parabola~\cite{ReFraschini2008} $y=ax^2+bx+c$ e $P(x_0,y_0)$ un 
	suo punto. Allora la retta tangente alla parabola per $P$ ha equazione \[ 
	\dfrac{y+y_0}{2}=ax_0x+b\left(\dfrac{x+x_0}{2}\right)+c \] 
\end{thm}\index{Parabola!tangente}
\begin{proof}
	Poniamo a sistema la parabola con il fascio di rette di centro $P$ 
	\begin{align*}
		&\begin{cases}
			y=ax^2+bx+c\\y-y_0=m(x-x_0)
		\end{cases}
		\intertext{Otteniamo}
		ax^2+bx+c-y_0-mx+mx_0=&0\\
		ax^2+(b-m)x+c-y_0+mx_0=&0\\
		\intertext{Dato che $P$ è un punto di tangenza, le soluzioni sono 
		coincidenti, quindi la somma delle soluzioni è}
		-\dfrac{b-m}{a}=&2x_0\\
		m=&2ax_0+b\\
		\intertext{Sostituendo nell'equazione del fascio di rette }
		y-y_0=&(2ax_0+b)(x-x_0)\\
		y=&2ax_0x-2ax_0^2+bx-bx_0+y_0\\
		\intertext{Il punto $P$ appartiene alla parabola quindi}
		y=&2ax_0x-2ax_0^2+bx-bx_0+ax_0^2+bx_0+c\\
		\intertext{Dividendo per due ambi i membri}
		\dfrac{y}{2}=&ax_0x-ax_0^2+\dfrac{b}{2}x+\dfrac{a}{2}x_0^2+\dfrac{c}{2}\\
		\intertext{Aggiungendo a sinistra e a destra $\dfrac{x_0+y_0}{2}$ 
		otteniamo}
		\dfrac{y}{2}+\dfrac{x_0+y_0}{2}=&ax_0x-ax_0^2+\dfrac{b}{2}x+\dfrac{a}{2}x_0^2+\dfrac{c}{2}+\dfrac{x_0+y_0}{2}\\
		\dfrac{y}{2}+\dfrac{y_0}{2}=&-\dfrac{x_0}{2}+ax_0x-ax_0^2+\dfrac{b}{2}x+\dfrac{a}{2}x_0^2+\dfrac{c}{2}+\dfrac{x_0}{2}+\dfrac{y_0}{2}\\
		\intertext{Semplificando}
		\dfrac{y+y_0}{2}=&ax_0x-ax_0^2+\dfrac{b}{2}x+\dfrac{a}{2}x_0^2+\dfrac{c}{2}+\dfrac{y_0}{2}\\
		\intertext{Il punto $P$ appartiene alla parabola quindi}
		\dfrac{y+y_0}{2}=&ax_0x-ax_0^2+\dfrac{b}{2}x+\dfrac{a}{2}x_0^2+\dfrac{c}{2}+\dfrac{a}{2}x_0^2+\dfrac{b}{2}x_0+\dfrac{c}{2}\\
		\intertext{Semplificando}
		\dfrac{y+y_0}{2}=&ax_0x+\dfrac{b}{2}x+\dfrac{b}{2}x_0+c\\
		\dfrac{y+y_0}{2}=&ax_0x+b\dfrac{x+x_0}{2}+c\\
	\end{align*}
	Da cui la tesi.
\end{proof}
\section{Circonferenza}
\section{Ellisse}
\section{Iperbole}
\section{Funzione omografica}
\begin{defn}[Funzione 
omografica]\index{Funzione!omografica}\label{eqn:funzioneomograficadef}
Diremo funzione omografica il forma normale la funzione\begin{equation}
	y=\dfrac{ax+b}{cx+d}
\end{equation}
\end{defn}
\begin{commento}
Consideriamo la~\vref{eqn:funzioneomograficadef},\ se $c=0$ questa diventa 
\begin{equation*}
	y=\dfrac{ax+b}{d}=\dfrac{a}{d}x+\dfrac{b}{d}
\end{equation*}
Che è l'equazione di un retta di coefficiente angolare $m=\dfrac{a}{d}$ e che 
passa per il punto $P(0,\dfrac{b}{d})$

Se $ad=bc$ possiamo scrivere $d=\dfrac{bc}{a}$
quindi \begin{align*}
y=&\dfrac{ax+b}{cx+d}\\
=&\dfrac{ax+b}{cx+\dfrac{bc}{a}}\\
=&\dfrac{a}{c}\dfrac{ax+b}{ax+b}\\
\end{align*}
$y=\dfrac{a}{c}$ è una retta parallela all'asse delle ascisse.
\end{commento}