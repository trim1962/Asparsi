% !TeX root = Asparsi.tex
% !BIB TS-program = biber
% !TeX encoding = UTF-8
% !TeX spellcheck = it_IT
\chapter{Goniometria}
\section{Formule parametriche}
\begin{thm}[Formule parametriche]\label{thm:formuleparametriche1}
	Se $\alpha$ è un angolo allora
	\begin{align*}
\sin\alpha=&\frac{2t}{1+t^2}\\
\cos\alpha=&\frac{1-t^2}{1+t^2}\\
t=&\tan\frac{\alpha}{2}&\alpha\neq\pi+2k\pi
	\end{align*}
\end{thm}
\begin{proof} Dalla somma di due angoli otteniamo
	\begin{align*}
	\intertext{Prima relazione}
	\sin2\alpha=&2\sin\alpha\cos\alpha\\
	=&\frac{2\sin\alpha\cos\alpha}{1}\\
	=&\frac{2\sin\alpha\cos\alpha}{\cos^2\alpha+\sin^2\alpha}\\
	=&\dfrac{\dfrac{2\sin\alpha\cos\alpha}{\cos^2\alpha}}{\dfrac{\cos^2\alpha+\sin^2\alpha}{cos^2\alpha}}&\alpha\neq\frac{\pi}{2}+k\pi\\
	=&\dfrac{\dfrac{2\sin\alpha}{\cos\alpha}}{\dfrac{\cos^2\alpha}{\cos^2\alpha}+\dfrac{\sin^2\alpha}{cos^2\alpha}}\\
	=&\dfrac{2\tan\alpha}{1+\tan^2\alpha}\\
	\alpha&\longmapsto\dfrac{\alpha}{2}\\
	\sin\alpha=&\dfrac{2\tan\dfrac{\alpha}{2}}{1+\tan^2\dfrac{\alpha}{2}}\\
	t=&\tan\frac{\alpha}{2}\\
	\sin\alpha=&\frac{2t}{1+t^2}\\
	\intertext{Seconda relazione}
	\cos 2\alpha=&\cos^2\alpha-\sin^2\alpha\\
	=&\dfrac{\cos^2\alpha-\sin^2\alpha}{1}\\
	=&\frac{\cos^2\alpha-\sin^2\alpha}{\cos^2\alpha+\sin^2\alpha}\\
	=&\frac{\dfrac{\cos^2\alpha-\sin^2\alpha}{\cos^2\alpha}}{\dfrac{\cos^2\alpha+\sin^2\alpha}{\cos^2\alpha}}&\alpha\neq\frac{\pi}{2}+k\pi\\ 
	=&\frac{1-\dfrac{\sin^2\alpha}{\cos^2\alpha}}{1+\dfrac{\sin^2\alpha}{\cos^2\alpha}}\\
	=&\frac{1-\tan^2\alpha}{1+\tan^2\alpha}\\
	\alpha&\longmapsto\dfrac{\alpha}{2}\\
	\cos 2\alpha=&\frac{1-\tan^2\dfrac{\alpha}{2}}{1+\tan^2\dfrac{\alpha}{2}}\\
	t=&\tan\frac{\alpha}{2}\\
	\cos\alpha=&\frac{1-t^2}{1+t^2}
	\end{align*}
\end{proof}