% !TeX root = Asparsi.tex
% !BIB TS-program = biber
% !TeX encoding = UTF-8
% !TeX spellcheck = it_IT
\chapter{Angoli valori numerici}\label{chap:Angoli_valori_numerici}
%\section{$\alpha=\ang{7.5}$}
\section{$\ang{7.5}$}
\begin{align*}
	\intertext{Utilizzando il~\vref{cor:Cosenoangolometa}}
	\cos^2\dfrac{\alpha}{2}=&\dfrac{1}{2}(\cos\alpha+1)\\
	\cos^2\ang{7.5}=&\dfrac{1}{2}(\cos\ang{15}+1)\\ 
	=&\dfrac{1}{2}(\dfrac{\sqrt{2}+\sqrt{6}}{4}+1)\\
	=&\dfrac{1}{8}(\sqrt{2}+\sqrt{6}+4)\\
	\cos\ang{7.5}=&\sqrt{\dfrac{1}{8}(\sqrt{2}+\sqrt{6}+4)}\\
\end{align*}
\begin{align*}
\intertext{Utilizzando il~\vref{cor:Senoangolometacoseno}}
\sin^2\dfrac{\alpha}{2}=&\dfrac{1}{2}(1-\cos\alpha)\\ 
\sin^2\ang{7.5}=&\dfrac{1}{2}(1-\dfrac{\sqrt{2}+\sqrt{6}}{4})\\ 
=&\dfrac{1}{2}(\dfrac{4-\sqrt{2}-\sqrt{6}}{4})\\ 
=&\dfrac{1}{8}(4-\sqrt{2}-\sqrt{6})\\ 
\sin\ang{7.5}=&\sqrt{\dfrac{1}{8}(4-\sqrt{2}-\sqrt{6})}
\end{align*}
\begin{align*}
	\tan\dfrac{\alpha}{2}=&\dfrac{\sin\dfrac{\alpha}{2}}{\cos\dfrac{\alpha}{2}}\\
	=&\dfrac{2\sin\dfrac{\alpha}{2}\sin\dfrac{\alpha}{2}}{2\cos\dfrac{\alpha}{2}\sin\dfrac{\alpha}{2}}\\
	=&\dfrac{2\sin^2\dfrac{\alpha}{2}}{2\cos\dfrac{\alpha}{2}\sin\dfrac{\alpha}{2}}\\
	\intertext{ma}
	2\sin^2\dfrac{\alpha}{2}=&1-\cos\alpha\\
	2\cos\dfrac{\alpha}{2}\sin\dfrac{\alpha}{2}=&\sin\alpha\\
	\tan\dfrac{\alpha}{2}=&\dfrac{1-\cos\alpha}{\sin\alpha}\\
\end{align*}
\begin{align*}
	\intertext{Quindi}
	\tan\ang{7.5}=&\dfrac{1-\cos\ang{15}}{\sin\ang{15}}\\
	=&\dfrac{1-\dfrac{\sqrt{2}+\sqrt{6}}{4}}{\dfrac{\sqrt{6}-\sqrt{2}}{4}}\\
	=&\dfrac{\dfrac{4-\sqrt{2}-\sqrt{6}}{4}}{\dfrac{\sqrt{6}-\sqrt{2}}{4}}\\
	=&\dfrac{4-\sqrt{2}-\sqrt{6}}{4}\dfrac{4}{\sqrt{6}-\sqrt{2}}\\
	=&\dfrac{4-\sqrt{2}-\sqrt{6}}{\sqrt{6}-\sqrt{2}}\\
	=&\dfrac{4-\sqrt{2}-\sqrt{6}}{\sqrt{6}-\sqrt{2}}\dfrac{\sqrt{6}+\sqrt{2}}{\sqrt{6}+\sqrt{2}}\\
	=&\dfrac{4\sqrt{6}+4\sqrt{2}-4\sqrt{3}-8}{4}\\
	=&\sqrt{6}+\sqrt{2}-\sqrt{3}-2
\end{align*}
\begin{align*}
	\cot\ang{7.5}=&\dfrac{1}{\tan\ang{7.5}}\\
	=&\dfrac{1}{\sqrt{6}+\sqrt{2}-\sqrt{3}-2}\\
	=&\dfrac{1}{\sqrt{6}+\sqrt{2}-\sqrt{3}-2}\dfrac{\sqrt{6}+\sqrt{2}+\sqrt{3}+2}{\sqrt{6}+\sqrt{2}+\sqrt{3}+2}\\
	=&\dfrac{\sqrt{6}+\sqrt{2}+\sqrt{3}+2}{1}\\
	=&\sqrt{6}+\sqrt{2}+\sqrt{3}+2\\
\end{align*}
\section{$\ang{9}$}
\begin{align*}
	\cos\ang{9}=&\sqrt{\dfrac{1+cos\ang{18}}{2}}\\
	=&\sqrt{\dfrac{1+\dfrac{\sqrt{10+2\sqrt{5}}}{4}}{2}}\\
	=&\sqrt{\dfrac{\dfrac{4+\sqrt{10+2\sqrt{5}}}{4}}{2}}\\
	=&\sqrt{\dfrac{4+\sqrt{10+2\sqrt{5}}}{8}}\\
	=&\sqrt{\dfrac{8+2\sqrt{10+2\sqrt{5}}}{16}}\\
	=&\sqrt{\dfrac{(5-\sqrt{5})+(3+\sqrt{5})+2\sqrt{10+2\sqrt{5}}}{16}}\\
	=&\sqrt{\dfrac{\left(\sqrt{5-\sqrt{5}}\right)^2+\left(\sqrt{3+\sqrt{5}}\right)^2+2\sqrt{(15+5\sqrt{5})-(5+3\sqrt{5})}}{16}}\\
	=&\sqrt{\dfrac{\left(\sqrt{5-\sqrt{5}}\right)^2+\left(\sqrt{3+\sqrt{5}}\right)^2+2\sqrt{5(3+\sqrt{5})-\sqrt{5}(\sqrt{5}+3)}}{16}}\\
	=&\sqrt{\dfrac{\left(\sqrt{5-\sqrt{5}}\right)^2+\left(\sqrt{3+\sqrt{5}}\right)^2+2\sqrt{(3+\sqrt{5})(5-\sqrt{5})}}{16}}\\
	=&\sqrt{\dfrac{\left(\sqrt{5-\sqrt{5}}+\sqrt{3+\sqrt{5}}\right)^2}{16}}\\
	=&\dfrac{\sqrt{5-\sqrt{5}}+\sqrt{3+\sqrt{5}}}{4}\\
\end{align*}
\begin{align*}
	\sin\ang{9}=&\sqrt{\dfrac{1-cos\ang{18}}{2}}\\
	=&\sqrt{\dfrac{1-\dfrac{\sqrt{10+2\sqrt{5}}}{4}}{2}}\\
	=&\sqrt{\dfrac{\dfrac{4-\sqrt{10+2\sqrt{5}}}{4}}{2}}\\
	=&\sqrt{\dfrac{4-\sqrt{10+2\sqrt{5}}}{8}}\\
	=&\sqrt{\dfrac{8-2\sqrt{10+2\sqrt{5}}}{16}}\\
	=&\sqrt{\dfrac{(5-\sqrt{5})+(3+\sqrt{5})+2\sqrt{10-2\sqrt{5}}}{16}}\\
	=&\sqrt{\dfrac{\left(\sqrt{5-\sqrt{5}}\right)^2+\left(\sqrt{3+\sqrt{5}}\right)^2-2\sqrt{(15+5\sqrt{5})-(5+3\sqrt{5})}}{16}}\\
	=&\sqrt{\dfrac{\left(\sqrt{5-\sqrt{5}}\right)^2+\left(\sqrt{3+\sqrt{5}}\right)^2-2\sqrt{5(3+\sqrt{5})-\sqrt{5}(\sqrt{5}+3)}}{16}}\\
	=&\sqrt{\dfrac{\left(\sqrt{5-\sqrt{5}}\right)^2+\left(\sqrt{3+\sqrt{5}}\right)^2-2\sqrt{(3+\sqrt{5})(5-\sqrt{5})}}{16}}\\
	=&\sqrt{\dfrac{\left(\sqrt{5-\sqrt{5}}-\sqrt{3+\sqrt{5}}\right)^2}{16}}\\
	=&\dfrac{\sqrt{5-\sqrt{5}}-\sqrt{3+\sqrt{5}}}{4}\\
\end{align*}
\section{$\ang{15}$}
\begin{align*}
	\cos\ang{15}=&\cos(\ang{60}-\ang{45})\\
	=&\difcos{60}{45}\\
	=&\cossessanta\cosquarantacinque+\sinsessanta\sinquarantecinque\\
	=&\dfrac{\sqrt{2}+\sqrt{6}}{4}\\
\end{align*}
\begin{align*}
	\sin\ang{15}=&\sin(\ang{60}-\ang{45})\\
	=&\difsin{60}{45}\\
	=&\sinsessanta\cosquarantacinque-\cossessanta\sinquarantecinque\\
	=&\dfrac{\sqrt{6}-\sqrt{2}}{4}
\end{align*}
\begin{align*}
	\tan\ang{15}=&\dfrac{\sin\ang{15}}{\cos\ang{15}}\\
	=&\dfrac{\dfrac{\sqrt{6}-\sqrt{2}}{4}}{\dfrac{\sqrt{2}+\sqrt{6}}{4}}\\
	=&\dfrac{\sqrt{6}-\sqrt{2}}{4}\dfrac{4}{\sqrt{2}+\sqrt{6}}\\
	=&\dfrac{\sqrt{6}-\sqrt{2}}{\sqrt{2}+\sqrt{6}}\\
	=&\dfrac{\sqrt{6}-\sqrt{2}}{\sqrt{2}+\sqrt{6}}\dfrac{\sqrt{6}-\sqrt{2}}{\sqrt{2}-\sqrt{6}}\\
	=&\dfrac{8-2\sqrt{12}}{6-2}\\
	=&2-\sqrt{3}
\end{align*}
\begin{align*}
	\cot\ang{15}=&\dfrac{1}{\tan\ang{15}}\\
	=&\dfrac{1}{2-\sqrt{3}}\\
	=&\dfrac{1}{2-\sqrt{3}}\dfrac{2+\sqrt{3}}{2+\sqrt{3}}\\
	=2+\sqrt{3}\\
\end{align*}
\section{$\ang{18}$}
\begin{align*}
	\sin\ang{18}\\
	a=&\ang{18}\\
	5a=&\ang{90}\\
	2a+3a=&\ang{90}\\
	2a=&\ang{90}-3a\\
	\sin 2a=&\sin(\ang{90}-3a)=\cos 3a\\
	\intertext{trasformiamo l'espressione}
	\sin 2a=&2\sin a\cos a\\
	\cos 3a=&\cos2a\cos a-\sin 2a\sin a\\
	 =&\cos a(2\cos^2 a -1)-2\sin^2 a\cos a\\
	 =&\cos^3 a -\cos a -2(1-cos^2 a)\cos a\\
	 =&\cos^3 a -\cos a -2\cos a +cos^3 a\\
	 =&4\cos^3 a-3\cos a
	 \intertext{Riprendendo}
	 2\sin a\cos a=&4\cos^3 a-3\cos a
	 \intertext{$\cos a\neq 0$}
	 2\sin a=&4\cos^2 a-3 \\
	 2\sin a-4\cos^2 a+3=&0 \\
	 2\sin a-4(1-\sin^2 a) a+3=&0 \\
	 4\sin^2 a+2\sin a-1=&0 \\
	 \sin a=&\dfrac{-2\pm\sqrt{4+16}}{8}\\
	 =&\dfrac{-2\pm\sqrt{20}}{8}\\
	 =&\dfrac{-2\pm2\sqrt{5}}{8}\\
	 =&\dfrac{-1\pm\sqrt{5}}{4}\\
	 \intertext{ma $\sin\ang{18}>0$}
	 \sin\ang{18}=&\dfrac{-1+\sqrt{5}}{4}
	\end{align*}
\begin{align*}
	 \cos a=&\pm\sqrt{1-\sin^2a}\\
	 \intertext{ma l'angolo è nel primo quadrante quindi}
	 \cos \ang{18}=&\sqrt{1-\left(\dfrac{-1+\sqrt{5}}{4}\right)^2}\\	
	 =&\sqrt{\dfrac{16-6+2\sqrt{5}}{16}}\\
	 =&\dfrac{\sqrt{10+2\sqrt{5}}}{4}
	\end{align*}
\begin{align*}
	 \tan\ang{18}=&\dfrac{\sin\ang{18}}{\cos\ang{18}}\\
	 =&\dfrac{\dfrac{-1+\sqrt{5}}{4}}{\dfrac{\sqrt{10+2\sqrt{5}}}{4}}\\
	 =&\dfrac{-1+\sqrt{5}}{4}\dfrac{4}{\sqrt{10+2\sqrt{5}}}\\
	 =&\dfrac{-1+\sqrt{5}}{\sqrt{10+2\sqrt{5}}}\\
	 =&\dfrac{-1+\sqrt{5}}{\sqrt{10+2\sqrt{5}}}\dfrac{\sqrt{10-2\sqrt{5}}}{\sqrt{10-2\sqrt{5}}}\\
	 &=\dfrac{\sqrt{(-1+\sqrt{5})^2(\sqrt{10-2\sqrt{5}})}}{\sqrt{100-20}}\\
	 &=\dfrac{\sqrt{(6-2\sqrt{5})(\sqrt{10-2\sqrt{5}})}}{\sqrt{80}}\\
	 =&\sqrt{\dfrac{60-12\sqrt{5}-20\sqrt{5}+20}{80}}\\
	 =&\sqrt{\dfrac{80-32\sqrt{5}}{80}}\\
	 =&\sqrt{\dfrac{5-2\sqrt{5}}{5}}
	 	\end{align*}
 	 \begin{align*}
 	 	\cot\ang{18}=&\dfrac{1}{\sqrt{\dfrac{5-2\sqrt{5}}{5}}}\\
 	 	=&\sqrt{\dfrac{5}{5-2\sqrt{5}}}\\
 	 	=&\sqrt{\dfrac{5}{5-2\sqrt{5}}}\sqrt{\dfrac{5+2\sqrt{5}}{5+2\sqrt{5}}}\\
 	 	=&\sqrt{\dfrac{5(5+2\sqrt{5})}{25-20}}\\
 	 	=&\sqrt{5+2\sqrt{5}}
\end{align*}
\section{$\ang{22;30;}$}
\begin{align*}
	\intertext{Dal~\vref{cor:Cosenoangolometa}}
	\cos^2\ang{22;30;}=&\dfrac{1}{2}\left(\cos\ang{45}+1\right)\\
	=&\dfrac{1}{2}\left(\dfrac{\sqrt{2}}{2}+1\right)\\
	=&\left(\dfrac{\sqrt{2}+2}{4}\right)\\
	\cos^2\ang{22;30;}=&\dfrac{\sqrt{2}+2}{4}\\
	\cos\ang{22;30;}=&\pm\sqrt{\dfrac{\sqrt{2}+2}{4}}
	\intertext{ma $\cos\ang{22;30;}>0$ }
	\cos\ang{22;30;}=&\dfrac{\sqrt{\sqrt{2}+2}}{2}
\end{align*}
\begin{align*}
	\intertext{Dal~\vref{cor:Senoangolometa}}
	\sin^2\ang{22;30;}=&\dfrac{1}{2}\left(1-\cos\ang{45}\right)\\
	=&\left(\dfrac{2-\sqrt{2}}{4}\right)\\
	\sin\ang{22;30;}=&\pm\sqrt{\dfrac{2-\sqrt{2}}{4}}
	\intertext{ma $\sin\ang{22;30;}>0$ }
	\cos\ang{22;30;}=&\dfrac{\sqrt{2- \sqrt{2}}}{2}
\end{align*}
\begin{align*}
	\tan\ang{22;30;}=&\dfrac{\sin\ang{22;30;}}{\cos\ang{22;30;}}\\
	=&\dfrac{\sqrt{2- \sqrt{2}}}{\sqrt{2+ \sqrt{2}}}\\
		=&\dfrac{\sqrt{2- \sqrt{2}}}{\sqrt{2+ \sqrt{2}}}\dfrac{\sqrt{2+\sqrt{2}}}{\sqrt{2+ \sqrt{2}}}\\
=&\dfrac{\sqrt{(2- \sqrt{2})(2+ \sqrt{2})}}{\sqrt{(2+ \sqrt{2})^2}}\\	
=&\dfrac{\sqrt{2}}{2+ \sqrt{2}}\\	
=&\dfrac{\sqrt{2}}{2+ \sqrt{2}}\dfrac{2-\sqrt{2}}{2-\sqrt{2}}\\
=&\dfrac{2\sqrt{2}-2}{4-2}\\
=&\sqrt{2}-1
\end{align*}
\begin{align*}
	\cot\ang{22;30;}=&\dfrac{1}{\tan\ang{22;30;}}\\
	=&\dfrac{1}{\sqrt{2}-1}\\
	=&\dfrac{1}{\sqrt{2}-1}\dfrac{\sqrt{2}+1}{\sqrt{2}+1}\\
	=&\dfrac{\sqrt{2}+1}{1}\\
	=&\sqrt{2}+1
\end{align*}
\section{$\ang{30}$}
Consideriamo la~\vref{fig:triangolosenotrenta}. Poniamo che l'angolo $BOC$ sia di ampiezza di \ang{30}. Ottengo il triangolo $OBA$. Ribalto la figura $OBA$ lungo il lato $OA$. Il triangolo $OBC$ è isoscele dato che ha i lati $OB$ e$OC$ uguali al raggio $R$. $OA$ è bisettrice dell'angolo $BOC$ e quindi è anche mediana ed altezza per il lato $BC$. Segue che l'angolo $BAO$ è retto. Se è retto allora l'angolo in $B$ è di \ang{60}. Segue che l'angolo in $C$ è di \ang{60}. Segue che il triangolo $BOC$ è equiangolo e quindi equilatero. Per cui il lato $AB$ è metà del raggio. Quindi 
\begin{align*}
	\sin\ang{30}=&\dfrac{AB}{OB}\\
	=&\dfrac{\dfrac{1}{2}R}{R}\\
	=&\dfrac{1}{2}\dfrac{R}{R}\\
	=&\dfrac{1}{2}
\end{align*}
\begin{align*}
	\cos\ang{30}=&\dfrac{OA}{OB}\\
	\intertext{ma}
	OA=&\sqrt{OB^2-AB^2}\\
	=&\sqrt{R^2-\dfrac{R^2}{4}}\\
	=&\dfrac{R}{2}\sqrt{3}
	\intertext{riprendendo}
	\cos\ang{30}=&\dfrac{R}{2}\sqrt{3}\dfrac{1}{R}\\
	=&\dfrac{\sqrt{3}}{2}
\end{align*}
\begin{align*}
	\sin\ang{30}=&2\sin\ang{15}\cos\ang{15}\\
	=&2\dfrac{\sqrt{6}+\sqrt{2}}{4}\dfrac{\sqrt{6}-\sqrt{2}}{4}\\
	=&2\dfrac{6-2}{16}\\
	=&\dfrac{1}{2}
\end{align*}
\begin{align*}
	\cos\ang{30}=&\cos^2\ang{15}-\sin\ang{15}\\
	=&\left(\dfrac{\sqrt{6}+\sqrt{2}}{4}\right)^2-\left(\dfrac{\sqrt{6}-\sqrt{2}}{4}\right)^2\\
	=&\dfrac{6+2+2\sqrt{12}}{16}-\dfrac{6+2-2\sqrt{12}}{16}\\
	=&\dfrac{4\sqrt{12}}{16}\\
	=&\dfrac{\sqrt{3}}{2}
\end{align*}
\begin{figure}
	\centering
	\includestandalone[width=0.3\linewidth]{geometria/triangolosenotrenta}
	\caption{$\alpha=\ang{30}$}
	\label{fig:triangolosenotrenta}
\end{figure}
\begin{align*}
	\tan\ang{30}=&\dfrac{\sin\ang{30}}{cos\ang{30}}\\
	=&\dfrac{\dfrac{1}{2}}{\dfrac{\sqrt{3}}{2}}\\
	=&\dfrac{1}{2}\dfrac{2}{\sqrt{3}}\\
		=&\dfrac{1}{\sqrt{3}}\\
		=&\dfrac{\sqrt{3}}{3}\\
\end{align*}
\begin{align*}
	\cot\ang{30}=&\dfrac{1}{\tan\ang{30}}\\
	=&\dfrac{1}{\dfrac{\sqrt{3}}{3}}\\
	=&\dfrac{3}{\sqrt{3}}\\
	=&\sqrt{3}
\end{align*}
\section{$\ang{36}$}
\begin{align*}
	\sin\ang{36}=&2\sin\ang{18}\cos\ang{18}\\
	=&2\dfrac{-1+\sqrt{5}}{4}\dfrac{\sqrt{10+2\sqrt{5}}}{4}\\
	=&\dfrac{\sqrt{(-1+\sqrt{5})^2(10+2\sqrt{5})}}{8}\\
	=&\dfrac{\sqrt{(6-2\sqrt{5})(10+2\sqrt{5})}}{8}\\
	=&\dfrac{\sqrt{60+12\sqrt{5}-20\sqrt{5}-20}}{8}\\
	=&\dfrac{\sqrt{40-8\sqrt{5}}}{8}\\
	=&\dfrac{\sqrt{10-2\sqrt{5}}}{4}\\
\end{align*}
\begin{align*}
	\cos\ang{36}=&\cos^2\ang{18}-\sin^2\ang{18}\\
	=&\left(\dfrac{\sqrt{10+2\sqrt{5}}}{4}\right)^2-\left(\dfrac{-1+\sqrt{5}}{4}\right)^2\\
	=&\dfrac{10+2\sqrt{5}}{16}-\dfrac{5+1-2\sqrt{5}}{16}\\
	=&\dfrac{10+2\sqrt{5}-1-5+2\sqrt{5}}{16}\\
		=&\dfrac{4+4\sqrt{5}}{16}\\
		=&\dfrac{1+\sqrt{5}}{4}\\
\end{align*}
\begin{align*}
	\tan\ang{36}=&\dfrac{\sin\ang{36}}{\cos\ang{36}}\\
	=&\dfrac{\dfrac{\sqrt{10-2\sqrt{5}}}{4}}{\dfrac{1+\sqrt{5}}{4}}\\
%	=&\dfrac{\sqrt{10-2\sqrt{5}}}{4}\dfrac{4}{1+\sqrt{5}}\\
	=&\dfrac{\sqrt{10-2\sqrt{5}}}{1+\sqrt{5}}\\
	=&\dfrac{\sqrt{10-2\sqrt{5}}}{1+\sqrt{5}}\dfrac{\sqrt{5}-1}{\sqrt{5}-1}\\
	=&\dfrac{\sqrt{(10-2\sqrt{5})(\sqrt{5}-1)^2}}{4}\\
	=&\dfrac{\sqrt{(10-2\sqrt{5})(6-2\sqrt{5})}}{4}\\
	=&\dfrac{\sqrt{84-32\sqrt{5}}}{4}\\
	=&\sqrt{5-2\sqrt{5}}
\end{align*}
\begin{align*}
	\cot\ang{36}=&\dfrac{1}{\tan\ang{36}}\\
	=&\dfrac{1}{\sqrt{5-2\sqrt{5}}}\\
	=&\dfrac{\sqrt{5-2\sqrt{5}}}{5-2\sqrt{5}}\\
	=&\dfrac{\sqrt{5-2\sqrt{5}}}{5-2\sqrt{5}}\dfrac{5+2\sqrt{5}}{5+2\sqrt{5}}\\
	=&\dfrac{\sqrt{(5-2\sqrt{5})(5+2\sqrt{5})^2}}{5}\\
	=&\dfrac{\sqrt{(5-2\sqrt{5})(45+20\sqrt{5})}}{5}\\
	=&\dfrac{\sqrt{25+10\sqrt{5}}}{5}\\
\end{align*}
\section{$\ang{45}$}
\begin{figure}
	\centering
	\includestandalone[width=0.3\linewidth]{geometria/triangolosenoquaratacinque}
	\caption{$\alpha=\ang{45}$}
	\label{fig:triangolosenoquaratacinque}
\end{figure}
Consideriamo la~\vref{fig:triangolosenoquaratacinque}. Poniamo che l'angolo $BOC$ sia di ampiezza di \ang{45}. Ottengo il triangolo $OBA$. Ribalto la figura $OBA$ lungo il lato $OA$. Il triangolo $OBC$ è isoscele dato che ha i lati $OB$ e$OC$ uguali al raggio $R$. $OA$ è bisettrice dell'angolo $BOC$ e quindi è anche mediana ed altezza per il lato $BC$. Segue che l'angolo $BAO$ è retto. Se è retto allora l'angolo in $B$ è di \ang{45}. Segue che l'angolo in $C$ è di \ang{45}. Segue che il triangolo $BOA$ è isoscele. Per cui il lato $AB$ è uguale a $OA$. Poniamo $AB=a$. Quindi
\begin{align*}
	R^2=&OA^2+AB^2\\
	R^2=&a^2+a^2\\
	R=&a\sqrt{2}\\
	\sin\ang{45}=&\dfrac{AB}{OB}\\
	=&\dfrac{a}{a\sqrt{2}}\\
	=&\dfrac{\sqrt{2}}{2}\\
	\cos\ang{45}=&\dfrac{OA}{OB}\\
	=&\dfrac{a}{a\sqrt{2}}\\
	=&\dfrac{\sqrt{2}}{2}
\end{align*} 
\begin{align*}
	2\cos^2\dfrac{\alpha}{2}=&\cos\alpha+1\\
	2\cos^2\ang{45}=&\cos\ang{90}+1\\
	2\cos^2\ang{45}=&1\\
	\cos\ang{45}=&\pm\sqrt{\dfrac{1}{2}}\\
	\intertext{ma $\cos\ang{45}>0$}
	\cos\ang{45}=&\dfrac{\sqrt{2}}{2}\\
\end{align*}
\begin{align*}
	2\sin^2\dfrac{\alpha}{2}=&1-\cos\alpha\\
	2\sin^2\ang{45}=&1-\sin\ang{90}\\
	2\sin^2\ang{45}=&1\\
	\sin\ang{45}=&\pm\sqrt{\dfrac{1}{2}}\\
	\intertext{ma $\sin\ang{45}>0$}
	\sin\ang{45}=&\dfrac{\sqrt{2}}{2}\\
\end{align*}
\begin{align*}
	\tan\ang{45}=&\dfrac{\sin\ang{45}}{\cos\ang{45}}\\
	=&\dfrac{\dfrac{\sqrt{2}}{2}}{\dfrac{\sqrt{2}}{2}}\\
	=&1
\end{align*}
\begin{align*}
	\cot\ang{45}=&\dfrac{\cos\ang{45}}{\sin\ang{45}}\\
	=&\dfrac{\dfrac{\sqrt{2}}{2}}{\dfrac{\sqrt{2}}{2}}\\
	=&1
\end{align*}
\section{$\ang{54}$}
\begin{align*}
	\cos\ang{54}=&\sin(\ang{90}-\ang{36})\\
	=&\sin\ang{36}\\
	=&\dfrac{\sqrt{10-2\sqrt{5}}}{4}\\
\end{align*}
\begin{align*}
\cos\ang{54}=&\cos(3\cdot\ang{18})
\intertext{utilizzando il~\vref{thm:Formulatriplicazionecoseno} otteniamo}
\cos 3x=&4cos^3 x-3\cos x\\
=&\cos\ang{18}(4\cos^2 \ang{18}-3)\\
=&\dfrac{\sqrt{10+2\sqrt{5}}}{4}\left(4\dfrac{10+2\sqrt{5}}{16}-3\right)\\
=&\dfrac{\sqrt{10+2\sqrt{5}}}{4}\left(\dfrac{10+2\sqrt{5}-12}{4}\right)\\
=&\dfrac{\sqrt{10+2\sqrt{5}}}{4}\left(\dfrac{\sqrt{5}-1}{2}\right)\\
=&\dfrac{\sqrt{(10+2\sqrt{5})(\sqrt{5}-1)^2}}{8}\\
=&\dfrac{\sqrt{(10+2\sqrt{5})(6-2\sqrt{5})}}{8}\\
=&\dfrac{\sqrt{40-8\sqrt{5}}}{8}\\
=&\dfrac{\sqrt{10-2\sqrt{5}}}{4}
\end{align*}
\begin{align*}
	\sin\ang{54}=&\cos(\ang{90}-\ang{54})\\
	=&\cos\ang{36}\\
	=&\dfrac{1+\sqrt{5}}{4}\\
\end{align*}
\begin{align*}
	\sin\ang{54}=&\sin(3\cdot\ang{18})
	\intertext{utilizzando il~\vref{thm:Formulatriplicazioneseno} otteniamo}
\sin 3x=&3\sin x-4\sin^3 x\\
=&\sin x(3-4sin^2 x)\\
=&\dfrac{-1+\sqrt{5}}{4}\left(3-4\dfrac{6-2\sqrt{5}}{16}\right)\\
=&\dfrac{-1+\sqrt{5}}{4}\left(\dfrac{6+2\sqrt{5}}{4}\right)\\
=&\dfrac{4+4\sqrt{5}}{16}\\
=&\dfrac{1+\sqrt{5}}{4}
\end{align*}
\begin{align*}
	\tan\ang{54}=&\cot(\ang{90}-\ang{36})\\
	=&\cot\ang{36}\\
	=&\dfrac{\sqrt{25+10\sqrt{5}}}{5}\\
\end{align*}
\begin{align*}
	\cot\ang{36}=&\tan(\ang{90}-\ang{36})\\
	=&\tan\ang{36}\\
	=&\sqrt{5-2\sqrt{5}}
\end{align*}
\section{$\ang{60}$}
\begin{align*}
	\cos\ang{60}=&\sin(\ang{90}-\ang{60})\\
	=&\sin\ang{30}\\
	=&\dfrac{1}{2}
\end{align*}
\begin{align*}
	\sin\ang{60}=&\cos(\ang{90}-\ang{60})\\
	=&\cos\ang{30}\\
	=&\dfrac{\sqrt{3}}{2}
\end{align*}
\begin{align*}
	\tan\ang{60}=&\cot(\ang{90}-\ang{60})\\
	=&\cot\ang{30}\\
	=&\sqrt{3}
\end{align*}
\begin{align*}
	\cot\ang{60}=&\tan(\ang{90}-\ang{60})\\
	=&\tan\ang{30}\\
	=&\dfrac{\sqrt{3}}{3}\\
\end{align*}
\section{$\ang{72}$}
\begin{align*}
	\cos\ang{72}=&\sin(\ang{90}-\ang{72})\\
	=&\sin\ang{18}\\
	=&\dfrac{-1+\sqrt{5}}{4}
\end{align*}
\begin{align*}
	\sin\ang{72}=&\cos(\ang{90}-\ang{72})\\
	=&\cos\ang{18}\\
	=&\dfrac{\sqrt{10+2\sqrt{5}}}{4}
\end{align*}
\begin{align*}
	\tan\ang{72}=&\dfrac{\sin\ang{72}}{\cos\ang{72}}\\
	=&\dfrac{\cos(\ang{90}-\ang{72})}{\sin(\ang{90}-\ang{72})}\\
	=&\dfrac{\cos(\ang{18})}{\sin\ang{18}}\\	
	=&\cot\ang{18}\\
	=&\sqrt{5+2\sqrt{5}}
\end{align*}
\begin{align*}
	\cot\ang{72}=&\dfrac{\cos\ang{72}}{\sin\ang{72}}\\
	=&\dfrac{\sin(\ang{90}-\ang{72})}{\cos(\ang{90}-\ang{72})}\\
	=&\dfrac{\sin(\ang{18})}{\cos\ang{18}}\\	
	=&\tan\ang{18}\\
	=&\sqrt{\dfrac{5-2\sqrt{5}}{5}}
\end{align*}
\section{$\ang{75}$}
\begin{align*}
	\cos\ang{75}=&\cos(\ang{30}+\ang{45})\\
	=&\sumcos{30}{45}\\
	=&\costrenta\cosquarantacinque-\sintrenta\sinquarantecinque\\
	=&\dfrac{\sqrt{6}}{4}-\dfrac{\sqrt{2}}{4}\\
	=&\dfrac{\sqrt{6}-\sqrt{2}}{4}
\end{align*}
\begin{align*}
	\sin\ang{75}=&\cos(\ang{90}-\ang{75})\\
	=&\cos\ang{15}\\
	=&\dfrac{\sqrt{6}+\sqrt{2}}{4}
\end{align*}
\begin{align*}
	\tan\ang{75}=&\dfrac{\sin\ang{75}}{\cos\ang{75}}\\
	=&\dfrac{\dfrac{\sqrt{6}+\sqrt{2}}{4}}{\dfrac{\sqrt{2}-\sqrt{6}}{4}}\\
	=&\dfrac{\sqrt{6}+\sqrt{2}}{4}\dfrac{4}{\sqrt{2}-\sqrt{6}}\\
	=&\dfrac{\sqrt{6}+\sqrt{2}}{\sqrt{2}-\sqrt{6}}\\
	=&\dfrac{\sqrt{6}+\sqrt{2}}{\sqrt{2}-\sqrt{6}}\dfrac{\sqrt{6}+\sqrt{2}}{\sqrt{2}+\sqrt{6}}\\
	=&\dfrac{8+2\sqrt{12}}{6-2}\\
	=&2+\sqrt{3}
\end{align*}
\begin{align*}
	\cot\ang{75}=&\dfrac{1}{\tan\ang{75}}\\
	=&\dfrac{1}{2+\sqrt{3}}\\
	=&\dfrac{1}{2+\sqrt{3}}\dfrac{2-\sqrt{3}}{2-\sqrt{3}}\\
	=&2-\sqrt{3}\\
\end{align*}

\section{$\ang{82.5}$}

\begin{align*}
	\cos\ang{82.5}&=\sin(\ang{90}-\ang{82.5})\\
	=&\sin\ang{7.5}\\
	=&\sqrt{\dfrac{1}{8}(4-\sqrt{2}-\sqrt{6})}
\end{align*}
\begin{align*}
	\sin\ang{82.5}&=\cos(\ang{90}-\ang{82.5})\\
	=&\cos\ang{7.5}\\
	=&\sqrt{\dfrac{1}{8}(\sqrt{2}+\sqrt{6}+4)}
\end{align*}
\begin{align*}
	\tan\ang{82.5}&=\cot(\ang{90}-\ang{82.5})\\
	=&\cot\ang{7.5}\\
	=&\sqrt{6}+\sqrt{2}+\sqrt{3}+2
\end{align*}
\begin{align*}
	\cot\ang{82.5}&=\tan(\ang{90}-\ang{82.5})\\
	=&\tan\ang{7.5}\\
	=&\sqrt{6}+\sqrt{2}-\sqrt{3}-2
\end{align*}