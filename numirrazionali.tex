% !TeX root = Asparsi.tex
% !BIB TS-program = biber
% !TeX encoding = UTF-8
% !TeX spellcheck = it_IT
\chapter{Numeri irrazionali}
\section{Preliminari}
\begin{lem}\label[lem]{lem:numirra1}
	Se il quadrato di un numero è pari allora quale numero è pari\index{Numero!pari}\index{Numero!quadrato}
	\begin{equation}
	a^2=2k \Rightarrow\qquad a=2m\quad a\neq 0
	\end{equation}
\end{lem}
\begin{proof}
P.A. Se $a$ non è pari allora $a$ è dispari quindi $a=2m+1$ segue che 
\begin{align*}
a^2=&(2m+1)^2\\
=&4m^2+4m+1\\
=&4m(m+1)+1
\end{align*}
Abbiamo dimostrato che $a^2$ è dispari, assurdo per ipotesi $a^2$ era pari. Segue che  $a$ è pari.
\end{proof}
\begin{lem}\label[lem]{lem:numirra2}
	S $a$ e $b$ sono coprimi allora $a^2$ e $b^2$ sono coprimi
\end{lem}
\section{Radice di due è irrazionale}
\begin{thm}
	Radice di due è irrazionale
\end{thm}
\begin{proof}
	Se radice di due non è irrazionale allora è razionale. Quindi esistono due numeri $a$ e$b$, primi fra loro, tali che
	\begin{align*}
	\frac{a}{b}=&\sqrt{2}\\
	\frac{a^2}{b^2}=&2\\
	a^2=&2b^2
	\intertext{Quindi per il~\vref{lem:numirra1} $a$ è pari, segue}
	a=&2k\\
	4k^2=&2b^2\\
	2k^2=&b^2\\
	\intertext{Per quanto detto prima anche $b$ è pari ma $a$ e $b$ erano primi fra di loro, assurdo}
	\end{align*}
\end{proof}