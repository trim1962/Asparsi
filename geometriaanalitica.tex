\chapter{Geometria analitica}
\section{Retta}
\subsection{Rette perpendicolari}
\begin{thm}[Rette perpendicolari]\label{thm:rette_perpendicolari}
	Due rette $y=m_1x+q_1$ e $y=m_2x+q_2$ incidenti sono perpendicolari se solo se \[m_1\cdot m_2=-1\]
\end{thm}
\begin{figure}
	\centering
	\includestandalone{geometria/retteperp1}
	\caption{Rette perpendicolari}
	\label{fig:retteperp1}
\end{figure}
\begin{proof}
	Consideriamo due rette perpendicolari~\cite{Dodero1999b}, trasportiamo il centro del sistema di riferimento in modo che coincida con il punto di inteserzione tra le due rette come nella~\cref{fig:retteperp1}. Le rette $y=m_1x+q_1$ e $y=m_2x+q_2$ diventano $y=m_1x$ e $y=m_2x$. Consideriamo due punti $C$ e $D$ di ascissa uno.  Il triangolo $ABC$ è retto in $\hat{A}$ di conseguenza vale il teorema di Pitagora\index{Teorema!Pitagora}
	\begin{align*}
	\overline{BC}^2=&\overline{AB}^2+\overline{AC}^2\\
	\intertext{Dato che}
	\overline{BC}^2=&\abs{m_2-m_1}^2\\
	\overline{AB}^2=&1+m_1^2\\
	\overline{AC}^2=&1+m_2^2
	\intertext{Quindi}
	\abs{m_2-m_1}^2=&1+m_1^2+1+m_2^2\\
	m_1^2+m_2^2-2m_1m_2=&1+m_1^2+1+m_2^2\\
	-2m_1m_2=2\\
	m_1m_2=&-1\\
	\end{align*}
	Dimostriamo il viceversa. Consideriamo la~\cref{fig:retteperp1}, le rette $y=m_1x$ e $y=m_2x$ sono tali che \[m_1m_2=-1\] Poniamo che $C(1,m_1)$, $H(1,0)$ e $B(1,m_2)$ 
	
Quindi 
\begin{align*}
\abs{m_1}\cdot\abs{m_2}=&\abs{m_1\cdot m_2}=1
\intertext{$ABH$ è retto in $\hat{H}$ quindi}
\overline{AB}^2=&\overline{HB}^2+\overline{HA}^2
\intertext{$ACH$ è retto in $\hat{H}$ quindi}
\overline{AC}^2=&\overline{HC}^2+\overline{HA}^2
\intertext{Per cui}
\overline{AB}^2+\overline{AC}^2=&\overline{HC}^2+\overline{HA}^2+\overline{HB}^2+\overline{HA}^2\\
=&\overline{HC}^2+\overline{HB}^2+2\overline{HA}^2\\
\intertext{Ma $\overline{HA}^2=\overline{HC}\cdot\overline{HB}$ quindi}
=&\overline{HC}^2+\overline{HB}^2+2\overline{HC}\cdot\overline{HB}\\
=&(\overline{HC}+\overline{HB})^2\\
=&\overline{CB}^2
\intertext{Segue}
\overline{AB}^2+\overline{AC}^2=&\overline{CB}^2
\end{align*}
come si voleva dimostrare.
\end{proof}
Altra dimostrazione della prima parte del~\cref{thm:rette_perpendicolari}
\begin{thm}[Rette perpendicolari]\label{thm:rette_perpendicolari1}
	Due rette $y=m_1x+q_1$ e $y=m_2x+q_2$ incidenti sono perpendicolari se \[m_1\cdot m_2=-1\]
\end{thm}
\begin{proof}
Procediamo come il~\cref{thm:rette_perpendicolari} e consideriamo la~\cref{fig:retteperp1}. Dato che il triangolo $ABC$ è retto vale il secondo teorema di Euclide\index{Teorema!Euclide}
\begin{align*}
\overline{HC}\cdot\overline{HB}=&\overline{HA}^2
\intertext{Dato che}
\overline{HC}=&m_1\\
\overline{HB}=&-m_2\\
\overline{HA}=&1\\
\intertext{otteniamo}
-m_2\cdot m_1=1\\
m_2\cdot m_1=-1\\ 
\end{align*}
Come si voleva dimostrare
\end{proof}
\section{Varie}
\subsection{Triangolo rettangolo}
\begin{thm}[Triangolo inscritto in una semicirconferenza]\label{thm:TriangoloRettangoloSemicirconferenza}
Ogni triangolo inscritto in una semicirconferenza è rettangolo.
\end{thm}
\begin{figure}
	\centering
	\includestandalone{geometria/triagolo_rettangolo_analitco}
	\caption{Triangolo rettangolo}
	\label{fig:triangolorettangoloinscritto}
\end{figure}
\begin{proof}
Consideriamo una circonferenza $x^2+y^2=1$ di raggio unitario e centro nell'origine degli assi come la~\cref{fig:triangolorettangoloinscritto}. Avremo che $A(1;0)$, $P(x;y)$ e $C(-1;0)$. Poniamo il punto $P$ sulla semi circonferenza positiva quindi ha coordinate $P(x;\sqrt{1-x^2})$. Avremo
\begin{align*}
m_{BC}=&\dfrac{\sqrt{1-x^2}-0}{x+1}\\
m_{BA}=&\dfrac{\sqrt{1-x^2}-0}{x-1}\\
\intertext{Resta da verificare che le rette $CB$ e $BA$ siano perpendicolari quindi }
m_{BC}\cdot m_{BA}=&\dfrac{\sqrt{1-x^2}-0}{x+1}\cdot\dfrac{\sqrt{1-x^2}-0}{x-1}\\
=&\dfrac{1-x^2}{x^2-1}\\
=&-1
\end{align*}
Le due rette sono perpendicolari da cui la tesi.
\end{proof}