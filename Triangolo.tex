
% !TeX root = Asparsi.tex
% !BIB TS-program = biber
% !TeX encoding = UTF-8
% !TeX spellcheck = it_IT
\chapter{Triangolo}\label{ch:triangolo}
\begin{thm}[Posizione punto circonferenza]\label{thm:Posizione-punto-circonferenza}
Data una circonferenza $C$ di diametro$AB$, valgono le seguenti affermazioni
\begin{itemize}
	\item Un punto $P$ appartiene a $C$ se e solo se $A\hat{P}B$ è retto.
	\item Un punto $P$ è interno a $C$ se e solo se $A\hat{P}B$ è ottuso.
	\item Un punto $P$ è esterno a $C$ se e solo se $A\hat{P}B$ è acuto
\end{itemize}
\end{thm}\index{Circonferenza}
\begin{proof} Consideriamo i vari casi\newline
	\begin{itemize}
		\item Un punto $P$ appartiene a $C$ se e solo se $A\hat{P}B$ è retto.\newline
		Se il punto è sulla circonferenza come nella~\cref{fig:circonferenza1} l'angolo in $\hat{P}$ è la metà di un angolo piatto quindi è retto. 
		\item Un punto $P$ è interno a $C$ se e solo se $A\hat{P}B$ è ottuso.\newline Consideriamo la~\cref{fig:circonferenza2}. Se il punto $P$ è interno prolunghiamo $AP$ in modo che incontri in $Q$ la circonferenza. Il Triangolo $AQB$ è retto in$Q$. L'angolo $A\hat{P}B$ è somma di $P\hat{Q}B$ e $P\hat{B}Q$, quindi è ottuso.
		\item Un punto $P$ è esterno a $C$  se e solo se $A\hat{P}B$ è acuto.\newline
		Prendiamo un punto $P$ punto esterno alla circonferenza $C$ come nella~\cref{fig:circonferenza3}. Se $AP$ è secante a $C$ in  $Q$ l'angolo $A\hat{Q}B$ è retto ed è la somma di $Q\hat{P}B$ e $Q\hat{B}P$. Quindi $A\hat{P}B$ è acuto.
	\end{itemize}
\end{proof}
\begin{figure}
	\centering
	\includestandalone{geometria/circonferenza1}
	\caption{Posizione punto circonferenza, angolo retto}
	\label{fig:circonferenza1}
\end{figure}
\begin{figure}
	\centering
	\includestandalone{geometria/circonferenza2}
	\caption{Posizione punto circonferenza, angolo ottuso}
	\label{fig:circonferenza2}
\end{figure}
\begin{figure}
	\centering
	\includestandalone{geometria/circonferenza3}
	\caption{Posizione punto circonferenza, angolo acuto}
	\label{fig:circonferenza3}
\end{figure}
\begin{defn}[Ceviana]\label{defn:ceviana1}
Una Ceviana è un segmento che congiunge un vertice di un triangolo con il suo lato opposto.
\end{defn}\index{Triangolo!ceviana}\index{Ceviana}
\section{Teorema di Stewart}\label{sec:teorema-di-stewart}
\begin{thm}[Teorema di Stewart]\label{thm:Stewart}
Dato un triangolo $ABC$ colleghiamo un vertice con il lato opposto come nella figura~\cref{fig:stewart1} allora vale la relazione 
\begin{align*}
c^2y+b^2x=&a(t^2+xy)\\     
a^2y+c^2x=&b(l^2+xy)\\        
b^2y+a^2x=&c(m^2+xy)
\end{align*}
\end{thm}\index{Triangolo!teorema di Stewart}
\begin{figure}
	\centering
	\includestandalone{geometria/Stewart1}
	\caption{Teorema di Stewart}
	\label{fig:stewart1}
\end{figure}
\begin{proof}
Poniamo
\begin{align*}
AB=&c&&AC=b&&BC=a\\
AP=&t&&BP=x&&PC=y
\end{align*}
Indichiamo con $\alpha$ l'angolo $A\hat{P}B$ Applichiamo il teorema di Carnot\index{Triangolo!Carnot} ai triangoli $APB$ e $PCA$ avremo 
\begin{align*}
c^2=&t^2+x^2-2xt\cos\alpha\\
b^2=&t^2+y^2-2ty\cos(\pi-\alpha)\\
c^2=&t^2+x^2-2xt\cos\alpha\\
b^2=&t^2+y^2+2ty\cos\alpha\\
\intertext{Moltiplichiamo la prima per $y$ e la seconda per $x$ ottengo}
c^2y=&t^2y+x^2y-2xyt\cos\alpha\\
b^2x=&t^2x+y^2x+2xyt\cos\alpha\\
\intertext{sommando otteniamo}
c^2y+b^2x=&t^2y+x^2y+t^2x+y^2x\\
c^2y+b^2x=&t^2(x+y)+xy(x+y)\\
c^2y+b^2x=&(x+y)(t^2+xy)\\
c^2y+b^2x=&a(t^2+xy)\\
\end{align*}
come si volevasi dimostrare
\end{proof}\index{Triangolo!Carnot}\index{Triangolo!ceviana}
\begin{cor}[Lunghezza ceviana]\label{cor:ceviana1}
In un trangolo le ceviane hanno lunhezza:
\begin{align*}
l=&\sqrt{\dfrac{a^2y-x(by-c^2)}{b}}\\
m=&\sqrt{\dfrac{b^2y-x(cy-a^2)}{c}}\\
t=&\sqrt{\dfrac{c^2y-x(ay-b^2)}{a}}\\
\end{align*}
\end{cor}
\section{Circocentro}\label{sec:circocentro}
\begin{defn}[Circonferenza circoscritta]\label{defn:CircCirc1}
Dato un triangolo, una circonferenza circoscritta è una circonferenza che passa per i vertici del triangolo\index{Triangolo!circonferenza!circoscritta}\index{Circonferenza!circoscritta}
\end{defn}
\begin{figure}
	\centering
\includestandalone{geometria/circumcerchio}
	\caption{Circonferenza circonscritta}
	\label{fig:circumcerchio}
\end{figure}
\begin{defn}[Circocentro]\label{defn:Circocentro1}
Il circocentro è il centro della circonferenza circonscritta. \index{Triangolo!circocentro}
\end{defn}
\begin{thm}[Assi dei lati e circocentro]\label{thm:CircoAsse1}
Il circocentro è il punto di incontro degli assi del segmento.\index{Triangolo!circocentro}\index{Triangolo!asse!lato}\index{Circocentro!triangolo}
\end{thm}
\begin{proof}
	Consideriamo il triangolo $ABC$ della figura~\cref{fig:circumcerchio2}, per comodità poniamo il vertice $A$ nell'origine degli assi e il vertice $B$ sull'asse delle $x$. Determiniamo la circonferenza che passa per i tre vertici.
	\begin{align*}
	\intertext{Considero la circonferenza generica}
	x^2+y^2+ax+by+c=&0\\
	\intertext{Passaggio per $A(0,0)$ }
	c=&0\\
	\intertext{Passaggio per $B(s,0)$}
	s^2+as+c=0\\
	\intertext{Passaggio per $C(t,r)$}
	t^2+r^2+at+br+c=&0\\
	\intertext{Risolvendo il sistema otteniamo}
	\begin{cases}	
		a=-s\\
		b=-\dfrac{t^2+r^2-ts}{r}\\
		c=0
	\end{cases}&\\
	\intertext{L'equazione cercata è}
	x^2+y^2-sx-\dfrac{t^2+r^2-ts}{r}y=&0
	\intertext{Il centro $D$ ha coordinate:}
	\begin{cases}
	x=\dfrac{s}{2}\\ \\
	y=\dfrac{t^2+r^2-ts}{2r}\\
	\end{cases}&\\
		\end{align*}
	Troviamo l'intersezione degli assi dei lati
	\begin{align*}
	\intertext{L'asse del lato $AB$ ha equazione}
	x=&\dfrac{s}{2}\\
	\intertext{Per l'asse del lato $BC$ procediamo come segue:}
	(x-s)^2+y^2=&(x-t)^2+(x-r)^2\\
	x(2t-2s)+2ry-t^2-r^2+s^2=&0\\
	y=&\dfrac{x(s-t)}{r}+\dfrac{t^{2}+r^{2}-s^{2}}{2r}
	\intertext{Risolvendo il sistema otteniamo}
	&\begin{cases}
	x=\dfrac{s}{2}\\ \\
	y=\dfrac{t^2+r^2-ts}{2r}\\
	\end{cases}
	\end{align*}
	Da cui la tesi.
\end{proof}
\begin{proof}
	Consideriamo il triangolo $ABC$ della figura~\cref{fig:circumcerchio2}, per comodità poniamo il vertice $A$ nell'origine degli assi e il vertice $B$ sull'asse delle $x$. Determiniamo la circonferenza che passa per i tre vertici.
	\begin{align*}
	\intertext{Considero la circonferenza generica}
	x^2+y^2+ax+by+c=&0\\
	\intertext{Passaggio per $A(0,0)$ }
	c=&0\\
	\intertext{Passaggio per $B(s,0)$}
	s^2+as+c=0\\
	\intertext{Passaggio per $C(t,r)$}
	t^2+r^2+at+br+c=&0\\
	\intertext{Risolvendo il sistema con il metodo di Cramer}
\Delta=&\begin{vmatrix}
0&0&1\\
s&0&1\\
t&r&1\\
\end{vmatrix}=rs\\
\Delta_a=&\begin{vmatrix}
0&0&1\\
-s^2&0&1\\
-t^2-r^2&r&1\\
\end{vmatrix}=-rs^2\\
\Delta_b=&\begin{vmatrix}
0&0&1\\
s&-s^2&1\\
t&-t^2-r^2&1\\
\end{vmatrix}=-s(t^2-st+r^2)\\
\Delta_c=&\begin{vmatrix}
0&0&0\\
s&0&-s^2\\
t&r&-t^2-r^2\\
\end{vmatrix}=0\\
a=&\dfrac{\Delta_a}{\Delta}=-\dfrac{rs^2}{rs}=-s\\
b=&\dfrac{\Delta_b}{\Delta}=-\dfrac{s(t^2-st+r^2)}{rs}=-\dfrac{t^2-st+r^2}{r}\\
c=&\dfrac{\Delta_c}{\Delta}=-\dfrac{0}{rs}=0
	\intertext{L'equazione cercata è}
&x^2+y^2-sx-\dfrac{t^2+r^2-ts}{r}y=0\\
\intertext{Il centro $D$ ha coordinate:}
&	\begin{cases}
x=\dfrac{s}{2}\\ \\
y=\dfrac{t^2+r^2-ts}{2r}\\
\end{cases}\\
	\end{align*}
	Troviamo l'intersezione degli assi dei lati
	\begin{align*}
	\intertext{L'asse del lato $AB$ ha equazione}
	x=&\dfrac{s}{2}\\
	\intertext{Per l'asse del lato $BC$ procediamo come segue:}
	(x-s)^2+y^2=&(x-t)^2+(x-r)^2\\
	2x(s-t)-2ry=&s^2-r^2-t^2
	\intertext{Risolvendo il sistema otteniamo}
	\Delta=&\begin{vmatrix}
	2(s-t)&-2r\\
	1&0\\
	\end{vmatrix}=2r\\
	\Delta_x=&\begin{vmatrix}
	s^2-r^2-t^2&-2r\\
	\dfrac{s}{2}&0\\
	\end{vmatrix}=rs\\
	\Delta_y=&\begin{vmatrix}
	2(s-t)&s^2-r^2-t^2\\
	1&0\\
	\end{vmatrix}=t^2+r^2-st\\
	x=&\dfrac{\Delta_x}{\Delta}=\dfrac{rs}{2r}=\dfrac{s}{2}\\
	y=&\dfrac{\Delta_y}{\Delta}=\dfrac{t^2+r^2-st}{2r}\\
	\end{align*}
	Da cui la tesi.
\end{proof}
\begin{figure}
	\centering
	\includestandalone{geometria/circumcerchio2}
	\caption{Circocentro ed assi segmento}
	\label{fig:circumcerchio2}
\end{figure}
\begin{cor}[Triangolo rettangolo]\label{cor:CircoAsse1}
	In un triangolo rettangolo il circocentro è il punto medio dell'ipotenusa.\index{Triangolo!rettangolo!circocentro}\index{Circocentro!triangolo!rettangolo}
\end{cor}
\begin{proof}
	Dal \cref{thm:CircoAsse1} sappiamo che il circocentro ha coordinate
	\[	\begin{cases}
	x=\dfrac{s}{2}\\ \\
	y=\dfrac{t^2+r^2-ts}{2r}\\
	\end{cases}\]
	Se come nella figura~\cref{fig:circumcerchio3} il triangolo è retto in $B$ quindi le coordinate del punto $C$ diventano $C(s,r)$ e di conseguenza quelle del punto $D$ si trasformano in 	\[\begin{cases}
	x=\dfrac{s}{2}\\ \\
	y=\dfrac{s^2+r^2-s^2}{2r}=\dfrac{r}{2}\\
	\end{cases}\]Per verificare che il punto $D$ è sull'ipotenusa\index{Triangolo!rettangolo!ipotenusa} basta trovare l'equazione di questa. Banalmente l'ipotenusa ha equazione:
	\begin{align*}
	y=&\dfrac{r}{s}x\\
	\intertext{sostiuendo le coordinate del centro $D$}
	\dfrac{r}{2}=&\dfrac{r}{2}
	\end{align*}
	Da cui la tesi.
\end{proof}
\begin{figure}
	\centering
	\includestandalone{geometria/circumcerchio3}
	\caption{Circocentro e triangolo rettangolo}
	\label{fig:circumcerchio3}
\end{figure}
\begin{cor}[Triangolo ottusangolo]\label{cor:CircoAsse2}
	In un triangolo ottusangolo il circocentro è esterno al triangolo.\index{Triangolo!ottusangolo!circocentro}\index{Circocentro!triangolo!ottusangolo}
\end{cor}
\begin{figure}
	\centering
	\includestandalone[width=0.9\linewidth]{geometria/incentro1}
	\caption{Incentro di un triangolo}
	\label{fig:incentro1}
\end{figure}
\section{Incentro}\label{sec:incentro}
\begin{defn}[Incentro]\label{defn:incentro1}
L'incentro è il punto di intersezione delle bisettrici di un triangolo
\end{defn}\index{Triangolo!incentro}\index{Incentro!triangolo}\index{Triangolo!bisettrice}
\begin{thm}[Incentro]\label{thm:incentro1}
	Le tre bisettrici si incontrano in un punto detto incentro.
\end{thm}
%\begin{proof}
%	Consideriamo la~\cref{fig:incentro1}. Poniamo $A(0,0)$, $B(s,0)$ e $C(t,r)$
%	\begin{align*}
%	\intertext{La retta $AC$ ha equazione:}
%	xr+y(r-t)=&0
%	\intertext{La retta $AB$ ha equazione:}	
%	y=&0
%	\intertext{La retta $BC$ ha equazione:}
%	xr+y(s-t)-sr=&0
%	\intertext{La bisettrice tra $AC$ e $AB$ è:}
%	\dfrac{xr+y(r-t)}{\sqrt{r^2+(r-t)^2}}=&\pm y
%	\intertext{La bisettrice tra $AB$ e $BC$ è:}
%	\dfrac{xr+y(s-t)-sr}{\sqrt{r^2+(s-t)^2}}=&\pm y
%	\intertext{Poniamo}
%	a=&\sqrt{r^2+(r-t)^2}\\
%	b=&\sqrt{r^2+(s-t)^2}
%	\intertext{otteniamo quattro sistemi}
%	&\begin{cases}
%\dfrac{xr+y(r-t)}{a}=\pm y\\
%\dfrac{xr+y(s-t)-sr}{b}=\pm y
%	\end{cases}\\
%	\intertext{il primo}
%	&\begin{cases}
%	\dfrac{xr+y(r-t)}{a}=y\\
%	\dfrac{xr+y(s-t)-sr}{b}=y
%	\end{cases}\\
%	\intertext{che ha per soluzione}
%	&\begin{cases}
%	x=\dfrac{s(a-r+t)}{a-b-r+s}\\
%	y=\dfrac{rs}{a-b-r+s}
%	\end{cases}\\
%		\intertext{il secondo}
%	&\begin{cases}
%	\dfrac{xr+y(r-t)}{a}=y\\
%	\dfrac{xr+y(s-t)-sr}{b}=-y
%	\end{cases}\\
%	\intertext{che ha per soluzione}
%	&\begin{cases}
%	x=\dfrac{s(a-r+t)}{a+b-r+s}\\
%	y=\dfrac{rs}{a+b-r+s}
%	\end{cases}\\
%			\intertext{il terzo}
%	&\begin{cases}
%	\dfrac{xr+y(r-t)}{a}=-y\\
%	\dfrac{xr+y(s-t)-sr}{b}=y
%	\end{cases}\\
%	\intertext{che ha per soluzione}
%	&\begin{cases}
%	x=\dfrac{s(a+r-t)}{a+b+r-s}\\
%	y=-\dfrac{rs}{a+b+r-s}
%	\end{cases}\\
%		\intertext{il quarto}
%	&\begin{cases}
%	\dfrac{xr+y(r-t)}{a}=-y\\
%	\dfrac{xr+y(s-t)-sr}{b}=-y
%	\end{cases}\\
%	\intertext{che ha per soluzione}
%	&\begin{cases}
%	x=\dfrac{s(a+r-t)}{a-b+r-s}\\
%	y=-\dfrac{rs}{a-b+r-s}
%	\end{cases}\\
%	\intertext{La bisettrice tra $AC$ e $AB$ è:}
%	\dfrac{xr+y(r-t)}{\sqrt{r^2+(r-t)^2}}=&\pm y
%	\intertext{La bisettrice tra $AC$ e $CB$ è:}
%	\dfrac{xr+y(r-t)}{\sqrt{r^2+(r-t)^2}}=&\pm\dfrac{xr+y(s-t)-sr}{\sqrt{r^2+(s-t)^2}}
%	\intertext{Poniamo}
%	a=&\sqrt{r^2+(r-t)^2}\\
%	b=&\sqrt{r^2+(s-t)^2}
%	\intertext{otteniamo quattro sistemi}
%	&\begin{cases}
%	\dfrac{xr+y(r-t)}{a}=\pm y\\
%	\dfrac{xr+y(r-t)}{a}=\pm\dfrac{xr+y(s-t)-sr}{b}
%	\end{cases}\\
%	\intertext{il primo}
%	&\begin{cases}
%	\dfrac{xr+y(r-t)}{a}=y\\
%	\dfrac{xr+y(r-t)}{a}=\dfrac{xr+y(s-t)-sr}{b}
%	\end{cases}\\
%	\intertext{che ha per soluzione}
%	&\begin{cases}
%	x=\dfrac{s(a-r+t)}{a-b-r+s}\\
%	y=\dfrac{rs}{a-b-r+s}
%	\end{cases}\\
%		\intertext{il secondo}
%	&\begin{cases}
%	\dfrac{xr+y(r-t)}{a}=y\\
%	\dfrac{xr+y(r-t)}{a}=-\dfrac{xr+y(s-t)-sr}{b}
%	\end{cases}\\
%	\intertext{che ha per soluzione}
%	&\begin{cases}
%	x=\dfrac{s(a-r+t)}{a+b-r+s}\\
%	y=\dfrac{rs}{a+b-r+s}
%	\end{cases}\\
%	\end{align*}
%\end{proof}
\section{Ortocentro}\label{sec:ortocentro}
\begin{defn}[Ortocentro]\label{defn:ortocentro1}
	L'ortocentro è il punto di intersezione delle altezze di un triangolo \index{Triangolo!ortocentro}\index{Ortocentro!triangolo}
\end{defn}
\begin{thm}[Ortocentro]\label{thm:ortocentro1}
	Le altezze di un triangolo passano tutte per lo stesso punto
\end{thm}
\begin{figure}
	\centering
	\includestandalone{geometria/ortocentro1}
	\caption{Ortocentro triangolo}
	\label{fig:ortocentro1}
\end{figure}
\begin{proof}
Consideriamo un triangolo come nella figura~\cref{fig:ortocentro1}. Poniamo che $A(0,0)$, $B(0,s)$ e $C(t,r)$ otteniamo che  le altezze relative ai lati sono:
\begin{align*}
y=&-\dfrac{t}{r}(x-s)\\
x=&t\\
y=&\dfrac{s-t}{r}x\\
\intertext{mettendo a sistema le prime due otteniamo}
&\begin{cases}
y=-\dfrac{t}{r}(x-s)\\
x=t
\end{cases}\\
\intertext{otteniamo:}
&\begin{cases}
y=\dfrac{t}{r}(s-t)\\
x=t
\end{cases}\\
\intertext{mettendo a sistema le rimanenti otteniamo}
&\begin{cases}
y=\dfrac{s-t}{r}x\\
x=t
\end{cases}\\
\intertext{otteniamo:}
&\begin{cases}
y=\dfrac{t}{r}(s-t)\\
x=t
\end{cases}\\
\end{align*}
\end{proof}
\begin{cor}[Triangolo rettangolo]
Nel triangolo rettangolo l'ortocentro coincide con il vertice dell'angolo retto.
\end{cor}
\begin{proof}
Dal~\cref{thm:ortocentro1} abbiamo 
\begin{align*}
&\begin{cases}
y=\dfrac{t}{r}(s-t)\\
x=t
\end{cases}
\intertext{basta porre $s=t$ per ottenere}
&\begin{cases}
y=0\\
x=s
\end{cases}
\end{align*}
\end{proof}
\section{Baricentro}\label{sec:baricentro}
\begin{defn}[Mediana]\label{defn:mediana1}
	In un triangolo è il segmento che unisce un vertice con il punto medio del lato opposto.\index{Triangolo!mediana}\index{Mediana!triangolo}
\end{defn}
\begin{thm}[Lunghezza mediana]\label{thm:mediana1}
Consideriamo il triangolo in~\cref{fig:mediana1} le mediane hanno lunghezza
\begin{align*}
m_a=&\dfrac{\sqrt{2(b^2+c^2)-a^2}}{2}\\
m_b=&\dfrac{\sqrt{2(a^2+c^2)-b^2}}{2}\\
m_c=&\dfrac{\sqrt{2(a^2+b^2)-c^2}}{2}\\
\end{align*}
\end{thm}
\begin{proof}
Diretta conseguenza del~\cref{cor:ceviana1}
\end{proof}
\begin{figure}
	\centering
	\includestandalone{geometria/mediana1}
	\caption{Lunghezza mediane}
	\label{fig:mediana1}
\end{figure}
 \begin{defn}[Baricentro]\label{defn:baricentro1}
	Il Baricentro è il punto di intersezione delle mediane di un triangolo \index{Triangolo!baricentro}\index{Baricentro!triangolo}
\end{defn}
\begin{thm}[Coordinate baricentro]\label{thm:baricentro1}
	In un triangolo di vertici $A(a,b)$, $B(c,d)$ e $C(e,f)$ il baricentro $M$ ha coordinate\[M\left(\frac{a+c+e}{3},\frac{b+d+f}{3}\right)\]
\end{thm}\index{Triangolo!baricentro!coordinate}\index{Baricentro!coordinate}	
\begin{proof}
Consideriamo il triangolo della~\cref{fig:baricentro1} e determiniamo le mediane
\begin{align*}
\intertext{La mediana $AF$ ha equazione}
y =& x\dfrac{2b-d-f}{2a-c-e}+\dfrac{a(d+f)-b(c+e)}{2a-c-e}\\
\intertext{La mediana $BD$ ha equazione}
y =& x\dfrac{b - 2d + f}{a - 2c + e} +\dfrac{d(a+e)-c(b+f)}{a - 2c + e}\\
\intertext{Per trovare il punto $M$ di intersezione le metto a sistema}
&\begin{cases}
y = x\dfrac{2b-d-f}{2a-c- e}+\dfrac{a(d+f)-b(c+e)}{2a-c-e}\\ \\
y = x\dfrac{b - 2d + f}{a - 2c + e}+\dfrac{d(a+e)-c(b+f)}{a-2c+e}
\end{cases}
\intertext{che risolto mi da}
&\begin{cases}
x=\dfrac{a+c+e}{3}\\ \\
y=\dfrac{b+d+f}{3}
\end{cases}
\end{align*}
Da cui la tesi
\end{proof}
\begin{figure}
	\centering
	\includestandalone{geometria/baricentro1}
	\caption{Baricentro di un triangolo}
	\label{fig:baricentro1}
\end{figure}
\begin{thm}[Teorema di Eulero]\label{thm:eulero1}
In un triangolo Ortocentro, Circocentro e Baricentro sono allineati
\end{thm}\index{Triangolo!Eulero}\index{Triangolo!ortocentro}\index{Triangolo!circocentro}\index{Circocentro!triangolo}\index{Ortocentro!triangolo}\index{Baricentro!triangolo}\index{Triangolo!baricentro}
\begin{proof}
	Consideriamo la retta che passa per il circocentro e per l'ortocentro. Utilizzando il~\cref{thm:CircoAsse1} e il~\cref{thm:baricentro1} otteniamo
	\[y=x\dfrac{[r^2-3t(s-t)]}{r(s-2t)}+\dfrac{t[r^2-(s+t)(s-t)]}{r(s-2t)}\]
Adattando il~\cref{thm:baricentro1} il Baricentro ha coordinate: \[M\left(\frac{s+t}{3},\frac{t}{3}\right)\] che verificano l'equazione.
\end{proof}
\begin{defn}[Retta di Eulero]\label{defn:rettaEulero1}
La retta che passa per Ortocentro, Circocentro e Baricentro  chiamata retta di Eulero\end{defn}\index{Retta!Eulero}
\section{Excentro}\label{sec:excentro}
\begin{defn}[Exentro]\label{defn:excentro}
Si dice excentro  di un triangolo il punto di incontro delle bisettrici di due angoli esterni e della bisettrice dell'angolo interno non adiacente ad essi.
\end{defn}\index{Triangolo!excentro}
\begin{figure}
	\centering
	\includestandalone[width=0.9\linewidth]{geometria/excentro1}
	\caption{Triangolo e circonferenze tangenti}
	\label{fig:excentro1}
\end{figure}
\section{Formula di Erone}\label{sec:Formula_Erone}
\begin{thm}[Formula di Erone]
Dato un triangolo~\cite{Dodero1999b}, se $2p=a+b+c$ allora l'area del triangolo è \[A=\sqrt{p(p-a)(p-b)(p-c)}\]
\end{thm}\index{Triangolo!Erone}\index{Teorema!Erone}
\begin{proof}
Consideriamo la~\cref{fig:erone1} se indichiamo con $h$ l'altezza del triangolo relativa al lato di lunghezza $a$ otteniamo \begin{align}
h^2=&b^2-x^2\label{eqn:erone1}\\
h^2=&c^2-(a-x)^2\nonumber
\intertext{uguagliando}
b^2-x^2=&c^2-(a-x)^2\nonumber\\
b^2-x^2=&c^2-a^2-x^2+2ax\nonumber\\
2ax=&b^2+a^2-c^2\nonumber\\
x=&\dfrac{b^2+a^2-c^2}{2a}\nonumber\\
\intertext{Per esprire l'altezza in funzione dei lati sostituiamo il risultato precedente nell'~\cref{eqn:erone1}}
h^2=&b^2-\dfrac{\left(b^2+a^2-c^2\right)^2}{4a^2}\nonumber\\
=&\dfrac{4a^2b^2-(b^2+a^2-c^2)^2}{4a^2}\nonumber\\
=&\dfrac{\left[2ab-\left(a^2+b^2-c^2\right)\right]\left[2ab+\left(a^2+b^2-c^2\right)\right]}{4a^2}\nonumber\\
=&\dfrac{\left(2ab-b^2-a^2+c^2\right)\left(2ab+b^2+a^2-c^2\right)}{4a^2}\nonumber\\
=&\dfrac{\left[c^2-\left(b-a\right)^2\right]\left[\left(a+b\right)^2-c^2\right]}{4a^2}\nonumber\\
=&\dfrac{(c-b+a)(c+b-a)(a+b-c)(a+b+c)}{4a^2}\label{eqn:erone2}
\intertext{Ponendo}
a+b+c=&2p\nonumber\\
c+b-a=&a+b+c-2a=2p-2a=2(p-a)\nonumber\\
a+c-b=&a+b+c-2b=2p-2b=2(p-b)\nonumber\\
a+b-c=&a+b+c-2c=2p-2c=2(p-c)\nonumber
\intertext{l'~\cref{eqn:erone2} diventa}
h^2=&\dfrac{2p2(p-a)2p(p-b)2(p-c)}{4a^2}\nonumber\\
=&\dfrac{4p(p-a)(p-b)(p-c)}{a^2}\nonumber\\
h=&\dfrac{2\sqrt{p(p-a)(p-b)(p-c)}}{a}\nonumber\\
\intertext{Quindi l'area del triangolo è}
A=&\dfrac{1}{2}ah\nonumber\\
=&\dfrac{1}{2}a\dfrac{2\sqrt{p(p-a)(p-b)(p-c)}}{a}\nonumber\\
=&\sqrt{p(p-a)(p-b)(p-c)}\nonumber
\end{align}
Come si voleva dimostrare.
\end{proof}
\begin{figure}
	\centering
	\includestandalone{geometria/erone1}
	\caption{Formula di Erone}
	\label{fig:erone1}
\end{figure}
\begin{thm}[Il raggio della circonferenza inscritta ad un triangolo]\label{thm:raggiocirconferenzainscrittatriangolo}
In un triangolo il raggio della circonferenza inscritta ad un triangolo
\[R=\dfrac{A}{p}\]
Dove $A$ è l'area del triangolo e $p$ il suo semiperimetro
\end{thm}\index{Circonferenza!inscritta}\index{Triangolo!area}\index{Triangolo!seiperimetro}
\begin{figure}
	\centering
	\includestandalone[width=0.9\linewidth]{geometria/triangolo_raggiocirconferenza_area_perimetro}
	\caption{Raggio della circonferenza inscritta ad un triangolo}
	\label{fig:raggiocirconferenzainscrittatriangolo}
\end{figure}
\begin{proof}
		Costruiamo la~\cref{fig:raggiocirconferenzainscrittatriangolo} l'area è la somma dell'area dei  triangoli $AOB$, $BOC$ e $COA$. Quindi dato che il raggio è perpendicolare al lato
		\begin{align*}
			A=&\dfrac{1}{2}Ra+\dfrac{1}{2}Rb+\dfrac{1}{2}Rc\\
			=&\dfrac{1}{2}\left(a+b+c\right)R\\
			2p=&a+b+c\\
			=&\dfrac{1}{2}2pR\\
			=&pR\\
			\intertext{Quindi}
			R=&\dfrac{A}{p}\\
		\end{align*}
	Come volevasi dimostrare.
\end{proof}
\section{Triangolo equilatero}\label{sec:triangolo-equilatero}
\begin{thm}[Ortocentro, circocentro e baricentro]\label{thm:triangoloequilatero1}
	In un triangolo equilatero Ortocentro, Circocentro e Baricentro coincidono.
\end{thm}\index{Triangolo!equilatero}\index{Triangolo!equilatero!ortocentro}\index{Triangolo!equilatero!circocentro}\index{Circocentro!triangolo!equilatero}\index{Ortocentro!triangolo!equilatero}\index{Triangolo!baricentro}\index{Triangolo!equilatero!baricentro}
\begin{figure}
	\centering
	\includestandalone{geometria/triangoloEquilatero1}
	\caption{Triangolo equilatero}
	\label{fig:triangoloequilatero1}
\end{figure}
\begin{proof}
	Costruiamo la~\cref{fig:triangoloequilatero1}, un triangolo equilatero di lato $s$. Le coordinate del suoi vertici sono $A(0,0)$, $B(s,0)$ e $C(\frac{s}{2},\frac{s}{2}\sqrt{3})$. 
	Per~\cref{thm:CircoAsse1} le coordinate del circocentro\index{Triangolo!circocentro}\index{Circocentro!triangolo!equilatero} sono
	\[\begin{cases}
		x=\dfrac{s}{2}\\ \\
		y=\dfrac{t^2+r^2-ts}{2r}\\
	\end{cases}\]
	Per il~\cref{thm:ortocentro1} le coordinate dell'ortocentro\index{Triangolo!ortocentro}\index{Ortocentro!triangolo!equilatero} sono \[\begin{cases}
		y=\dfrac{t}{r}(s-t)\\
		x=t
	\end{cases}\] 
	Per il~\cref{thm:baricentro1} le coordinate del baricentro\index{Triangolo!baricentro}\index{Baricentro!triangolo!equilatero} sono \[\begin{cases}
		x=\dfrac{a+c+e}{3}\\ \\
		y=\dfrac{b+d+f}{3}
	\end{cases} \]
	Adattando le tre formule al triangolo equilatero otteniamo:
	\[\begin{cases}
		x=\dfrac{s}{2}\\ \\
		y=\dfrac{s}{6}\sqrt{3}
	\end{cases}\]
\end{proof}
\begin{thm}[Raggio della circonferenza circoscritta]\label{thm:raggiocirconferenzacircoscritta}
	In un triangolo equilatero di lato $l$, il raggio della circonferenza circoscritta è \[R=\dfrac{\sqrt{3}}{3}l\]
\end{thm}\index{Circonferenza!circoscritta}
\begin{figure}
	\centering
	\includestandalone[width=0.7\linewidth]{geometria/triangolo_equi_raggiocirconferenza_circoscritta}
	\caption{Raggio della circonferenza circoscritta}
	\label{fig:raggiocirconferenzacircoscritta}
\end{figure}
\begin{proof}
	Costruiamo la~\cref{fig:raggiocirconferenzacircoscritta}
	\begin{align*}
		\intertext{Consideriamo il triangolo $ABC$}
		h^2=&l^2-\dfrac{l^2}{4}\\
		=&\dfrac{4l^2-l^2}{4}\\
		=&\dfrac{3}{4}l^2\\
		\intertext{Quindi}
		h=&\dfrac{l}{2}\sqrt{3}\\
		\intertext{Consideriamo il triangolo $OBC$}
		x^2=&r^2-\dfrac{l^2}{4}\\
		x=&h-R\\
		(h-R)^2=&r^2-\dfrac{l^2}{4}\\
		2hR-h^2=&\dfrac{l^2}{4}\\
		2(\dfrac{l}{2}\sqrt{3})R-(\dfrac{l}{2}\sqrt{3})^2=&\dfrac{l^2}{4}\\
		\dfrac{3}{4}l^2-l\sqrt{3}R=&-\dfrac{l^2}{4}\\
		\dfrac{3}{4}l^2+\dfrac{l^2}{4}-l\sqrt{3}R=&0\\
		l^2-l\sqrt{3}R=&0\\
		l=&0\\
		l-\sqrt{3}R=&0\\
		l=&\sqrt{3}R\\
		R=&\dfrac{\sqrt{3}}{3}l
	\end{align*}
	Come volevasi dimostrare.
\end{proof}
\begin{thm}[Raggio della circonferenza inscritta]\label{thm:raggiocirconferenzainscritta}
	In un triangolo equilatero di lato $l$, il raggio della circonferenza inscritta è \[R=\dfrac{\sqrt{3}}{6}l\]
\end{thm}\index{Circonferenza!inscritta}
\begin{figure}
	\centering
	\includestandalone[width=0.7\linewidth]{geometria/triangolo_equi_raggiocirconferenza_inscritta}
	\caption{Raggio della circonferenza inscritta}
	\label{fig:raggiocirconferenzainscritta}
\end{figure}
\begin{proof}
Costruiamo la~\cref{fig:raggiocirconferenzainscritta}	
utilizzando il~\cref{thm:raggiocirconferenzainscrittatriangolo}
\begin{align*}
	A=&pR\\
	p=&\frac{3}{2}l\\
	A=&\frac{3}{2}lR\\
	\intertext{ma}
	A=&\frac{lh}{2}\\
	\intertext{in un triangolo equilatero}
	h=&\frac{\sqrt{3}}{2}l\\
	A=&\frac{\sqrt{3}}{4}l^2\\
	\frac{\sqrt{3}}{4}l^2=&\frac{3}{2}lR\\
	\sqrt{3}l^2=&6lR\\
	\sqrt{3}l^2-6lR=&0\\
	\intertext{ma $l\neq0$}
	\sqrt{3}l-6R=&0\\
	l=&\dfrac{6R}{\sqrt{3}}\\
	l=&2\sqrt{3}R\\
	R=&\dfrac{\sqrt{3}}{6}l
\end{align*}
\end{proof}