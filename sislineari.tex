% !TeX root = Asparsi.tex
% !BIB TS-program = biber
% !TeX encoding = UTF-8
% !TeX spellcheck = it_IT
\chapter{Sistemi lineari}\label{ch:sistemi-lineari}
\section{Metodo di Cramer}\label{sec:metodo-di-cramer}\index{Sistemi!lineari!Cramer}\index{Cramer}
\begin{thm}[Metodo di Cramer]\label[thm]{thm:teoCramer1a}
	Dato un sistema lineare del tipo
\begin{equation}
\begin{cases}
a_1x+b_1y=c_1\\
a_2x+b_2y=c_2
\end{cases}\quad a_1\neq 0\;b_1\neq 0\quad a_2\neq 0\;b_2\neq 0
\end{equation}\label[sistema]{sistema:teoCramer1}
allora 
\begin{align*}
x=&\frac{c_{1} b_{2} - c_{2} b_{1}}{a_{1} b_{2} - a_{2} b_{1}}=\frac{\begin{vmatrix}
	c_{1} & b_{1}  \\
	c_{2} & b_{2}
	\end{vmatrix} }{\begin{vmatrix}
	a_{1} & b_{1}  \\
	a_{2} & b_{2}  \\
	\end{vmatrix}}\\
y=&\frac{c_{2} a_{1} - c_{1} a_{2}}{a_{1} b_{2} - a_{2} b_{1}}=\frac{\begin{vmatrix}
	a_{1} & c_{1}  \\
	a_{2} & c_{2}
	\end{vmatrix} }{\begin{vmatrix}
	a_{1} & b_{1}  \\
	a_{2} & b_{2}  \\
	\end{vmatrix}}\\
&&a_{1} b_{2} - a_{2} b_{1}\neq& 0\\
\end{align*}
\end{thm}
\begin{proof}
	Consideriamo il~\cref{sistema:teoCramer1} moltiplicando la prima riga per $b_2$ e la seconda riga per $b_1$ otteniamo:
\begin{equation}\label[sistema]{si:teoCramer2}
\begin{cases}
a_1b_2x+b_1b_2y=c_1b_2\\
a_2b_1x+b_2b_1y=c_2b_1
\end{cases}
\end{equation}
sottraendo otteniamo
\begin{align*}
a_1b_2x-a_2b_1x+0=&b_2c_1-b_1c_2\\
x=&\frac{b_2c_1-b_1c_2}{a_1b_2-a_2b_1}
\end{align*}
Ripartiamo dal~\cref{sistema:teoCramer1} moltiplicando la prima riga per $a_2$ e la seconda riga per $a_1$ otteniamo:
\begin{equation}\label[sistema]{si:teoCramer3}
\begin{cases}
a_1a_2x+b_1a_2y=c_1a_2\\
a_2a_1x+b_2a_1y=c_2a_1
\end{cases}
\end{equation}
sottraendo otteniamo
\begin{align*}
0+a_1b_2y-a_2b_1y=&a_2c_1-a_1c_2\\
y=&\frac{c_2a_1-c_1a_2}{a_1b_2-a_2b_1}
\end{align*}
\end{proof}
\section{Interpretazione geometrica}
\begin{thm}[Posizioni di due rette]
	Date due rette~\cite{Zwirner1988}c $a_1x+b_1y=c_1$ e $a_2x+b_2y=c_2$ o in forma esplicita $y=m_1x+q_1$ e $y=m_2x+q_2$ allora le due rette sono:
\begin{align*}
&\text{Incidenti}&&\dfrac{a_1}{a_2}\neq\dfrac{b_1}{b_2}&m_1\neq& m_2\\
&\text{Parallele e distinte}&&\dfrac{a_1}{a_2}=\dfrac{b_1}{b_2}\neq\dfrac{c_1}{c_2}&m_1=& m_2\quad q_1\neq q_2\\
&\text{Coincidenti}&&\dfrac{a_1}{a_2}=\dfrac{b_1}{b_2}=\dfrac{c_1}{c_2}&m_1=& m_2\quad q_1= q_2\\
\end{align*}
\end{thm}\index{Retta!parallela distinta}\index{Retta!incidente}\index{Retta!coincidenti}
\begin{proof}
Se $\dfrac{a_1}{a_2}\neq\dfrac{b_1}{b_2}$ allora $a_1b_2-a_2b_1\neq 0$ quindi per il~\cref{thm:teoCramer1a}, il~\cref{sistema:teoCramer1} ha soluzione per cui le due rette hanno un punto in comune.

Se $\dfrac{a_1}{a_2}=\dfrac{b_1}{b_2}\neq\dfrac{c_1}{c_2}$ allora $a_1b_2-a_2b_1= 0$ quindi per il~\cref{thm:teoCramer1a}, il~\cref{sistema:teoCramer1} non ha soluzione per cui le due rette non hanno punti in comune. Inoltre $\dfrac{a_1}{a_2}=\dfrac{b_1}{b_2}$ implica che $a_1=ka_2$ e $b_1=kb_2$. Di conseguenza le due rette diventano $ka_2x+kb_2y=c_1$ e $a_2x+b_2y=c_2$ cioè riscrivendole otteniamo $a_2x+b_2y=\dfrac{c_1}{k}$ $a_2x+b_2y=c_2$ ma per ipotesi $c_2\neq\dfrac{c_1}{k}$, quindi le due rette non coincidono.

Se $\dfrac{a_1}{a_2}=\dfrac{b_1}{b_2}=\dfrac{c_1}{c_2}$ allora il sistema non ha soluzione e $c_1=kc_2$ quindi le  due rette coincidono.

Se le due rette sono in forma esplicita allora 
\begin{align*}
m_1=&\dfrac{a_1}{b_1}\\
m_2=&\dfrac{a_2}{b_2}\\
q_1=&\dfrac{c_1}{b_1}\\
q_2=&\dfrac{c_2}{b_2}
\end{align*} 
Quindi il teorema.
\end{proof}