% !TeX root = Asparsi.tex
% !BIB TS-program = biber
% !TeX encoding = UTF-8
% !TeX spellcheck = it_IT
\chapter{Goniometria}
\section{Somma sottrazione di angoli}
\begin{figure}
	\centering
\includestandalone[scale=0.68]{geometria/cosenosommadifferenza1}
	\caption{Coseno della differenza}
	\label{fig:cosenosommadifferenza1}
\end{figure}
\begin{thm}[Coseno della differenza di due angoli]\label{thm:Cosenodelladifferenza}
Dati due angoli $\alpha$ e $\beta$ allora
\[\cos(\alpha-\beta)=\cos\alpha\cos\beta+\sin\alpha\sin\beta  \]
\end{thm}\index{Coseno!differenza}
\begin{proof}
Consideriamo due angoli $\alpha$ e $\beta$ con $\alpha>\beta$. Otteniamo la~\vref{fig:cosenosommadifferenza1}. Costruiamo una angolo di ampiezza $\alpha-\beta$ con vertice in $O$ e un lato $AD$. Otteniamo quattro punti $A(\cos\alpha;\sin\alpha)$, $B (\cos\beta;\sin\beta)$, $C(\cos\left(\alpha-\beta\right);\sin\left(\alpha-\beta\right))$ e $D(1;0)$. I triangoli $AOB$ e $COD$ sono congruenti in particolare abbiamo: \begin{align*}
\widebar{CD}=&\widebar{AB}\\
\left[\cos(\alpha-\beta)-1\right]^2+\sin^2(\alpha-\beta)=&\left[\cos\alpha-\cos\beta\right]^2+\left[\sin\alpha-\sin\beta\right]^2\\
\cos^2(\alpha-\beta)-2\cos(\alpha-\beta)+1+\sin^2(\alpha-\beta)=&\cos^2\alpha+\cos^2\beta-2\cos\alpha\cos\beta+\sin^2\alpha+\sin^2\beta-2\sin\alpha\sin\beta\\
\intertext{ma}
\cos^2(\alpha-\beta)+\sin^2(\alpha-\beta)=&1\\
\cos^2\alpha+\sin^2\alpha=&1\\
\cos^2\beta+\sin^2\beta=&1\\
\intertext{quindi}
1+1-2\cos(\alpha-\beta)=&1+1-2\cos\alpha\cos\beta-2\sin\alpha\sin\beta\\
2\cos(\alpha-\beta)=&-2\cos\alpha\cos\beta-2\sin\alpha\sin\beta\\
\intertext{semplifichiamo}
\cos(\alpha-\beta)=&\cos\alpha\cos\beta+\sin\alpha\sin\beta 
\end{align*}
Da cui la tesi.
\end{proof}
\begin{cor}[Coseno della somma di due angoli]\label{thm:Cosenodellasomma}
Dati due angoli $\alpha$ e $\beta$ allora
\[\cos(\alpha+\beta)=\cos\alpha\cos\beta-\sin\alpha\sin\beta  \]
\end{cor}\index{Coseno!somma}
\begin{proof}
	Dal~\vref{thm:Cosenodelladifferenza}
	\begin{align*}
	\cos(\alpha+\beta)=&\cos\left[\alpha-(-\beta)\right]\\
	\cos\left[\alpha-(-\beta)\right]=&\cos\alpha\cos(-\beta)+\sin\alpha\sin(-\beta)\\
	\intertext{ma}
	\cos(-\beta)=&\cos\beta\\
	\sin(-\beta)=&-\sin\beta
	\intertext{quindi}
	\cos(\alpha+\beta)=&\cos\alpha\cos\beta-\sin\alpha\sin\beta
	\end{align*}
	Da cui la tesi
\end{proof}
\section{Formule parametriche}
\begin{thm}[Formule parametriche]\label{thm:formuleparametriche1}
	Se $\alpha$ è un angolo allora
	\begin{align*}
\sin\alpha=&\frac{2t}{1+t^2}\\
\cos\alpha=&\frac{1-t^2}{1+t^2}\\
t=&\tan\frac{\alpha}{2}&\alpha\neq\pi+2k\pi
	\end{align*}\index{Seno}\index{Coseno}\index{Tangente}\index{Formule!parametriche}
\end{thm}
\begin{proof} Dalla somma di due angoli otteniamo le seguenti relazioni
	\begin{align*}
	\intertext{Prima relazione}
	\sin2\alpha=&2\sin\alpha\cos\alpha\\
	=&\frac{2\sin\alpha\cos\alpha}{1}\\
	=&\frac{2\sin\alpha\cos\alpha}{\cos^2\alpha+\sin^2\alpha}\\
	=&\dfrac{\dfrac{2\sin\alpha\cos\alpha}{\cos^2\alpha}}{\dfrac{\cos^2\alpha+\sin^2\alpha}{cos^2\alpha}}&\alpha\neq\frac{\pi}{2}+k\pi\\
	=&\dfrac{\dfrac{2\sin\alpha}{\cos\alpha}}{\dfrac{\cos^2\alpha}{\cos^2\alpha}+\dfrac{\sin^2\alpha}{cos^2\alpha}}\\
	=&\dfrac{2\tan\alpha}{1+\tan^2\alpha}\\
	\alpha&\longmapsto\dfrac{\alpha}{2}\\
	\sin\alpha=&\dfrac{2\tan\dfrac{\alpha}{2}}{1+\tan^2\dfrac{\alpha}{2}}\\
	t=&\tan\frac{\alpha}{2}\\
	\sin\alpha=&\frac{2t}{1+t^2}\\
	\end{align*}
	\begin{align*}
	\intertext{Seconda relazione}
	\cos 2\alpha=&\cos^2\alpha-\sin^2\alpha\\
	=&\dfrac{\cos^2\alpha-\sin^2\alpha}{1}\\
	=&\frac{\cos^2\alpha-\sin^2\alpha}{\cos^2\alpha+\sin^2\alpha}\\
	=&\frac{\dfrac{\cos^2\alpha-\sin^2\alpha}{\cos^2\alpha}}{\dfrac{\cos^2\alpha+\sin^2\alpha}{\cos^2\alpha}}&\alpha\neq\frac{\pi}{2}+k\pi\\ 
	=&\frac{1-\dfrac{\sin^2\alpha}{\cos^2\alpha}}{1+\dfrac{\sin^2\alpha}{\cos^2\alpha}}\\
	=&\frac{1-\tan^2\alpha}{1+\tan^2\alpha}\\
	\alpha&\longmapsto\dfrac{\alpha}{2}\\
	\cos\alpha=&\frac{1-\tan^2\dfrac{\alpha}{2}}{1+\tan^2\dfrac{\alpha}{2}}\\
	t=&\tan\frac{\alpha}{2}\\
	\cos\alpha=&\frac{1-t^2}{1+t^2}
	\end{align*}
	da cui la tesi.
\end{proof}