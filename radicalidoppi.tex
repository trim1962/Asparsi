% !TeX root = Asparsi.tex
% !BIB TS-program = biber
% !TeX encoding = UTF-8
% !TeX spellcheck = it_IT
\chapter{Radicali doppi}\label{ch:radicalidoppi}
\section{Radicali doppi}
\begin{thm}[Radicali doppi]\label{thm:Radicalidoppi1}
Se  $a$ e $b$ sono numeri interi allora
\begin{align*}
		\sqrt{a+\sqrt{b}}=&\sqrt{\dfrac{a+\sqrt{a^2-b}}{2}}+\sqrt{\dfrac{a-\sqrt{a^2-b}}{2}}\\
		\sqrt{a-\sqrt{b}}=&\sqrt{\dfrac{a+\sqrt{a^2-b}}{2}}-\sqrt{\dfrac{a-\sqrt{a^2-b}}{2}}
		\\
\end{align*}
\end{thm}\index{Radicali!doppi}
\begin{proof}
	\begin{align*}
	\intertext{Dimostriamo la prima relazione}
	\sqrt{a+\sqrt{b}}=&\sqrt{x}+\sqrt{y}\\
	\intertext{Elevando al quadrato}
	a+\sqrt{b}=&x+y+2\sqrt{xy}\\
	a+\sqrt{b}=&x+y+\sqrt{4xy}\\
	\intertext{Otteniamo il sistema simmetrico}
	&\begin{cases}
	x+y=a\\
	xy=\dfrac{b}{4}
	\end{cases}\\
	t^2-at+\dfrac{b}{4}=&0\\
	\intertext{Che ha per soluzioni}
	t_{1,2}=&\dfrac{4a\pm\sqrt{16a^2-16b}}{8}\\
	=&\dfrac{4a\pm 4\sqrt{a^2-b}}{8}\\
	=&\dfrac{a\pm\sqrt{a^2-b}}{2}\\
	&\begin{cases}
	x=\dfrac{a+\sqrt{a^2-b}}{2}\\
	\\
	y=\dfrac{a-\sqrt{a^2-b}}{2}\\
	\end{cases}
	\intertext{Quindi}
	\sqrt{a+\sqrt{b}}=&\sqrt{\dfrac{a+\sqrt{a^2-b}}{2}}+\sqrt{\dfrac{a-\sqrt{a^2-b}}{2}}\\
	\end{align*}
		\begin{align*}
	\intertext{Dimostriamo la seconda relazione}
	\sqrt{a-\sqrt{b}}=&\sqrt{x}-\sqrt{y}\\
	\intertext{Elevando al quadrato}
	a+\sqrt{b}=&x+y-2\sqrt{xy}\\
	a+\sqrt{b}=&x+y-\sqrt{4xy}\\
	\intertext{Otteniamo il sistema simmetrico}
	&\begin{cases}
	x+y=a\\
	xy=\dfrac{b}{4}
	\end{cases}\\
	t^2-at+\dfrac{b}{4}=&0\\
	\intertext{Che ha per soluzioni}
	t_{1,2}=&\dfrac{4a\pm\sqrt{16a^2-16b}}{8}\\
	=&\dfrac{4a\pm 4\sqrt{a^2-b}}{8}\\
	=&\dfrac{a\pm\sqrt{a^2-b}}{2}\\
	&\begin{cases}
	x=\dfrac{a+\sqrt{a^2-b}}{2}\\
	\\
	y=\dfrac{a-\sqrt{a^2-b}}{2}\\
	\end{cases}
	\intertext{Quindi}
	\sqrt{a-\sqrt{b}}=&\sqrt{\dfrac{a+\sqrt{a^2-b}}{2}}-\sqrt{\dfrac{a-\sqrt{a^2-b}}{2}}\\
	\end{align*}
\end{proof}  
\begin{cor}[Quadrato perfetto]\label{cor:quadratoperfetto}
	Dato $\sqrt{a+\sqrt{b}}$ e $\sqrt{a-\sqrt{b}}$ se $a^2-b=x^2$ è un quadrato perfetto allora
	\begin{align*}
	\sqrt{a+\sqrt{b}}=&\sqrt{\dfrac{a+x}{2}}+\sqrt{\dfrac{a-x}{2}}\\
	\sqrt{a-\sqrt{b}}=&\sqrt{\dfrac{a+x}{2}}-\sqrt{\dfrac{a-x}{2}}\\
	\end{align*}
\end{cor}\index{Quadrato!perfetto}