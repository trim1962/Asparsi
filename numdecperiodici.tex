% !TeX root = Asparsi.tex
% !BIB TS-program = biber
% !TeX encoding = UTF-8
% !TeX spellcheck = it_IT
\chapter{Numeri}\label{ch:numeri-primi}
\section{I numeri primi sono infiniti}\label{sec:i-numeri-primi}
\begin{thm}[Numeri primi]\label{thm:numeriprimiinfiniti1}
I numeri primi sono infiniti
\end{thm}\index{Numero!primo!infiniti}
\chapter{Numeri pari}
\section{Numeri pari per differenza}\label{sec:quadratodifferenza}
\begin{thm}[Quadrato meno differenza]\label{thm:quadratodifferenza}
	Se $x$ è un numero naturale allora $x^2-x$ è pari. 
\end{thm}\index{Numero!pari}
\begin{proof}
		\begin{align*}
		\intertext{Se $x$ è pari allora:}
		x=&2n\\
		x^2-x=&4n^2-2n\\
		=&2n(2n-1)\\
		\intertext{pari}
		\intertext{Se $x$ è dispari allora:}
		x=&2n+1\\
		x^2-x=&4n^2-4n+1-(2n+1)\\
		=&4n^2-4n+1-2n-1\\
		=&4n^2-4n-2n\\
		=&4n^2-6n\\
		=&2n(2n-3)
		\intertext{pari}
	\end{align*}
come si voleva dimostrare
\end{proof}
\chapter{Numeri decimali periodici}\label{ch:numeri-decimali-periodici}
\section{Numero decimale periodico puro}\label{sec:numerodecimaleperiodico-puro}
\begin{thm}[Numero decimale periodico puro]
	La frazione generatrice di un numero decimale periodico puro ha per numeratore la differenza fra in numero formato dalla parte intere e il periodo e il numero formato dalla parte intera. Per denominatore un numero formato da tanti nove per quanto è lungo il periodo.  
\end{thm}\index{Numero!decimale!infinito}
\begin{proof}
	Sia $x=a_1\dots a_m,\overline{c_1\dots c_n}$ dove $a_1\dots a_m$ è la parte intera\index{Numero!parte intera} e $c_1\dots c_n$ il periodo\index{Numero!periodo}. 
	\begin{align*}
	10^nx=&a_1\dots a_m c_1\dots c_n,\overline{c_1\dots c_n}\\
	10^n1x-x=&a_1\dots a_m b_1\dots b_p c_1\dots c_n-a_1\dots a_m b_1\dots b_p\\
	(10^n-1)x=&a_1\dots a_m c_1\dots c_n-a_1\dots a_m\\
	x=&\frac{a_1\dots a_m c_1\dots c_n-a_1\dots a_m}{10^n-1}
	\end{align*}
\end{proof}
\section{Numero decimale periodico misto}\label{sec:numero-decimale-periodico-misto}
\begin{thm}[Numero decimale periodico misto]\label{thm:Numero-decimale-periodico-misto}
La frazione generatrice di un numero decimale periodico misto ha per numeratore la differenza fra in numero formato dalla parte intere, l'antiperiodo e il periodo e il numero formato dalla parte intera e l'antiperiodo. Per denominatore un numero formato da tanti nove per quanto è lungo il periodo e tanti zero per quanto è lungo l'antiperiodo.  
\end{thm}
\begin{proof}
	Sia $x=a_1\dots a_m,b_1\dots b_p \overline{c_1\dots c_n}$ dove $a_1\dots a_m$ è la parte intera\index{Numero!parte intera}, $b_1\dots b_p$ l'antiperiodo\index{Numero!antiperiodo} e $c_1\dots c_n$ il periodo\index{Numero!periodo}. 
	\begin{align*}
	10^px=&a_1\dots a_m b_1\dots b_p,\overline{c_1\dots c_n}\\
	10^n10^px=&a_1\dots a_m b_1\dots b_p c_1\dots c_n,\overline{c_1\dots c_n}\\
	10^n10^px-10^px=&a_1\dots a_m b_1\dots b_p c_1\dots c_n-a_1\dots a_m b_1\dots b_p\\
	10^px(10^n-1)=&a_1\dots a_m b_1\dots b_p c_1\dots c_n-a_1\dots a_m b_1\dots b_p\\
	x=&\frac{a_1\dots a_m b_1\dots b_p c_1\dots c_n-a_1\dots a_m b_1\dots b_p}{10^p(10^n-1)}
	\end{align*}
\end{proof}