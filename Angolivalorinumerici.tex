% !TeX root = Asparsi.tex
% !BIB TS-program = biber
% !TeX encoding = UTF-8
% !TeX spellcheck = it_IT
\chapter{Angoli valori numerici}\label{chap:Angoli_valori_numerici}
\section{$alpha=\ang{7.5}$}
\begin{align*}
	\intertext{Utilizzando il~\vref{cor:Cosenoangolometa}}
	\cos^2\dfrac{\alpha}{2}=&\dfrac{1}{2}(\cos\alpha+1)\\
	\cos^2\ang{7.5}=&\dfrac{1}{2}(\cos\ang{15}+1)\\ 
	=&\dfrac{1}{2}(\dfrac{\sqrt{2}+\sqrt{6}}{4}+1)\\
	=&\dfrac{1}{8}(\sqrt{2}+\sqrt{6}+4)\\
	\cos\ang{7.5}=&\sqrt{\dfrac{1}{8}(\sqrt{2}+\sqrt{6}+4)}\\
\end{align*}
\begin{align*}
\intertext{Utilizzando il~\vref{cor:Senoangolometacoseno}}
\sin^2\dfrac{\alpha}{2}=&\dfrac{1}{2}(1-\cos\alpha)\\ 
\sin^2\ang{7.5}=&\dfrac{1}{2}(1-\dfrac{\sqrt{2}+\sqrt{6}}{4})\\ 
=&\dfrac{1}{2}(\dfrac{4-\sqrt{2}-\sqrt{6}}{4})\\ 
=&\dfrac{1}{8}(4-\sqrt{2}-\sqrt{6})\\ 
\sin\ang{7.5}=&\sqrt{\dfrac{1}{8}(4-\sqrt{2}-\sqrt{6})}
\end{align*}
\begin{align*}
	\tan\dfrac{\alpha}{2}=&\dfrac{\sin\dfrac{\alpha}{2}}{\cos\dfrac{\alpha}{2}}\\
	=&\dfrac{2\sin\dfrac{\alpha}{2}\sin\dfrac{\alpha}{2}}{2\cos\dfrac{\alpha}{2}\sin\dfrac{\alpha}{2}}\\
	=&\dfrac{2\sin^2\dfrac{\alpha}{2}}{2\cos\dfrac{\alpha}{2}\sin\dfrac{\alpha}{2}}\\
	\intertext{ma}
	2\sin^2\dfrac{\alpha}{2}=&1-\cos\alpha\\
	2\cos\dfrac{\alpha}{2}\sin\dfrac{\alpha}{2}=&\sin\alpha\\
	\tan\dfrac{\alpha}{2}=&\dfrac{1-\cos\alpha}{\sin\alpha}\\
\end{align*}
\begin{align*}
	\intertext{Quindi}
	\tan\ang{7.5}=&\dfrac{1-\cos\ang{15}}{\sin\ang{15}}\\
	=&\dfrac{1-\dfrac{\sqrt{2}+\sqrt{6}}{4}}{\dfrac{\sqrt{6}-\sqrt{2}}{4}}\\
	=&\dfrac{\dfrac{4-\sqrt{2}-\sqrt{6}}{4}}{\dfrac{\sqrt{6}-\sqrt{2}}{4}}\\
	=&\dfrac{4-\sqrt{2}-\sqrt{6}}{4}\dfrac{4}{\sqrt{6}-\sqrt{2}}\\
	=&\dfrac{4-\sqrt{2}-\sqrt{6}}{\sqrt{6}-\sqrt{2}}\\
	=&\dfrac{4-\sqrt{2}-\sqrt{6}}{\sqrt{6}-\sqrt{2}}\dfrac{\sqrt{6}+\sqrt{2}}{\sqrt{6}+\sqrt{2}}\\
	=&\dfrac{4\sqrt{6}+4\sqrt{2}-4\sqrt{3}-8}{4}\\
	=&\sqrt{6}+\sqrt{2}-\sqrt{3}-2
\end{align*}
\section{$alpha=\ang{15}$}
\begin{align*}
	\cos\ang{15}=&\cos(\ang{60}-\ang{45})\\
	=&\difcos{60}{45}\\
	=&\cossessanta\cosquarantacinque+\sinsessanta\sinquarantecinque\\
	=&\dfrac{\sqrt{2}+\sqrt{6}}{4}\\
\end{align*}
\begin{align*}
	\sin\ang{15}=&\sin(\ang{60}-\ang{45})\\
	=&\difsin{60}{45}\\
	=&\sinsessanta\cosquarantacinque-\cossessanta\sinquarantecinque\\
	=&\dfrac{\sqrt{6}-\sqrt{2}}{4}
\end{align*}
\begin{align*}
	\tan\ang{15}=&\dfrac{\sin\ang{15}}{\cos\ang{15}}\\
	=&\dfrac{\dfrac{\sqrt{6}-\sqrt{2}}{4}}{\dfrac{\sqrt{2}+\sqrt{6}}{4}}\\
	=&\dfrac{\sqrt{6}-\sqrt{2}}{4}\dfrac{4}{\sqrt{2}+\sqrt{6}}\\
	=&\dfrac{\sqrt{6}-\sqrt{2}}{\sqrt{2}+\sqrt{6}}\\
	=&\dfrac{\sqrt{6}-\sqrt{2}}{\sqrt{2}+\sqrt{6}}\dfrac{\sqrt{6}-\sqrt{2}}{\sqrt{2}-\sqrt{6}}\\
	=&\dfrac{8-2\sqrt{12}}{6-2}\\
	=&2-\sqrt{3}
\end{align*}
\begin{align*}
	\cot\ang{15}=&\dfrac{1}{\tan\ang{15}}\\
	=&\dfrac{1}{2-\sqrt{3}}\\
	=&\dfrac{1}{2-\sqrt{3}}\dfrac{2+\sqrt{3}}{2+\sqrt{3}}\\
	=2+\sqrt{3}\\
\end{align*}
\section{$alpha=\ang{75}$}
\begin{align*}
\cos\ang{75}=&\cos(\ang{30}+\ang{45})\\
=&\sumcos{30}{45}\\
=&\costrenta\cosquarantacinque-\sintrenta\sinquarantecinque\\
=&\dfrac{\sqrt{6}}{4}-\dfrac{\sqrt{2}}{4}\\
=&\dfrac{\sqrt{6}-\sqrt{2}}{4}
\end{align*}
\begin{align*}
\sin\ang{75}=&\cos(\ang{90}-\ang{75})\\
=&\cos\ang{15}\\
=&\dfrac{\sqrt{6}+\sqrt{2}}{4}
\end{align*}
\begin{align*}
\tan\ang{75}=&\dfrac{\sin\ang{75}}{\cos\ang{75}}\\
=&\dfrac{\dfrac{\sqrt{6}+\sqrt{2}}{4}}{\dfrac{\sqrt{2}-\sqrt{6}}{4}}\\
=&\dfrac{\sqrt{6}+\sqrt{2}}{4}\dfrac{4}{\sqrt{2}-\sqrt{6}}\\
=&\dfrac{\sqrt{6}+\sqrt{2}}{\sqrt{2}-\sqrt{6}}\\
=&\dfrac{\sqrt{6}+\sqrt{2}}{\sqrt{2}-\sqrt{6}}\dfrac{\sqrt{6}+\sqrt{2}}{\sqrt{2}+\sqrt{6}}\\
=&\dfrac{8+2\sqrt{12}}{6-2}\\
=&2+\sqrt{3}
\end{align*}
\begin{align*}
\cot\ang{75}=&\dfrac{1}{\tan\ang{75}}\\
=&\dfrac{1}{2+\sqrt{3}}\\
=&\dfrac{1}{2+\sqrt{3}}\dfrac{2-\sqrt{3}}{2-\sqrt{3}}\\
=&2-\sqrt{3}\\
\end{align*}
\section{$alpha=\ang{18}$}
\begin{align*}
	\sin\ang{18}\\
	a=&\ang{18}\\
	5a=&\ang{90}\\
	2a+3a=&\ang{90}\\
	2a=&\ang{90}-3a\\
	\sin 2a=&\sin(\ang{90}-3a)=\cos 3a\\
	\intertext{trasformiamo l'espressione}
	\sin 2a=&2\sin a\cos a\\
	\cos 3a=&\cos2a\cos a-\sin 2a\sin a\\
	 =&\cos a(2\cos^2 a -1)-2\sin^2 a\cos a\\
	 =&\cos^3 a -\cos a -2(1-cos^2 a)\cos a\\
	 =&\cos^3 a -\cos a -2\cos a +cos^3 a\\
	 =&4\cos^3 a-3\cos a
	 \intertext{Riprendendo}
	 2\sin a\cos a=&4\cos^3 a-3\cos a
	 \intertext{$\cos a\neq 0$}
	 2\sin a=&4\cos^2 a-3 \\
	 2\sin a-4\cos^2 a+3=&0 \\
	 2\sin a-4(1-\sin^2 a) a+3=&0 \\
	 4\sin^2 a+2\sin a-1=&0 \\
	 \sin a=&\dfrac{-2\pm\sqrt{4+16}}{8}\\
	 =&\dfrac{-2\pm\sqrt{20}}{8}\\
	 =&\dfrac{-2\pm2\sqrt{5}}{8}\\
	 =&\dfrac{-1\pm\sqrt{5}}{4}\\
	 \intertext{ma $\sin\ang{18}>0$}
	 \sin a=&\dfrac{-1+\sqrt{5}}{4}
	\end{align*}
\begin{align*}
	 \cos a=&\pm\sqrt{1-\sin^2a}\\
	 \intertext{ma l'angolo è nel primo quadrante quindi}
	 \cos a=&\sqrt{1-\left(\dfrac{-1+\sqrt{5}}{4}\right)^2}\\	
	 =&\sqrt{\dfrac{16-6+2\sqrt{5}}{16}}\\
	 =&\dfrac{\sqrt{10+2\sqrt{5}}}{4}
	\end{align*}
\begin{align*}
	 \tan\ang{18}=&\dfrac{\sin\ang{18}}{\cos\ang{18}}\\
	 =&\dfrac{\dfrac{-1+\sqrt{5}}{4}}{\dfrac{\sqrt{10+2\sqrt{5}}}{4}}\\
	 =&\dfrac{-1+\sqrt{5}}{4}\dfrac{4}{\sqrt{10+2\sqrt{5}}}\\
	 =&\dfrac{-1+\sqrt{5}}{\sqrt{10+2\sqrt{5}}}\\
	 =&\dfrac{-1+\sqrt{5}}{\sqrt{10+2\sqrt{5}}}\dfrac{\sqrt{10-2\sqrt{5}}}{\sqrt{10-2\sqrt{5}}}\\
	 &=\dfrac{\sqrt{(-1+\sqrt{5})^2(\sqrt{10-2\sqrt{5}})}}{\sqrt{100-20}}\\
	 &=\dfrac{\sqrt{(6-2\sqrt{5})(\sqrt{10-2\sqrt{5}})}}{\sqrt{80}}\\
	 =&\sqrt{\dfrac{60-12\sqrt{5}-20\sqrt{5}+20}{80}}\\
	 =&\sqrt{\dfrac{80-32\sqrt{5}}{80}}\\
	 =&\sqrt{\dfrac{5-2\sqrt{5}}{5}}
	 	\end{align*}
 	 \begin{align*}
 	 	\cot\ang{18}=&\dfrac{1}{\sqrt{\dfrac{5-2\sqrt{5}}{5}}}\\
 	 	=&\sqrt{\dfrac{5}{5-2\sqrt{5}}}\\
 	 	=&\sqrt{\dfrac{5}{5-2\sqrt{5}}}\sqrt{\dfrac{5+2\sqrt{5}}{5+2\sqrt{5}}}\\
 	 	=&\sqrt{\dfrac{5(5+2\sqrt{5})}{25-20}}\\
 	 	=&\sqrt{5+2\sqrt{5}}
\end{align*}
\section{$alpha=\ang{72}$}
\begin{align*}
	\cos\ang{72}=&\sin(\ang{90}-\ang{72})\\
	=&\sin\ang{18}\\
	=&\dfrac{-1+\sqrt{5}}{4}
\end{align*}
\begin{align*}
	\sin\ang{72}=&\cos(\ang{90}-\ang{72})\\
	=&\cos\ang{18}\\
	=&\dfrac{\sqrt{10+2\sqrt{5}}}{4}
\end{align*}
\begin{align*}
	\tan\ang{72}=&\dfrac{\sin\ang{72}}{\cos\ang{72}}\\
	=&\dfrac{\cos(\ang{90}-\ang{72})}{\sin(\ang{90}-\ang{72})}\\
	=&\dfrac{\cos(\ang{18})}{\sin\ang{18}}\\	
	=&\cot\ang{18}\\
	=&\sqrt{5+2\sqrt{5}}
\end{align*}
\begin{align*}
	\cot\ang{72}=&\dfrac{\cos\ang{72}}{\sin\ang{72}}\\
	=&\dfrac{\sin(\ang{90}-\ang{72})}{\cos(\ang{90}-\ang{72})}\\
	=&\dfrac{\sin(\ang{18})}{\cos\ang{18}}\\	
	=&\tan\ang{18}\\
	=&\sqrt{\dfrac{5-2\sqrt{5}}{5}}
\end{align*}