% !TeX root = Asparsi.tex
% !BIB TS-program = biber
% !TeX encoding = UTF-8
% !TeX spellcheck = it_IT
\chapter{Pitagora}
\begin{thm}[Teorema di Pitagora]
In un triangolo rettangolo\index{Triangolo!rettangolo}\index{Pitagora} di cateti $a$, $b$ e ipotenusa $c$ avremo\[c^2=a^2+b^2\]
\end{thm}
% TODO: \usepackage{graphicx} required
\begin{figure}
	\centering
	\includestandalone{geometria/teoremapitagorauno}
	\caption{Teorema di Pitagora}
	\label{fig:teoremapitagorauno}
\end{figure}
\begin{proof}
	Consideriamo la~\vref{fig:teoremapitagorauno}, il quadrato ha il lato di lunghezza $a+b$.  Il poligono $EFGH$ è un quadrato ha i lati che sono le ipotenuse di quattro triangoli rettangoli uguali. Ha inoltre tutti gli angoli retti. Dimostriamolo per l'angolo $EHG$. Dato che la somma di $CHG$ e $EHD$ è  retto e visto che $DHE+GHC+EHG=DHC$ segue che $EHG$ è retto.\par 
	L'area del quadrato $ABCD$ è uguale alla somma dell'area del quadrato $EFGH$ e quelle dei triangoli rettangoli $AFE$, $FBG$, $GCH$ e $HDE$.\par Quindi
	\begin{align*}
	(a+b)^2=&c^2+4\dfrac{1}{2}ab\\
	a^2+b^2+2ab=&c^2+2ab\\
	c^2=&a^2+b^2
	\end{align*}
	Come volevasi dimostrare.
\end{proof}