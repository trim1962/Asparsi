\subsection{Tangente formula di sdoppiamento}
\begin{thm}[Formula di sdoppiamento]\label{thm:Formulasdoppiamento_parab}
	Data una parabola~\cite{ReFraschini2008} $y=ax^2+bx+c$ e $P(x_0,y_0)$ un suo punto. Allora la retta tangente alla parabola per $P$ ha equazione \[ \dfrac{y+y_0}{2}=ax_0x+b\left(\dfrac{x+x_0}{2}\right)+c \] 
\end{thm}\index{Parabola!tangente}
\begin{proof}
	Poniamo a sistema la parabola con il fascio di rette di centro $P$ 
	\begin{align*}
	&\begin{cases}
	y=ax^2+bx+c\\y-y_0=m(x-x_0)
	\end{cases}
	\intertext{Otteniamo}
	ax^2+bx+c-y_0-mx+mx_0=&0\\
	ax^2+(b-m)x+c-y_0+mx_0=&0\\
	\intertext{Dato che $P$ è un punto di tangenza, le soluzioni sono coincidenti, quindi la somma delle soluzioni è}
	-\dfrac{b-m}{a}=&2x_0\\
	m=&2ax_0+b\\
	\intertext{Sostituendo nell'equazione del fascio di rette }
	y-y_0=&(2ax_0+b)(x-x_0)\\
	y=&2ax_0x-2ax_0^2+bx-bx_0+y_0\\
	\intertext{Il punto $P$ appartiene alla parabola quindi}
		y=&2ax_0x-2ax_0^2+bx-bx_0+ax_0^2+bx_0+c\\
\intertext{Dividendo per due ambi i membri}
\dfrac{y}{2}=&ax_0x-ax_0^2+\dfrac{b}{2}x+\dfrac{a}{2}x_0^2+\dfrac{c}{2}\\
\intertext{Aggiungendo a sinistra e a destra $\dfrac{x_0+y_0}{2}$ otteniamo}
\dfrac{y}{2}+\dfrac{x_0+y_0}{2}=&ax_0x-ax_0^2+\dfrac{b}{2}x+\dfrac{a}{2}x_0^2+\dfrac{c}{2}+\dfrac{x_0+y_0}{2}\\
\dfrac{y}{2}+\dfrac{y_0}{2}=&-\dfrac{x_0}{2}+ax_0x-ax_0^2+\dfrac{b}{2}x+\dfrac{a}{2}x_0^2+\dfrac{c}{2}+\dfrac{x_0}{2}+\dfrac{y_0}{2}\\
\intertext{Semplificando}
\dfrac{y+y_0}{2}=&ax_0x-ax_0^2+\dfrac{b}{2}x+\dfrac{a}{2}x_0^2+\dfrac{c}{2}+\dfrac{y_0}{2}\\
\intertext{Il punto $P$ appartiene alla parabola quindi}
\dfrac{y+y_0}{2}=&ax_0x-ax_0^2+\dfrac{b}{2}x+\dfrac{a}{2}x_0^2+\dfrac{c}{2}+\dfrac{a}{2}x_0^2+\dfrac{b}{2}x_0+\dfrac{c}{2}\\
\intertext{Semplificando}
\dfrac{y+y_0}{2}=&ax_0x+\dfrac{b}{2}x+\dfrac{b}{2}x_0+c\\
\dfrac{y+y_0}{2}=&ax_0x+b\dfrac{x+x_0}{2}+c\\
	\end{align*}
	Da cui la tesi.
\end{proof}