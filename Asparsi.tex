% !TeX root = Asparsi.tex
% !BIB TS-program = biber
% !TeX encoding = UTF-8
% !TeX spellcheck = it_IT

\documentclass[a4paper,oneside]{book}%
\usepackage{cmap}
\frenchspacing%
\usepackage{amsmath}

\usepackage{amssymb}
\usepackage[italian]{babel}
\usepackage[thmmarks,hyperref]{ntheorem}
\usepackage{miamatematica}
\usepackage{copyright}
\usepackage{lmodern} % load vector font
\usepackage[T1]{fontenc} % font encoding
\usepackage[utf8]{inputenc} % input encoding
%\usepackage{noto}
\usepackage[babel=true]{microtype}
%\usepackage{geometry}
\usepackage{textcomp}
\usepackage[big]{layaureo}
%\geometry{top=1.5cm,bottom=1.5cm}
\usepackage{grafica}

%Teorema
\theoremstyle{marginbreak}
\theoremheaderfont{\normalfont\bfseries}\theorembodyfont{\slshape}
\theoremsymbol{\ensuremath{\diamondsuit}}
\theoremseparator{:} %
\newtheorem{thm}{Teorema}[section]
%Proprietà
\theoremstyle{marginbreak}
\theoremheaderfont{\normalfont\bfseries}\theorembodyfont{\slshape}
\theoremsymbol{\ensuremath{\diamondsuit}}
\theoremseparator{:}
\newtheorem{prop}{Proprietà}[section]
%lemma
\theoremstyle{changebreak}
\theoremsymbol{\ensuremath{\heartsuit}}
\theoremindent0.5cm
\theoremnumbering{greek}
\newtheorem{lem}[thm]{Lemma}
%corollario
\theoremindent0cm
\theoremsymbol{\ensuremath{\spadesuit}}
\theoremnumbering{arabic}
\newtheorem{cor}[thm]{Corollario}
%esempio
\theoremstyle{change}
\theorembodyfont{\upshape}
\theoremsymbol{\ensuremath{\ast}}
\theoremseparator{}
\newtheorem{exmp}{Esempio}[section]
%controesempio
\theoremstyle{change}
\theorembodyfont{\upshape}
\theoremsymbol{\ensuremath{\odot}}
\theoremseparator{}
\newtheorem{cexmp}{Contro esempio}[section]
%definizione
\theoremstyle{plain}
\theoremsymbol{\ensuremath{\clubsuit}}
\theoremseparator{.}
\theoremstyle{marginbreak}
\theoremprework{\hrule\bigskip}
\theorempostwork{\hrule\bigskip}
\newtheorem{defn}{Definizione}[section]
%commento
\theoremstyle{plain}
\theorembodyfont{\upshape}
\theoremsymbol{\ensuremath{\blacklozenge}}
\theoremseparator{:}
\newtheorem{commento}{Commento}
%dimostrazione

\theoremstyle{plain}
\theoremheaderfont{\sc}
\theorembodyfont{\bfseries}
\theoremstyle{nonumberplain}
%^{}\theoremseparator{.}

\theoremsymbol{\ensuremath{\blacksquare}}
\theoremheaderfont{\bfseries}
%\theoremstyle{nonumberplain}
%\theoremstyle{marginbreak}
\theorembodyfont{\normalfont}
\newtheorem{proof}{Dimostrazione}
%\input{../Mod_base/tabelle}



%\usepackage{adjustbox}
%\input{../Mod_base/stand_class}
\usepackage{pagina}

\setlength{\headheight}{13pt}
\usepackage{indice}
\usepackage{date}
\usepackage{unita_misura}

\usepackage{imakeidx}
\makeindex[options=-s ../Mod_base/oldclaudio.sti]

\usepackage{diagbox}
%\include{simboli_operatori}

\usepackage{stand_class}

\newcommand{\HRule}{\rule{\linewidth}{0.5mm}}


 \makeatletter
 \renewcommand\frontmatter{%
 	\cleardoublepage
 	\@mainmatterfalse
 	%\pagenumbering{roman}
 }
 \renewcommand\mainmatter{%
 	\cleardoublepage
 	\@mainmattertrue
 	%\pagenumbering{arabic}
 }
 \makeatother

\usepackage[grumpy,mark,markifdirty,raisemark=0.95\paperheight]{gitinfo2}
% 10/02/2018 :: 20:05:58 :: \usepackage{parskip}
\usepackage[toc,page]{appendix}

\renewcommand{\appendixtocname}{Appendici}

\renewcommand{\appendixpagename}{Appendici}


\usepackage[style=italian]{csquotes}
\usepackage[%
style=philosophy-modern,
annotation=true,
hyperref,
backend=biber,
backref]{biblatex}
\addbibresource{formulario.bib}
\usepackage[italian]{varioref}
\usepackage{hyperxmp}
\usepackage[pdfpagelabels]{hyperref}
\usepackage[italian,noabbrev]{cleveref}
\crefname{defn}{definizione}{definizioni}
\Crefname{defn}{Definizione}{Definizioni}
\crefname{thm}{teorema}{teoremi}
\Crefname{thm}{Teorema}{Teoremi}
\crefname{cor}{corollario}{corollari}
\Crefname{cor}{Corollario}{Corollari}
\crefname{equation}{equazione}{equazioni}
\Crefname{equation}{Equazione}{Equazioni}
\crefname{sistema}{sistema}{sistemi}
\Crefname{sistema}{Sistema}{Sistemi}
\crefname{lem}{lemma}{lemmi}
\Crefname{lem}{Lemma}{Lemmi}
\creflabelformat{equation}{#2\textup{#1}#3}

\usepackage{tcolorboxgest}
\title{Zibaldone di pensieri}
\author{Claudio Duchi}
\date{\datetime}
\hypersetup{%
pdfencoding=auto,
urlcolor={blue},
pdftitle={Appunti sparsi},
pdfsubject={Per non dimenticare},
pdfstartview={FitH},
pdfpagemode={UseOutlines},
pdflicenseurl={http://creativecommons.org/licenses/by-nc-nd/3.0/},
pdflang={it},
pdfmetalang={it},
pdfkeywords={Algebra, geometria, analisi},
pdfcopyright={Copyright (C) 2019, Claudio Duchi},
pdfcontacturl={http://breviariomatematico.altervista.org},
pdfcontactpostcode={06128},
pdfcontactphone={},
pdfcontactemail={claduc},
pdfcontactcountry={Italy},
pdfcontactcity={Perugia},
pdfcontactaddress={},
pdfcaptionwriter={Claudio Duchi},
pdfauthortitle={},%
pdfauthor={Claudio Duchi},
linkcolor={blue},
colorlinks=true,
citecolor={red},
breaklinks,
bookmarksopen,
verbose,
baseurl={http://breviariomatematico.altervista.org}
}

% !TeX root = Asparsi.tex
% !BIB TS-program = biber
% !TeX encoding = UTF-8
% !TeX spellcheck = it_IT
\includeonly{%
simmetrie,
equasecondo,
scompsecgrado,
numdecperiodici,
numirrazionali,
sislineari,
brafibo,
Triangolo,
Progressioni,
NumComplFormGonio,
%parabola,
contro,
geometriaanalitica,
ParabolaFormSdopp,
goniometria,
trigonometria,
radicalidoppi,
}


%patch allieamento lista teoremi
\usepackage{regexpatch}
\makeatletter
%\xpatchcmd*{\thm@@thmline}{2.3em}{5em}{}{} % not really needed
\xpatchcmd*{\thm@@thmline@name}{2.3em}{5em}{}{} 
\xpatchcmd*{\thm@@thmline@noname}{2.3em}{5em}{}{}
\makeatother
%fine patch allieamento lista teoremi
\usepackage{CDloghi}
\listfiles
\begin{document}
\setcounter{page}{2}
		\frontmatter
		\begin{titlepage}\parindent=0pt
			\centering
			\parbox{0.8\textwidth}{\centering
%			\begin{center}
				\Lgrandedue\\[1cm]   
				\textsc{\LARGE Claudio Duchi}\\[1.5cm]
				\HRule \\[0.4cm]
				{ \huge \bfseries Zibaldone}\\[0.4cm]
				{ \large \bfseries di}\\[0.4cm]
				{ \large \bfseries pensieri}\\[0.4cm]
				\HRule \\[1.5cm]
				\vfill
			\begin{tikzpicture}
			\renewcommand*{\VertexBallColor}{green!50!black}
			\GraphInit[vstyle=Art]
\grCirculant[RA=6]{5}{1,3,10}%
%			\grComplete[RA=6]{20}
			\end{tikzpicture}
%		\end{center}

\makebox[\linewidth]{\centering
		Release:\gitReln\ (\gitAbbrevHash)\ Autore:\gitAuthorName\ 
  \gitCommitterDate}
}
		\end{titlepage}
	\hypersetup{pageanchor=true}
		\CDcopyright
		\tableofcontents
		\chapter*{Lista dei teoremi}
		\theoremlisttype{allname}
		\listtheorems{thm,defn,cor,comm,lem}
	\addcontentsline{toc}{chapter}{\listfigurename}%
		\listoffigures
%	\addcontentsline{toc}{chapter}{\listtablename}%
%			\listoftables
			\mainmatter

% !TeX root = Asparsi.tex
% !BIB TS-program = biber
% !TeX encoding = UTF-8
% !TeX spellcheck = it_IT

% !TeX root = Asparsi.tex
% !BIB TS-program = biber
% !TeX encoding = UTF-8
% !TeX spellcheck = it_IT
\chapter{Simmetrie}
\section{Simmetria assiale}
\begin{defn}[Simmetria assiale]\label{defn:Sassiale1}
Ho una simmetria assiale\index{Simmetria!assiale} di asse $r$ quando presi due punti $P$ e $Q$:
\begin{enumerate}
	\item Il punto medio $M$ del segmento $PQ\in r$
	\item Il segmento $PQ$ è perpendicolare ad $r$
\end{enumerate} 
\end{defn}
\begin{thm}\label{thm:Sassiale1}
In una simmetria assiale di asse $r:y=mx+q$ la trasformazione che lega i punti della simmetria ha equazione:
\[\begin{cases}
x_0=\frac{2m(y_1-q)+(1-m^2)x_1}{1+m^2}\\
\\
y_0=\frac{2(mx_1+q)+(m^2-1)y_1}{1+m^2}
\end{cases}\]
\end{thm}
\begin{proof}
	Supponiamo di avere due punti $P(x_1,y_1)$ e $Q(x_0,y_0)$ con $x_1\neq x_0$
	
	 Dalla~\cref{defn:Sassiale1} se $P$ e $Q$ sono simmetrici rispetto alla retta $r$ 
	  \begin{equation}
	 y=mx+q\label[equation]{eq:ass1}
	 \end{equation}
	 allora il punto \[M\left(\frac{x_1+x_0}{2},\frac{y_1+y_0}{2}\right)\] appartiene a $r$ quindi avremo 
	 \begin{equation}
	 \dfrac{y_1+y_0}{2}=m\dfrac{x_1+x_0}{2}+q\label[equation]{eq:ass2}
	 \end{equation}
	 
	 Dalla~\cref{defn:Sassiale1} la retta $r$ è perpendicolare a $PQ$ quindi se $m$ è il coefficiente angolare della retta di~\cref{eq:ass1} avremo:\begin{equation}
	 \dfrac{y_1-y_0}{x_1-x_0}=-\dfrac{1}{m}\label[equation]{eq:ass3}
	 \end{equation}
	 Utilizzando \crefrange{eq:ass2}{eq:ass3} otteniamo il sistema
	 \begin{equation}
	 \begin{cases}
	 \dfrac{y_1+y_0}{2}=m\dfrac{x_1+x_0}{2}+q\\[0.4cm]
	 \dfrac{y_1-y_0}{x_1-x_0}=-\dfrac{1}{m}
	 \end{cases}\label[equation]{eq:ass4}
	 \end{equation}
	 che diventa
	  \begin{align*}
	 & \begin{cases}
	 y_1+y_0=mx_1+mx_0+2q\\
	 y_1-y_0=-\dfrac{1}{m}x_1+\dfrac{1}{m}x_0
	 \end{cases}\\
%	  & \begin{cases}
%	 y_0=mx_1+mx_0+2q-y_1\\
%	 y_1-mx_1-mx_0-2q+y_1=-\dfrac{1}{m}x_1+\dfrac{1}{m}x_0
%	 \end{cases}\\
	 & \begin{cases}
	 y_0=mx_1+mx_0+2q-y_1\\
	 \dfrac{1}{m}x_0+mx_0= y_1-mx_1-2q+y_1+\dfrac{1}{m}x_1
	 \end{cases}\\
	 & \begin{cases}
	 y_0=mx_1+mx_0+2q-y_1\\
	 x_0+m^2x_0= my_1-m^2x_1-2mq+my_1+x_1
	 \end{cases}\\
%	 & \begin{cases}
%	 y_0=mx_1+mx_0+2q-y_1\\
%	 x_0=\dfrac{2my_1-m^2x_1-2mq+x_1}{1+m^2}
%	 \end{cases}\\
	 & \begin{cases}
	 y_0=mx_1+\dfrac{2m^2y_1-m^3x_1-2m^2q+mx_1}{1+m^2}+2q-y_1\\
	 x_0=\dfrac{2my_1-m^2x_1-2mq+x_1}{1+m^2}
	 \end{cases}\\
%	  & \begin{cases}
%	 y_0=\dfrac{(1+m^2)mx_1+2m^2y_1-m^3x_1-2m^2q+mx_1+2(1+m^2)q-(1+m^2)y_1}{1+m?"}\\
%	 x_0=\dfrac{2my_1-m^2x_1-2mq+x_1}{1+m^2}
%	 \end{cases}\\
	 & \begin{cases}
	 	y_0=\dfrac{mx_1+m^3x_1+2m^2y_1-m^3x_1-2m^2q+mx_1+2q+2m^2q-y_1-m^2y_1}{1+m?"}\\
	 	x_0=\dfrac{2my_1-m^2x_1-2mq+x_1}{1+m^2}
	 \end{cases}\\
	 \intertext{quindi}
	 &\begin{cases}
	 x_0=\frac{2m(y_1-q)+(1-m^2)x_1}{1+m^2}\\[0.4cm]
	 y_0=\frac{2(mx_1+q)+(m^2-1)y_1}{1+m^2}
	 \end{cases}
 \end{align*}
 \end{proof}
Discutendo il risultato del~\cref{thm:Sassiale1} otteniamo i seguenti corollari:
\begin{cor}
	Per $m=1$ il sistema diviene 
	\begin{equation}
	\begin{cases}
	x_0=y_1-q\\
	y_1=x_1+q
	\end{cases}
	\end{equation}
\end{cor}
\begin{cor}
	Per $m=-1$ il sistema diviene 
	\begin{equation}
	\begin{cases}
	x_0=-y_1-q\\
	y_1=-x_1+q
	\end{cases}
	\end{equation}
\end{cor}
\begin{cor}
	Per $m=0$ il sistema diviene 
	\begin{equation}
	\begin{cases}
	x_0=x_1\\
	y_1=2q-y_1
	\end{cases}
	\end{equation}
\end{cor}			

\nocite{*}
 \addcontentsline{toc}{chapter}{\bibname}
\printbibliography
 \addcontentsline{toc}{chapter}{\indexname}
 \printindex
 \appendix
 \chapter{Mezzi usati}
 \CDMezziUsati
\end{document}
