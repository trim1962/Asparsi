% !TeX root = Asparsi.tex
% !BIB TS-program = biber
% !TeX encoding = UTF-8
% !TeX spellcheck = it_IT
\chapter{Poligoni}\label{ch:poligoni}
\begin{defn}[Diagonali]\label{defn:poligonidiagonale}
In un poligono regolare una diagonale è il segmento che unisce due vertici non consecutivi.
\end{defn}\index{Diagonale}\index{Poligono!diagonale}
\begin{defn}[Apotema]\label{defn:poligoniapotema}
	In un poligono regolare l'apotema è il raggio della circonferenza inscritta.
\end{defn}\index{Apotema}\index{Poligono!apotema}
\begin{thm}[Area poligono circoscritto]\label{thm:areapoligonocircoscritto}
Dato un poligono di perimetro $2p$ ed area $A$ allora 
\[A=pa\]
dove $a$ è l'apotema.
\end{thm}\index{Poligono!area}\index{Poligono!apotema}\index{Poligono!perimetro}
\begin{figure}
	\centering
	\includestandalone{geometria/circonferenzainscrittapoligono}
	\caption{Area poligono circoscritto}
	\label{fig:circonferenzainscrittapoligono}
\end{figure}
\begin{proof}
Consideriamo la~\cref{fig:circonferenzainscrittapoligono}. Avremo che
\begin{align*}
	A_{ABCD}=&A_{AOB}+A_{BOC}+A_{COD}+A_{AOD}\\
	=&\dfrac{aR}{2}+\dfrac{bR}{2}+\dfrac{cR}{2}+\dfrac{dR}{2}\\
	=&(a+b+c+d)\dfrac{R}{2}\\
	=&2p\dfrac{R}{2}\\
	=&pR
\end{align*}
Come volevasi dimostrare.
\end{proof}
\begin{thm}[Rapporto tra apotema e lato]\label{thm:rapportoapotemalato}
In un poligono regolare il rapporto tra apotema e lato è costante.
\[\dfrac{a}{l}=\dfrac{1}{2\tan{\dfrac{\pi}{n}}}=\phi\]
\end{thm}\index{Poligono!regolare}\index{Apotema}\index{Poligono!lato}
\begin{figure}
	\centering
	\includestandalone{geometria/latoapotema}
	\caption{Rapporto lato apotema}
	\label{fig:latoapotema}
\end{figure}
\begin{proof}
	Costruiamo la~\cref{fig:latoapotema}.  Ogni poligono regolare è formato da $n$ triangoli isosceli formati da i raggi della circonferenza circoscritta e dal lato $r$. Consideriamo il triangolo $ADC$. Questo triangolo è retto in D. L'angolo in $C$ è $\dfrac{2\pi}{2n}=\dfrac{\pi}{n}=\phi$. Applicando la trigonometria abbiamo:
\begin{align*}
\dfrac{l}{2}=&R\sin\dfrac{\pi}{n}\\
R=&\dfrac{l}{2\sin\dfrac{\pi}{n}}\\
a=&R\cos\dfrac{\pi}{n}
\intertext{ma semplificando}
a=&\dfrac{l}{2\sin\dfrac{\pi}{n}}\cos\dfrac{\pi}{n}\\
=&\dfrac{l}{2\tan{\dfrac{\pi}{n}}}\\
\dfrac{a}{l}=&\dfrac{1}{2\tan{\dfrac{\pi}{n}}}=\phi_n
\end{align*}
Come volevasi dimostrare.
\end{proof}
\begin{defn}[Numero fisso poligono]
	Dato un poligono di $n$ lati chiamo numero fisso del poligono
	\[\phi_n=\dfrac{1}{2\tan{\dfrac{\pi}{n}}}\]
\end{defn}\index{Poligono!numero fisso}
\begin{thm}[Rapporto tra area e lato]
In un poligono regolare il rapporto tra l'area e il quadrato del lato è costante.
\[\dfrac{A}{l^2}=\dfrac{n}{2}\phi\]
\end{thm}
\begin{proof}
	\begin{align*}
		\intertext{Per il~\cref{thm:areapoligonocircoscritto}}
		A=&pa\\
		\intertext{Per il~\cref{thm:rapportoapotemalato}}
		a=&l\phi_n\\
		A=&pl\phi_n\\
		\intertext{ma}
		p=&\dfrac{nl}{2}\\
		\intertext{quindi}
		A=&\dfrac{nl}{2}l\phi\\
		A=&\dfrac{nl^2}{2}\phi\\
		\dfrac{A}{l^2}=&\dfrac{n}{2}\phi_n\\
	\end{align*}
Come volevasi dimostrare.
\end{proof}
\section{Esagono}
\begin{figure}
	\centering
	\includestandalone{geometria/esagono}
	\caption{Esagono}
	\label{fig:esagono1}
\end{figure}
\begin{thm}[Diagonali]
	Un esagono ha nove diagonali.
\end{thm}\index{Esagono!diagonale}\index{Diagonale!esagono}

\begin{figure}
	\centering
	\includestandalone{geometria/esagono_diagonali}
	\caption{Esagono diagonali}
	\label{fig:esagonodiagonali}
\end{figure}
\begin{proof}
	Consideriamo la~\cref{fig:esagonodiagonali}. Da ogni vertice escono tre diagonali i vertici sono sei quindi abbiamo \num{18} diagonali ma ogni diagonale collega due vertici quindi il numero totale delle diagonali è nove. 
	
	Come volevasi dimostrare.
\end{proof}
\begin{thm}[Lato esagono]\label{thm:latoesagonoinscritto}
	L'esagono inscritto in una circonferenza ha i lati uguali al raggio della circonferenza.
\end{thm}\index{Esagono!lato}
\begin{proof}
	Consideriamo la~\cref{fig:esagono1}. Il triangolo $OCD$ è isoscele di lati uguali $OC$ e $OD$. Quindi gli angoli in $C$ e in $D$ sono uguali. L'angolo in $O$ è la sesta parte di un angolo giro. Quindi
	\begin{align*} 
		2x+\ang{60}=&\ang{180}\\
		x=&\ang{60}
	\end{align*} 
Dato che un triangolo equiangolo è equilatero i tre lati sono uguali.

Come volevasi dimostrare.
\end{proof}
\begin{thm}[Area esagono]
	L'area dell'esagono di lato $R$ è \[A=\dfrac{3}{2}R^2\sqrt{3}\]
\end{thm}\index{Esagono!area}
\begin{proof}
Consideriamo la~\cref{fig:esagono1}. Calcoliamo l'area del triangolo $OCD$. 
\begin{align*}
	A_{OCD}=&Ra\dfrac{1}{2}
	\intertext{Abbiamo:}
	a=&\sqrt{\dfrac{4R^2-R^2}{4}}\\
	=&	\sqrt{\dfrac{4R^2-R^2}{4}}\\
	=&\dfrac{R}{2}\sqrt{3}\\
	\intertext{sostituendo}
	A_{OCD}=&\dfrac{R^2}{4}\sqrt{3}\\
	\intertext{L'esagono è formato da sei triangoli quindi:}
	A=&6\dfrac{R^2}{4}\sqrt{3}\\
	=&\dfrac{3}{2}R^2\sqrt{3}
\end{align*}
Come volevasi dimostrare.
\end{proof}
\begin{thm}[Apotema dell'esagono]
	in un esagono l'apotema è:\[a=\dfrac{R}{2}\sqrt{3} \]
\end{thm}
\begin{figure}
	\centering
	\includestandalone{geometria/esagono_inscritta_circoscritta}
	\caption{Esagono inscritto e circoscritto}
	\label{fig:esagonoinscrittocircoscritto}
\end{figure}
