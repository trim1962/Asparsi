% !TeX root = Asparsi.tex
% !BIB TS-program = biber
% !TeX encoding = UTF-8
% !TeX spellcheck = it_IT
\chapter{Angoli valori numerici}\label{chap:Angoli_valori_numerici}
\section{$alpha=\ang{15}$}
\begin{align*}
	\cos\ang{15}=&\cos(\ang{60}-\ang{45})\\
	=&\difcos{60}{45}\\
	=&\cossessanta\cosquarantacinque+\sinsessanta\sinquarantecinque\\
	=&\dfrac{\sqrt{2}+\sqrt{6}}{4}\\
	\sin\ang{15}=&\sin(\ang{60}-\ang{45})\\
	=&\difsin{60}{45}\\
	=&\sinsessanta\cosquarantacinque-\cossessanta\sinquarantecinque\\
	=&\dfrac{\sqrt{6}-\sqrt{2}}{4}\\
	\tan\ang{15}=&\dfrac{\sin\ang{15}}{\cos\ang{15}}\\
	=&\dfrac{\dfrac{\sqrt{6}-\sqrt{2}}{4}}{\dfrac{\sqrt{2}+\sqrt{6}}{4}}\\
	=&\dfrac{\sqrt{6}-\sqrt{2}}{4}\dfrac{4}{\sqrt{2}+\sqrt{6}}\\
	=&\dfrac{\sqrt{6}-\sqrt{2}}{\sqrt{2}+\sqrt{6}}\\
	=&\dfrac{\sqrt{6}-\sqrt{2}}{\sqrt{2}+\sqrt{6}}\dfrac{\sqrt{6}-\sqrt{2}}{\sqrt{2}-\sqrt{6}}\\
	=&\dfrac{8-2\sqrt{12}}{6-2}\\
	=&2-\sqrt{3}\\
	\cot\ang{15}=&\dfrac{1}{\tan\ang{15}}\\
	=&\dfrac{1}{2-\sqrt{3}}\\
	=&\dfrac{1}{2-\sqrt{3}}\dfrac{2+\sqrt{3}}{2+\sqrt{3}}\\
	=2+\sqrt{3}\\
\end{align*}
\section{$alpha=\ang{75}$}
\begin{align*}
\cos\ang{75}=&\cos(\ang{30}+\ang{45})\\
=&\sumcos{30}{45}\\
=&\costrenta\cosquarantacinque-\sintrenta\sinquarantecinque\\
=&\dfrac{\sqrt{6}}{4}-\dfrac{\sqrt{2}}{4}\\
=&\dfrac{\sqrt{6}-\sqrt{2}}{4}\\
\sin\ang{75}=&\cos(\ang{90}-\ang{75})\\
=&\cos\ang{15}\\
=&\dfrac{\sqrt{6}+\sqrt{2}}{4}\\
\tan\ang{75}=&\dfrac{\sin\ang{15}}{\cos\ang{15}}\\
=&\dfrac{\dfrac{\sqrt{6}+\sqrt{2}}{4}}{\dfrac{\sqrt{2}-\sqrt{6}}{4}}\\
=&\dfrac{\sqrt{6}+\sqrt{2}}{4}\dfrac{4}{\sqrt{2}-\sqrt{6}}\\
=&\dfrac{\sqrt{6}+\sqrt{2}}{\sqrt{2}-\sqrt{6}}\\
=&\dfrac{\sqrt{6}+\sqrt{2}}{\sqrt{2}-\sqrt{6}}\dfrac{\sqrt{6}+\sqrt{2}}{\sqrt{2}+\sqrt{6}}\\
=&\dfrac{8+2\sqrt{12}}{6-2}\\
=&2+\sqrt{3}\\
\cot\ang{75}=&\dfrac{1}{\tan\ang{75}}\\
=&\dfrac{1}{2+\sqrt{3}}\\
=&\dfrac{1}{2+\sqrt{3}}\dfrac{2-\sqrt{3}}{2-\sqrt{3}}\\
=&2-\sqrt{3}\\
\end{align*}
\section{$alpha=\ang{18}$}
\begin{align*}
	\sin\ang{18&}
	a=&\ang{18}\\
	5a=&\ang{90}\\
	2a+3a=&\ang{90}\\
	2a=&\ang{90}-3a\\
	\sin 2a=&\sin(\ang{90}-3a)=\cos 3a\\
	\intertext{trasformiamo l'espressione}
	\sin 2a=&2\sin a\cos a\\
	\cos 3a=&\cos2a\cos a-\sin 2a\sin a\\
	 =&\cos a(2\cos^2 a -1)-2\sin^2 a\cos a\\
	 =&\cos^3 a -\cos a -2(1-cos^2 a)\cos a\\
	 =&\cos^3 a -\cos a -2\cos a +cos^3 a\\
	 =&4\cos^3 a-3\cos a
	 \intertext{Riprendendo}
	 2\sin a\cos a=&4\cos^3 a-3\cos a
	 \intertext{$\cos a\neq 0$}
	 2\sin a=&4\cos^2 a-3 \\
	 2\sin a-4\cos^2 a+3=&0 \\
	 2\sin a-4(1-\sin^2 a) a+3=&0 \\
	 4\sin^2 a+2\sin a-1=&0 \\
	 \sin a=&\dfrac{-2\pm\sqrt{4+16}}{8}\\
	 =&\dfrac{-2\pm\sqrt{20}}{8}\\
	 =&\dfrac{-2\pm2\sqrt{5}}{8}\\
	 =&\dfrac{-1\pm\sqrt{5}}{4}\\
	 \intertext{ma $\sin\ang{18}>0$}
	 \sin a=&\dfrac{-1+\sqrt{5}}{4}\\
\end{align*}