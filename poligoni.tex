% !TeX root = Asparsi.tex
% !BIB TS-program = biber
% !TeX encoding = UTF-8
% !TeX spellcheck = it_IT
\chapter{Poligoni}\label{ch:poligoni}
\begin{defn}[Diagonali]\label{defn:poligonidiagonale}
In un poligono regolare una diagonale è il segmento che unisce due vertici non consecutivi.
\end{defn}\index{Diagonale}\index{Poligono!diagonale}
\begin{defn}[Apotema]\label{defn:poligoniapotema}
	In un poligono regolare l'apotema è il raggio del cerchio inscritto.
\end{defn}\index{Apotema}\index{Poligono!apotema}
\section{Esagono}\label{sec:esagono }
\begin{figure}
	\centering
	\includestandalone{geometria/esagono}
	\caption{Esagono}
	\label{fig:esagono1}
\end{figure}
\begin{thm}[Diagonali]
	Un esagono ha nove diagonali.
\end{thm}\index{Esagono!diagonale}\index{Diagonale!esagono}
\begin{figure}
	\centering
	\includestandalone{geometria/esagono_diagonali}
	\caption{Esagono diagonali}
	\label{fig:esagonodiagonali}
\end{figure}
\begin{proof}
	Consideriamo la~\cref{fig:esagonodiagonali}. Da ogni vertice escono tre diagonali i vertici sono sei quindi abbiamo \num{18} diagonali ma ogni diagonale collega due vertici quindi il numero totale delle diagonali è nove. 
	
	Come volevasi dimostrare.
\end{proof}
\begin{thm}[Lato esagono]\label{thm:latoesagonoinscritto}
	L'esagono inscritto in una circonferenza ha i lati uguali al raggio della circonferenza.
\end{thm}\index{Esagono!lato}
\begin{proof}
	Consideriamo la~\cref{fig:esagono1}. Il triangolo $OCD$ è isoscele di lati uguali $OC$ e $OD$. Quindi gli angoli in $C$ e in $D$ sono uguali. L'angolo in $O$ è la sesta parte di un angolo giro. Quindi
	\begin{align*} 
		2x+\ang{60}=&\ang{180}\\
		x=&\ang{60}
	\end{align*} 
Dato che un triangolo equiangolo è equilatero i tre lati sono uguali.

Come volevasi dimostrare.
\end{proof}
\begin{thm}[Area esagono]
	L'area dell'esagono di lato $R$ è \[A=\dfrac{3}{2}R^2\sqrt{3}\]
\end{thm}\index{Esagono!area}
\begin{proof}
Consideriamo la~\cref{fig:esagono1}. Calcoliamo l'area del triangolo $OCD$. 
\begin{align*}
	A=&Ra\dfrac{1}{2}
	\intertext{Abbiamo:}
	a=&\sqrt{\dfrac{4R^2-R^2}{4}}\\
	=&	\sqrt{\dfrac{4R^2-R^2}{4}}\\
	=&\dfrac{R}{2}\sqrt{3}\\
	\intertext{sostituendo}
	=&\dfrac{R^2}{4}\sqrt{3}\\
	\intertext{L'esagono è formato da sei triangoli quindi:}
	A=&6\dfrac{R^2}{4}\sqrt{3}\\
	=&\dfrac{3}{2}R^2\sqrt{3}
\end{align*}
Come volevasi dimostrare.
\end{proof}
\begin{figure}
	\centering
	\includestandalone{geometria/esagono_inscritta_circoscritta}
	\caption{Esagono inscritto e circoscritto}
	\label{fig:esagonoinscrittocircoscritto}
\end{figure} 