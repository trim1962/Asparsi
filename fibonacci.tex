% !TeX root = Asparsi.tex
% !BIB TS-program = biber
% !TeX encoding = UTF-8
% !TeX spellcheck = it_IT
\chapter{Numeri di Fibonacci e Lucas}
\section{Definizione}
\begin{defn}[Numeri di Fibonacci]\index{Fibonacci!definizione}
	\begin{align}
		\Fib{0}={}&0\notag\\
		\Fib{1}={}&1\notag\\
		\Fib{n}={}&\Fib{n-1}+\Fib{n-2}\quad n>2\label{eqn:Fibodef}
	\end{align}
\end{defn}
\begin{defn}[Numeri di Lucas]\index{Lucas!definizione}
	\begin{align}
		\Luc{0}={}&2\notag\\
		\Luc{1}={}&1\notag\\
		\Luc{n}={}&\Luc{n-1}+\Luc{n-2}\quad n>2\label{eqn:Lucadef}
	\end{align}
\end{defn}
\section{Sezione aurea}
\begin{defn}[Sezione aurea]
	Date due quantità $a$ e $b$ con $a>b>0$ diremo sezione aurea il rapporto
	\begin{equation}
	a+b:a=a:b=\varphi	
	\end{equation}\label{eqn:FibAureaDef}\index{Sezione!aurea}
\end{defn}
\begin{prop}
	Dalla~\vref{eqn:FibAureaDef} abbiamo
	\begin{align}
		\dfrac{a}{b}={}&\dfrac{a+b}{a}\notag\\
		={}&1+\dfrac{b}{a}\notag\\
		={}&1+\dfrac{1}{\frac{a}{b}}\notag\\
		\intertext{quindi}
		\varphi={}&1+\dfrac{1}{\varphi}\label{eqn:FibPhiProp}\\
		\varphi^2={}&\varphi+1\notag
	\end{align}
\end{prop}
	Dalla~\vref{eqn:FibPhiProp} segue
	\begin{prop}
	\begin{align}
		\varphi^2={}&\varphi+1\notag\\
		\varphi^2-\varphi-1={}&0\notag\\
		x^2-x-1={}&0\label{eqn:FibValPhiEqua}\\
		x_1={}&\dfrac{1+\sqrt{5}}{2}\label{eqn:FibValPhi}\\
		x_1={}&\dfrac{1-\sqrt{5}}{2}\notag
	\end{align}
\end{prop}
Il risultato~\ref{eqn:FibValPhi} permette di scrivere la seguente definizione
\begin{defn}[Sezione aurea]
La sezione aurea è: 
\begin{align*}
	\varphi={}&\dfrac{1+\sqrt{5}}{2}\\
\intertext{inoltre vale}
	1-	\varphi={}&\dfrac{1-\sqrt{5}}{2}\\
\end{align*}
\end{defn}\index{Sezione!aurea!valore}
\begin{lem}[Proprietà]\label{lem:FibpropPhi}
	Se $a=\varphi$ e  $b=1-\varphi$ allora
	\begin{align*}
		ab={}&-1\\
		a^2+1={}&(a-b)a\\
		b^2+1={}&(b-a)b\\
		a^2={}&a+1\\
		b^2={}&b+1\\
		a+b={}&1\\
		a^3={}&a^2+a\\
		b^3={}&b^2+b\\
		a^2+b^2={}&a+b+2\\
	a-b={}&\sqrt{5}\\
	a^2b+b={}&-a+b\\
	-b^2a-a={}&b-a\\
		\end{align*}
\end{lem}
\begin{proof}
	\begin{align*}
		ab={}&\dfrac{1+\sqrt{5}}{2}\dfrac{1-\sqrt{5}}{2}\\
		={}&\dfrac{1-5}{4}\\
		={}&-1
	\end{align*}
	\begin{align*}
		(a-b)a={}&a^2-ab=a^2+1\\
		(b-a)b={}&b^2-ab=b^2+1\\
	\end{align*}
	cvd.
\end{proof}
\section{Formula di Binet}
\begin{thm}[Formula di Binet]\index{Formula!Binet!Fibonacci}
	Se $\Fib{n}$ è la successione di Fibonacci allora avremo:
	\begin{equation}
		\Fib{n}=\dfrac{1}{\sqrt{5}}\left[\left(\dfrac{1+\sqrt{5}}{2}\right)^n-\left(\dfrac{1-\sqrt{5}}{2}\right)^n\right]
	\end{equation}\label{eqn:FinBinet}
\end{thm}~\cite{Conti2020}
\begin{proof}
	Poniamo $a=\varphi$ e con $b=1-\varphi$ allora
	\begin{align*}
		\intertext{Poniamo}
		a^1={}&a\Fib{1}+\Fib{0}\\
%		a^3={}&a^2a\\
%		={}&a(a+1)\\
%		={}&a^2+a\\
%		={}&a+1+a\\
%		={}&2a+1\\
%		={}&a\Fib{3}+\Fib{2}\\
		\intertext{Per induzione su $n$}
		a^{n-1}={}&a\Fib{n-1}+\Fib{n-2}\\
		a^{n-1}a={}&a^2\Fib{n-1}+a\Fib{n-2}\\
		a^{n}={}&(a+1)\Fib{n-1}+a\Fib{n-2}\\
		a^{n}={}&a\Fib{n-1}+\Fib{n-1}+a\Fib{n-2}\\
		a^{n}={}&a(\Fib{n-1}+\Fib{n-2})+\Fib{n-1}\\
		a^{n}={}&a\Fib{n}+\Fib{n-1}\\
		\intertext{Analogamente si dimostra}
		b^{n}={}&b\Fib{n}+\Fib{n-1}\\
		\intertext{Sottraendo}
		a^n-b^n={}&a\Fib{n}-b\Fib{n}\\
		a^n-b^n={}&\Fib{n}(a-b)\\
		\Fib{n}={}&\dfrac{a^n-b^n}{a-b}
	\end{align*}
quindi
\begin{equation}
\Fib{n}=\dfrac{1}{\sqrt{5}}\left[\left(\dfrac{1+\sqrt{5}}{2}\right)^n-\left(\dfrac{1-\sqrt{5}}{2}\right)^n\right]
\end{equation}
\end{proof}
Altra dimostrazione
\begin{proof}
	\begin{align*}
		\intertext{Poniamo:}
		&\Fib{n}={}xa^n+yb^n\\
		&\left\{
		\begin{array}{l}
			a^0x+b^0y=0\\ a^1y+b^1y=1
		\end{array}\right.
	\intertext{Otteniamo}
	&\left\{
	\begin{array}{l}
	x=\dfrac{1}{a-b}\\ y=-\dfrac{1}{a-b}
	\end{array}\right.\\
%\intertext{per il~\vref{lem:FibpropPhi}}
%&\left\{
%	\begin{array}{l}
%	x=\dfrac{a}{a-b}\\ y=-\dfrac{b}{a-b}
%\end{array}\right.\\ 
	&\Fib{n}={}\dfrac{a^n-b^n}{a-b}
	\end{align*}
quindi
\begin{equation}
	\Fib{n}=\dfrac{1}{\sqrt{5}}\left[\left(\dfrac{1+\sqrt{5}}{2}\right)^n-\left(\dfrac{1-\sqrt{5}}{2}\right)^n\right]
\end{equation}
\end{proof}
\begin{thm}[Formula di Binet per lucas]\index{Formula!Binet!Lucas}
	Se $\Luc{n}$ è la successione di Fibonacci allora avremo:
	\begin{equation}
		\Luc{n}=\left(\dfrac{1+\sqrt{5}}{2}\right)^{}+\left(\dfrac{1-\sqrt{5}}{2}\right)^{n}
	\end{equation}\label{eqn:LucBinet}
\end{thm}
\begin{proof}
	\begin{align*}
		\intertext{Poniamo:}
		&\Luc{n}={}xa^n+yb^n\\
		&\left\{
		\begin{array}{l}
			a^0x+b^0y=2\\ a^1y+b^1y=1
		\end{array}\right.
		\intertext{Otteniamo}
		&\left\{
		\begin{array}{l}
			x=\dfrac{1-2b}{a-b}\\ y=\dfrac{2a-1}{a-b}
		\end{array}\right.
		\intertext{per il~\vref{lem:FibpropPhi}}
		&\left\{
		\begin{array}{l}
			x=1\\ y=1
		\end{array}\right.\\ 
		&\Luc{n}={}a^{n}+b^{n}
	\end{align*}
	quindi
	\begin{equation}
		\Luc{n}=\left(\dfrac{1+\sqrt{5}}{2}\right)^n+\left(\dfrac{1-\sqrt{5}}{2}\right)^n
	\end{equation}
\end{proof}
\section{Proprietà}
\begin{thm}[Limite successione]
Se $\Fib{n}$ è la successione di Fibonacci allora 
\begin{equation}
	\lim_{n\to\infty}\dfrac{\Fib{n+1}}{\Fib{n}}=\varphi
\end{equation}\label{eqn:FibLimRap}
\end{thm}
\begin{proof}
\begin{align*}
	\lim_{n\to\infty}\dfrac{\Fib{n+1}}{\Fib{n}}={}&\lim_{n\to\infty}\dfrac{\varphi^{n+1}-(1-\varphi)^{n+1}}{\varphi^n-(1-\varphi)^n}\\
	={}&\lim_{n\to\infty}\dfrac{\varphi^{n+1}\left[1-\left(\dfrac{1-\varphi}{\varphi}\right)^{n+1}\right]}{\varphi^{n}\left[1-\left(\dfrac{1-\varphi}{\varphi}\right)^{n}\right]}\\
	={}&\lim_{n\to\infty}\dfrac{\varphi\left[1-\left(\dfrac{1-\varphi}{\varphi}\right)^{n+1}\right]}{1-\left(\dfrac{1-\varphi}{\varphi}\right)^{n}}\\
	={}&\varphi
	\intertext{dato che}
	\lim_{n\to\infty}\left(\dfrac{1-\varphi}{\varphi}\right)^{n}={}&0
\end{align*}cvd.
\end{proof}
\begin{thm}[Identità di Cassini]\label{thm:fibQuadrato}\index{Fibonacci!identità!Cassini}
	Se $\Fib{n}$ è la successione di Fibonacci allora 
	\begin{equation}
		\Fib{n-1}\cdot\Fib{n+1}=\Fib{n}^2+(-1)^n
	\end{equation}\label{eqn:FibQuadrato}
\end{thm}
\begin{proof}
Riscriviamo il risultato del~\vref{eqn:FinBinet} 
\begin{align*}
	\Fib{n}={}&\dfrac{1}{a-b}\left(a^n-b^n\right)
	\intertext{da cui}
	\Fib{n}^2={}&\dfrac{a^2n}{(a-b)^2}-2\dfrac{a^nb^n}{(a-b)^n}+\dfrac{b^2n}{(a-b)^2}\\
	\intertext{ma}
	\Fib{n-1}={}&\dfrac{1}{a-b}\left(a^{n-1}-b^{n-1}\right)\\
	\Fib{n+1}={}&\dfrac{1}{a-b}\left(a^{n+1}-b^{n+1}\right)\\
	\intertext{quindi}
	\Fib{n-1}\cdot\Fib{n+1}={}&\dfrac{a^2n}{(a-b)^2}-2\dfrac{a^nb^n}{(a-b)^n}+\dfrac{b^2n}{(a-b)^2}-a^{n-1}b^{n-1}\\
	={}&	\Fib{n}^2-a^{n-1}b^{n-1}
		\intertext{ma}
		-a^{n-1}b^{n-1}={}&-\left(\dfrac{1+\sqrt{5}}{2}\right)^{n-1}\cdot\left(\dfrac{1-\sqrt{5}}{2}\right)^{n-1}=(-1)^{n}\\
		\Fib{n-1}\cdot\Fib{n+1}={}&\Fib{n}^2+(-1)^n
\end{align*}
cvd.
\end{proof}
\begin{thm}[Identità di Catalan]
	Se $\Fib{n}$ è la successione di Fibonacci allora 
	\begin{equation}
	\Fib{n}^2-	\Fib{n-r}\cdot\Fib{n+r}=(-1)^{n-r}\Fib{r}^2
	\end{equation}\label{eqn:FibCatalan}
\end{thm}\index{Fibonacci!identità!Catalan}
\begin{proof}
Riscriviamo il risultato del~\vref{eqn:FinBinet} 
\begin{align*}
	\Fib{n}={}&\dfrac{1}{a-b}\left(a^n-b^n\right)
\intertext{da cui}
\Fib{n}^2={}&\dfrac{a^{2n}-2a^nb^n+b^{2n}}{(a-b)^2}\\\Fib{n-r}={}&\dfrac{1}{a-b}\left(a^{n-r}-b^{n-r}\right)\\
\Fib{n+r}={}&\dfrac{1}{a-b}\left(a^{n+r}-b^{n+r}\right)\\
	\Fib{n}^2-	\Fib{n-r}\cdot\Fib{n+r}={}&\dfrac{a^{n+r}b^{n-r}}{(a-b)^2}+\dfrac{a^{n-r}b^{n+r}}{(a-b)^2}-\dfrac{2a^{n}b^{n}}{(a-b)^2}\\
={}&\dfrac{a^{n-r}b^{n-r}(a^r-b^r)^2}{(a-b)^2}
	\intertext{per~\vref{lem:FibpropPhi}}
={}&\dfrac{(-1)^{n-r}(a^r-b^r)^2}{(a-b)^2}\\
={}&(-1)^{n-r}\Fib{r}^2\\
\end{align*}
cvd
\end{proof}
\begin{thm}[Identità di Vajada]
	Se $\Fib{n}$ è la successione di Fibonacci allora 
	\begin{equation}
		\Fib{n+i}\cdot\Fib{n+j}-\Fib{n}\Fib{n+i+j}=(-1)^{n}\Fib{i}\Fib{j}
	\end{equation}\label{eqn:FibVajada}
\end{thm}\index{Fibonacci!identità!Vajada}
\begin{proof}
	Riscriviamo il risultato del~\vref{eqn:FinBinet} 
	\begin{align*}
		\Fib{n}={}&\dfrac{1}{a-b}\left(a^n-b^n\right)
		\intertext{da cui}
		\Fib{n+i}={}&\dfrac{1}{a-b}\left(a^{n+i}-b^{n+i}\right)\\
		\Fib{n+j}={}&\dfrac{1}{a-b}\left(a^{n+j}-b^{n+j}\right)\\
		\Fib{n+i+j}={}&\dfrac{1}{a-b}\left(a^{n+i+j}-b^{n+i+j}\right)\\
\Fib{n+i}\Fib{n+j}-\Fib{n}\Fib{n+i+j}={}&\dfrac{a^{n}b^{n}(a^i-b^i)(a^j-b^j)}{(a-b)^2}\\
		\intertext{per~\vref{lem:FibpropPhi}}
	={}&\dfrac{(-1)^{n}(a^i-b^i)(a^j-b^j)}{(a-b)^2}\\
		={}&(-1)^{n}\Fib{i}\Fib{j}\\
	\end{align*}
cvd
\end{proof}
\begin{thm}[Dispari]\label{thm:Fibdispari}
	Se $\Fib{n}$ è la successione di Fibonacci allora 
	\begin{equation}
		\Fib{n}^2+\Fib{n+1}^2=\Fib{2n+1}
	\end{equation}\label{eqn:FibDispari}
\end{thm}
\begin{proof}
	Riscriviamo il risultato del~\vref{eqn:FinBinet} 
	\begin{align*}
		\Fib{n}={}&\dfrac{1}{a-b}\left(a^n-b^n\right)
		\intertext{da cui}
		\Fib{n}^2={}&\dfrac{a^{2n}-2a^nb^n+b^{2n}}{(a-b)^2}\\
			\Fib{n+1}^2={}&\dfrac{a^{2(n+1)}-2a^{n+1}b^{n+1}+b^{2(n+1)}}{(a-b)^2}\\
		\Fib{n}^2+\Fib{n+1}^2={}&\dfrac{a^{2n}(a^2+1)-2a^nb^n(ab+1)+b^{2n}(b^2+1)}{(a-b)^2}
		\intertext{per~\vref{lem:FibpropPhi}}		
		={}&\dfrac{a^{2n}(a-b)a+b^{2n}(b-a)b}{(a-b)^2}\\
		={}&\dfrac{1}{a-b}\left(a^{2n+1}-b^{2n+1}\right)\\
		={}&\Fib{2n+1}
	\end{align*}
	cvd.
\end{proof}
\begin{thm}[Quattro numeri consecutivi]\label{thm:FibConsecutivi}
	Se $\Fib{n}$ è la successione di Fibonacci allora 
	\begin{equation}
		\Fib{n+2}^2-\Fib{n+1}^2=\Fib{n}\cdot\Fib{n+3}
	\end{equation}\label{eqn:FibConsecutivi}
\end{thm}
\begin{proof}
	Riscriviamo il risultato del~\vref{eqn:FinBinet} 
	\begin{align*}
		\Fib{n}={}&\dfrac{1}{a-b}\left(a^n-b^n\right)
		\intertext{da cui}
		\Fib{n+2}^2-\Fib{n+1}^2={}&\dfrac{a^{2(n+2)}-2(ab)^{n+2}+b^{2(n+2)}}{(a-b)^2}-\dfrac{a^{2(n+1)}-2(ab)^{n+1}+b^{2(n+1)}}{(a-b)^2}\\
		={}&\dfrac{a^{2(n+1)}(a^2-1)+2(ab)^{n+1}(1-ab)+b^{2(n+1)}(b^2-1)}{(a-b)^2}
		\intertext{per~\vref{lem:FibpropPhi}}
			={}&\dfrac{a^{2(n+1)}a+2(ab)^{n+1}2+b^{2(n+1)}b}{(a-b)^2}\\
				={}&\dfrac{a^{2n+3}+4(ab)^{n+1}+b^{2n+3}b}{(a-b)^2}\\
		\Fib{n}\cdot\Fib{n+3}={}&\dfrac{1}{a-b}\left(a^n-b^n\right)\cdot\dfrac{1}{a-b}\left(a^{n+3}-b^{n+3}\right)\\
		={}&\dfrac{(a^n-b^n)(a^{n+3}-b^{n+3})}{(a-b)^2}\\
		={}&\dfrac{(a^n-b^n)(a^{n}a^3-b^{n}b^3)}{(a-b)^2}
		\intertext{per~\vref{lem:FibpropPhi}}
		={}&\dfrac{(a^n-b^n)[a^{n}(a^2+a)-b^{n}(b^2+b)]}{(a-b)^2}\\
		={}&\dfrac{a^{2n+1}(a+1)-(ab)^n(a^2+a+b^2+b)+b^{2n+1}(b+1)}{(a-b)^2}\\
		={}&\dfrac{a^{2n+1}a^2-(ab)^n(a+b+2+a+b)+b^{2n+1}b^2}{(a-b)^2}\\
		={}&\dfrac{a^{2n+3}+4ab(ab)^n+b^{2n+3}}{(a-b)^2}\\
		={}&\dfrac{a^{2n+3}+4(ab)^{n+1}+b^{2n+3}}{(a-b)^2}\\
	\end{align*}
	cvd.
\end{proof}
\begin{thm}[Tre numeri consecutivi]
	Tre numeri di Fibonacci consecutivi non possono essere i lati di un triangolo.
\end{thm}\index{Fibonacci!triangolo}
\begin{proof}
	Supponiamo che $\Fib{n}$, $\Fib{n+1}$ e $\Fib{n+2}$ siano i lati di un triangolo. Allora per il~\vref{thm:Formula_Erone} avremo, indicando con $p$ il semiperimetro e $a$, $b$ e $c$ i lati:
\begin{align*}
	S^2={}&p(p-a)(p-b)(p-c)\\
	\intertext{ma}
	p={}&\dfrac{\Fib{n}+\Fib{n+1}+\Fib{n+2}}{2}\\
	p={}&\dfrac{\Fib{n+2}+\Fib{n+2}}{2}\\
	p={}&\Fib{n+2}\\
	p-c={}&\Fib{n+2}-\Fib{n+2}=0\\
\end{align*}
Quindi il triangolo ha area zero, assurdo. 
\end{proof}
\begin{thm}[Quadrati consecutivi]
		Se $\Fib{n}$ è la successione di Fibonacci allora:
	\begin{equation}
		\Fib{n}^2+\Fib{n+1}^2=\Fib{n}\cdot\Fib{n+2}+\Fib{n}\cdot\Fib{n+2}
	\end{equation}\label{eqn:FibquadratiConsecutivi}
\end{thm}
\begin{proof}
		Riscriviamo il risultato del~\vref{eqn:FinBinet} 
\begin{align*}
	\Fib{n}={}&\dfrac{1}{a-b}\left(a^n-b^n\right)
	\intertext{da cui}
	\Fib{n}^2={}&\dfrac{a^{2n}-2(ab)^n+b^{2n}}{(a-b)^2}\\
	\Fib{n+1}^2={}&\dfrac{a^{2(n+1)}-2(ab)^{(n+1)}+b^{2(n+1)}}{(a-b)^2}\\
	\Fib{n}^2+\Fib{n+1}^2={}&\dfrac{a^{2n}(a^2+1)-2(ab)^n(ab+1)+b^{2n}(b^2+1)}{(a-b)^2}\\
	\intertext{per~\vref{lem:FibpropPhi}}
	={}&\dfrac{a^{2n}(a^2+1)+b^{2n}(b^2+1)}{(a-b)^2}\\
	\Fib{n}\cdot\Fib{n+1}={}&\dfrac{(a^{n}-b^{n})((a^{n+2}-b^{n+2})}{(a-b)^2}\\
	\Fib{n-1}\cdot\Fib{n+1}={}&\dfrac{(a^{n+1}-b^{n+1})((ba^{n}-ab^{n})}{(a-b)^2}\\
	\Fib{n}\cdot\Fib{n+1}+\Fib{n-1}\cdot\Fib{n+1}={}&
	\dfrac{ba^{2n+1}(a^2+1)-(ab)^n(a^2+b^2)(ab+1)+ab^{2n+1}(b^2+1)}{(a-b)^2}\\
	\intertext{per~\vref{lem:FibpropPhi}}
	={}&\dfrac{a^{2n}(a^2+1)+b^{2n}(b^2+1)}{(a-b)^2}\\
\end{align*}
Da cui la tesi.
\end{proof}
\begin{thm}[Somme di numeri di Fibonacci]
	Se $\Fib{n}$ è la successione di Fibonacci allora:
	\begin{equation}
		\sum_{k=1}^{n}\Fib{k}=\Fib{n+2}-1
	\end{equation}\label{eqn:FibSommaNumeri}
\end{thm}\index{Fibonacci!somma}
\begin{proof}
\begin{align*}
\intertext{Per induzione su $n$}
\Fib{1}={}&\Fib{1+2}-1=1\\
\intertext{Supponiamola vera per $n-1$ proviamola per $n$}
\sum_{k=1}^{n-1}\Fib{k}+\Fib{n}={}&\Fib{n+1}+\Fib{n}-1\\
\sum_{k=1}^{n}\Fib{k}={}&\Fib{n+2}-1\\
\end{align*}
cvd.
\end{proof}
\begin{thm}[Fibonacci e Lucas negativo]
Se $\Fib{n}$ è la successione di Fibonacci e  $\Luc{n}$ è quella di Lucas allora:
\begin{equation}
	\Fib{-n}=(-1)^{n+1}\Fib{n}
\end{equation}\label{eqn:FibNegate}
\begin{equation}
	\Luc{-n}=(-1)^{n}\Luc{n}
\end{equation}\label{eqn:LucNegate}
\end{thm}\cite{Rabinowitz_1996}
\begin{proof}
	\proofpart{Fibonacci}
	Riscriviamo il risultato del~\vref{eqn:FinBinet}
	\begin{align*}
		\Fib{n}={}&\dfrac{1}{a-b}\left(a^n-b^n\right)\\
		\Fib{-n}={}&\dfrac{1}{a-b}\left(a^{-n}-b^{-n}\right)\\
	={}&\dfrac{(ab)^{-n}}{b-a}[a^n-b^n]
		\intertext{per il~\vref{lem:FibpropPhi}}
	={}&\dfrac{(-1)^{n+1}}{a-b}[a^n-b^n]\\
	={}&(-1)^{n+1}\Fib{n}
	\end{align*}
	\proofpart{Lucas}
	Riscriviamo il risultato del~\vref{eqn:LucBinet}
\begin{align*}
	\Luc{n}={}&a^{n}+b^{n}\\
	\Luc{-n}={}&a^{-n}-b^{-n}\\
	={}&(ab)^{-n}[a^n-b^n]
	\intertext{per il~\vref{lem:FibpropPhi}}
	={}&(-1)^{n+1}[a^n-b^n]\\
	={}&(-1)^{n}\Luc{n}
\end{align*}
\end{proof}
\section{Addizione}
\begin{thm}[Formule di addizione]\label{thm:LucFibSommaprodotto}
	Se $\Fib{n}$ è la successione di Fibonacci e  $\Luc{n}$ è quella di Lucas allora:
	\begin{equation}
		\Fib{n+m}=\dfrac{\Fib{m}\Luc{n}+\Luc{m}\Fib{n}}{2}
	\end{equation}\label{eqn:FibLucSommaprodotto}
\begin{equation}
	\Luc{n+m}=\dfrac{5\Fib{m}\Fib{n}+\Luc{m}\Luc{n}}{2}
\end{equation}\label{eqn:LucFibSommaprodotto}
\end{thm}\cite{Rabinowitz_1996}
\begin{proof}
		\proofpart{Fibonacci}
	\begin{align*}
	\Fib{m}\Luc{n}+\Luc{m}\Fib{n}={}&\dfrac{1}{a-b}[a^m-b^m][a^n+b^n]+[a^m+b^m]\dfrac{1}{a-b}[a^n-b^n]\\
	={}&\dfrac{2}{a-b}[a^{n+m}-b^{n+m}]\\
	=&2\Fib{m+n}\\
	\end{align*}
	\proofpart{Lucas}
\begin{align*}
	5\Fib{m}\Fib{n}+\Luc{m}\Luc{n}=
	&\dfrac{a^{n+m}[(a-b)^2+5]}{(a-b)^2}+\dfrac{a^{m}b^{n}[(a-b)^2-5]}{(a-b)^2}\\
	+&\dfrac{a^{n}b^{m}[(a-b)^2-5]}{(a-b)^2}+\dfrac{b^{n+m}[(a-b)^2+5]}{(a-b)^2}
	\intertext{per il~\vref{lem:FibpropPhi}}
	={}&\dfrac{a^{n+m}[5+5]}{5}+\dfrac{b^{n+m}[5+5]}{5}\\
	={}&2(a^{n+m}+b^{n+m})\\
	=2\Luc{n+m}\\ 
\end{align*}
cvd.
\end{proof}
Una conseguenza del~\vref{thm:LucFibSommaprodotto} è il seguente
\begin{cor}[Formule moltiplicazione scalare]\label{cor:LucFibmoltscalare}
	Se $\Fib{n}$ è la successione di Fibonacci e  $\Luc{n}$ è quella di Lucas allora:
	\begin{equation}
		\Fib{kn}=\dfrac{\Fib{(k-1)n}\Luc{n}+\Luc{(k-1)n}\Fib{n}}{2}
	\end{equation}\label{eqn:Fibmoltiplicazione scalare}
	\begin{equation}
		\Luc{kn}=\dfrac{5\Fib{(k-1)n}\Fib{n}+\Luc{(k-1)n}\Luc{n}}{2}
	\end{equation}\label{eqn:Lucmoltiplicazionescalare}
\end{cor}\cite{Rabinowitz_1996}
\begin{proof}
	\begin{align*}
		\intertext{Utilizzando il~\vref{thm:LucFibSommaprodotto}}
		\Fib{n+m}={}&\dfrac{\Fib{m}\Luc{n}+\Luc{m}\Fib{n}}{2}\\
		\Luc{n+m}={}&\dfrac{5\Fib{m}\Fib{n}+\Luc{m}\Luc{n}}{2}
		\intertext{Ponendo}
		kn={}&kn-n+n=n(k-1)+n\\
		m={}&n(k-1)\\
		n={}&n\\
		\Fib{kn}={}&\dfrac{\Fib{n(k-1)}\Luc{n}+\Luc{n(k-1)}\Fib{n}}{2}\\
		\Luc{kn}={}&\dfrac{5\Fib{n(k-1)}\Fib{n}+\Luc{n(k-1)}\Luc{n}}{2}
	\end{align*}
\end{proof}
\begin{thm}[Relazione fondamentale fra Lucas e Fibonacci]
	Se $\Fib{n}$ è la successione di Fibonacci e  $\Luc{n}$ è quella di Lucas allora:
	\begin{equation}
		\Luc{n}^2-5\Fib{n}^2=4(-1)^{n}
	\end{equation}\label{eqn:FibLucFondamentale}
\end{thm}\cite{Rabinowitz_1996}
\begin{proof}
\begin{align*}
	\Luc{n}^2={}&a^{2n}+2a^nb^n+b^{2n}\\
	\Fib{n}^2={}&\dfrac{a^{2n}}{(a-b)^2}+\dfrac{2a^{n}b^{n}}{(a-b)^2}+\dfrac{b^{2n}}{(a-b)^2}\\
	\Luc{n}^2-5\Fib{n}^2={}&\dfrac{(a^2-2ab+b^2-5)}{(a-b)^2}\\
	={}&\dfrac{(a^2-2ab+b^2-5)(a^{2n}+b^{2n})+2a^nb^n(a^2+2ab+b^2+5)}{(a-b)^2}\\
	={}&\dfrac{[(a-b)^2-5](a^{2n}+b^{2n})+2a^nb^n[(a-b)^2+5]}{(a-b)^2}\\
	\intertext{per il~\vref{lem:FibpropPhi}}
	={}&\dfrac{2(-1)^n[5+5]}{5}\\
	={}&4(-1)^n\\
\end{align*}
Da cui la tesi.
\end{proof}
\section{Conversione}
\begin{thm}[Converti in Fibonacci o in Lucas]
	Se $\Fib{n}$ è la successione di Fibonacci e  $\Luc{n}$ è quella di Lucas allora:
	\begin{equation}
		\Luc{n}=\Fib{n+1}+\Fib{n-1}
	\end{equation}\label{eqn:LucasConvertiinFib}
\begin{equation}
	\Fib{n}=\dfrac{\Luc{n+1}+\Luc{n-1}}{5}
\end{equation}\label{eqn:FibConvertiinLucas}
\end{thm}\cite{Rabinowitz_1996}
\begin{proof}
		\proofpart{Fibonacci}
	\begin{align*}
	\Fib{n+1}={}&\dfrac{1}{a-b}\left(a^{n+1}-b^{n+1}\right)\\
	\Fib{n-1}={}&\dfrac{1}{a-b}\left(a^{n-1}-b^{n-1}\right)\\
	\Fib{n+1}+\Fib{n-1}={}&\dfrac{ba^n(a^2+1)-ab^n(b^2+1)}{ab(a-b)}
		\intertext{per il~\vref{lem:FibpropPhi}}
		={}&-\dfrac{a^n+b^n}{ab}\\
		={}&a^n+b^n
	\end{align*}
	\proofpart{Lucas}
	\begin{align*}
		\Luc{n+1}={}&a^{n+1}-b^{n+1}\\
		\Luc{n-1}={}&a^{n-1}-b^{n-1}\\
		\dfrac{\Luc{n+1}+\Luc{n-1}}{5}={}&
		\dfrac{a^{n+1}-b^{n+1}+a^{n-1}-b^{n-1}}{(a-b)^2}\\
		={}&\dfrac{ba^n(a^2+1)+ab^n(b^2+1)}{ab(a-b)^2}
		\intertext{per il~\vref{lem:FibpropPhi}}
		={}&\dfrac{a^n(-a+b)+b^n(-b+a)}{ab(a-b)^2}\\
		={}&\dfrac{a^n-b^n}{ab(b-a)}\\
		={}&\dfrac{a^n-b^n}{a-b}\\
	\end{align*}
	Da cui la tesi.
\end{proof}

\begin{thm}[Prodotto in somma]\label{thm:FibProdSomma}
	Se $\Fib{n}$ è la successione di Fibonacci e  $\Luc{n}$ è quella di Lucas allora:
	\begin{equation}
		\Fib{m}\Fib{n}=\dfrac{\Luc{n+m}-(-1)^n\Luc{m-n}}{5}
	\end{equation}\label{eqn:FibProdSomma}
	\begin{equation}
	\Luc{m}\Luc{n}=\Luc{n+m}+(-1)^n\Luc{m-n}
	\end{equation}\label{eqn:LucProdSomma}
\begin{equation}
	\Fib{m}\Luc{n}=\Fib{n+m}+(-1)^n\Fib{m-n}
\end{equation}\label{eqn:LucFibProdSomma}
\end{thm}\cite{Rabinowitz_1996}
\begin{proof}
		\proofpart{Fibonacci}
\begin{align*}
	\Fib{m}\Fib{n}=&\dfrac{a^{m+n}}{(a-b)^2}-\dfrac{a^mb^n}{(a-b)^2}-\dfrac{a^nb^m}{(a-b)^2}+\dfrac{b^{m+n}}{(a-b)^2}\\
	=&\dfrac{a^{m+n}}{(a-b)^2}+\dfrac{b^{m+n}}{(a-b)^2}-\dfrac{a^mb^n}{(a-b)^2}-\dfrac{a^nb^m}{(a-b)^2}\\
	=&\dfrac{\Luc{m+n}}{(a-b)^2}-\dfrac{a^mb^n+a^nb^m}{(a-b)^2}\\
	\Luc{m-n}=&a^{m-n}-b^{m-n}=a^{-n}b^{-n}(a^mb^n+a^nb^m)\\
	\intertext{ma}
	a^mb^n+a^nb^m=&a^{n}b^{n}\Luc{m-n}\\
	\intertext{quindi}
	=&\dfrac{\Luc{m+n}}{(a-b)^2}-\dfrac{a^{n}b^{n}\Luc{m-n}}{(a-b)^2}\\
	\intertext{per il~\vref{lem:FibpropPhi}}
	=&\dfrac{\Luc{m+n}+(-1)^{n}\Luc{m-n}}{5}
\end{align*}
	\proofpart{Lucas}
\begin{align*}
	\Luc{m}\Luc{n}=&a^{m+n}+a^mb^n+a^nb^m+b^{m+n}\\
	=&a^{m+n}+b^{m+n}+a^mb^n+a^nb^m\\
	=&\Luc{m+n}+(a^mb^n+a^nb^m)\\
	\Luc{m-n}=&a^{m-n}-b^{m-n}=a^{-n}b^{-n}(a^mb^n+a^nb^m)\\
	\intertext{ma}
	a^mb^n+a^nb^m=&a^{n}b^{n}\Luc{m-n}\\
	\intertext{quindi}
	=&\Luc{m+n}+a^{n}b^{n}\Luc{m-n}\\
	\intertext{per il~\vref{lem:FibpropPhi}}
	=&\Luc{m+n}+(-1)^{n}\Luc{m-n}
\end{align*}
	\proofpart{Fibonacci Lucas}
\begin{align*}
	\Fib{m}\Luc{n}=&\dfrac{a^{m+n}}{a-b}+\dfrac{a^mb^n}{a-b}+\dfrac{a^nb^m}{b-a}+\dfrac{b^{m+n}}{b-a}\\
	=&\dfrac{a^{m+n}}{a-b}+\dfrac{b^{m+n}}{b-a}+\dfrac{a^mb^n}{a-b}+\dfrac{a^nb^m}{b-a}\\
	=&\Fib{m+n}+\dfrac{a^mb^n-a^nb^m}{a-b}\\
	\Fib{m-n}=&\dfrac{a^{m-n}}{a-b}+\dfrac{b^{m-n}}{b-a}=a^{-n}b^{-n}\left(\dfrac{a^mb^n+a^nb^m}{a-b}\right)\\
	\intertext{ma}
	\dfrac{a^mb^n+a^nb^m}{a-b}=&a^{n}b^{n}\Fib{m-n}\\
	\intertext{quindi}
	=&\Fib{m+n}+a^{n}b^{n}\Fib{m-n}\\
	\intertext{per il~\vref{lem:FibpropPhi}}
	=&\Fib{m+n}+(-1)^{n}\Fib{m-n}
\end{align*}
cvd.
\end{proof}
Conseguenza del~\vref{thm:FibProdSomma} è il seguente
\begin{cor}[Potenze in somma]\label{cor:FibpotSomma}
	Se $\Fib{n}$ è la successione di Fibonacci e  $\Luc{n}$ è quella di Lucas allora:
	\begin{equation}
		\Fib{n}^2=\dfrac{\Luc{2n}-2(-1)^n}{5}
	\end{equation}\label{eqn:FibQuadSomma}
	\begin{equation}
		\Luc{n}^2=\Luc{2n}+2(-1)^n
	\end{equation}\label{eqn:LucQuadSomma}
\end{cor}\cite{Rabinowitz_1996}
\begin{proof}
		\proofpart{Fibonacci}
\begin{align*}
\intertext{Per il~\vref{thm:FibProdSomma}}
\Fib{m}\Fib{n}=&\dfrac{\Luc{n+m}-(-1)^n\Luc{m-n}}{5}
\intertext{posto}
m=&n\\
\Fib{n}\Fib{n}=&\dfrac{\Luc{n+n}-(-1)^n\Luc{n-n}}{5}\\
\Fib{n}^2=&\dfrac{\Luc{2n}-(-1)^n\Luc{0}}{5}\\
\Fib{n}^2=&\dfrac{\Luc{2n}-2(-1)^n}{5}\\
\end{align*}
	\proofpart{Lucas }
\begin{align*}
	\intertext{Per il~\vref{thm:FibProdSomma}}
\Luc{m}\Luc{n}=&\Luc{n+m}+(-1)^n\Luc{m-n}
	\intertext{posto}
	m=&n\\
	\Luc{n}\Luc{n}=&\Luc{n+n}+(-1)^n\Luc{n-n}\\
	\Luc{n}^2=&\Luc{2n}+(-1)^n\Luc{0}\\
	\Luc{n}^2=&\Luc{2n}+2(-1)^n\\
\end{align*}
\end{proof}
Simili sono  i seguenti risultati
\begin{thm}[Potenze in somma]\label{thm:FibpotSommadue}
Se $\Fib{n}$ è la successione di Fibonacci e  $\Luc{n}$ è quella di Lucas allora:
\begin{equation}
	\Fib{n}^3=\dfrac{\Fib{3n}-3(-1)^n\Fib{n}}{5}
\end{equation}\label{eqn:FibCubSomma}
\begin{equation}
	\Luc{n}^3=\Luc{3n}+3(-1)^n\Luc{n}
\end{equation}\label{eqn:LucCubSomma}
\begin{equation}
	\Fib{n}^4=\dfrac{\Luc{4n}-4(-1)^n\Luc{2n}+6}{25}
\end{equation}\label{eqn:FibQuartaSomma}
\end{thm}
\begin{proof}
		\proofpart{Cubo Fibonacci}
\begin{align*}
\Fib{n}^3=&\dfrac{a^{3n}}{(a-b)^3}+3\dfrac{a^{2n}b^n}{(b-a)^3}+3\dfrac{a^nb^{2n}}{(a-b)^3}+\dfrac{b^{3n}}{(b-a)^3}\\
=&+3\dfrac{a^{2n}b^n}{(b-a)^3}+3\dfrac{a^nb^{2n}}{(a-b)^3}\\
=&\dfrac{a^{3n}}{(a-b)^3}+\dfrac{b^{3n}}{(b-a)^3}+3\dfrac{a^nb^n(a^n-b^n)}{(b-a)^3}\\
\intertext{per il~\vref{lem:FibpropPhi}}
=&\dfrac{\Fib{3n}-3(-1)^n\Fib{n}}{5}\\
\end{align*}
	\proofpart{Cubo Lucas}
\begin{align*}
	\Luc{n}^3=&a^{3n} + 3a^{n}b^{2n} + 3a^{2n}b^{n} + b^{3n}\\
	=&a^{3n}  + b^{3n}+ 3a^{n}b^{2n} + 3a^{2n}b^{n}\\
	=&a^{3n}  + b^{3n}+ 3a^nb^n(a^n+b^n)\\
	\intertext{per il~\vref{lem:FibpropPhi}} 
	=&\Luc{3n}+3(-1)^n\Luc{n}
\end{align*}
\proofpart{Fibonacci Quarta}
\begin{align*}
	\Fib{n}^4=&\dfrac{a^{4n}}{(a-b)^4}-4\dfrac{a^{3n}b^{n}}{(a-b)^4}+6\dfrac{a^{2n}b^{2n}}{(a-b)^4}-4\dfrac{a^{n}b^{3n}}{(a-b)^4}+\dfrac{b^{4n}}{(a-b)^4}\\
	=&\dfrac{a^{4n}}{(a-b)^4}+\dfrac{b^{4n}}{(a-b)^4}-4\dfrac{a^{3n}b^{n}}{(a-b)^4}-4\dfrac{a^{n}b^{3n}}{(a-b)^4}+6\dfrac{a^{2n}b^{2n}}{(a-b)^4}\\
	=&\dfrac{a^{4n}}{(a-b)^4}+\dfrac{b^{4n}}{(a-b)^4}-4\dfrac{a^{n}b^{n}(a^{2n}+b^{2n})}{(a-b)^4}+6\dfrac{a^{2n}b^{2n}}{(a-b)^4}\\
	\intertext{per il~\vref{lem:FibpropPhi}} 
	=&\dfrac{\Luc{4n}-4(-1)^n\Luc{2n}+6}{25}
\end{align*}
\proofpart{Lucas Quarta}
\begin{align*}
	\Luc{n}^4=&a^{4n}+4a^{3n}b^{n}+6a^{2n}b^{2n}+4a^{n}b^{3n}+b^{4n}\\
	=&a^{4n}+b^{4n}+4a^{n}b^{n}(a^{2n}+b^{2n})+6a^{2n}b^{2n}\\
\end{align*}
\end{proof}