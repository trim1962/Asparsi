\chapter{Fibonacci}
\section{Definizione}
\begin{defn}[Numeri di Fibonacci]\index{Fibonacci!definizione}
	\begin{align}
		\Fib{1}=&1\notag\\
		\Fib{2}=&1\notag\\
		\Fib{n}=&\Fib{n-1}+\Fib{n-2}\quad n>2\label{eqn:Fibodef}
	\end{align}
\end{defn}
\section{Sezione aurea}
\begin{defn}[Sezione aurea]
	Date due quantità $a$ e $b$ con $a>b>0$ diremo sezione aurea il rapporto
	\begin{equation}
	a+b:a=a:b=\phi	
	\end{equation}\label{eqn:FibAureaDef}\index{Sezione!aurea}
\end{defn}
\begin{prop}
	Dalla~\vref{eqn:FibAureaDef} abbiamo
	\begin{align}
		\dfrac{a}{b}=&\dfrac{a+b}{a}\notag\\
		=&1+\dfrac{b}{a}\notag\\
		=&1+\dfrac{1}{\frac{a}{b}}\notag\\
		\intertext{quindi}
		\phi=&1+\dfrac{1}{\phi}\label{eqn:FibPhiProp}\\
		\phi^2=&\phi+1\notag
	\end{align}
\end{prop}
	Dalla~\vref{eqn:FibPhiProp} segue
	\begin{prop}
	\begin{align}
		\phi^2=&\phi+1\notag\\
		\phi^2-\phi-1=&0\notag\\
		x^2-x-1=&0\label{eqn:FibValPhiEqua}\\
		x_1=&\dfrac{1+\sqrt{5}}{2}\label{eqn:FibValPhi}\\
		x_1=&\dfrac{1-\sqrt{5}}{2}\notag
	\end{align}
\end{prop}
Il risultato~\ref{eqn:FibValPhi} permette di scrivere la seguente definizione
\begin{defn}[Sezione aurea]
La sezione aurea è: \[\phi=\dfrac{1+\sqrt{5}}{2}\]\end{defn}
\section{Formula di Binet}
\begin{lem}[Sezione aurea]
	contenuto...
\end{lem}
\begin{thm}[Formula di Binet]
	Se $\Fib{n}$ è la successione di Fibonacci allora avremo:
	\begin{equation}
		\Fib{n}=\dfrac{1}{\sqrt{5}}\left[\left(\dfrac{1+\sqrt{5}}{2}\right)^n-\left(\dfrac{1-\sqrt{5}}{2}\right)^n\right]
	\end{equation}\label{eqn:FinBinet}
\end{thm}
\begin{proof}
	contenuto...
\end{proof}\index{Sezione!aurea!definizione}