
% !TeX root = Asparsi.tex
% !BIB TS-program = biber
% !TeX encoding = UTF-8
% !TeX spellcheck = it_IT
\chapter{Simmetrie}\label{ch:simmetrie} 
\section{Simmetria assiale}\label{sec:simmetria-assiale}
\begin{defn}[Simmetria assiale]\label{defn:Sassiale1}
Abbiamo una simmetria assiale\index{Simmetria!assiale} di asse $r$ quando, presi due punti $P$ e $Q$:
\begin{enumerate}
	\item Il punto medio $M$ del segmento $PQ\in r$
	\item Il segmento $PQ$ è perpendicolare ad $r$
\end{enumerate} 
\end{defn}
\begin{thm}[Simmetria assiale]\label{thm:Sassiale1}
In una simmetria assiale di asse $r:y=mx+q$ la trasformazione che lega i punti della simmetria ha equazione:
\[\begin{cases}
x_0=\frac{2m(y_1-q)+(1-m^2)x_1}{1+m^2}\\
\\
y_0=\frac{2(mx_1+q)+(m^2-1)y_1}{1+m^2}
\end{cases}\]
\end{thm}
\begin{proof}
	Supponiamo di avere due punti $P(x_1,y_1)$ e $Q(x_0,y_0)$ con $x_1\neq x_0$
	
	 Dalla~\vref{defn:Sassiale1} se $P$ e $Q$ sono simmetrici rispetto alla retta $r$ 
	  \begin{equation}
	 y=mx+q
	 \end{equation}\label[equation]{eq:ass1}
	 allora il punto \[M\left(\frac{x_1+x_0}{2},\frac{y_1+y_0}{2}\right)\] appartiene a $r$ quindi avremo 
	 \begin{equation}
	 \dfrac{y_1+y_0}{2}=m\dfrac{x_1+x_0}{2}+q
	 \end{equation}\label[equation]{eq:ass2}
	 
	 Dalla~\vref{defn:Sassiale1} la retta $r$ è perpendicolare a $PQ$. Quindi se $m$ è il coefficiente angolare della retta di~\vref{eq:ass1} avremo:\begin{equation}
	 \dfrac{y_1-y_0}{x_1-x_0}=-\dfrac{1}{m}
	 \end{equation}\label[equation]{eq:ass3}
	 Utilizzando \vrefrange{eq:ass2}{eq:ass3} otteniamo il sistema
	 \begin{equation}
	 \begin{cases}
	 \dfrac{y_1+y_0}{2}=m\dfrac{x_1+x_0}{2}+q\\[0.4cm]
	 \dfrac{y_1-y_0}{x_1-x_0}=-\dfrac{1}{m}
	 \end{cases}
	 \end{equation}\label[equation]{eq:ass4}
	 che diventa
	  \begin{align*}
	 & \begin{cases}
	 y_1+y_0=mx_1+mx_0+2q\\
	 y_1-y_0=-\dfrac{1}{m}x_1+\dfrac{1}{m}x_0
	 \end{cases}\\
%	  & \begin{cases}
%	 y_0=mx_1+mx_0+2q-y_1\\
%	 y_1-mx_1-mx_0-2q+y_1=-\dfrac{1}{m}x_1+\dfrac{1}{m}x_0
%	 \end{cases}\\
	 & \begin{cases}
	 y_0=mx_1+mx_0+2q-y_1\\
	 \dfrac{1}{m}x_0+mx_0= y_1-mx_1-2q+y_1+\dfrac{1}{m}x_1
	 \end{cases}\\
	 & \begin{cases}
	 y_0=mx_1+mx_0+2q-y_1\\
	 x_0+m^2x_0= my_1-m^2x_1-2mq+my_1+x_1
	 \end{cases}\\
%	 & \begin{cases}
%	 y_0=mx_1+mx_0+2q-y_1\\
%	 x_0=\dfrac{2my_1-m^2x_1-2mq+x_1}{1+m^2}
%	 \end{cases}\\
	 & \begin{cases}
	 y_0=mx_1+\dfrac{2m^2y_1-m^3x_1-2m^2q+mx_1}{1+m^2}+2q-y_1\\
	 x_0=\dfrac{2my_1-m^2x_1-2mq+x_1}{1+m^2}
	 \end{cases}\\
%	  & \begin{cases}
%	 y_0=\dfrac{(1+m^2)mx_1+2m^2y_1-m^3x_1-2m^2q+mx_1+2(1+m^2)q-(1+m^2)y_1}{1+m?"}\\
%	 x_0=\dfrac{2my_1-m^2x_1-2mq+x_1}{1+m^2}
%	 \end{cases}\\
	 & \begin{cases}
	 	y_0=\dfrac{mx_1+m^3x_1+2m^2y_1-m^3x_1-2m^2q+mx_1+2q+2m^2q-y_1-m^2y_1}{1+m?"}\\
	 	x_0=\dfrac{2my_1-m^2x_1-2mq+x_1}{1+m^2}
	 \end{cases}\\
	 \intertext{quindi}
	 &\begin{cases}
	 x_0=\frac{2m(y_1-q)+(1-m^2)x_1}{1+m^2}\\[0.4cm]
	 y_0=\frac{2(mx_1+q)+(m^2-1)y_1}{1+m^2}
	 \end{cases}
 \end{align*}
 \end{proof}
Discutendo il risultato del~\vref{thm:Sassiale1} otteniamo i seguenti corollari:
\begin{cor}[Bisettrice primo quadrante]
	Per $m=1$ il sistema diviene 
	\begin{equation}
	\begin{cases}
	x_0=y_1-q\\
	y_1=x_1+q
	\end{cases}
	\end{equation}
\end{cor}
\begin{cor}[Bisettrice secondo quadrante]
	Per $m=-1$ il sistema diviene 
	\begin{equation}
	\begin{cases}
	x_0=-y_1-q\\
	y_1=-x_1+q
	\end{cases}
	\end{equation}
\end{cor}
\begin{cor}[Asse x]
	Per $m=0$ il sistema diviene 
	\begin{equation}
	\begin{cases}
	x_0=x_1\\
	y_1=2q-y_1
	\end{cases}
	\end{equation}
\end{cor}