\chapter{Fibonacci generalizzato}
\section{Definizione}
\begin{defn}[Numeri di Fibonacci generalizzati]\index{Fibonacci!generalizzati!definizione}
	\begin{align}
		\Gib{1}={}&h\notag\\
		\Gib{2}={}&k\notag\\
		\Gib{n}={}&\Gib{n-1}+\Gib{n-2}\quad n>2\label{eqn:FiboGendef}
	\end{align}
\end{defn}
\section{Formula di Binet generalizzata}
\begin{thm}[Formula di Binet generalizzata]\index{Formula!Binet!generalizzata}
Se $\Gib{1}={}h$ e $\Gib{2}={}k$ allora \begin{equation}
	\Gib{n}={}\dfrac{k-bh}{a+2}a^n+\dfrac{k-ah}{b+2}b^n
\end{equation}\label{eqn:FormulaBinetgeneralizzata}
\end{thm}
\begin{proof}
Poniamo che \begin{equation}
	\Gib{n}=xa^n+yb^n
\end{equation} Se
\begin{equation}
	\left\{
	\begin{array}{l}
		ax+by=h\\ a^2x+b^2y=k
	\end{array}
	\right.
\end{equation}
\end{proof}
\section{Proprietà}
\begin{thm}[Derivazione]
	Se $\Gib{n}$ è una successione di Fibonacci generalizzata  con $\Gib{1}=a$ e $\Gib{2}=b$ allora
	\begin{equation}
		\Gib{n}=b\Fib{n-1}+a\Fib{n-2}\quad n>3
	\end{equation}\label{thm:FibGenDer}
	Dove $\Fib{n}$ è la successione di Fibonacci
\end{thm}