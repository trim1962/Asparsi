\chapter{Fibonacci}
\section{Definizione}
\begin{defn}[Numeri di Fibonacci]\index{Fibonacci!definizione}
	\begin{align}
		\Fib{1}=&1\notag\\
		\Fib{2}=&1\notag\\
		\Fib{n}=&\Fib{n-1}+\Fib{n-2}\quad n>2\label{eqn:Fibodef}
	\end{align}
\end{defn}
\section{Sezione aurea}
\begin{defn}[Sezione aurea]
	Date due quantità $a$ e $b$ con $a>b>0$ diremo sezione aurea il rapporto
	\begin{equation}
	a+b:a=a:b=\phi	
	\end{equation}\label{eqn:FibAureaDef}\index{Sezione!aurea}
\end{defn}
\begin{prop}
	Dalla~\vref{eqn:FibAureaDef} abbiamo
	\begin{align}
		\dfrac{a}{b}=&\dfrac{a+b}{a}\notag\\
		=&1+\dfrac{b}{a}\notag\\
		=&1+\dfrac{1}{\frac{a}{b}}\notag\\
		\intertext{quindi}
		\phi=&1+\dfrac{1}{\phi}\label{eqn:FibPhiProp}\\
		\phi^2=&\phi+1\notag
	\end{align}
\end{prop}
	Dalla~\vref{eqn:FibPhiProp} segue
	\begin{prop}
	\begin{align}
		\phi^2=&\phi+1\notag\\
		\phi^2-\phi-1=&0\notag\\
		x^2-x-1=&0\label{eqn:FibValPhiEqua}\\
		x_1=&\dfrac{1+\sqrt{5}}{2}\label{eqn:FibValPhi}\\
		x_1=&\dfrac{1-\sqrt{5}}{2}\notag
	\end{align}
\end{prop}
Il risultato~\ref{eqn:FibValPhi} permette di scrivere la seguente definizione
\begin{defn}[Sezione aurea]
La sezione aurea è: 
\begin{align*}
	\phi=&\dfrac{1+\sqrt{5}}{2}\\
\intertext{inoltre vale}
	1-	\phi=&\dfrac{1-\sqrt{5}}{2}\\
\end{align*}
\end{defn}\index{Sezione!aurea!valore}
\section{Formula di Binet}
\begin{lem}[Sezione aurea]\label{lem:SezioneAurea}
	Se indichiamo con $a=\phi$ e con $b=1-\phi$
	allora
	\begin{align*}
		a^2=&a+1\\
		b^2=&b+1
	\end{align*}
\end{lem}
\begin{thm}[Formula di Binet]
	Se $\Fib{n}$ è la successione di Fibonacci allora avremo:
	\begin{equation}
		\Fib{n}=\dfrac{1}{\sqrt{5}}\left[\left(\dfrac{1+\sqrt{5}}{2}\right)^n-\left(\dfrac{1-\sqrt{5}}{2}\right)^n\right]
	\end{equation}\label{eqn:FinBinet}
\end{thm}~\cite{Conti2020}
\begin{proof}
	\begin{align*}
		\intertext{Poniamo}
		a^2=&a\Fib{2}+\Fib{1}\\
		a^3=&a^2a\\
		=&a(a+1)\\
		=&a^2+a\\
		=&a+1+a\\
		=&2a+1\\
		=&a\Fib{3}+\Fib{2}\\
		\intertext{Per induzione su $n$}
		a^{n-1}=&\Fib{n-1}a+\Fib{n-2}\\
		a^{n-1}a=&\Fib{n-1}a^2+a\Fib{n-2}\\
		a^{n}=&\Fib{n-1}(a+1)+a\Fib{n-2}\\
		a^{n}=&a\Fib{n-1}+\Fib{n-1}+a\Fib{n-2}\\
		a^{n}=&a(\Fib{n-1}+\Fib{n-2})+\Fib{n-1}\\
		a^{n}=&a\Fib{n}+\Fib{n-1}\\
	\end{align*}
Analogamente si dimostra
\[b^{n}=b\Fib{n}+\Fib{n-1} \]
\end{proof}\index{Sezione!aurea!definizione}