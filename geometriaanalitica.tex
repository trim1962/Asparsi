\chapter{Geometria analitica}
\section{Retta}
\subsection{Rette perpendicolari}
\begin{thm}[Rette perpendicolari]\label{thm:rette_perpendicolari}
	Due rette $y=m_1x+q_1$ e $y=m_2x+q_2$ incidenti sono perpendicolari se solo se \[m_1\cdot m_2=-1\]
\end{thm}
\begin{figure}
	\centering
	\includestandalone{geometria/retteperp1}
	\caption{Rette perpendicolari}
	\label{fig:retteperp1}
\end{figure}
\begin{proof}
	Consideriamo due rette perpendicolari~\cite{Dodero1999b}, trasportiamo il centro del sistema di riferimento in modo che coincida co il punto di inteserzione trala due rette come nella~\cref{fig:retteperp1}. Le rette $y=m_1x+q_1$ e $y=m_2x+q_2$ diventano $y=m_1x$ e $y=m_2x$. Consideriamo due punti $C$ e $D$ di ascissa uno.  Il triangolo $ABC$ è retto in $\hat{A}$ di conseguenza vale il teorema di Pitagora\index{Teorema!Pitagora}
	\begin{align*}
	\overline{BC}^2=&\overline{AB}^2+\overline{AC}^2\\
	\intertext{Dato che}
	\overline{BC}^2=&\abs{m_2-m_1}^2\\
	\overline{AB}^2=&1+m_1^2\\
	\overline{AC}^2=&1+m_2^2
	\intertext{Quindi}
	\abs{m_2-m_1}^2=&1+m_1^2+1+m_2^2\\
	m_1^2+m_2^2-2m_1m_2=&1+m_1^2+1+m_2^2\\
	-2m_1m_2=2\\
	m_1m_2=&-1\\
	\end{align*}
	Dimostriamo il viceversa. Consideriamo la~\cref{fig:retteperp1}, le rette $y=m_1x$ e $y=m_2x$ sono tali che \[m_1m_2=-1\] Poniamo che $C(1,m_1)$, $H(1,0)$ e $B(1,m_2)$ 
	
Quindi 
\begin{align*}
\abs{m_1}\cdot\abs{m_2}=&\abs{m_1\cdot m_2}=1
\intertext{$ABH$ è retto in $\hat{H}$ quindi}
\overline{AB}^2=&\overline{HB}^2+\overline{HA}^2
\intertext{$ACH$ è retto in $\hat{H}$ quindi}
\overline{AC}^2=&\overline{HC}^2+\overline{HA}^2
\intertext{Per cui}
\overline{AB}^2+\overline{AC}^2=&\overline{HC}^2+\overline{HA}^2+\overline{HB}^2+\overline{HA}^2\\
=&\overline{HC}^2+\overline{HB}^2+2\overline{HA}^2\\
\intertext{Ma $\overline{HA}^2=\overline{HC}\cdot\overline{HB}$ quindi}
=&\overline{HC}^2+\overline{HB}^2+2\overline{HC}\cdot\overline{HB}\\
=&(\overline{HC}+\overline{HB})^2\\
=&\overline{CB}^2
\intertext{Segue}
\overline{AB}^2+\overline{AC}^2=&\overline{CB}^2
\end{align*}
come si voleva dimostrare.
\end{proof}
Altra dimostrazione della prima parte del~\cref{thm:rette_perpendicolari}
\begin{thm}[Rette perpendicolari]\label{thm:rette_perpendicolari1}
	Due rette $y=m_1x+q_1$ e $y=m_2x+q_2$ incidenti sono perpendicolari se \[m_1\cdot m_2=-1\]
\end{thm}
\begin{proof}
Procediamo come il~\cref{thm:rette_perpendicolari} e consideriamo la~\cref{fig:retteperp1}. Dato che il triangolo $ABC$ è retto vale il secondo teorema di Euclide\index{Teorema!Euclide}
\begin{align*}
\overline{HC}\cdot\overline{HB}=&\overline{HA}^2
\intertext{Dato che}
\overline{HC}=&m_1\\
\overline{HB}=&-m_2\\
\overline{HA}=&1\\
\intertext{otteniamo}
-m_2\cdot m_1=1\\
m_2\cdot m_1=-1\\ 
\end{align*}
Come si voleva dimostrare
\end{proof}
\section{Parabola}
\begin{defn}[Definizione parabola]
	Definisco parabola l'equazione $y=ax^2+bx+c$ $a\neq0$
\end{defn}\index{Parabola}
\subsection{Proprietà}
\begin{thm}[Complemento al quadrato]\label{thm:Parabola_complemento}
	La parabola $y=ax^2+bx+c$ $a\neq0$ è equivalente a \begin{equation*}
	y+\dfrac{\Delta}{4a}=a\left(x+\dfrac{b}{2a}\right)^2\quad\Delta=b^2-4ac
	\end{equation*}\label{equa:Parabola_scomposizione}
\end{thm}\index{Parabola!complemento!quadrato}
\begin{proof}
	\begin{align*}
	y=&ax^2+bx+c\\
	y-c=&ax^2+bx\\
	y+\dfrac{b^2}{4a}-c=&ax^2+bx+\dfrac{b^2}{4a}\\
	y+\dfrac{b^2-4ac}{4a}=&a\dfrac{4a^2x^2+4abx+b^2}{4a^2}\\
	y+\dfrac{\Delta}{4a}=&a\left(\dfrac{2ax+b}{2a}\right)^2\\
	y+\dfrac{\Delta}{4a}=&a\left(x+\dfrac{b}{2a}\right)^2
	\end{align*}
	Da cui la tesi.
\end{proof}
\subsection{Concavità}
\begin{lem}
	Se $m>0$ $\forall n$ $m-n>-n$ se $m<0$ $\forall n$ $m-n<-n$
\end{lem}
\begin{thm}[Concavità]\label{thm:concavitaparabola}
	Data una parabola~\cite{Zwirner1988} $y=ax^2+bx+c$ $a\neq0$ allora se $a>0$ la parabola ha la concavità rivolata verso l'alto e il vertice è il punto di minima ordinata, se $a<0$ la parabola ha la concavità rivolata verso il basso e il vertice è il punto di massima ordinata
\end{thm}
\begin{proof}
	Dal~\cref{thm:Parabola_complemento} abbiamo
	\begin{align*}
	y=&a\left(x+\dfrac{b}{2a}\right)^2-\dfrac{b^2-4ac}{4a}& x\neq&\dfrac{b}{2a}\\
	\intertext{Il vertice della parabola ha ordinata}
	y_v=&-\dfrac{b^2-4ac}{4a}
	\intertext{Se $a>0$ }
	a\left(x+\dfrac{b}{2a}\right)^2>&0\\
	\intertext{Quindi}
	a\left(x+\dfrac{b}{2a}\right)^2-\dfrac{b^2-4ac}{4a}>&-\dfrac{b^2-4ac}{4a}\\
	\intertext{Analogamente}
	\intertext{Se $a<0$ }
	a\left(x+\dfrac{b}{2a}\right)^2<&0\\
	\intertext{Quindi}
	a\left(x+\dfrac{b}{2a}\right)^2-\dfrac{b^2-4ac}{4a}<&-\dfrac{b^2-4ac}{4a}\\
	\end{align*}
	Da cui la tesi
\end{proof}