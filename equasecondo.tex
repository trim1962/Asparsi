% !TeX root = Asparsi.tex
% !BIB TS-program = biber
% !TeX encoding = UTF-8
% !TeX spellcheck = it_IT
\chapter{Equazioni secondo grado}\label{ch:equazioni-secondo-grado}
\section{Risoluzione}\label{sec:risoluzione}
\begin{thm}[Formula risolutiva]\label{thm:Equasec1}
	Una equazione di secondo grado $ax^2+bx+c=0$ con $a\neq 0$ ha soluzione \begin{equation}\label[equation]{eq:equasec1}
	x_{1,2}=\dfrac{-b\pm\sqrt{b^2-4ac}}{2a}
	\end{equation}\index{Equazione!secondo grado!soluzioni}\index{Triniomio!secondo!grado}
\end{thm}
\begin{proof}
	\begin{align*}
	ax^2+bx+c=&0
	\intertext{$a\neq 0$}
	ax^2+bx+\frac{b^2}{4a}+c=&\frac{b^2}{4a}\\
	ax^2+bx+\frac{b^2}{4a}=&\frac{b^2}{4a}-c\\
	\frac{4a^2x^2+4abx+b^2}{4a}=&\frac{b^2-4ac}{4a}\\
	\left(2ax+b\right)^2=&b^2-4ac
	\intertext{Otteniamo un'equazione di primo grado che risolviamo separando le variabili}
	2ax+b=&\pm\sqrt{b^2-4ac}\\
	2ax=&-b\pm\sqrt{b^2-4ac}\\
	x=&\frac{-b\pm\sqrt{b^2-4ac}}{2a}
	\end{align*}
\end{proof}