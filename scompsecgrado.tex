% !TeX root = Asparsi.tex
% !BIB TS-program = biber
% !TeX encoding = UTF-8
% !TeX spellcheck = it_IT
\chapter{Scomposizioni trinomio}
\section{Formula generale}
\begin{thm}\label{thm:scompsecgrad1}
Dato un trinomio di secondo grado\index{Triniomio!secondo!grado} $ax^2+bx+c=0$ con $a\neq 0$ avremo:\begin{equation}\label[equation]{eq:scompsecgrad1}
ax^2+bx+c=a\left[\left(\dfrac{2ax+b}{2a}\right)^2-\dfrac{\Delta}{4a^2}\right]\qquad\Delta=b^2-4ac
\end{equation}
\end{thm}
\begin{proof}
	\begin{align*}
	ax^2+bx+c=&0\\
	\intertext{raccolgo $a$}
	=&a\left(x^2+\frac{b}{a}x+\frac{c}{a}\right)\\
	=&a\left(x^2+\frac{b}{a}x+\frac{b^2}{4a^2}+\frac{c}{a}-\frac{b^2}{4a^2}\right)\\
	=&a\left[\left(\frac{4a^2x^2+4abx+b^2}{4a^2}\right)+\left(\frac{4ac-b^2}{4a^2}\right)\right]\\
	=&a\left[\left(\frac{2ax+b}{2a}\right)^2-\left(\frac{b^2-4ac}{4a^2}\right)\right]\\
	\intertext{$\Delta=b^2-4ac$}
	=&a\left[\left(\frac{2ax+b}{2a}\right)^2-\frac{\Delta}{4a^2}\right]
	\end{align*}
\end{proof}
\begin{comm}
	Utilizzando il~\cref{thm:scompsecgrad1} è possibile ottenere la risoluzione di un'equazione di secondo grado completa. Avremo tre casi:
	\begin{description}
		\item[$\Delta>0$]In questo caso l'\cref{eq:scompsecgrad1}, ponendola uguale a zero, diventa:  
		\begin{equation*}
		\left(\frac{2ax+b}{2a}-\frac{\sqrt{\Delta}}{2a}\right)\left(\frac{2ax+b}{2a}+\frac{\sqrt{\Delta}}{2a}\right)=0
		\end{equation*}
		\begin{align*}
		\frac{2ax+b}{2a}-\frac{\sqrt{\Delta}}{2a}=&0&\frac{2ax+b}{2a}+\frac{\sqrt{\Delta}}{2a}=&0\\
		2ax+b-\sqrt{\Delta}=&0&2ax+b+\sqrt{\Delta}=&0\\
		2ax=&-b-\sqrt{\Delta}&2ax=&-b+\sqrt{\Delta}\\
		x=&\frac{-b-\sqrt{\Delta}}{2a}&x=&\frac{-b+\sqrt{\Delta}}{2a}\\
		\end{align*}
	\item[$\Delta=0$]In questo caso l'\cref{eq:scompsecgrad1}, ponendola uguale a zero, diventa:
	\begin{equation}
	\left(\dfrac{2ax+b}{2a}\right)^2=0
	\end{equation}Quindi:
	\begin{align*}
	2ax+b=&0\\
	2ax=&-b\\
	x=&-\frac{b}{2a}
	\end{align*}
	\item[$\Delta<0$]In questo caso l'\cref{eq:scompsecgrad1}, ponendola uguale a zero, diventa:
	\begin{equation*}
	\left(\dfrac{2ax+b}{2a}\right)^2-\dfrac{\Delta}{4a^2}=0
	\end{equation*}
	Dato che $\Delta$ è negativo abbiamo la somma di due quantità positive, questa somma può essere scomposta nel campo complesso \[a^2+b^2=(a+ib)(a-ib)\] Quindi
	
		\begin{equation*}
	\left(\frac{2ax+b}{2a}-\frac{i\sqrt{\Delta}}{2a}\right)\left(\frac{2ax+b}{2a}+\frac{i\sqrt{\Delta}}{2a}\right)=0
	\end{equation*}
		\begin{align*}
	\frac{2ax+b}{2a}-\frac{i\sqrt{\Delta}}{2a}=&0&\frac{2ax+b}{2a}+\frac{i\sqrt{\Delta}}{2a}=&0\\
	2ax+b-i\sqrt{\Delta}=&0&2ax+b+i\sqrt{\Delta}=&0\\
	2ax=&-b-i\sqrt{\Delta}&2ax=&-b+i\sqrt{\Delta}\\
	x=&\frac{-b-i\sqrt{\Delta}}{2a}&x=&\frac{-b+i\sqrt{\Delta}}{2a}\\
	\end{align*}

	\end{description}
\end{comm}