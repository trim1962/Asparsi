% !TeX root = Asparsi.tex
% !BIB TS-program = biber
% !TeX encoding = UTF-8
% !TeX spellcheck = it_IT
\chapter{Disuguaglianze triangolari}
\section{Triangolo qualunque}
\begin{thm}[Disuguaglianza triangolare]\label{thm:disuguaglianza_triangolare}
	In un triangolo qualunque  di lati $a$, $b$ e $c$ abbiamo\index{Triangolo!qualunque}
	\begin{align*}
		a<&b+c\\
		b<&a+c\\
		c<&a+b\\
	\end{align*}
\end{thm}
\begin{thm}[Triangolo qualunque]\label{thm:triangolo_qualunque}
In un triangolo qualunque se $\alpha > \beta$ allora $a>\dfrac{1}{2}c$\index{Triangolo!qualunque}
\end{thm}
\begin{figure}
	\centering
	\includestandalone{geometria/triangolo_qualunque}
	\caption{Triangolo qualunque}
	\label{fig:triangoloqualunque}
\end{figure}
\begin{figure}
	\centering
	\includestandalone{geometria/triangolo_qualunque_mediana}
	\caption{Triangolo qualunque mediana}
	\label{fig:triangoloqualunquemediana}
\end{figure}
\begin{proof}
	Consideriamo la figura~\vref{fig:triangoloqualunque} se  $\alpha > \beta$ allora 
	\begin{align*}
		a>&b\\
			a+a>&a+b\\
			2a>&a+b\\
			\intertext{ma}
			a+b>&c\\
			\intertext{quindi}
				2a>&c\\
				a>&\dfrac{1}{2}c
	\end{align*}
cvd.
\end{proof}
\begin{thm}[Triangolo qualunque mediana]\label{thm:triangolo_qualunque_mediana}
	In un triangolo qualunque, se $m_a$, $m_b$ e $m_c$ sono le lunghezze delle mediane  di un triangolo di lati $a$, $b$ e $c$ allora\index{Triangolo!qualunque}\index{Mediana!triangolo}\index{Triangolo!mediana}
	\begin{align*}
		m_a>&\dfrac{b+c-a}{2}\\
		m_b>&\dfrac{a+c-b}{2}\\
		m_c>&\dfrac{a+b-c}{2}\\
	\end{align*}
\end{thm}
\begin{proof}
	Dimostriamo la prima disuguaglianza essendo le altre analoghe. Consideriamo il triangolo $ABC$ della figura~\vref{fig:triangoloqualunquemediana}
	\begin{align*}
		\intertext{Nel triangolo $AMB$}
		m_a+\dfrac{a}{2}>&c\\
		\intertext{Nel triangolo $AMC$}
		m_a+\dfrac{a}{2}>&b\\
		2m_a+a>&b+c\\
		m_a>&\dfrac{b+c-a}{2}
	\end{align*}
\end{proof}
cvd.
\begin{thm}[Prodotti dei lati]\label{thm:triangoloprodottolati}
	In un triangolo qualunque  di lati $a$, $b$ e $c$ abbiamo\index{Triangolo!qualunque}
	\begin{align*}
	(p-a)(p-b)<&ab\\
	(p-a)(p-c)<&ac\\
	(p-b)(p-c)<&bc\\
	\end{align*}
\end{thm}
\begin{proof}
	Dimostriamo la prima visto che le altre sono analoghe::
	\begin{align*}
	c<&a+b\\
a+b+c<&a+b+a+b\\
a+b+c<&2(a+b)\\
p<&a+b\\
p^2<&p(a+b)\\
p^2-p(a+b)<&0\\
p^2-pa-pb<&0\\
p^2-pa-pb+ab<&ab\\
p(p-a)-b(p-a)<&ab\\
(p-a)(p-b)<&ab\\
\end{align*}
\end{proof}
cvd.
\begin{thm}[Prodotti negativi]\label{thm:Prodotti_negativi}
In un triangolo di lati $a$, $b$, $c$ se $ab=a^2+b^2-c^2$ allora $(a-b)(b-c)<0$
\end{thm}
\begin{proof}
	Il segno del prodotto  dipende dal prodotto dei segni
	
	\begin{center}
		\begin{tabular}{ccc}.
	$(a-b)$&$(b-c)$	&  \\
	\midrule
	$+$&$+$&$+$\\
	$-$&$-$	&$+$ \\
	$+$&$-$	&$-$ \\
	$-$&$+$	&$-$ \\
	\bottomrule
	\end{tabular}
	\end{center}
Supponiamo che
\begin{align*}
	\begin{cases}
		a-b>0\\
		b-c>0
	&\end{cases}\\
\begin{cases}
	a>b\\
	b>c
	&\end{cases}\\
\intertext{Se}
a>&b\\
a^2>&ab
\intertext{ma}
ab=&a^2+b^2-c^2\\
a^2>&a^2+b^2-c^2\\
c^2-b^2>&0\\
(c-b)(c+b)>&0
\intertext{Se per ipotesi $b-c>0$ allora $c-b<0$ Quindi il prodotto non può essere positivo.}
\intertext{Se}
b>&c\\
b^2>&c^2\\
b^2-c^2>&0\\
\intertext{ma}
b^2-a^2-b^2+ab>&0\\
ab-a^2>&0\\
a(b-a)>&0
\intertext{Se per ipotesi $a-b>0$ allora $b-a<0$ Quindi il prodotto non può essere positivo.}
\end{align*}
Quindi supporre che i fattori di $(a-b)(b-c)$  sono entrambi positivi porta ad una contraddizione.

Supponiamo che
\begin{align*}
	\begin{cases}
		a-b<0\\
		b-c<0
		&\end{cases}\\
	\begin{cases}
		a<b\\
		b<c
		&\end{cases}\\
	\intertext{Se}
	b<&c\\
	b^2<&c^2
	\intertext{ma}
	b^2=&ab+c^2-a^2\\
	ab+c^2-a^2<&c^2\\
	ab-a^2<&0\\
	a(b-a)<&0
	\intertext{Se per ipotesi $a-b<0$ allora $b-a>0$ Quindi il prodotto non può essere negativo.}
	\intertext{Se}
	a<&b\\
	a^2<&ab
	\intertext{ma}
	a^2<&a^2+b^2-c^2\\
	a^2-a^2-b^2+c^2<&0\\
	c^2-b^2<&0\\
	(c+b)(c-b)<&0\\
	\intertext{Se per ipotesi $b-c<0$ allora $c-b>0$ Quindi il prodotto non può essere negativo.}
\end{align*}
Quindi supporre che i fattori di $(a-b)(b-c)$  sono entrambi negativi porta ad una contraddizione.

Supponiamo che
\begin{align*}
	\begin{cases}
		a-b<0\\
		b-c>0
		&\end{cases}\\
	\begin{cases}
		a<b\\
		b>c
		&\end{cases}\\
	\intertext{Se}
	b>&c\\
	b^2>&c^2
	\intertext{ma}
	c^2=&ab+b^2-a^2\\
	b^2>&a^2+b^2-ab\\
	ab-a^2>&0\\
	a(b-a)> &0
	\intertext{Se per ipotesi $a-b<0$ allora $b-a>0$ Quindi il prodotto è positivo.}
	\intertext{Se}
	a<&b\\
	a^2<&ab
	\intertext{ma}
	a^2<&a^2+b^2-c^2\\
	a^2-a^2-b^2+c^2<&0\\
	c^2-b^2<&0\\
	(c+b)(c-b)<&0\\
	\intertext{Se per ipotesi $b-c>0$ allora $c-b<0$ Quindi il prodotto è negativo.}
\end{align*}
Quindi supporre che i fattori di $(a-b)(b-c)$  sono discordi porta ad un'identità da cui la tesi.

Supponiamo che
\begin{align*}
	\begin{cases}
		a-b>0\\
		b-c<0
		&\end{cases}\\
	\begin{cases}
		a>b\\
		b<c
		&\end{cases}\\
	\intertext{Se}
	b<&c\\
	b^2<&c^2
	\intertext{ma}
	c^2=&ab+c^2-a^2\\
	b^2<&a^2+b^2-ab\\
	ab-a^2<&0\\
	a(b-a)< &0
	\intertext{Se per ipotesi $a-b>0$ allora $b-a<0$ Quindi il prodotto è negativo.}
	\intertext{Se}
	a>&b\\
	a^2>&ab
	\intertext{ma}
	a^2>&a^2+b^2-c^2\\
	a^2-a^2-b^2+c^2>&0\\
	c^2-b^2>&0\\
	(c+b)(c-b)>&0\\
	\intertext{Se per ipotesi $b-c<0$ allora $c-b>0$ Quindi il prodotto è positivo.}
\end{align*}
Quindi supporre che i fattori di $(a-b)(b-c)$  sono discordi porta ad un'identità da cui la tesi.
\end{proof}