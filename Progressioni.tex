
% !TeX root = Asparsi.tex
% !BIB TS-program = biber
% !TeX encoding = UTF-8
% !TeX spellcheck = it_IT
\chapter{Progressioni}
\section{Progressione aritmetica}
\begin{defn}[Progressione aritmetica]\label{defn:ProgAritm1}
	Successione di numeri $a_1,a_2,a_3,\dots,a_n$ in cui la differenza tra un termine e il suo precedente è costante. Tale termine $d=a_r-a_{r-1}$ è chiamato ragione della successione.
\end{defn}\index{Progressione!aritmetica}\index{Progressione!aritmetica!ragione}
\begin{prop}
	Data una successione aritmetica $a_1,a_2,a_3,\dots,a_n$ di ragione $d$ allora\[a_r=a_1+(r-1)d\]
\end{prop}
\begin{thm}[Somma progressione aritmetica]\label{thm:SommaProgAritm1}
Data una progressione aritmetica $a_1,a_2,a_3,\dots,a_n$ allora
\begin{align*}
	S_n=&a_1+a_2+a_3+\dots+a_n\\
	=&\dfrac{a_1+a_n}{2}n\\
\end{align*}
\end{thm}\index{Progressione!aritmetica!somma}
\begin{proof}
\begin{align*}
\intertext{Indichiamo con}
S_n=&a_1+a_2+\dots+a_{n-1}+a_n\\
S_n=&a_n+a_{n-1}+\dots+a_2+a_1\\
S_n+S_n=&(a_1+a_n)+(a_+a_{n-1})+\dots+(a_{n-1}+a_2)+(a_n+a_1)\\
\intertext{I termini $(a_1+a_n)$ per definizione sono uguali quindi}
2S_n=&n(a_1+a_n)\\
S_n=&\dfrac{a_1+a_n}{2}n
\end{align*}
\end{proof}
\section{Progressione geometrica}
\begin{defn}[Progressione geometrica]\label{defn:ProgGeom1}
		Successione di numeri $a_1,a_2,a_3,\dots,a_n$ in cui il rapporto tra un termine e il suo precedente è costante. Tale termine $d=\dfrac{a_r}{a{r-1}}$ è chiamato ragione della successione.
\end{defn}\index{Progressione!geometrica}\index{Progressione!geometrica!ragione}
\begin{prop}
	Data una successione geometrica $a_1,a_2,a_3,\dots,a_n$ di ragione $d$ allora\[a_r=a_1d^{r-1}\]
\end{prop}
\begin{thm}[Somma progressione geometrica]\label{thm:SommaProgGeo}
		Data una successione geometrica $a_1,a_2,a_3,\dots,a_n$ di ragione $d$ allora 
		\begin{align*}
			S_n=&a_1+a_2+a_3+\dots+a_n\\
			=&a_1\dfrac{d^n-1}{d-1}
		\end{align*}
\end{thm}\index{Progressione!geometrica!somma}
\begin{proof}
\begin{align*}
\intertext{Indichiamo con}
S_n=&a_1+a_2+\dots+a_{n-1}+a_n\\
\intertext{moltiplico ogni termine per la ragione $d$}
S_nd=&a_1d+a_2d+\dots+a_{n-1}d+a_nd\\
\intertext{Sottraiamo i termini}
S_nd-S_n=&-a_1+a_nd\\
a_n=&a_1d^{n-1}\\
S_nd-S_n=&a_1d^{n-1}d-a_1\\
S_nd-S_n=&a_1d^{n}-a_1\\
S_n(d-1)=&a_1(d^{n}-1)\\
S_n=&\dfrac{d^{n}-1}{d-1}a_1\\
\end{align*}
\end{proof}