% !TeX root = Asparsi.tex
% !BIB TS-program = biber
% !TeX encoding = UTF-8
% !TeX spellcheck = it_IT
\chapter{Numeri di Fibonacci e Lucas}
\section{Definizione}
\begin{defn}[Numeri di Fibonacci]\index{Fibonacci!definizione}
	\begin{align}
		\Fib{0}={}&0\notag\\
		\Fib{1}={}&1\notag\\
		\Fib{n}={}&\Fib{n-1}+\Fib{n-2}\quad n>2\label{eqn:Fibodef}
	\end{align}
\end{defn}
\begin{defn}[Numeri di Lucas]\index{Lucas!definizione}
	\begin{align}
		\Luc{0}={}&2\notag\\
		\Luc{1}={}&1\notag\\
		\Luc{n}={}&\Luc{n-1}+\Luc{n-2}\quad n>2\label{eqn:Lucadef}
	\end{align}
\end{defn}
\section{Sezione aurea}
\begin{defn}[Sezione aurea]
	Date due quantità $a$ e $b$ con $a>b>0$ diremo sezione aurea il rapporto
	\begin{equation}
	a+b:a=a:b=\varphi	
	\end{equation}\label{eqn:FibAureaDef}\index{Sezione!aurea}
\end{defn}
\begin{prop}
	Dall'~\vref{eqn:FibAureaDef} abbiamo
	\begin{align}
		\dfrac{a}{b}={}&\dfrac{a+b}{a}\notag\\
		={}&1+\dfrac{b}{a}\notag\\
		={}&1+\dfrac{1}{\frac{a}{b}}\notag\\
		\intertext{Quindi}
		\varphi={}&1+\dfrac{1}{\varphi}\label{eqn:FibPhiProp}\\
		\varphi^2={}&\varphi+1\notag
	\end{align}
\end{prop}
	Dall'~\vref{eqn:FibPhiProp} segue
	\begin{prop}
	\begin{align}
		\varphi^2={}&\varphi+1\notag\\
		\varphi^2-\varphi-1={}&0\notag\\
		x^2-x-1={}&0\label{eqn:FibValPhiEqua}\\
		x_1={}&\dfrac{1+\sqrt{5}}{2}\label{eqn:FibValPhi}\\
		x_1={}&\dfrac{1-\sqrt{5}}{2}\notag
	\end{align}
\end{prop}
Il risultato~\ref{eqn:FibValPhi} permette di scrivere la seguente definizione
\begin{defn}[Sezione aurea]
La sezione aurea è: 
\begin{align*}
	\varphi={}&\dfrac{1+\sqrt{5}}{2}\\
\intertext{inoltre vale}
	1-	\varphi={}&\dfrac{1-\sqrt{5}}{2}\\
\end{align*}
\end{defn}\index{Sezione!aurea!valore}
\begin{lem}[Proprietà]\label{lem:FibpropPhi}
	Se $a=\varphi$ e  $b=1-\varphi$ allora
	\begin{align*}
		ab={}&-1\\
		a^2+1={}&(a-b)a\\
		b^2+1={}&(b-a)b\\
		a^2={}&a+1\\
		b^2={}&b+1\\
		a+b={}&1\\
		a^3={}&a^2+a\\
		b^3={}&b^2+b\\
		a^2+b^2={}&a+b+2\\
	a-b={}&\sqrt{5}\\
	a^2b+b={}&-a+b\\
	-b^2a-a={}&b-a\\
		\end{align*}
\end{lem}
\begin{proof}
	\begin{align*}
		ab={}&\dfrac{1+\sqrt{5}}{2}\dfrac{1-\sqrt{5}}{2}\\
		={}&\dfrac{1-5}{4}\\
		={}&-1
	\end{align*}
	\begin{align*}
		(a-b)a={}&a^2-ab=a^2+1\\
		(b-a)b={}&b^2-ab=b^2+1\\
	\end{align*}
	cvd.
\end{proof}
\section{Formula di Binet}
\begin{thm}[Formula di 
Binet]~\cite{Conti2020}\label{thm:FibFormulaBinet}\index{Formula!Binet!Fibonacci}
	Se $\Fib{n}$ è la successione di Fibonacci allora avremo:
	\begin{equation}
		\Fib{n}=\dfrac{1}{\sqrt{5}}\left[\left(\dfrac{1+\sqrt{5}}{2}\right)^n-\left(\dfrac{1-\sqrt{5}}{2}\right)^n\right]
	\end{equation}\label{eqn:FinBinet}
\end{thm}
\begin{proof}
	Poniamo $a=\varphi$ e con $b=1-\varphi$ allora
	\begin{align*}
		\intertext{Poniamo}
		a^1={}&a\Fib{1}+\Fib{0}\\
%		a^3={}&a^2a\\
%		={}&a(a+1)\\
%		={}&a^2+a\\
%		={}&a+1+a\\
%		={}&2a+1\\
%		={}&a\Fib{3}+\Fib{2}\\
		\intertext{Per induzione su $n$}
		a^{n-1}={}&a\Fib{n-1}+\Fib{n-2}\\
		a^{n-1}a={}&a^2\Fib{n-1}+a\Fib{n-2}\\
		a^{n}={}&(a+1)\Fib{n-1}+a\Fib{n-2}\\
		a^{n}={}&a\Fib{n-1}+\Fib{n-1}+a\Fib{n-2}\\
		a^{n}={}&a(\Fib{n-1}+\Fib{n-2})+\Fib{n-1}\\
		a^{n}={}&a\Fib{n}+\Fib{n-1}\\
		\intertext{Analogamente si dimostra}
		b^{n}={}&b\Fib{n}+\Fib{n-1}\\
		\intertext{Sottraendo}
		a^n-b^n={}&a\Fib{n}-b\Fib{n}\\
		a^n-b^n={}&\Fib{n}(a-b)\\
		\Fib{n}={}&\dfrac{a^n-b^n}{a-b}
	\end{align*}
quindi
\begin{equation}
\Fib{n}=\dfrac{1}{\sqrt{5}}\left[\left(\dfrac{1+\sqrt{5}}{2}\right)^n-\left(\dfrac{1-\sqrt{5}}{2}\right)^n\right]
\end{equation}
cvd.
\end{proof}
Altra dimostrazione
\begin{proof}
	\begin{align*}
		\intertext{Poniamo:}
		&\Fib{n}={}xa^n+yb^n\\
		&\left\{
		\begin{array}{l}
			a^0x+b^0y=0\\ a^1y+b^1y=1
		\end{array}\right.
	\intertext{Otteniamo}
	&\left\{
	\begin{array}{l}
	x=\dfrac{1}{a-b}\\ y=-\dfrac{1}{a-b}
	\end{array}\right.\\
%\intertext{Per il~\vref{lem:FibpropPhi}}
%&\left\{
%	\begin{array}{l}
%	x=\dfrac{a}{a-b}\\ y=-\dfrac{b}{a-b}
%\end{array}\right.\\ 
	&\Fib{n}={}\dfrac{a^n-b^n}{a-b}
	\end{align*}
quindi
\begin{equation}
	\Fib{n}=\dfrac{1}{\sqrt{5}}\left[\left(\dfrac{1+\sqrt{5}}{2}\right)^n-\left(\dfrac{1-\sqrt{5}}{2}\right)^n\right]
\end{equation}
cvd.
\end{proof}
\begin{thm}[Formula di Binet per 
Lucas]\label{thm:LucFormulaBinet}\index{Formula!Binet!Lucas}
	Se $\Luc{n}$ è la successione di Fibonacci allora avremo:
	\begin{equation}
		\Luc{n}=\left(\dfrac{1+\sqrt{5}}{2}\right)^{}+\left(\dfrac{1-\sqrt{5}}{2}\right)^{n}
	\end{equation}\label{eqn:LucBinet}
\end{thm}
\begin{proof}
	\begin{align*}
		\intertext{Poniamo:}
		&\Luc{n}={}xa^n+yb^n\\
		&\left\{
		\begin{array}{l}
			a^0x+b^0y=2\\ a^1y+b^1y=1
		\end{array}\right.
		\intertext{Otteniamo}
		&\left\{
		\begin{array}{l}
			x=\dfrac{1-2b}{a-b}\\ y=\dfrac{2a-1}{a-b}
		\end{array}\right.
		\intertext{Per il~\vref{lem:FibpropPhi}}
		&\left\{
		\begin{array}{l}
			x=1\\ y=1
		\end{array}\right.\\ 
		&\Luc{n}={}a^{n}+b^{n}
	\end{align*}
	quindi
	\begin{equation}
		\Luc{n}=\left(\dfrac{1+\sqrt{5}}{2}\right)^n+\left(\dfrac{1-\sqrt{5}}{2}\right)^n
	\end{equation}
cvd.
\end{proof}
\section{Formule di addizione}
\begin{thm}[Formule di addizione]~\cite{Rabinowitz_1996}\label{thm:LucFibSommaprodotto}
	Se $\Fib{n}$ è la successione di Fibonacci e  $\Luc{n}$ è quella di Lucas 
	allora:
	\begin{align}
\Fib{n+m}={}&{}\dfrac{\Fib{m}\Luc{n}+\Luc{m}\Fib{n}}{2}\label{eqn:FibLucSommaprodotto}\\
\Luc{n+m}={}&{}\dfrac{5\Fib{m}\Fib{n}+\Luc{m}\Luc{n}}{2}\label{eqn:LucFibSommaprodotto}
	\end{align}
\end{thm}
\begin{proof}
		\proofpart{Fibonacci}
	\begin{align*}
	\Fib{m}\Luc{n}+\Luc{m}\Fib{n}={}&\dfrac{1}{a-b}[a^m-b^m][a^n+b^n]+[a^m+b^m]\dfrac{1}{a-b}[a^n-b^n]\\
	={}&\dfrac{2}{a-b}[a^{n+m}-b^{n+m}]\\
	={}&2\Fib{m+n}\\
	\end{align*}
	\proofpart{Lucas}
\begin{align*}
	5\Fib{m}\Fib{n}+\Luc{m}\Luc{n}=
	&\dfrac{a^{n+m}[(a-b)^2+5]}{(a-b)^2}+\dfrac{a^{m}b^{n}[(a-b)^2-5]}{(a-b)^2}\\
	+&\dfrac{a^{n}b^{m}[(a-b)^2-5]}{(a-b)^2}+\dfrac{b^{n+m}[(a-b)^2+5]}{(a-b)^2}
	\intertext{Per il~\vref{lem:FibpropPhi}}
	={}&\dfrac{a^{n+m}[5+5]}{5}+\dfrac{b^{n+m}[5+5]}{5}\\
	={}&2(a^{n+m}+b^{n+m})\\
={}&{}{}{}{}2\Luc{n+m}\\ 
\end{align*}
cvd.
\end{proof}
Una conseguenza del~\vref{thm:LucFibSommaprodotto} è il seguente
\begin{cor}[Formule moltiplicazione scalare]~\cite{Rabinowitz_1996}\label{cor:LucFibmoltscalare}
	Se $\Fib{n}$ è la successione di Fibonacci e  $\Luc{n}$ è quella di Lucas allora:
	\begin{equation}
		\Fib{kn}=\dfrac{\Fib{(k-1)n}\Luc{n}+\Luc{(k-1)n}\Fib{n}}{2}
	\end{equation}\label{eqn:Fibmoltiplicazionescalare}
	\begin{equation}
		\Luc{kn}=\dfrac{5\Fib{(k-1)n}\Fib{n}+\Luc{(k-1)n}\Luc{n}}{2}
	\end{equation}\label{eqn:Lucmoltiplicazionescalare}
\end{cor}
\begin{proof}
	\begin{align*}
		\intertext{Utilizzando il~\vref{thm:LucFibSommaprodotto}}
		\Fib{n+m}={}&\dfrac{\Fib{m}\Luc{n}+\Luc{m}\Fib{n}}{2}\\
		\Luc{n+m}={}&\dfrac{5\Fib{m}\Fib{n}+\Luc{m}\Luc{n}}{2}
		\intertext{Ponendo}
		kn={}&kn-n+n=n(k-1)+n\\
		m={}&n(k-1)\\
		n={}&n\\
		\Fib{kn}={}&\dfrac{\Fib{n(k-1)}\Luc{n}+\Luc{n(k-1)}\Fib{n}}{2}\\
		\Luc{kn}={}&\dfrac{5\Fib{n(k-1)}\Fib{n}+\Luc{n(k-1)}\Luc{n}}{2}
	\end{align*}
cvd.
\end{proof}
\begin{thm}[Relazione fondamentale fra Lucas e 
Fibonacci]\label{thm:FibLucFondamentale}~\cite{Rabinowitz_1996}
	Se $\Fib{n}$ è la successione di Fibonacci e  $\Luc{n}$ è quella di Lucas allora:
	\begin{equation}
		\Luc{n}^2-5\Fib{n}^2=4(-1)^{n}
	\end{equation}\label{eqn:FibLucFondamentale}
\end{thm}
\begin{proof}
\begin{align*}
	\Luc{n}^2={}&a^{2n}+2a^nb^n+b^{2n}\\
	\Fib{n}^2={}&\dfrac{a^{2n}}{(a-b)^2}+\dfrac{2a^{n}b^{n}}{(a-b)^2}+\dfrac{b^{2n}}{(a-b)^2}\\
	\Luc{n}^2-5\Fib{n}^2={}&\dfrac{(a^2-2ab+b^2-5)}{(a-b)^2}\\
	={}&\dfrac{(a^2-2ab+b^2-5)(a^{2n}+b^{2n})+2a^nb^n(a^2+2ab+b^2+5)}{(a-b)^2}\\
	={}&\dfrac{[(a-b)^2-5](a^{2n}+b^{2n})+2a^nb^n[(a-b)^2+5]}{(a-b)^2}\\
	\intertext{Per il~\vref{lem:FibpropPhi}}
	={}&\dfrac{2(-1)^n[5+5]}{5}\\
	={}&4(-1)^n\\
\end{align*}
Da cui la tesi.
\end{proof}
\section{Formule di conversione}
\begin{thm}[Fibonacci e Lucas 
	negativo]~\cite{Rabinowitz_1996}\label{thm:FibLucNeg}
	Se $\Fib{n}$ è la successione di Fibonacci e  $\Luc{n}$ è quella di Lucas 
	allora:
	\begin{align}
		\Fib{-n}={}&(-1)^{n+1}\Fib{n}\label{eqn:FibNegate}\\
		\Luc{-n}={}&(-1)^{n}\Luc{n}\label{eqn:LucNegate}
	\end{align}
\end{thm}
\begin{proof}
	\proofpart{Fibonacci}
	Riscriviamo il risultato dell'~\vref{eqn:FinBinet}
	\begin{align*}
		\Fib{n}={}&\dfrac{1}{a-b}\left(a^n-b^n\right)\\
		\Fib{-n}={}&\dfrac{1}{a-b}\left(a^{-n}-b^{-n}\right)\\
		={}&\dfrac{(ab)^{-n}}{b-a}[a^n-b^n]
		\intertext{Per il~\vref{lem:FibpropPhi}}
		={}&\dfrac{(-1)^{n+1}}{a-b}[a^n-b^n]\\
		={}&(-1)^{n+1}\Fib{n}
	\end{align*}
	\proofpart{Lucas}
	Riscriviamo il risultato dell'~\vref{eqn:LucBinet}
	\begin{align*}
		\Luc{n}={}&a^{n}+b^{n}\\
		\Luc{-n}={}&a^{-n}-b^{-n}\\
		={}&(ab)^{-n}[a^n-b^n]
		\intertext{Per il~\vref{lem:FibpropPhi}}
		={}&(-1)^{n+1}[a^n-b^n]\\
		={}&(-1)^{n}\Luc{n}
	\end{align*}
cvd.
\end{proof}  
\begin{thm}[Converti in Fibonacci o in 
Lucas]\label{thm:LucasToFibFibToLuc}~\cite{Rabinowitz_1996}
	Se $\Fib{n}$ è la successione di Fibonacci e  $\Luc{n}$ è quella di Lucas allora:
	\begin{align}
		\Luc{n}={}&\Fib{n+1}+\Fib{n-1}\label{eqn:LucasConvertiinFib}\\
			\Fib{n}={}&\dfrac{\Luc{n+1}+\Luc{n-1}}{5}\label{eqn:FibConvertiinLucas}
	\end{align}
\end{thm}
\begin{proof}
		\proofpart{Fibonacci}
		\begin{align*}
			\intertext{Per il~\vref{thm:LucFibSommaprodotto}}
			\Fib{n+m}={}&\dfrac{\Fib{m}\Luc{n}+\Luc{m}\Fib{n}}{2}\\
			\Fib{n+1}={}&\dfrac{\Fib{1}\Luc{n}+\Luc{1}\Fib{n}}{2}\\
			\Fib{n-1}={}&\dfrac{\Fib{-1}\Luc{n}+\Luc{-1}\Fib{n}}{2}\\
				\Fib{n+1}+\Fib{n-1}={}&\dfrac{\Fib{1}\Luc{n}+\Luc{1}\Fib{n}+\Fib{-1}\Luc{n}+\Luc{-1}\Fib{n}}{2}\\
				\intertext{Per il~\vref{thm:FibLucNeg}}
		\Fib{-1}={}&(-1)^{-1+1}\Fib{1}=\Fib{1}\\	
		\Luc{-1}={}&(-1)^{1}\Luc{1}=-	\Luc{1}
		\intertext{Quindi}
		={}&\dfrac{\Fib{1}\Luc{n}+\Luc{1}\Fib{n}+\Fib{1}\Luc{n}-\Luc{1}\Fib{n}}{2}\\
		={}&\dfrac{\Fib{1}\Luc{n}+\Fib{1}\Luc{n}}{2}\\
		={}&\Luc{n}
		\end{align*}
	\proofpart{Lucas}
	\begin{align*}
		\intertext{Per il~\vref{thm:LucFibSommaprodotto}}
		\Luc{n+m}={}&\dfrac{5\Fib{m}\Fib{n}+\Luc{m}\Luc{n}}{2}\\
		\Luc{n+1}={}&\dfrac{5\Fib{1}\Fib{n}+\Luc{1}\Luc{n}}{2}\\
		\Luc{n-1}={}&\dfrac{5\Fib{-1}\Fib{n}+\Luc{-1}\Luc{n}}{2}\\
		\Luc{n+1}+\Luc{n-1}={}&\dfrac{5\Fib{1}\Fib{n}+\Luc{1}\Luc{n}+5\Fib{-1}\Fib{n}+\Luc{-1}\Luc{n}}{2}\\
			\intertext{Per il~\vref{thm:FibLucNeg}}
		\Fib{-1}={}&(-1)^{-1+1}\Fib{1}=\Fib{1}\\	
		\Luc{-1}={}&(-1)^{1}\Luc{1}=-	\Luc{1}
		\intertext{Quindi}
		={}&\dfrac{5\Fib{1}\Fib{n}+\Luc{1}\Luc{n}+5\Fib{1}\Fib{n}-\Luc{1}\Luc{n}}{2}\\
		={}&\dfrac{5\Fib{1}\Fib{n}+5\Fib{1}\Fib{n}}{2}\\
		={}&\dfrac{10\Fib{1}\Fib{n}}{2}\\
		={}&5\Fib{n}
	\end{align*}
	Da cui la tesi.
\end{proof}
\begin{thm}[Prodotto in somma]~\cite{Rabinowitz_1996}\label{thm:FibProdSomma}
	Se $\Fib{n}$ è la successione di Fibonacci e  $\Luc{n}$ è quella di Lucas allora:
	\begin{align}
	\Fib{m}\Fib{n}={}&\dfrac{\Luc{n+m}-(-1)^n\Luc{m-n}}{5}\label{eqn:FibProdSomma}\\
	\Luc{m}\Luc{n}={}&\Luc{n+m}+(-1)^n\Luc{m-n}\label{eqn:LucProdSomma}\\
	\Fib{m}\Luc{n}={}&\Fib{n+m}+(-1)^n\Fib{m-n}\label{eqn:LucFibProdSomma}
	\end{align}
\end{thm}
\begin{proof}
		\proofpart{Fibonacci}
\begin{align*}
	\Fib{m}\Fib{n}={}&\dfrac{a^{m+n}}{(a-b)^2}-\dfrac{a^mb^n}{(a-b)^2}-\dfrac{a^nb^m}{(a-b)^2}+\dfrac{b^{m+n}}{(a-b)^2}\\
	={}&\dfrac{a^{m+n}}{(a-b)^2}+\dfrac{b^{m+n}}{(a-b)^2}-\dfrac{a^mb^n}{(a-b)^2}-\dfrac{a^nb^m}{(a-b)^2}\\
	={}&\dfrac{\Luc{m+n}}{(a-b)^2}-\dfrac{a^mb^n+a^nb^m}{(a-b)^2}\\
	\Luc{m-n}={}&a^{m-n}-b^{m-n}=a^{-n}b^{-n}(a^mb^n+a^nb^m)\\
	\intertext{Ma}
	a^mb^n+a^nb^m={}&a^{n}b^{n}\Luc{m-n}\\
	\intertext{Quindi}
	={}&\dfrac{\Luc{m+n}}{(a-b)^2}-\dfrac{a^{n}b^{n}\Luc{m-n}}{(a-b)^2}\\
	\intertext{Per il~\vref{lem:FibpropPhi}}
	={}&\dfrac{\Luc{m+n}+(-1)^{n}\Luc{m-n}}{5}
\end{align*}
	\proofpart{Lucas}
\begin{align*}
	\Luc{m}\Luc{n}={}&a^{m+n}+a^mb^n+a^nb^m+b^{m+n}\\
	={}&a^{m+n}+b^{m+n}+a^mb^n+a^nb^m\\
	={}&\Luc{m+n}+(a^mb^n+a^nb^m)\\
	\Luc{m-n}={}&a^{m-n}-b^{m-n}=a^{-n}b^{-n}(a^mb^n+a^nb^m)\\
	\intertext{Ma}
	a^mb^n+a^nb^m={}&a^{n}b^{n}\Luc{m-n}\\
	\intertext{Quindi}
	={}&\Luc{m+n}+a^{n}b^{n}\Luc{m-n}\\
	\intertext{Per il~\vref{lem:FibpropPhi}}
	={}&\Luc{m+n}+(-1)^{n}\Luc{m-n}
\end{align*}
	\proofpart{Fibonacci Lucas}
\begin{align*}
	\Fib{m}\Luc{n}={}&\dfrac{a^{m+n}}{a-b}+\dfrac{a^mb^n}{a-b}+\dfrac{a^nb^m}{b-a}+\dfrac{b^{m+n}}{b-a}\\
	={}&\dfrac{a^{m+n}}{a-b}+\dfrac{b^{m+n}}{b-a}+\dfrac{a^mb^n}{a-b}+\dfrac{a^nb^m}{b-a}\\
	={}&\Fib{m+n}+\dfrac{a^mb^n-a^nb^m}{a-b}\\
	\Fib{m-n}={}&\dfrac{a^{m-n}}{a-b}+\dfrac{b^{m-n}}{b-a}=a^{-n}b^{-n}\left(\dfrac{a^mb^n+a^nb^m}{a-b}\right)\\
	\intertext{Ma}
	\dfrac{a^mb^n+a^nb^m}{a-b}={}&a^{n}b^{n}\Fib{m-n}\\
	\intertext{Quindi}
	={}&\Fib{m+n}+a^{n}b^{n}\Fib{m-n}\\
	\intertext{Per il~\vref{lem:FibpropPhi}}
	={}&\Fib{m+n}+(-1)^{n}\Fib{m-n}
\end{align*}
cvd.
\end{proof}
Conseguenza del~\vref{thm:FibProdSomma} è il seguente
\begin{cor}[Potenze in somma]~\cite{Rabinowitz_1996}\label{cor:FibpotSomma}
	Se $\Fib{n}$ è la successione di Fibonacci e  $\Luc{n}$ è quella di Lucas allora:
	\begin{align}
		\Fib{n}^2={}&\dfrac{\Luc{2n}-2(-1)^n}{5}\label{eqn:FibQuadSomma}\\
		\Luc{n}^2={}&\Luc{2n}+2(-1)^n\label{eqn:LucQuadSomma}
	\end{align}
\end{cor}
\begin{proof}
		\proofpart{Fibonacci}
\begin{align*}
\intertext{Per il~\vref{thm:FibProdSomma}}
\Fib{m}\Fib{n}={}&\dfrac{\Luc{n+m}-(-1)^n\Luc{m-n}}{5}
\intertext{posto}
m={}&n\\
\Fib{n}\Fib{n}={}&\dfrac{\Luc{n+n}-(-1)^n\Luc{n-n}}{5}\\
\Fib{n}^2={}&\dfrac{\Luc{2n}-(-1)^n\Luc{0}}{5}\\
\Fib{n}^2={}&\dfrac{\Luc{2n}-2(-1)^n}{5}\\
\end{align*}
	\proofpart{Lucas }
\begin{align*}
	\intertext{Per il~\vref{thm:FibProdSomma}}
\Luc{m}\Luc{n}={}&\Luc{n+m}+(-1)^n\Luc{m-n}
	\intertext{posto}
	m={}&n\\
	\Luc{n}\Luc{n}={}&\Luc{n+n}+(-1)^n\Luc{n-n}\\
	\Luc{n}^2={}&\Luc{2n}+(-1)^n\Luc{0}\\
	\Luc{n}^2={}&\Luc{2n}+2(-1)^n\\
\end{align*}
cvd.
\end{proof}
Simili al precedente sono  i seguenti risultati
\begin{thm}[Potenze in somma]~\cite{Rabinowitz_1996}\label{thm:FibpotSommadue}
Se $\Fib{n}$ è la successione di Fibonacci e  $\Luc{n}$ è quella di Lucas allora:
\begin{align}
\Fib{n}^3={}&\dfrac{\Fib{3n}-3(-1)^n\Fib{n}}{5}\label{eqn:FibCubSomma}\\
\Luc{n}^3={}&\Luc{3n}+3(-1)^n\Luc{n}\label{eqn:LucCubSomma}\\
	\Fib{n}^4={}&\dfrac{\Luc{4n}-4(-1)^n\Luc{2n}+6}{25}\label{eqn:FibQuartaSomma}\\
	\Luc{n}^4={}&\Luc{4n}+4(-1)^n\Luc{2n}+6\label{eqn:LucQuartaSomma}\\
	\Fib{n}^5={}&\dfrac{\Fib{5n}-5(-1)^n\Fib{3n}+10\Fib{n}}{25}\label{eqn:FibQuintaSomma}\\
	\Luc{n}^5={}&\Luc{5n}+5(-1)^n\Luc{3n}+10\Luc{n}\label{eqn:LucQuintaSomma}\\
	\Fib{n}^6={}&\dfrac{\Luc{6n}-6(-1)^n\Luc{4n}+15\Luc{2n}-20(-1)^n}{125}\label{eqn:FibSestsaSomma}\\
	\Luc{n}^6={}&\Luc{6n}+6(-1)^n\Luc{4n}+15\Luc{2n}+20(-1)^n\label{eqn:LucSestsaSomma}
\end{align} 
\end{thm}
\begin{proof}
		\proofpart{Cubo Fibonacci}
\begin{align*}
	\Fib{n}^3={}&\Fib{n}^2\Fib{n}\\
	\intertext{Per il~\vref{cor:FibpotSomma}}
	={}&\dfrac{\Luc{2n}-2(-1)^n}{5}\Fib{n}\\
	={}&\dfrac{\Luc{2n}\Fib{n}-2(-1)^n\Fib{n}}{5}\\
	\intertext{Per il~\vref{thm:FibProdSomma}}
	\Fib{m}\Luc{n}={}&\Fib{n+m}+(-1)^n\Fib{m-n}
	\intertext{Quindi}
		\Fib{n}\Luc{2n}={}&\Fib{n+2n}+(-1)^{2n}\Fib{n-2n}\\
		={}&\Fib{3n}+\Fib{-n}\\
	\intertext{Per il~\vref{thm:FibLucNeg}}
	={}&\Fib{3n}+(-1)^{n+1}\Fib{-n}\\
	={}&\Fib{3n}-(-1)^{n}\Fib{n}\\
\intertext{Quindi}
={}&\dfrac{\Fib{3n}-(-1)^{n}\Fib{n} -2(-1)^n\Fib{n}}{5}\\
={}&\dfrac{\Fib{3n}-3(-1)^n\Fib{n}}{5}\\
\end{align*}
	\proofpart{Cubo Lucas}
\begin{align*}
	\Luc{n}^3={}&\Luc{n}^2\Luc{n}\\
	\intertext{Per il~\vref{cor:FibpotSomma}}
	={}&(\Luc{2n}+2(-1)^n)\Luc{n}\\
	={}&\Luc{2n}\Luc{n}+2(-1)^n\Luc{n}\\
	\intertext{Per il~\vref{thm:FibProdSomma}}
	\Luc{2n}\Luc{n}={}&\Luc{2n+n}+(-1)^n\Luc{2n-n}\\
	={}&\Luc{3n}+(-1)^n\Luc{n}\\	
	\intertext{Quindi}	
	={}&\Luc{3n}+(-1)^n\Luc{n}+2(-1)^n\Luc{n}\\
={}&\Luc{3n}+3(-1)^n\Luc{n}\\
\end{align*}
\proofpart{Fibonacci Quarta}
\begin{align*}
\Fib{n}^4={}&\Fib{n}^3\Fib{n}\\
\intertext{Per l'~\vref{eqn:FibCubSomma}}
={}&\dfrac{\Fib{3n}-3(-1)^n\Fib{n}}{5}\Fib{n}\\
={}&\dfrac{\Fib{3n}\Fib{n}-3(-1)^n\Fib{n}^2}{5}\\
\intertext{Per il~\vref{thm:FibProdSomma}}
\Fib{3n}\Fib{n}={}&\dfrac{\Luc{3n+n}-(-1)^n\Luc{3n-n}}{5}\\
\Fib{3n}\Fib{n}={}&\dfrac{\Luc{4n}-(-1)^n\Luc{2n}}{5}\\
\Fib{n}^2={}&\dfrac{\Luc{2n}-2(-1)^n}{5}\\
\intertext{Quindi}
={}&\dfrac{\Luc{4n}-(-1)^n\Luc{2n}-3(-1)^n[\Luc{2n}-2(-1)^n]}{25}\\
={}&\dfrac{\Luc{4n}-(-1)^n\Luc{2n}-3(-1)^n\Luc{2n}+6(-1)^{2n}}{25}\\
={}&\dfrac{\Luc{4n}-4(-1)^n\Luc{2n}+6}{25}\\
\end{align*}
\proofpart{Lucas Quarta}
\begin{align*}
\Luc{n}^4={}&\Luc{n}^3\Luc{n}\\
\intertext{Per l'~\vref{eqn:LucCubSomma}}
={}&[\Luc{3n}+3(-1)^n\Luc{n}]\Luc{n}\\
={}&\Luc{3n}\Luc{n}+3(-1)^n\Luc{n}^2\\
\intertext{Per il~\vref{thm:FibProdSomma}}
\Luc{3n}\Luc{n}={}&\Luc{3n+n}+(-1)^n\Luc{3n-n}\\
={}&\Luc{4n}+(-1)^n\Luc{2n}\\
\Luc{n}^2={}&\Luc{2n}+2(-1)^n\\
\intertext{Quindi}
={}&\Luc{4n}+(-1)^n\Luc{2n}+3(-1)^n[\Luc{2n}+2(-1)^n]\\
={}&\Luc{4n}+(-1)^n\Luc{2n}+3(-1)^n\Luc{2n}+6(-1)^{2n}\\
={}&\Luc{4n}+4(-1)^n\Luc{2n}+6\\
\end{align*}
\proofpart{Fibonacci Quinta}
\begin{align*}
	\Fib{n}^5={}&\Fib{n}^4\Fib{n}\\
	\intertext{Per l'~\vref{eqn:FibQuartaSomma}}
	={}&\dfrac{\Luc{4n}-4(-1)^n\Luc{2n}+6}{25}\Fib{n}\\
	={}&\dfrac{\Luc{4n}\Fib{n}-4(-1)^n\Luc{2n}\Fib{n}+6\Fib{n}}{25}\\
	\intertext{Per il~\vref{thm:FibProdSomma}}
	\Fib{m}\Luc{n}={}&\Fib{n+m}+(-1)^n\Fib{m-n}\\
	\Fib{n}\Luc{4n}={}&\Fib{4n+n}+(-1)^{4n}\Fib{n-4n}\\
	={}&\Fib{5n}+\Fib{-3n}\\
	={}&\Fib{5n}+(-1)^{3n+1}\Fib{3n}\\
	\Fib{n}\Luc{2n}={}&\Fib{2n+n}+(-1)^{2n}\Fib{n-2n}\\
	={}&\Fib{3n}+(-1)^{2n}\Fib{n-2n}\\
	={}&\Fib{3n}+\Fib{-n}\\
	={}&\Fib{3n}+(-1)^{n+1}\Fib{n}\\
	\intertext{Quindi}
	={}&\dfrac{\Fib{5n}+(-1)^{3n+1}\Fib{3n}-4(-1)^n[\Fib{3n}+(-1)^{n+1}\Fib{n}]+6\Fib{n}}{25}\\
	={}&\dfrac{\Fib{5n}+(-1)^{3n+1}\Fib{3n}-4(-1)^n\Fib{3n}-4(-1)^{2n+1}\Fib{n}+6\Fib{n}}{25}\\
	={}&\dfrac{\Fib{5n}-(-1)^{n}\Fib{3n}-4(-1)^n\Fib{3n}-4(-1)^{1}\Fib{n}+6\Fib{n}}{25}\\
	={}&\dfrac{\Fib{5n}-5(-1)^n\Fib{3n}+10\Fib{n}}{25}\\
\end{align*}
\proofpart{Lucas Quinta}
\begin{align*}
	\Luc{n}^5={}&\Luc{n}^4\Luc{n}\\
	\intertext{Per l'~\vref{eqn:LucQuartaSomma}}
	={}&(\Luc{4n}+4(-1)^n\Luc{2n}+6)\Luc{n}\\
	={}&\Luc{4n}\Luc{n}+4(-1)^n\Luc{2n}\Luc{n}+6\Luc{n}\\
	\intertext{Per il~\vref{thm:FibProdSomma}}
	\Luc{m}\Luc{n}={}&\Luc{m+n}+(-1)^{n}\Luc{m-n}\\
	\Luc{4n}\Luc{n}={}&\Luc{4n+n}+(-1)^{n}\Luc{4n-n}\\
	={}&\Luc{5n}+(-1)^{n}\Luc{3n}\\
	\Luc{2n}\Luc{n}={}&\Luc{2n+n}+(-1)^{n}\Luc{2n-n}\\
	={}&\Luc{3n}+(-1)^{n}\Luc{n}\\
	\intertext{Quindi}
	={}&\Luc{5n}+(-1)^{n}\Luc{3n}+4(-1)^n[\Luc{3n}+(-1)^{n}\Luc{n}]+6\Luc{n}\\
	={}&\Luc{5n}+(-1)^{n}\Luc{3n}+4(-1)^n\Luc{3n}+4(-1)^{2n}\Luc{n}+6\Luc{n}\\
	={}&\Luc{5n}+5(-1)^n\Luc{3n}+4\Luc{n}+6\Luc{n}\\
	={}&\Luc{5n}+5(-1)^n\Luc{3n}+10\Luc{n}\\
\end{align*}
\proofpart{Fibonacci Sesta}
\begin{align*}
	\Fib{n}^6={}&\Fib{n}^4\Fib{n}\\
	\intertext{Per l'~\vref{eqn:FibQuintaSomma}}
	={}&\dfrac{\Fib{5n}\Fib{n}-5(-1)^n\Fib{3n}\Fib{n}+10\Fib{n}^2}{25}\\
	\Fib{m}\Fib{n}={}&\dfrac{\Luc{n+m}-(-1)^n\Luc{m-n}}{5}\\
	\Fib{5m}\Fib{n}={}&\dfrac{\Luc{5n+n}-(-1)^n\Luc{5n-n}}{5}\\
	={}&\dfrac{\Luc{6n}-(-1)^n\Luc{4n}}{5}\\
	\Fib{3n}\Fib{n}={}&\dfrac{\Luc{3n+n}-(-1)^n\Luc{3n-n}}{5}\\
	={}&\dfrac{\Luc{4n}-(-1)^n\Luc{2n}}{5}\\
	\Luc{n}^2={}&\Luc{2n}+2(-1)^n\\
	\intertext{Quindi}
	={}&\dfrac{\Luc{6n}-(-1)^n\Luc{4n}-5(-1)^n\Luc{4n}+5(-1)^{2n}\Luc{2n}+10\Luc{2n}+20(-1)^n}{25}\\
	={}&\dfrac{\Luc{6n}-6(-1)^n\Luc{4n}+15(-1)^{2n}\Luc{2n}+20(-1)^n}{25}\\
\end{align*}
\proofpart{Lucas Sesta}
\begin{align*}
\Luc{n}^6={}&\Luc{n}^5\Luc{n}\\
\intertext{Per l'~\vref{eqn:LucQuintaSomma}}
={}&[\Luc{5n}+5(-1)^n\Luc{3n}+10\Luc{n}]\Luc{n}\\
={}&\Luc{5n}\Luc{n}+5(-1)^n\Luc{3n}\Luc{n}+10\Luc{n}^2\\
\intertext{Per il~\vref{thm:FibProdSomma}}
\Luc{m}\Luc{n}={}&\Luc{m+n}+(-1)^{n}\Luc{m-n}\\
\Luc{5n}\Luc{n}={}&\Luc{5n+n}+(-1)^{n}\Luc{5n-n}\\
={}&\Luc{4n}+(-1)^{n}\Luc{4n}\\
\Luc{3n}\Luc{n}={}&\Luc{3n+n}+(-1)^{n}\Luc{3n-n}\\
={}&\Luc{4n}+(-1)^{n}\Luc{2n}\\
\Luc{n}^2={}&\Luc{2n}+2(-1)^n
\intertext{Quindi}
={}&\Luc{5n}+(-1)^{n}\Luc{4n}+5(-1)^n[\Luc{4n}+(-1)^{n}\Luc{2n}]+10[\Luc{4n}+(-1)^{n}]\\
={}&\Luc{6n}+(-1)^{n}\Luc{4n}+5(-1)^n\Luc{4n}+5(-1)^{2n}\Luc{2n}+10\Luc{4n}+20(-1)^{n}\\
={}&\Luc{6n}+6(-1)^{n}\Luc{4n}+15\Luc{2n}+20(-1)^{n}\\
\end{align*}
cvd.
\end{proof}
\begin{thm}[Algoritmo per rimuovere a e 
b]~\cite{Rabinowitz_1996}\label{thm:FibLucRimuoviab}
	Se $\Fib{n}$ è la successione di Fibonacci e  $\Luc{n}$ è quella di Lucas 
	allora:\begin{equation}
		\left\{\begin{aligned}
				a^n={}&{}{}\dfrac{\Luc{n}+(a-b)\Fib{n}}{2}\\
				b^n={}&{}{}\dfrac{\Luc{n}-(a-b)\Fib{n}}{2}\\
			\end{aligned}
			\right.
	\end{equation}
\end{thm}
\begin{proof}
	Utilizzando il~\vref{thm:FibFormulaBinet} e il~\vref{thm:LucFormulaBinet} 
	possiamo scrivere
	\begin{equation*}
		\left\{
		\begin{aligned}
			\Fib{n}={}&{}{}\dfrac{1}{a-b}\left(a^n-b^n\right)\\
			\Luc{n}={}&{}{}a^n+b^n
		\end{aligned}
		\right.
	\end{equation*}
che risolto rispetto $a^n$ e $b^n$ da la tesi.
\end{proof}
\section{Formule di sottrazione}
\begin{thm}[Formule di 
sottrazione]~\cite{Rabinowitz_1996}\label{thm:FibLucFormSottazione}
	Se $\Fib{n}$ è la successione di Fibonacci e  $\Luc{n}$ è quella di Lucas 
allora:
\begin{align}
	\Fib{m-n}={}&{}(-1)^n\dfrac{\Fib{m}\Luc{n}-\Luc{m}\Fib{n}}{2}\\
	\Luc{m-n}={}&{}(-1)^n\dfrac{\Luc{m}\Luc{n}-5\Fib{m}\Fib{n}}{2}
\end{align}
\end{thm}
\begin{proof}
\proofpart{Fibonacci}
Utilizzando l'~\vref{eqn:FibLucSommaprodotto}
\begin{align*}
	\Fib{m+n}={}&{}\dfrac{\Fib{m}\Luc{n}+\Luc{m}\Fib{n}}{2}\\
\intertext{Posto}
n={}&{}-n\\
\Fib{m-n}={}&{}\dfrac{\Fib{m}\Luc{-n}+\Luc{m}\Fib{-n}}{2}\\
\intertext{Utilizzando il~\vref{thm:FibLucNeg} }
={}&{}\dfrac{\Fib{m}(-1)^n\Luc{n}+\Luc{m}(-1)^{n+1}\Fib{n}}{2}\\
={}&{}(-1)^n\dfrac{\Fib{m}\Luc{n}-\Luc{m}\Fib{n}}{2}
\end{align*}
\proofpart{Lucas}
Utilizzando l'~\vref{eqn:LucFibSommaprodotto}
\begin{align*}
\Luc{m+n}={}&{}\dfrac{5\Fib{m}\Fib{n}+\Luc{m}\Luc{n}}{2}\\
\intertext{Posto}
n={}&{}-n\\
\Luc{m-n}={}&{}\dfrac{5\Fib{m}\Fib{-n}+\Luc{m}\Luc{-n}}{2}\\
\intertext{Utilizzando il~\vref{thm:FibLucNeg} }
={}&{}\dfrac{5\Fib{m}\Fib{n}(-1)^{n+1}+\Luc{m}(-1)^n\Luc{n}}{2}\\
={}&{}(\Luc{m}\Luc{n}-1^n)\dfrac{-5\Fib{m}\Fib{n}}{2}\\
\end{align*}
cvd.
\end{proof}
\section{Limiti}
\begin{thm}[Limite successione]
	Se $\Fib{n}$ è la successione di Fibonacci allora 
	\begin{equation}
		\lim_{n\to\infty}\dfrac{\Fib{n+1}}{\Fib{n}}=\varphi
	\end{equation}\label{eqn:FibLimRap}
\end{thm}
\begin{proof}
	\begin{align*}
		\lim_{n\to\infty}\dfrac{\Fib{n+1}}{\Fib{n}}={}&\lim_{n\to\infty}\dfrac{\varphi^{n+1}-(1-\varphi)^{n+1}}{\varphi^n-(1-\varphi)^n}\\
		={}&\lim_{n\to\infty}\dfrac{\varphi^{n+1}\left[1-\left(\dfrac{1-\varphi}{\varphi}\right)^{n+1}\right]}{\varphi^{n}\left[1-\left(\dfrac{1-\varphi}{\varphi}\right)^{n}\right]}\\
		={}&\lim_{n\to\infty}\dfrac{\varphi\left[1-\left(\dfrac{1-\varphi}{\varphi}\right)^{n+1}\right]}{1-\left(\dfrac{1-\varphi}{\varphi}\right)^{n}}\\
		={}&\varphi
		\intertext{dato che}
		\lim_{n\to\infty}\left(\dfrac{1-\varphi}{\varphi}\right)^{n}={}&0
	\end{align*}
cvd.
\end{proof}
\section{Identità}
\begin{thm}[Identità di 
Cassini]\label{thm:FibCassini}\index{Fibonacci!identità!Cassini}
	Se $\Fib{n}$ è la successione di Fibonacci allora 
	\begin{equation}
		\Fib{n-1}\Fib{n+1}=\Fib{n}^2+(-1)^n
	\end{equation}\label{eqn:FibCassini}
\end{thm}
\begin{proof}
	
	\proofpart{Prima dimostrazione}
	Riscriviamo il risultato dell'~\vref{eqn:FinBinet} 
	\begin{align*}
		\Fib{n}={}&\dfrac{1}{a-b}\left(a^n-b^n\right)
		\intertext{da cui}
		\Fib{n}^2={}&\dfrac{a^2n}{(a-b)^2}-2\dfrac{a^nb^n}{(a-b)^n}+\dfrac{b^2n}{(a-b)^2}\\
		\intertext{Ma}
		\Fib{n-1}={}&\dfrac{1}{a-b}\left(a^{n-1}-b^{n-1}\right)\\
		\Fib{n+1}={}&\dfrac{1}{a-b}\left(a^{n+1}-b^{n+1}\right)\\
		\intertext{Quindi}
		\Fib{n-1}\Fib{n+1}={}&\dfrac{a^2n}{(a-b)^2}-2\dfrac{a^nb^n}{(a-b)^n}+\dfrac{b^2n}{(a-b)^2}-a^{n-1}b^{n-1}\\
		={}&	\Fib{n}^2-a^{n-1}b^{n-1}
		\intertext{Ma}
		-a^{n-1}b^{n-1}={}&-\left(\dfrac{1+\sqrt{5}}{2}\right)^{n-1}\left(\dfrac{1-\sqrt{5}}{2}\right)^{n-1}=(-1)^{n}\\
		\Fib{n-1}\Fib{n+1}={}&\Fib{n}^2+(-1)^n
	\end{align*}
	cvd.
	\proofpart{Seconda dimostrazione}
	\begin{align*}
\Fib{n-1}\Fib{n+1}={}&{}{}{}\\ 
\intertext{Per il~\vref{thm:LucFibSommaprodotto}}
\Fib{n+m}={}&{}{}{}\dfrac{\Fib{m}\Luc{n}+\Luc{m}\Fib{n}}{2}\\
\Fib{n-1}={}&{}{}{}\dfrac{\Fib{-1}\Luc{n}+\Luc{-1}\Fib{n}}{2}\\
\intertext{Per il~\vref{thm:FibLucNeg}}
\Fib{-1}={}&(-1)^{-1+1}\Fib{1}=\Fib{1}\\	
\Luc{-1}={}&(-1)^{1}\Luc{1}=-	\Luc{1}\\
={}&{}{}{}\dfrac{\Fib{1}\Luc{n}-\Luc{1}\Fib{n}}{2}\\
={}&{}{}{}\dfrac{\Luc{n}-\Fib{n}}{2}\\
\Fib{n+1}={}&{}{}{}\dfrac{\Luc{n}+\Fib{n}}{2}\\
\intertext{Quindi}
={}&{}{}{}\dfrac{(\Luc{n}-\Fib{n})(\Luc{n}+\Fib{n})}{2}\\
={}&{}{}{}\dfrac{\Luc{n}^2-\Fib{n}^2}{4}\\
\intertext{Per il~\vref{thm:FibLucFondamentale}}
={}&{}{}{}\dfrac{5\Fib{n}^2+4(-1)^{n}-\Fib{n}^2}{4}\\
={}&{}{}{}\Fib{n}^2+(-1)^{n}
	\end{align*}
cvd.
\end{proof}

\begin{thm}[Identità di Catalan]
	Se $\Fib{n}$ è la successione di Fibonacci allora 
	\begin{equation}
		\Fib{n}^2-\Fib{n-r}\Fib{n+r}=(-1)^{n-r}\Fib{r}^2
	\end{equation}\label{eqn:FibCatalan}
\end{thm}\index{Fibonacci!identità!Catalan}
\begin{proof}
	
	\proofpart{Prima dimostrazione}
	Riscriviamo il risultato dell'~\vref{eqn:FinBinet} 
	\begin{align*}
		\Fib{n}={}&\dfrac{1}{a-b}\left(a^n-b^n\right)
		\intertext{da cui}
		\Fib{n}^2={}&\dfrac{a^{2n}-2a^nb^n+b^{2n}}{(a-b)^2}\\\Fib{n-r}={}&\dfrac{1}{a-b}\left(a^{n-r}-b^{n-r}\right)\\
		\Fib{n+r}={}&\dfrac{1}{a-b}\left(a^{n+r}-b^{n+r}\right)\\
		\Fib{n}^2-	
		\Fib{n-r}\Fib{n+r}={}&\dfrac{a^{n+r}b^{n-r}}{(a-b)^2}+\dfrac{a^{n-r}b^{n+r}}{(a-b)^2}-\dfrac{2a^{n}b^{n}}{(a-b)^2}\\
		={}&\dfrac{a^{n-r}b^{n-r}(a^r-b^r)^2}{(a-b)^2}
		\intertext{Per~\vref{lem:FibpropPhi}}
		={}&\dfrac{(-1)^{n-r}(a^r-b^r)^2}{(a-b)^2}\\
		={}&(-1)^{n-r}\Fib{r}^2\\
	\end{align*}
	cvd
	
	\proofpart{Seconda dimostrazione}
	\begin{align*}
		\intertext{Per~\vref{thm:FibLucFormSottazione}}
		\Fib{n-r}={}&{}(-1)^r\dfrac{\Fib{n}\Luc{r}-\Luc{n}\Fib{r}}{2}\\
		\intertext{Per~\vref{thm:LucFibSommaprodotto}}
		\Fib{n+r}={}&{}\dfrac{\Fib{n}\Luc{r}+\Luc{n}\Fib{r}}{2}\\
\Fib{n-r}\Fib{n+r}
={}&{}\dfrac{(-1)^r}{4}\left(\Fib{n}\Luc{r}-\Luc{n}\Fib{r}\right)\left(\Fib{n}\Luc{r}+\Luc{n}\Fib{r}\right)\\
={}&{}\dfrac{(-1)^r}{4}\left(\Fib{n}^2\Luc{r}^2-\Luc{n}^2\Fib{r}^2\right)\\
\intertext{Per il~\vref{cor:FibpotSomma}}
\Fib{n}^2={}&\dfrac{\Luc{2n}-2(-1)^n}{5}\\
\Fib{r}^2={}&\dfrac{\Luc{2r}-2(-1)^r}{5}\\
\Luc{n}^2={}&\Luc{2n}+2(-1)^n\\
\Luc{r}^2={}&\Luc{2r}+2(-1)^r\\
\Fib{n}^2\Luc{r}^2={}&{}\dfrac{1}{5}\left(\Luc{2n}-2(-1)^n\right)\left(\Luc{2r}+2(-1)^r\right)\\
 ={}&{}\dfrac{1}{5}[\Luc{2n}\Luc{2r}+2(-1)^r\Luc{2n}-2(-1)^n\Luc{2r}-4(-1)^{n+r}]\\
\Luc{n}^2\Fib{r}^2={}&{}\dfrac{1}{5}\left(\Luc{2n}+2(-1)^n\right)\left(\Luc{2r}-2(-1)^r
 \right)\\
 ={}&{}\dfrac{1}{5}[\Luc{2n}\Luc{2r}-2(-1)^r\Luc{2n}+2(-1)^n\Luc{2r}-4(-1)^{n+r}]\\
 \intertext{Quindi}
 \dfrac{(-1)^r}{4}\left(\Fib{n}^2\Luc{r}^2-\Luc{n}^2\Fib{r}^2\right)={}&{}\\
 ={}&{} 
 \dfrac{(-1)^r}{4}\dfrac{1}{5}[\Luc{2n}\Luc{2r}+2(-1)^r\Luc{2n}-2(-1)^n\Luc{2r}-4(-1)^{n+r}\\
 -&\Luc{2n}\Luc{2r}+2(-1)^r\Luc{2n}-2(-1)^n\Luc{2r}+4(-1)^{n+r}]\\
  ={}&{} \dfrac{(-1)^r}{4}\dfrac{1}{5}[4(-1)^r\Luc{2n}-4(-1)^n\Luc{2r}]\\
  ={}&{} \dfrac{(-1)^r}{5}[(-1)^r\Luc{2n}-(-1)^n\Luc{2r}]\\
\intertext{Ricapitolando}
	\Fib{n}^2-\Fib{n-r}\Fib{n+r}={}&{}
	\intertext{Per il~\vref{cor:FibpotSomma}}
={}&{}\dfrac{\Luc{2n}-2(-1)^n}{5}-\dfrac{(-1)^r}{5}[(-1)^r\Luc{2n}-(-1)^n\Luc{2r}]\\
={}&{}\dfrac{\Luc{2n}-2(-1)^n- (-1)^{2r}\Luc{2n}+(-1)^{n+r}\Luc{2r}}{5}\\
={}&{}\dfrac{-2(-1)^n+(-1)^{n+r}\Luc{2r}}{5}\\
={}&{}(-1)^{n-r}\dfrac{-2(-1)^r+(-1)^{2r}\Luc{2r}}{5}\\
={}&{}(-1)^{n-r}\dfrac{-2(-1)^r+\Luc{2r}}{5}\\
	\intertext{Per~\vref{lem:FibpropPhi}}
		={}&(-1)^{n-r}\Fib{r}^2\\
\end{align*}
cvd.
\end{proof}
\begin{thm}[Identità di Vajada]\label{thm:FibVajada}
	Se $\Fib{n}$ è la successione di Fibonacci allora 
	\begin{equation}
		\Fib{n+i}\Fib{n+j}-\Fib{n}\Fib{n+i+j}=(-1)^{n}\Fib{i}\Fib{j}
	\end{equation}
\end{thm}\index{Fibonacci!identità!Vajada}
\begin{proof}
	
	\proofpart{Prima dimostrazione}
	Riscriviamo il risultato dell'~\vref{eqn:FinBinet} 
	\begin{align*}
		\Fib{n}={}&\dfrac{1}{a-b}\left(a^n-b^n\right)
		\intertext{da cui}
		\Fib{n+i}={}&\dfrac{1}{a-b}\left(a^{n+i}-b^{n+i}\right)\\
		\Fib{n+j}={}&\dfrac{1}{a-b}\left(a^{n+j}-b^{n+j}\right)\\
		\Fib{n+i+j}={}&\dfrac{1}{a-b}\left(a^{n+i+j}-b^{n+i+j}\right)\\
		\Fib{n+i}\Fib{n+j}-\Fib{n}\Fib{n+i+j}={}&\dfrac{a^{n}b^{n}(a^i-b^i)(a^j-b^j)}{(a-b)^2}\\
		\intertext{Per~\vref{lem:FibpropPhi}}
		={}&\dfrac{(-1)^{n}(a^i-b^i)(a^j-b^j)}{(a-b)^2}\\
		={}&(-1)^{n}\Fib{i}\Fib{j}\\
	\end{align*}
	cvd.
	
	\proofpart{Seconda dimostrazione}
	\begin{align*}
		\intertext{Per il~\vref{thm:FibProdSomma}}
		\Fib{m}\Fib{n}={}&\dfrac{\Luc{n+m}-(-1)^n\Luc{m-n}}{5}\\
		\intertext{Quindi}
		\Fib{n+i}\Fib{n+j}={}&\dfrac{\Luc{n+i+n+j}-(-1)^{n+j}\Luc{n+i-n-j}}{5}\\
	={}&\dfrac{\Luc{2n+i+j}-(-1)^{n+j}\Luc{i-j}}{5}\\
		\Fib{n}\Fib{n+i+j}={}&\dfrac{\Luc{n+i+n+j}-(-1)^{n+i+j}\Luc{n-n-i-j}}{5}\\
	={}&\dfrac{\Luc{2n+i+j}-(-1)^{n+i+j}\Luc{-i-j}}{5}\\
		\Fib{n+i}\Fib{n+j}-\Fib{n}\Fib{n+i+j}={}&\dfrac{\Luc{2n+i+j}-(-1)^{n+j}\Luc{i-j}-\Luc{2n+i+j}+(-1)^{n+i+j}\Luc{-i-j}}{5}\\
	={}&\dfrac{-(-1)^{n+j}\Luc{i-j}+(-1)^{n+i+j}\Luc{-i-j}}{5}\\
		\intertext{Per il~\vref{thm:FibLucNeg}}
		\Luc{-i-j}={}&(-1)^{i+j}\Luc{i+j}\\
	={}&\dfrac{-(-1)^{n+j}\Luc{i-j}+(-1)^{n+i+j}(-1)^{i+j}\Luc{i+j}}{5}\\
	={}&\dfrac{-(-1)^{n+j}\Luc{i-j}+(-1)^{n+2i+2j}\Luc{i+j}}{5}\\
	={}&\dfrac{-(-1)^{n+j}\Luc{i-j}+(-1)^{n}\Luc{i+j}}{5}\\	
	={}&(-1)^{n}\dfrac{\Luc{i+j}-(-1)^{j}\Luc{i-j}}{5}\\	
	={}&(-1)^{n}\Fib{i}\Fib{j}\\
	\end{align*}
cvd.
\end{proof}
\begin{thm}[Identità di 
	d'Ocagne]\label{thm:FibdOcagne}\index{Fibonacci!identità!d'Ocagne}
	Se $\Fib{n}$ è la successione di Fibonacci allora 
	\begin{equation}
		\Fib{m}\Fib{n+1}-\Fib{n}\Fib{m+1}=(-1)^n\Fib{m-n}
	\end{equation}\label{eqn:FibdOcagne}
\end{thm}
\begin{proof}
\begin{align*}
	\intertext{Per il~\vref{thm:FibProdSomma}}
	\Fib{m}\Fib{n}={}&\dfrac{\Luc{n+m}-(-1)^n\Luc{m-n}}{5}\\
	\intertext{Quindi}
		\Fib{m}\Fib{n+1}={}&\dfrac{\Luc{n+m+1}-(-1)^{n+1}\Luc{m-n-1}}{5}\\
		\Fib{n}\Fib{m+1}={}&\dfrac{\Luc{n+m+1}-(-1)^{n}\Luc{m-n+1}}{5}\\
		\Fib{m}\Fib{n+1}-\Fib{n}\Fib{m+1}
	={}&{}\dfrac{1}{5}\left[\Luc{n+m+1}-(-1)^{n+1}\Luc{m-n-1}-\Luc{n+m+1}+(-1)^{n}\Luc{m-n+1}\right]\\
={}&\dfrac{1}{5}\left[(-1)^{n}\Luc{m-n-1}+(-1)^{n}\Luc{m-n+1}\right]\\
={}&\dfrac{(-1)^{n}}{5}\left[\Luc{m-n+1}+\Luc{m-n-1}\right]\\
\intertext{Per l'~\vref{eqn:FibConvertiinLucas}}
={}&(-1)^{n}\Fib{m-n}\\
\end{align*}
cvd.
\end{proof}
\begin{thm}[Identità di 
	Honsberger]\label{thm:FibHonsberger}\index{Fibonacci!identità!Honsberger}
	Se $\Fib{n}$ è la successione di Fibonacci allora 
	\begin{equation}
		\Fib{k-1}\Fib{n}+\Fib{k}\Fib{n+1}=\Fib{n+k}
	\end{equation}\label{eqn:FibHonsberger}
\end{thm}
\begin{proof}
	\begin{align*}
		\intertext{Per il~\vref{thm:FibProdSomma}}
	\Fib{m}\Fib{n}={}&\dfrac{\Luc{n+m}-(-1)^n\Luc{m-n}}{5}\\
	\intertext{Quindi}
	\Fib{k-1}\Fib{n}={}&\dfrac{\Luc{n+k-1}-(-1)^n\Luc{k-n-1}}{5}\\
	\Fib{k}\Fib{n+1}={}&\dfrac{\Luc{n+k+1}-(-1)^{n+1}\Luc{k-n-1}}{5}\\
	\Fib{k-1}\Fib{n}-\Fib{k}\Fib{n+1}=
	&\dfrac{1}{5}\left[\Luc{n+k-1}-(-1)^n\Luc{k-n-1}+\Luc{n+k+1}-(-1)^{n+1}\Luc{k-n-1}\right]\\
={}&{}\dfrac{1}{5}\left[\Luc{n+k-1}-(-1)^n\Luc{k-n-1}+\Luc{n+k+1}+(-1)^{n}\Luc{k-n-1}\right]\\
={}&{}\dfrac{1}{5}\left[\Luc{n+k-1}+\Luc{n+k+1}\right]\\
	\intertext{Per l'~\vref{eqn:FibConvertiinLucas}}
={}&{}\Fib{n+k}
	\end{align*}
cvd.
\end{proof}
\begin{thm}[Identità di Gelin 
Cesàro]\label{thm:FibGelinCesaro}\index{Fibonacci!identità!Gelin Cesàro}
	Se $\Fib{n}$ è la successione di Fibonacci allora 
	\begin{equation}
	\Fib{n}^4-\Fib{n-2}\Fib{n-1}\Fib{n+1}\Fib{n+2}=1
	\end{equation}\label{eqn:FibGelinCesaro}
\end{thm}
\begin{proof}
\begin{align*}
		\intertext{Per il~\vref{thm:FibProdSomma}}
	\Fib{m}\Fib{n}={}&\dfrac{\Luc{n+m}-(-1)^n\Luc{m-n}}{5}\\
	\intertext{Quindi}
	\Fib{n-2}\Fib{n+2}={}&\dfrac{\Luc{n-2+n+2}-(-1)^{n-2}\Luc{n+2-n+2}}{5}\\
={}&\dfrac{\Luc{2n}-(-1)^{n-2}\Luc{4}}{5}\\
	\Fib{n-1}\Fib{n+1}={}&\dfrac{\Luc{n-1+n+1}-(-1)^{n-1}\Luc{n+1-n+1}}{5}\\
={}&\dfrac{\Luc{2n}-(-1)^{n-1}\Luc{2}}{5}\\
	\Fib{n-2}\Fib{n-1}\Fib{n+1}\Fib{n+2}
={}&\dfrac{1}{25}\left[\Luc{2n}-(-1)^{n-1}\Luc{2}\right]\left[\Luc{2n}-(-1)^{n-2}\Luc{4}\right]\\
={}&\dfrac{1}{25}\left[\Luc{2n}^2-(-1)^{n-2}\Luc{4}\Luc{2n}-(-1)^{n-1}\Luc{2}\Luc{2n}-(-1)^{n-1+n-2}\Luc{2}\Luc{4}\right]\\
={}&\dfrac{1}{25}\left[\Luc{2n}^2-7(-1)^{n}\Luc{2n}+3(-1)^{n}\Luc{2n}-21\right]\\
={}&\dfrac{1}{25}\left[\Luc{2n}^2-4(-1)^{n}\Luc{2n}-21\right]\\
	\intertext{Per l'~\vref{eqn:FibQuartaSomma}}
	\Fib{n}^4={}&\dfrac{\Luc{4n}-4(-1)^n\Luc{2n}+6}{25}\\
	\intertext{Ricapitolando}
={}&\dfrac{1}{25}\left[\Luc{4n}-4(-1)^n\Luc{2n}+6-\Luc{2n}^2+4(-1)^{n}\Luc{2n}+21\right]\\
={}&\dfrac{1}{25}\left[\Luc{4n}-\Luc{2n}^2+27\right]\\
	\intertext{Ma}
	\Luc{2n}^2=&\Luc{4n}+2\\
={}&\dfrac{1}{25}\left[\Luc{4n}-\Luc{4n}-2+27\right]\\
={}&\dfrac{1}{25}25=1\\
\end{align*}
cvd.
\end{proof}
\section{Proprietà}
\begin{thm}[Dispari]\label{thm:Fibdispari}
	Se $\Fib{n}$ è la successione di Fibonacci allora 
	\begin{equation}
		\Fib{n}^2+\Fib{n+1}^2=\Fib{2n+1}
	\end{equation}\label{eqn:FibDispari}
\end{thm}
\begin{proof}
	
	\proofpart{Prima dimostrazione}
	
	Riscriviamo il risultato dell'~\vref{eqn:FinBinet} 
	\begin{align*}
		\Fib{n}={}&\dfrac{1}{a-b}\left(a^n-b^n\right)
		\intertext{da cui}
		\Fib{n}^2={}&\dfrac{a^{2n}-2a^nb^n+b^{2n}}{(a-b)^2}\\
		\Fib{n+1}^2={}&\dfrac{a^{2(n+1)}-2a^{n+1}b^{n+1}+b^{2(n+1)}}{(a-b)^2}\\
		\Fib{n}^2+\Fib{n+1}^2={}&\dfrac{a^{2n}(a^2+1)-2a^nb^n(ab+1)+b^{2n}(b^2+1)}{(a-b)^2}
		\intertext{Per~\vref{lem:FibpropPhi}}		
		={}&\dfrac{a^{2n}(a-b)a+b^{2n}(b-a)b}{(a-b)^2}\\
		={}&\dfrac{1}{a-b}\left(a^{2n+1}-b^{2n+1}\right)\\
		={}&\Fib{2n+1}
	\end{align*}
	cvd.
	
	\proofpart{Seconda dimostrazione}
	
	\begin{align*}
		\intertext{Per il~\vref{thm:FibCassini}}
		\Fib{n}^2={}&\Fib{n-1}\Fib{n+1}-(-1)^n\\
		\intertext{Quindi}
		\Fib{n}^2+\Fib{n+1}^2={}&\Fib{n-1}\Fib{n+1}-(-1)^n+\Fib{n+1}^2\\
	={}&\Fib{n+1}\left[\Fib{n-1}+\Fib{n+1}\right]-(-1)^n\\
		\intertext{ma per il~\vref{thm:LucasToFibFibToLuc}}
			\Luc{n}={}&\Fib{n+1}+\Fib{n-1}
		\intertext{Quindi}
	={}&\Fib{n+1}\Luc{n}-(-1)^n\\
		\intertext{Per il~\vref{thm:FibProdSomma}}
		={}&\Fib{n+1+n}+(-1)^n\Fib{n+1-n}-(-1)^n\\
			={}&\Fib{2n+1}+(-1)^n\Fib{1}-(-1)^n\\
				={}&\Fib{2n+1}\\
	\end{align*}
\end{proof}
\begin{thm}[Quattro numeri consecutivi]\label{thm:FibConsecutivi}
	Se $\Fib{n}$ è la successione di Fibonacci allora 
	\begin{equation}
		\Fib{n+2}^2-\Fib{n+1}^2=\Fib{n}\Fib{n+3}
	\end{equation}\label{eqn:FibConsecutivi}
\end{thm}
\begin{proof}
	Riscriviamo il risultato dell'~\vref{eqn:FinBinet} 
	\begin{align*}
		\Fib{n}={}&\dfrac{1}{a-b}\left(a^n-b^n\right)
		\intertext{da cui}
		\Fib{n+2}^2-\Fib{n+1}^2={}&\dfrac{a^{2(n+2)}-2(ab)^{n+2}+b^{2(n+2)}}{(a-b)^2}-\dfrac{a^{2(n+1)}-2(ab)^{n+1}+b^{2(n+1)}}{(a-b)^2}\\
		={}&\dfrac{a^{2(n+1)}(a^2-1)+2(ab)^{n+1}(1-ab)+b^{2(n+1)}(b^2-1)}{(a-b)^2}
		\intertext{Per~\vref{lem:FibpropPhi}}
		={}&\dfrac{a^{2(n+1)}a+2(ab)^{n+1}2+b^{2(n+1)}b}{(a-b)^2}\\
		={}&\dfrac{a^{2n+3}+4(ab)^{n+1}+b^{2n+3}b}{(a-b)^2}\\
		\Fib{n}\Fib{n+3}={}&\dfrac{1}{a-b}\left(a^n-b^n\right)\dfrac{1}{a-b}\left(a^{n+3}-b^{n+3}\right)\\
		={}&\dfrac{(a^n-b^n)(a^{n+3}-b^{n+3})}{(a-b)^2}\\
		={}&\dfrac{(a^n-b^n)(a^{n}a^3-b^{n}b^3)}{(a-b)^2}
		\intertext{Per~\vref{lem:FibpropPhi}}
		={}&\dfrac{(a^n-b^n)[a^{n}(a^2+a)-b^{n}(b^2+b)]}{(a-b)^2}\\
		={}&\dfrac{a^{2n+1}(a+1)-(ab)^n(a^2+a+b^2+b)+b^{2n+1}(b+1)}{(a-b)^2}\\
		={}&\dfrac{a^{2n+1}a^2-(ab)^n(a+b+2+a+b)+b^{2n+1}b^2}{(a-b)^2}\\
		={}&\dfrac{a^{2n+3}+4ab(ab)^n+b^{2n+3}}{(a-b)^2}\\
		={}&\dfrac{a^{2n+3}+4(ab)^{n+1}+b^{2n+3}}{(a-b)^2}\\
	\end{align*}
	cvd.
\end{proof}
\begin{thm}[Tre numeri consecutivi]
	Tre numeri di Fibonacci consecutivi non possono essere i lati di un 
	triangolo.
\end{thm}\index{Fibonacci!triangolo}
\begin{proof}
	Supponiamo che $\Fib{n}$, $\Fib{n+1}$ e $\Fib{n+2}$ siano i lati di un 
	triangolo. Allora per il~\vref{thm:Formula_Erone} avremo, indicando con $p$ 
	il semiperimetro e $a$, $b$ e $c$ i lati:
	\begin{align*}
		S^2={}&p(p-a)(p-b)(p-c)\\
		\intertext{Ma}
		p={}&\dfrac{\Fib{n}+\Fib{n+1}+\Fib{n+2}}{2}\\
		p={}&\dfrac{\Fib{n+2}+\Fib{n+2}}{2}\\
		p={}&\Fib{n+2}\\
		p-c={}&\Fib{n+2}-\Fib{n+2}=0\\
	\end{align*}
	Quindi il triangolo ha area zero, assurdo. 
\end{proof}
\begin{thm}[Quadrati consecutivi]
	Se $\Fib{n}$ è la successione di Fibonacci allora:
	\begin{equation}
		\Fib{n}^2+\Fib{n+1}^2=\Fib{n}\Fib{n+2}+\Fib{n}\Fib{n+2}
	\end{equation}\label{eqn:FibquadratiConsecutivi}
\end{thm}
\begin{proof}
	Riscriviamo il risultato dell'~\vref{eqn:FinBinet} 
	\begin{align*}
		\Fib{n}={}&\dfrac{1}{a-b}\left(a^n-b^n\right)
		\intertext{da cui}
		\Fib{n}^2={}&\dfrac{a^{2n}-2(ab)^n+b^{2n}}{(a-b)^2}\\
		\Fib{n+1}^2={}&\dfrac{a^{2(n+1)}-2(ab)^{(n+1)}+b^{2(n+1)}}{(a-b)^2}\\
		\Fib{n}^2+\Fib{n+1}^2={}&\dfrac{a^{2n}(a^2+1)-2(ab)^n(ab+1)+b^{2n}(b^2+1)}{(a-b)^2}\\
		\intertext{Per~\vref{lem:FibpropPhi}}
		={}&\dfrac{a^{2n}(a^2+1)+b^{2n}(b^2+1)}{(a-b)^2}\\
		\Fib{n}\Fib{n+1}={}&\dfrac{(a^{n}-b^{n})((a^{n+2}-b^{n+2})}{(a-b)^2}\\
		\Fib{n-1}\Fib{n+1}={}&\dfrac{(a^{n+1}-b^{n+1})((ba^{n}-ab^{n})}{(a-b)^2}\\
		\Fib{n}\Fib{n+1}+\Fib{n-1}\Fib{n+1}={}&
		\dfrac{ba^{2n+1}(a^2+1)-(ab)^n(a^2+b^2)(ab+1)+ab^{2n+1}(b^2+1)}{(a-b)^2}\\
		\intertext{Per~\vref{lem:FibpropPhi}}
		={}&\dfrac{a^{2n}(a^2+1)+b^{2n}(b^2+1)}{(a-b)^2}\\
	\end{align*}
	Da cui la tesi.
\end{proof}
\begin{thm}[Somme di numeri di Fibonacci]
	Se $\Fib{n}$ è la successione di Fibonacci allora:
	\begin{equation}
		\sum_{k=1}^{n}\Fib{k}=\Fib{n+2}-1
	\end{equation}\label{eqn:FibSommaNumeri}
\end{thm}\index{Fibonacci!somma}
\begin{proof}
	\begin{align*}
		\intertext{Per induzione su $n$}
		\Fib{1}={}&\Fib{1+2}-1=1\\
		\intertext{Supponiamola vera per $n-1$ proviamola per $n$}
		\sum_{k=1}^{n-1}\Fib{k}+\Fib{n}={}&\Fib{n+1}+\Fib{n}-1\\
		\sum_{k=1}^{n}\Fib{k}={}&\Fib{n+2}-1\\
	\end{align*}
	cvd.
\end{proof}
