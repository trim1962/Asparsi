\chapter{Fibonacci}
\section{Definizione}
\begin{defn}[Numeri di Fibonacci]\index{Fibonacci!definizione}
	\begin{align}
		\Fib{1}={}&1\notag\\
		\Fib{2}={}&1\notag\\
		\Fib{n}={}&\Fib{n-1}+\Fib{n-2}\quad n>2\label{eqn:Fibodef}
	\end{align}
\end{defn}
\section{Sezione aurea}
\begin{defn}[Sezione aurea]
	Date due quantità $a$ e $b$ con $a>b>0$ diremo sezione aurea il rapporto
	\begin{equation}
	a+b:a=a:b=\varphi	
	\end{equation}\label{eqn:FibAureaDef}\index{Sezione!aurea}
\end{defn}
\begin{prop}
	Dalla~\vref{eqn:FibAureaDef} abbiamo
	\begin{align}
		\dfrac{a}{b}={}&\dfrac{a+b}{a}\notag\\
		={}&1+\dfrac{b}{a}\notag\\
		={}&1+\dfrac{1}{\frac{a}{b}}\notag\\
		\intertext{quindi}
		\varphi={}&1+\dfrac{1}{\varphi}\label{eqn:FibPhiProp}\\
		\varphi^2={}&\varphi+1\notag
	\end{align}
\end{prop}
	Dalla~\vref{eqn:FibPhiProp} segue
	\begin{prop}
	\begin{align}
		\varphi^2={}&\varphi+1\notag\\
		\varphi^2-\varphi-1={}&0\notag\\
		x^2-x-1={}&0\label{eqn:FibValPhiEqua}\\
		x_1={}&\dfrac{1+\sqrt{5}}{2}\label{eqn:FibValPhi}\\
		x_1={}&\dfrac{1-\sqrt{5}}{2}\notag
	\end{align}
\end{prop}
Il risultato~\ref{eqn:FibValPhi} permette di scrivere la seguente definizione
\begin{defn}[Sezione aurea]
La sezione aurea è: 
\begin{align*}
	\varphi={}&\dfrac{1+\sqrt{5}}{2}\\
\intertext{inoltre vale}
	1-	\varphi={}&\dfrac{1-\sqrt{5}}{2}\\
\end{align*}
\end{defn}\index{Sezione!aurea!valore}
\begin{lem}[Proprietà della sezione aurea]\label{lem:FibpropPhi}
	Se $a=\varphi$ e  $b=1-\varphi$ allora
	\begin{align*}
		ab={}&-1\\
		a^2+1={}&(a-b)a\\
		b^2+1={}&(b-a)b\\
		a^2={}&a+1\\
		b^2={}&b+1
		a+b={}&1\\
		a^3={}&a^2+a\\
		b^3={}&b^2+b\\
		a^2+b^2={}&a+b+2
	\end{align*}
\end{lem}
\begin{proof}
	\begin{align*}
		ab={}&\dfrac{1+\sqrt{5}}{2}\dfrac{1-\sqrt{5}}{2}\\
		={}&\dfrac{1-5}{4}\\
		={}&-1
	\end{align*}
	\begin{align*}
		(a-b)a={}&a^2-ab=a^2+1\\
		(b-a)b={}&b^2-ab=b^2+1\\
	\end{align*}
	cvd.
\end{proof}
\section{Formula di Binet}
\begin{thm}[Formula di Binet]
	Se $\Fib{n}$ è la successione di Fibonacci allora avremo:
	\begin{equation}
		\Fib{n}=\dfrac{1}{\sqrt{5}}\left[\left(\dfrac{1+\sqrt{5}}{2}\right)^n-\left(\dfrac{1-\sqrt{5}}{2}\right)^n\right]
	\end{equation}\label{eqn:FinBinet}
\end{thm}~\cite{Conti2020}
\begin{proof}
	Poniamo $a=\varphi$ e con $b=1-\varphi$ allora
	\begin{align*}
		\intertext{Poniamo}
		a^2={}&a\Fib{2}+\Fib{1}\\
		a^3={}&a^2a\\
		={}&a(a+1)\\
		={}&a^2+a\\
		={}&a+1+a\\
		={}&2a+1\\
		={}&a\Fib{3}+\Fib{2}\\
		\intertext{Per induzione su $n$}
		a^{n-1}={}&a\Fib{n-1}+\Fib{n-2}\\
		a^{n-1}a={}&a^2\Fib{n-1}+a\Fib{n-2}\\
		a^{n}={}&(a+1)\Fib{n-1}+a\Fib{n-2}\\
		a^{n}={}&a\Fib{n-1}+\Fib{n-1}+a\Fib{n-2}\\
		a^{n}={}&a(\Fib{n-1}+\Fib{n-2})+\Fib{n-1}\\
		a^{n}={}&a\Fib{n}+\Fib{n-1}\\
		\intertext{Analogamente si dimostra}
		b^{n}={}&b\Fib{n}+\Fib{n-1}\\
		\intertext{Sottraendo}
		a^n-b^n={}&a\Fib{n}-b\Fib{n}\\
		a^n-b^n={}&\Fib{n}(a-b)\\
		\Fib{n}={}&\dfrac{a^n-b^n}{a-b}
	\end{align*}
quindi
\begin{equation}
\Fib{n}=\dfrac{1}{\sqrt{5}}\left[\left(\dfrac{1+\sqrt{5}}{2}\right)^n-\left(\dfrac{1-\sqrt{5}}{2}\right)^n\right]
\end{equation}
\end{proof}
Altra dimostrazione
\begin{proof}
	\begin{align*}
		\intertext{Poniamo:}
		\Fib{n}={}&xa^n+yb^n\\
		\left\{
		\begin{array}{l}
			a^1x+b^1y=1\\ a^2y+b^2y=1
		\end{array}\right.&
	\intertext{Otteniamo}
	\left\{
	\begin{array}{l}
	x=\dfrac{1-b}{a(a-b)}\\ y=\dfrac{a-1}{b(a-b)}
	\end{array}\right.&
	\end{align*}
\end{proof}
\begin{thm}[Limite successione]
Se $\Fib{n}$ è la successione di Fibonacci allora 
\begin{equation}
	\lim_{n\to\infty}\dfrac{\Fib{n+1}}{\Fib{n}}=\varphi
\end{equation}\label{eqn:FibLimRap}
\end{thm}
\begin{proof}
\begin{align}
	\lim_{n\to\infty}\dfrac{\Fib{n+1}}{\Fib{n}}={}&\lim_{n\to\infty}\dfrac{\varphi^{n+1}-(1-\varphi)^{n+1}}{\varphi^n-(1-\varphi)^n}\\
	={}&\lim_{n\to\infty}\dfrac{\varphi^{n+1}\left[1-\left(\dfrac{1-\varphi}{\varphi}\right)^{n+1}\right]}{\varphi^{n}\left[1-\left(\dfrac{1-\varphi}{\varphi}\right)^{n}\right]}\\
	={}&\lim_{n\to\infty}\dfrac{\varphi\left[1-\left(\dfrac{1-\varphi}{\varphi}\right)^{n+1}\right]}{1-\left(\dfrac{1-\varphi}{\varphi}\right)^{n}}\\
	={}&\varphi
	\intertext{dato che}
	\lim_{n\to\infty}\left(\dfrac{1-\varphi}{\varphi}\right)^{n}={}&0
\end{align}cvd
\end{proof}
\begin{thm}[Quadrato]\label{thm:fibQuadrato}
	Se $\Fib{n}$ è la successione di Fibonacci allora 
	\begin{equation}
		\Fib{n-1}\cdot\Fib{n+1}=\Fib{n}^2+(-1)^n
	\end{equation}\label{eqn:FibQuadrato}
\end{thm}
\begin{proof}
Riscriviamo il risultato del~\vref{eqn:FinBinet} 
\begin{align*}
	\Fib{n}={}&\dfrac{1}{a-b}\left(a^n-b^n\right)
	\intertext{da cui}
	\Fib{n}^2={}&\dfrac{a^2n}{(a-b)^2}-2\dfrac{a^nb^n}{(a-b)^n}+\dfrac{b^2n}{(a-b)^2}\\
	\intertext{ma}
	\Fib{n-1}={}&\dfrac{1}{a-b}\left(a^{n-1}-b^{n-1}\right)\\
	\Fib{n+1}={}&\dfrac{1}{a-b}\left(a^{n+1}-b^{n+1}\right)\\
	\intertext{quindi}
	\Fib{n-1}\cdot\Fib{n+1}={}&\dfrac{a^2n}{(a-b)^2}-2\dfrac{a^nb^n}{(a-b)^n}+\dfrac{b^2n}{(a-b)^2}-a^{n-1}b^{n-1}\\
	={}&	\Fib{n}^2-a^{n-1}b^{n-1}
		\intertext{ma}
		-a^{n-1}b^{n-1}={}&-\left(\dfrac{1+\sqrt{5}}{2}\right)^{n-1}\cdot\left(\dfrac{1-\sqrt{5}}{2}\right)^{n-1}=(-1)^{n}\\
		\Fib{n-1}\cdot\Fib{n+1}={}&\Fib{n}^2+(-1)^n
\end{align*}
cvd.
\end{proof}

\begin{thm}[Dispari]\label{thm:Fibdispari}
	Se $\Fib{n}$ è la successione di Fibonacci allora 
	\begin{equation}
		\Fib{n}^2+\Fib{n+1}^2=\Fib{2n+1}
	\end{equation}\label{eqn:FibDispari}
\end{thm}
\begin{proof}
	Riscriviamo il risultato del~\vref{eqn:FinBinet} 
	\begin{align*}
		\Fib{n}={}&\dfrac{1}{a-b}\left(a^n-b^n\right)
		\intertext{da cui}
		\Fib{n}^2={}&\dfrac{a^{2n}-2a^nb^n+b^{2n}}{(a-b)^2}\\
			\Fib{n+1}^2={}&\dfrac{a^{2(n+1)}-2a^{n+1}b^{n+1}+b^{2(n+1)}}{(a-b)^2}\\
		\Fib{n}^2+\Fib{n+1}^2={}&\dfrac{a^{2n}(a^2+1)-2a^nb^n(ab+1)+b^{2n}(b^2+1)}{(a-b)^2}
		\intertext{per~\vref{lem:FibpropPhi}}		
		={}&\dfrac{a^{2n}(a-b)a+b^{2n}(b-a)b}{(a-b)^2}\\
		={}&\dfrac{1}{a-b}\left(a^{2n+1}-b^{2n+1}\right)\\
		={}&\Fib{2n+1}
	\end{align*}
	cvd.
\end{proof}
\begin{thm}[Quattro numeri consecutivi]\label{thm:FibConsecutivi}
	Se $\Fib{n}$ è la successione di Fibonacci allora 
	\begin{equation}
		\Fib{n+2}^2-\Fib{n+1}^2=\Fib{n}\cdot\Fib{n+3}
	\end{equation}\label{eqn:FibConsecutivi}
\end{thm}
\begin{proof}
	Riscriviamo il risultato del~\vref{eqn:FinBinet} 
	\begin{align*}
		\Fib{n}={}&\dfrac{1}{a-b}\left(a^n-b^n\right)
		\intertext{da cui}
		\Fib{n+2}^2-\Fib{n+1}^2={}&\dfrac{a^{2(n+2)}-2(ab)^{n+2}+b^{2(n+2)}}{(a-b)^2}-\dfrac{a^{2(n+1)}-2(ab)^{n+1}+b^{2(n+1)}}{(a-b)^2}\\
		={}&\dfrac{a^{2(n+1)}(a^2-1)+2(ab)^{n+1}(1-ab)+b^{2(n+1)}(b^2-1)}{(a-b)^2}
		\intertext{per~\vref{lem:FibpropPhi}
			={}&\dfrac{a^{2(n+1)}a+2(ab)^{n+1}2+b^{2(n+1)}b}{(a-b)^2}\\
				={}&\dfrac{a^{2n+3}+4(ab)^{n+1}+b^{2n+3}b}{(a-b)^2}\\
		\Fib{n}\cdot\Fib{n+3}={}&\dfrac{1}{a-b}\left(a^n-b^n\right)\cdot\dfrac{1}{a-b}\left(a^{n+3}-b^{n+3}\right)\\
		={}&\dfrac{(a^n-b^n)(a^{n+3}-b^{n+3})}{(a-b)^2}\\
		={}&\dfrac{(a^n-b^n)(a^{n}a^3-b^{n}b^3)}{(a-b)^2}
		\intertext{per~\vref{lem:FibpropPhi}}
		={}&\dfrac{(a^n-b^n)[a^{n}(a^2+a)-b^{n}(b^2+b)]}{(a-b)^2}\\
		={}&\dfrac{a^{2n+1}(a+1)-(ab)^n(a^2+a+b^2+b)+b^{2n+1}(b+1)}{(a-b)^2}\\
		={}&\dfrac{a^{2n+1}a^2-(ab)^n(a+b+2+a+b)+b^{2n+1}b^2}{(a-b)^2}\\
		={}&\dfrac{a^{2n+3}+4ab(ab)^n+b^{2n+3}}{(a-b)^2}\\
		={}&\dfrac{a^{2n+3}+4(ab)^{n+1}+b^{2n+3}}{(a-b)^2}\\
	\end{align*}
	cvd.
\end{proof}
\begin{thm}[Tre numeri consecutivi]
	Tre numeri di Fibonacci consecutivi non possono essere i lati di un triangolo.
\end{thm}
\begin{proof}
	Supponiamo che $\Fib{n}$, $\Fib{n+1}$ e $\Fib{n+2}$ sono i lati di un triangolo. Allora per il~\vref{thm:Formula_Erone} avremo indicando con $p$ il semiperimetro e $a$, $b$ e $c$ i lati:
\begin{align*}
	S^2={}&p(p-a)(p-b)(p-c)\\
	\intertext{ma}
	p={}&\dfrac{\Fib{n}+\Fib{n+1}+\Fib{n+2}}{2}\\
	p={}&\dfrac{\Fib{n+2}+\Fib{n+2}}{2}\\
	p={}&\Fib{n+2}\\
	p-c={}&\Fib{n+2}-\Fib{n+2}=0\\
\end{align*}
Quindi il triangolo ha area zero, assurdo. 
\end{proof}
\chapter{Fibonacci generalizzato}
\section{Definizione}
\begin{defn}[Numeri di Fibonacci generalizzati]\index{Fibonacci!generalizzati!definizione}
	\begin{align}
		\Gib{1}={}&a\notag\\
		\Gib{2}={}&b\notag\\
		\Gib{n}={}&\Gib{n-1}+\Gib{n-2}\quad n>2\label{eqn:FiboGendef}
	\end{align}
\end{defn}
\section{Proprietà}
\begin{thm}[Derivazione]
	Se $\Gib{n}$ è una successione di Fibonacci generalizzata  con $\Gib{1}=a$ e $\Gib{2}=b$ allora
\begin{equation}
	\Gib{n}=b\Fib{n-1}+a\Fib{n-2}\quad n>3
\end{equation}\label{thm:FibGenDer}
Dove $\Fib{n}$ è la successione di Fibonacci
\end{thm}