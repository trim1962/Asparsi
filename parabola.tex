\chapter{Parabola}
\section{Definizione}
\begin{defn}[Definizione parabola]
Definisco parabola l'equazione $y=ax^2+bx+c$ $a\neq0$
\end{defn}\index{Parabola}
\section{Proprietà}
\begin{thm}[Complemento al quadrato]\label{thm:Parabola_complemento}
	La parabola $y=ax^2+bx+c$ $a\neq0$ è equivalente a \begin{equation*}
	y+\dfrac{\Delta}{4a}=a\left(x+\dfrac{b}{2a}\right)^2\quad\Delta=b^2-4ac
	\end{equation*}\label{equa:Parabola_scomposizione}
\end{thm}\index{Parabola!complemento!quadrato}
\begin{proof}
	\begin{align*}
	y=&ax^2+bx+c\\
	y-c=&ax^2+bx\\
	y+\dfrac{b^2}{4a}-c=&ax^2+bx+\dfrac{b^2}{4a}\\
	y+\dfrac{b^2-4ac}{4a}=&a\dfrac{4a^2x^2+4abx+b^2}{4a^2}\\
	y+\dfrac{\Delta}{4a}=&a\left(\dfrac{2ax+b}{2a}\right)^2\\
	y+\dfrac{\Delta}{4a}=&a\left(x+\dfrac{b}{2a}\right)^2
	\end{align*}
	Da cui la tesi.
\end{proof}
\section{Concavità}
\begin{lem}
	Se $m>0$ $\forall n$ $m-n>-n$ se $m<0$ $\forall n$ $m-n<-n$
\end{lem}
\begin{thm}[Concavità]
Data una parabola $y=ax^2+bx+c$ $a\neq0$ allora se $a>0$ la parabola ha la concavità rivolata verso l'alto e il vertice è il punto di minima ordinata, se $a<0$ la parabola ha la concavità rivolata verso il basso e il vertice è il punto di massima ordinata
\end{thm}
\begin{proof}
	Dal~\cref{thm:Parabola_complemento} abbiamo
	\begin{align*}
y=&a\left(x+\dfrac{b}{2a}\right)^2-\dfrac{b^2-4ac}{4a}& x\neq&\dfrac{b}{2a}\\
\intertext{Il vertice della parabola ha ordinata}
y_v=&-\dfrac{b^2-4ac}{4a}
\intertext{Se $a>0$ }
a\left(x+\dfrac{b}{2a}\right)^2>&0\\
\intertext{Quindi}
a\left(x+\dfrac{b}{2a}\right)^2-\dfrac{b^2-4ac}{4a}>&-\dfrac{b^2-4ac}{4a}\\
\intertext{Analogamente}
\intertext{Se $a<0$ }
a\left(x+\dfrac{b}{2a}\right)^2<&0\\
\intertext{Quindi}
a\left(x+\dfrac{b}{2a}\right)^2-\dfrac{b^2-4ac}{4a}<&-\dfrac{b^2-4ac}{4a}\\
	\end{align*}
Da cui la tesi
\end{proof}