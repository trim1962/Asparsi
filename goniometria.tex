% !TeX root = Asparsi.tex
% !BIB TS-program = biber
% !TeX encoding = UTF-8
% !TeX spellcheck = it_IT
\chapter{Goniometria}
\section{Somma sottrazione di angoli}
\begin{figure}
	\centering
\includestandalone[scale=0.68]{geometria/cosenosommadifferenza1}
	\caption{Coseno della differenza}
	\label{fig:cosenosommadifferenza1}
\end{figure}
\begin{thm}[Coseno della differenza di due angoli]\label{thm:Cosenodelladifferenza}
Dati due angoli $\alpha$ e $\beta$ allora
\[\cos(\alpha-\beta)=\cos\alpha\cos\beta+\sin\alpha\sin\beta  \]
\end{thm}\index{Coseno!differenza}
\begin{proof}
Consideriamo due angoli $\alpha$ e $\beta$ con $\alpha>\beta$. Otteniamo la~\vref{fig:cosenosommadifferenza1}. Costruiamo una angolo di ampiezza $\alpha-\beta$ con vertice in $O$ e un lato $AD$. Otteniamo quattro punti $A(\cos\alpha;\sin\alpha)$, $B (\cos\beta;\sin\beta)$, $C(\cos\left(\alpha-\beta\right);\sin\left(\alpha-\beta\right))$ e $D(1;0)$. I triangoli $AOB$ e $COD$ sono congruenti in particolare abbiamo: \begin{align*}
\widebar{CD}=&\widebar{AB}\\
\left[\cos(\alpha-\beta)-1\right]^2+\sin^2(\alpha-\beta)=&\left[\cos\alpha-\cos\beta\right]^2+\left[\sin\alpha-\sin\beta\right]^2\\
\cos^2(\alpha-\beta)-2\cos(\alpha-\beta)+1+\sin^2(\alpha-\beta)=&\cos^2\alpha+\cos^2\beta-2\cos\alpha\cos\beta+\sin^2\alpha+\sin^2\beta-2\sin\alpha\sin\beta\\
\intertext{ma}
\cos^2(\alpha-\beta)+\sin^2(\alpha-\beta)=&1\\
\cos^2\alpha+\sin^2\alpha=&1\\
\cos^2\beta+\sin^2\beta=&1\\
\intertext{quindi}
1+1-2\cos(\alpha-\beta)=&1+1-2\cos\alpha\cos\beta-2\sin\alpha\sin\beta\\
2\cos(\alpha-\beta)=&-2\cos\alpha\cos\beta-2\sin\alpha\sin\beta\\
\intertext{semplifichiamo}
\cos(\alpha-\beta)=&\cos\alpha\cos\beta+\sin\alpha\sin\beta 
\end{align*}
Da cui la tesi.
\end{proof}
\begin{cor}[Coseno della somma di due angoli]\label{cor:Cosenodellasomma}
Dati due angoli $\alpha$ e $\beta$ allora
\[\cos(\alpha+\beta)=\cos\alpha\cos\beta-\sin\alpha\sin\beta  \]
\end{cor}\index{Coseno!somma}
\begin{proof}
	Dal~\vref{thm:Cosenodelladifferenza}
	\begin{align*}
	\cos(\alpha+\beta)=&\cos\left[\alpha-(-\beta)\right]\\
	\cos\left[\alpha-(-\beta)\right]=&\cos\alpha\cos(-\beta)+\sin\alpha\sin(-\beta)\\
	\intertext{ma}
	\cos(-\beta)=&\cos\beta\\
	\sin(-\beta)=&-\sin\beta
	\intertext{quindi}
	\cos(\alpha+\beta)=&\cos\alpha\cos\beta-\sin\alpha\sin\beta
	\end{align*}
	Da cui la tesi.
\end{proof}
\begin{cor}[Seno della differenza di due angoli]\label{cor:Senodelladifferenza}
Dati due angoli $\alpha$ e $\beta$ allora
\[\sin(\alpha-\beta)=\sin\alpha\cos\beta-\cos\alpha\sin\beta  \]
\end{cor}\index{Seno!differenza}
\begin{proof}
		Dal~\vref{thm:Cosenodelladifferenza}
		\begin{align*}
		\sin(\alpha-\beta)=&\cos\left[\dfrac{\pi}{2}-\left(\alpha-\beta\right)\right]\\
		=&\cos\left[\left(\dfrac{\pi}{2}-\alpha\right)+\beta\right]\\
		=&\cos\left(\dfrac{\pi}{2}-\alpha\right)\cos\beta-\sin\left(\dfrac{\pi}{2}-\alpha\right)\sin\beta\\
		\intertext{ma}
		\cos\left(\dfrac{\pi}{2}-\alpha\right)=&\sin\alpha\\
		\sin\left(\dfrac{\pi}{2}-\alpha\right)=&\cos\alpha\\
		\intertext{quindi}
		\sin(\alpha-\beta)=&\sin\alpha\cos\beta-\cos\alpha\sin\beta\\
		\end{align*}
		Da cui la tesi.
\end{proof}
\begin{cor}[Seno della somma di due angoli]\label{cor:senosommaangoli}
	Dati due angoli $\alpha$ e $\beta$ allora
	\[\sin(\alpha+\beta)=\sin\alpha\cos\beta+\cos\alpha\sin\beta  \]
\end{cor}\index{Seno!somma}
\begin{proof}
	Dal~\vref{cor:Cosenodellasomma}
	\begin{align*}
	\sin(\alpha+\beta)=&\cos\left[\dfrac{\pi}{2}-\left(\alpha+\beta\right)\right]\\
	=&\cos\left[\left(\dfrac{\pi}{2}-\alpha\right)-\beta\right]\\
	=&\cos\left(\dfrac{\pi}{2}-\alpha\right)\cos\beta+\sin\left(\dfrac{\pi}{2}-\alpha\right)\sin\beta\\
	\intertext{ma}
	\cos\left(\dfrac{\pi}{2}-\alpha\right)=&\sin\alpha\\
	\sin\left(\dfrac{\pi}{2}-\alpha\right)=&\cos\alpha\\
	\intertext{quindi}
	\sin(\alpha\beta)=&\sin\alpha\cos\beta+\cos\alpha\sin\beta\\
	\end{align*}
	Da cui la tesi.
\end{proof}
\begin{thm}[Tangente differenza di angoli]\label{thm:tangentedifferenza}
Dati due angoli $\alpha$ e $\beta$ allora\[\tan\left(\alpha-\beta\right)=\dfrac{\tan\alpha-\tan\beta}{1+\tan\alpha\tan\beta}\qquad\alpha,\beta,(\alpha-\beta)\neq\dfrac{\pi}{2}+k\pi\]
\end{thm}\index{Tangente!differenza}
\begin{proof}
	Dalla definizione, dal~\vref{thm:Cosenodelladifferenza} e dal~\vref{cor:Senodelladifferenza}
	\begin{align*}
	\tan\left(\alpha-\beta\right)=&\dfrac{\sin\left(\alpha-\beta\right)}{\cos\left(\alpha-\beta\right)}\\
	=&\dfrac{\sin\alpha\cos\beta-\cos\alpha\sin\beta}{\cos\alpha\cos\beta+\sin\alpha\sin\beta}\\
	\intertext{Dato che $\cos\alpha\cos\beta\neq 0$}
	=&\dfrac{\dfrac{\sin\alpha\cos\beta}{\cos\alpha\cos\beta}-\dfrac{\cos\alpha\sin\beta}{\cos\alpha\cos\beta}}{\dfrac{\cos\alpha\cos\beta}{\cos\alpha\cos\beta}+\dfrac{\sin\alpha\sin\beta}{\cos\alpha\cos\beta}}\\
	=&\dfrac{\tan\alpha-\tan\beta}{1+\tan\alpha\tan\beta}\\
	\end{align*}
	Da cui la tesi.
\end{proof}
\begin{cor}[Tangente somma di angoli]\label{cor:Tangentesommadiangoli}
	Dati due angoli $\alpha$ e $\beta$ allora\[\tan\left(\alpha+\beta\right)=\dfrac{\tan\alpha+\tan\beta}{1-\tan\alpha\tan\beta}\qquad\alpha,\beta,(\alpha-\beta)\neq\dfrac{\pi}{2}+k\pi\]
\end{cor}\index{Tangente!somma}
\begin{proof}
	Dal~\vref{thm:tangentedifferenza} avremo:
	\begin{align*}
	\tan\left(\alpha+\beta\right)=&\tan\left[\alpha-\left(-\beta\right)\right]\\
	=&\dfrac{\tan\alpha-\tan\left(-\beta\right)}{1+\tan\alpha\tan\left(-\beta\right)}
	\intertext{ma}
	\tan\left(-\beta\right)=&-\tan\beta\\
	\intertext{quindi}
	=&\dfrac{\tan\alpha+\tan\beta}{1-\tan\alpha\tan\beta}\\
	\end{align*}
	Da cui la tesi.
\end{proof}
\begin{thm}[Cotangente differenza di angoli]\label{thm:cotangentedifferenza}
	Dati due angoli $\alpha$ e $\beta$ allora\[\cot\left(\alpha-\beta\right)=\dfrac{\cot\alpha\cot\beta+1}{\cot\beta-\cot\alpha}\qquad\alpha,\beta,(\alpha-\beta)\neq\ k\pi\]
\end{thm}\index{Cotangente!differenza}
\begin{proof}
	Dalla definizione dal~\vref{thm:Cosenodelladifferenza} e dal~\vref{cor:Senodelladifferenza}
	\begin{align*}
	\cot\left(\alpha-\beta\right)=&\dfrac{\cos\left(\alpha-\beta\right)}{\sin\left(\alpha-\beta\right)}\\
	=&\dfrac{\cos\alpha\cos\beta+\sin\alpha\sin\beta}{\sin\alpha\cos\beta-\cos\alpha\sin\beta}\\
	\intertext{Dato che $\sin\alpha\sin\beta\neq 0$}
	=&\dfrac{\dfrac{\cos\alpha\cos\beta}{[\sin\alpha\sin\beta}+\dfrac{\sin\alpha\sin\beta}{[\sin\alpha\sin\beta}}{\dfrac{\sin\alpha\cos\beta}{[\sin\alpha\sin\beta}-\dfrac{\cos\alpha\sin\beta}{[\sin\alpha\sin\beta}}\\
	=&\dfrac{\cot\alpha\cot\beta+1}{\cot\beta-\cot\alpha}\\
	\end{align*}
	Da cui la tesi.
\end{proof}
\begin{cor}[Cotangente somma di angoli]\label{cor:Cotangentesommadiangoli}
	Dati due angoli $\alpha$ e $\beta$ allora\[\cot\left(\alpha+\beta\right)=\dfrac{\cot\alpha\cot\beta-1}{\cot\alpha+\cot\beta}\qquad\alpha,\beta,(\alpha-\beta)\neq\ k\pi\]
\end{cor}\index{Cotangente!somma}
\begin{proof}
	Dal~\vref{thm:cotangentedifferenza} avremo:
	\begin{align*}
	\cot\left(\alpha+\beta\right)=&\cot\left[\alpha-\left(-\beta\right)\right]\\
	=&\dfrac{\cot\alpha\cot\left(-\beta\right)+1}{\cot\left(-\beta\right)-\cot\alpha}
	\intertext{ma}
	\cot\left(-\beta\right)=&-\cot\beta\\
	\intertext{quindi}
	=&\dfrac{-\cot\alpha\cot\beta+1}{-\cot\beta-\cot\alpha}\\
	=&\dfrac{\cot\alpha\cot\beta-1}{\cot\alpha+\cot\beta}\\
	\end{align*}
	Da cui la tesi.
\end{proof}
\section{Formule di duplicazione}
\begin{thm}[Formule di duplicazione coseno]\label{thm:Formulediduplicazionecoseno}
Dato un angolo $\alpha$ avremo:
\begin{align*}
\cos 2\alpha=&\cos^2\alpha-\sin^2\cos\alpha\\
\cos 2\alpha=&1-2\sin^2\alpha\\
\cos 2\alpha=&2\cos^2\alpha-1\\
\end{align*}
\end{thm}\index{Coseno!duplicazione}
\begin{proof}
\begin{align*}
\intertext{Dal~\vref{cor:Cosenodellasomma}}
\cos\left(2\alpha\right)=&\cos\left(2\alpha\right)\\
=&\cos\alpha\cos\alpha-\sin\alpha\sin\alpha\\
=&\cos^2\alpha-\sin^2\alpha\\
\intertext{ma}
\cos^2\alpha=&1-\sin^2\alpha\\
\sin^2\alpha=&1-\cos^2\alpha\\
\intertext{quindi}
=&\cos^2\alpha-\sin^2\alpha\\
=&1-\sin^2\alpha-\sin^2\alpha\\
=&1-2\sin^2\alpha\\
=&\cos^2\alpha-\sin^2\alpha\\
=&\cos^2\alpha-1+\cos^2\alpha\\
=&2\cos^2\alpha-1\\
\end{align*}
Da cui la tesi.
\end{proof}
\begin{thm}[Formule di duplicazione seno]\label{thm:Formulediduplicazioneseno}
	Dato un angolo $\alpha$ avremo:
	\[\sin\left(2\alpha\right)=2\sin\alpha\cos\alpha\]
\end{thm}\index{Seno!duplicazione}
\begin{proof}
	Dal~\vref{cor:senosommaangoli} abbiamo:
	\begin{align*}
	\sin\left(2\alpha\right)=&\sin\left(\alpha+\alpha\right)\\
	=&\sin\alpha\cos\alpha+\cos\alpha\sin\alpha\\
	=&2\sin\alpha\cos\alpha\\
	\end{align*}
	Da cui la tesi.
\end{proof}
\begin{thm}[Formule di duplicazione tangente]\label{thm:Formulediduplicazionetangente}
	Dato un angolo $\alpha$ avremo:
	\[\tan\left(2\alpha\right)=\dfrac{2\tan\alpha}{1-\tan^2\alpha}\qquad\begin{cases}
	\alpha\neq\dfrac{\pi}{2}+k\pi\\
	\\
	\alpha\neq\dfrac{\pi}{4}+k\dfrac{\pi}{2}\\
	\end{cases}\]
\end{thm}\index{Tangente!duplicazione}
\begin{proof}
Dal~\vref{cor:Tangentesommadiangoli} abbiamo
\begin{align*}
\tan\left(2\alpha\right)=&\tan\left(\alpha+\alpha\right)\\
=&\dfrac{\tan\alpha+\tan\alpha}{1-\tan\alpha\tan\alpha}\\
=&\dfrac{2\tan\alpha}{1-\tan^2\alpha}\\
\end{align*}
Da cui la tesi.
\end{proof}
\begin{thm}[Formule di duplicazione cotangente]\label{thm:Formulediduplicazionecotangente}
	Dato un angolo $\alpha$ avremo:
	\[\cot\left(2\alpha\right)=\dfrac{\cot^2\alpha-1}{2\cot\alpha}\qquad\alpha\neq\dfrac{\pi}{2}k\]
\end{thm}\index{Cotangente!duplicazione}
\begin{proof}
	Dal~\vref{cor:Cotangentesommadiangoli} abbiamo
	\begin{align*}
	\cot\left(2\alpha\right)=&\cot\left(\alpha+\alpha\right)\\
	=&\dfrac{1-\cot\alpha\tan\alpha}{\cot\alpha+\cot\alpha}\\
	=&\dfrac{1-\cot^2\alpha}{2\cot\alpha}\\
	\end{align*}
	Da cui la tesi.
\end{proof}
\section{Formule parametriche}
\begin{thm}[Funzione seno in funzione della tangente dell'angolo metà]\label{thm:senotangentemeta}
	Dato una angolo $\alpha$  allora:
\[\sin\alpha=\dfrac{2\tan\dfrac{\alpha}{2}}{1+\tan^2\dfrac{\alpha}{2}}\qquad\alpha\neq\pi+2k\pi \]
\end{thm}\index{Seno!tangente!metà}
\begin{proof}
	Dal~\vref{thm:Formulediduplicazioneseno} il seno della somma di due angoli 
	\begin{align*}
	\sin2\alpha=&2\sin\alpha\cos\alpha\\
	=&\frac{2\sin\alpha\cos\alpha}{1}\\
	=&\frac{2\sin\alpha\cos\alpha}{\cos^2\alpha+\sin^2\alpha}\\
	=&\dfrac{\dfrac{2\sin\alpha\cos\alpha}{\cos^2\alpha}}{\dfrac{\cos^2\alpha+\sin^2\alpha}{cos^2\alpha}}&\alpha\neq\frac{\pi}{2}+k\pi\\
	=&\dfrac{\dfrac{2\sin\alpha}{\cos\alpha}}{\dfrac{\cos^2\alpha}{\cos^2\alpha}+\dfrac{\sin^2\alpha}{cos^2\alpha}}\\
	=&\dfrac{2\tan\alpha}{1+\tan^2\alpha}\\
	\alpha&\longmapsto\dfrac{\alpha}{2}\\
	\sin\alpha=&\dfrac{2\tan\dfrac{\alpha}{2}}{1+\tan^2\dfrac{\alpha}{2}}\\
	\end{align*}
Da cui la tesi:
\end{proof}
\begin{thm}[Formule parametriche]\label{thm:formuleparametriche1}
	Dato una angolo $\alpha$  allora
\begin{align*}
\sin\alpha=&\frac{2t}{1+t^2}\\
\cos\alpha=&\frac{1-t^2}{1+t^2}\\
t=&\tan\frac{\alpha}{2}&\alpha\neq\pi+2k\pi
\end{align*}\index{Seno}\index{Coseno}\index{Tangente}\index{Formule!parametriche}
\end{thm}
\begin{proof} Dal~\vref{thm:Formulediduplicazioneseno} somma di due angoli otteniamo le seguenti relazioni
	\begin{align*}
	\intertext{Prima relazione}
	\sin2\alpha=&2\sin\alpha\cos\alpha\\
	=&\frac{2\sin\alpha\cos\alpha}{1}\\
	=&\frac{2\sin\alpha\cos\alpha}{\cos^2\alpha+\sin^2\alpha}\\
	=&\dfrac{\dfrac{2\sin\alpha\cos\alpha}{\cos^2\alpha}}{\dfrac{\cos^2\alpha+\sin^2\alpha}{cos^2\alpha}}&\alpha\neq\frac{\pi}{2}+k\pi\\
	=&\dfrac{\dfrac{2\sin\alpha}{\cos\alpha}}{\dfrac{\cos^2\alpha}{\cos^2\alpha}+\dfrac{\sin^2\alpha}{cos^2\alpha}}\\
	=&\dfrac{2\tan\alpha}{1+\tan^2\alpha}\\
	\alpha&\longmapsto\dfrac{\alpha}{2}\\
	\sin\alpha=&\dfrac{2\tan\dfrac{\alpha}{2}}{1+\tan^2\dfrac{\alpha}{2}}\\
	t=&\tan\frac{\alpha}{2}\\
	\sin\alpha=&\frac{2t}{1+t^2}\\
	\end{align*}
	Dal~\vref{thm:Formulediduplicazionecoseno} otteniamo:
	\begin{align*}
	\intertext{Seconda relazione}
	\cos 2\alpha=&\cos^2\alpha-\sin^2\alpha\\
	=&\dfrac{\cos^2\alpha-\sin^2\alpha}{1}\\
	=&\frac{\cos^2\alpha-\sin^2\alpha}{\cos^2\alpha+\sin^2\alpha}\\
	=&\frac{\dfrac{\cos^2\alpha-\sin^2\alpha}{\cos^2\alpha}}{\dfrac{\cos^2\alpha+\sin^2\alpha}{\cos^2\alpha}}&\alpha\neq\frac{\pi}{2}+k\pi\\ 
	=&\frac{1-\dfrac{\sin^2\alpha}{\cos^2\alpha}}{1+\dfrac{\sin^2\alpha}{\cos^2\alpha}}\\
	=&\frac{1-\tan^2\alpha}{1+\tan^2\alpha}\\
	\alpha&\longmapsto\dfrac{\alpha}{2}\\
	\cos\alpha=&\frac{1-\tan^2\dfrac{\alpha}{2}}{1+\tan^2\dfrac{\alpha}{2}}\\
	t=&\tan\frac{\alpha}{2}\\
	\cos\alpha=&\frac{1-t^2}{1+t^2}
	\end{align*}
	Da cui la tesi.
\end{proof}