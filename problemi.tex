\chapter{Problemi}
\section{Somme}
\begin{prob}[Somme di potenze]
Se $a+b+c=1$ $a^2+b^2+c^2=2$ e $a^3+b^3+c^3=3$ allora quanto vale $a^4+b^4+c^4$?
\end{prob}~\cite{Gregorio2021}
\begin{proof}
consideriamo la relazione \[a^4+b^4+c^4=(a^2+b^2+c^2)^2-2(ab+bc+ca)^2+4abc(a+b+c)\] abbiamo già dimostrato l'~\vref{eq:ascomp15}
Nella relazione bisogna determinare il valore di $abc$ e di $ab+bc+ca$.
Dall'~\vref{eq:ascomp13}
\end{proof}