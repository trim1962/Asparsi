% !TeX root = Asparsi.tex
% !BIB TS-program = biber
% !TeX encoding = UTF-8
% !TeX spellcheck = it_IT

\documentclass[a4paper,oneside]{book}%

\usepackage{cmap}
\usepackage[big]{layaureo}
\usepackage{copyright}
\frenchspacing%
\usepackage{amsmath}

\usepackage{amssymb}
\usepackage[italian]{babel}
\usepackage[thmmarks,hyperref]{ntheorem}
\usepackage{miamatematica}

\usepackage{lmodern} % load vector font
\usepackage[T1]{fontenc} % font encoding
\usepackage[utf8]{inputenc} % input encoding
%\usepackage{noto}
\usepackage[babel=true]{microtype}
%\usepackage{geometry}
\usepackage{textcomp}

%\geometry{top=1.5cm,bottom=1.5cm}
\usepackage{grafica}

%Teorema
\theoremstyle{marginbreak}
\theoremheaderfont{\normalfont\bfseries}\theorembodyfont{\slshape}
\theoremsymbol{\ensuremath{\diamondsuit}}
\theoremseparator{:} %
\newtheorem{thm}{Teorema}[section]
%Proprietà
\theoremstyle{marginbreak}
\theoremheaderfont{\normalfont\bfseries}\theorembodyfont{\slshape}
\theoremsymbol{\ensuremath{\diamondsuit}}
\theoremseparator{:}
\newtheorem{prop}{Proprietà}[section]
%lemma
\theoremstyle{changebreak}
\theoremsymbol{\ensuremath{\heartsuit}}
\theoremindent0.5cm
\theoremnumbering{greek}
\newtheorem{lem}[thm]{Lemma}
%corollario
\theoremindent0cm
\theoremsymbol{\ensuremath{\spadesuit}}
\theoremnumbering{arabic}
\newtheorem{cor}[thm]{Corollario}
%esempio
\theoremstyle{change}
\theorembodyfont{\upshape}
\theoremsymbol{\ensuremath{\ast}}
\theoremseparator{}
\newtheorem{exmp}{Esempio}[section]
%controesempio
\theoremstyle{change}
\theorembodyfont{\upshape}
\theoremsymbol{\ensuremath{\odot}}
\theoremseparator{}
\newtheorem{cexmp}{Contro esempio}[section]
%definizione
\theoremstyle{plain}
\theoremsymbol{\ensuremath{\clubsuit}}
\theoremseparator{.}
\theoremstyle{marginbreak}
\theoremprework{\hrule\bigskip}
\theorempostwork{\hrule\bigskip}
\newtheorem{defn}{Definizione}[section]
%commento
\theoremstyle{plain}
\theorembodyfont{\upshape}
\theoremsymbol{\ensuremath{\blacklozenge}}
\theoremseparator{:}
\newtheorem{commento}{Commento}
%dimostrazione

\theoremstyle{plain}
\theoremheaderfont{\sc}
\theorembodyfont{\bfseries}
\theoremstyle{nonumberplain}
%^{}\theoremseparator{.}

\theoremsymbol{\ensuremath{\blacksquare}}
\theoremheaderfont{\bfseries}
%\theoremstyle{nonumberplain}
%\theoremstyle{marginbreak}
\theorembodyfont{\normalfont}
\newtheorem{proof}{Dimostrazione}
%\input{../Mod_base/tabelle}
\newtheorem{prob}{Problema}[section]
%Proprietà
\theoremstyle{marginbreak}
\theoremheaderfont{\normalfont\bfseries}\theorembodyfont{\slshape}
\theoremsymbol{\ensuremath{\diamondsuit}}
\theoremseparator{:}
\newcommand{\difcos}[2]{\cos\ang{#1}\cos\ang{#2}+\sin\ang{#1}\sin\ang{#2}}
\newcommand{\difsin}[2]{\sin\ang{#1}\cos\ang{#2}-\cos\ang{#1}\sin\ang{#2}}
\newcommand{\sumcos}[2]{\cos\ang{#1}\cos\ang{#2}-\sin\ang{#1}\sin\ang{#2}}
\newcommand{\sumsin}[2]{\sin\ang{#1}\cos\ang{#2}+\cos\ang{#1}\sin\ang{#2}}
\newcommand{\cossessanta}{\dfrac{1}{2}}
\newcommand{\cosquarantacinque}{\dfrac{\sqrt{2}}{2}}
\newcommand{\sinquarantecinque}{\dfrac{\sqrt{2}}{2}}
\newcommand{\sinsessanta}{\dfrac{\sqrt{3}}{2}}
\newcommand{\costrenta}{\dfrac{\sqrt{3}}{2}}
\newcommand{\sintrenta}{\dfrac{1}{2}}
\DeclareMathOperator{\Fi}{F}
\newcommand{\Fib}[1]{\Fi_{#1}}
\usepackage{pagina}

\setlength{\headheight}{13pt}
\usepackage{indice}
\usepackage{date}
\usepackage{unita_misura}

\usepackage{imakeidx}
\makeindex[options=-s ../Mod_base/oldclaudio.sti]

\usepackage{diagbox}
%\include{simboli_operatori}

\usepackage{stand_class}

\newcommand{\HRule}{\rule{\linewidth}{0.5mm}}


 \makeatletter
 \renewcommand\frontmatter{%
 	\cleardoublepage
 	\@mainmatterfalse
 	%\pagenumbering{roman}
 }
 \renewcommand\mainmatter{%
 	\cleardoublepage
 	\@mainmattertrue
 	%\pagenumbering{arabic}
 }
 \makeatother

\usepackage[grumpy,mark,markifdirty,raisemark=0.95\paperheight]{gitinfo2}
% 10/02/2018 :: 20:05:58 :: \usepackage{parskip}
\usepackage[toc,page]{appendix}

\renewcommand{\appendixtocname}{Appendici}

\renewcommand{\appendixpagename}{Appendici}


\usepackage[style=italian]{csquotes}
\usepackage[%
style=philosophy-modern,
annotation=true,
hyperref,
backend=biber,
backref]{biblatex}
\addbibresource{formulario.bib}
\usepackage[italian]{varioref}
\usepackage{hyperxmp}
\usepackage[pdfpagelabels]{hyperref}
\usepackage[italian,noabbrev]{cleveref}
\crefname{defn}{definizione}{definizioni}
\Crefname{defn}{Definizione}{Definizioni}
\crefname{thm}{teorema}{teoremi}
\Crefname{thm}{Teorema}{Teoremi}
\crefname{cor}{corollario}{corollari}
\Crefname{cor}{Corollario}{Corollari}
\crefname{equation}{equazione}{equazioni}
\Crefname{equation}{Equazione}{Equazioni}
\creflabelformat{equation}{#2\textup{#1}#3}
\crefname{sistema}{sistema}{sistemi}
\Crefname{sistema}{Sistema}{Sistemi}
\crefname{lem}{lemma}{lemmi}
\Crefname{lem}{Lemma}{Lemmi}
\crefname{prob}{problema}{problemi}
\Crefname{prob}{Problema}{Problemi}


%\usepackage{tcolorboxgest}
\title{Zibaldone di pensieri}
\author{Claudio Duchi}
\date{\datetime}
\hypersetup{%
pdfencoding=auto,
urlcolor={blue},
pdftitle={Appunti sparsi},
pdfsubject={Per non dimenticare},
pdfstartview={FitH},
pdfpagemode={UseOutlines},
pdflicenseurl={http://creativecommons.org/licenses/by-nc-nd/3.0/},
pdflang={it},
pdfmetalang={it},
pdfkeywords={Algebra, geometria, analisi},
pdfcopyright={Copyright (C) 2021, Claudio Duchi},
pdfcontacturl={http://breviariomatematico.altervista.org},
pdfcontactpostcode={06128},
pdfcontactphone={},
pdfcontactemail={claduc},
pdfcontactcountry={Italy},
pdfcontactcity={Perugia},
pdfcontactaddress={},
pdfcaptionwriter={Claudio Duchi},
pdfauthortitle={},%
pdfauthor={Claudio Duchi},
linkcolor={blue},
colorlinks=true,
citecolor={red},
breaklinks,
bookmarksopen,
verbose,
baseurl={http://breviariomatematico.altervista.org}
}

% !TeX root = Asparsi.tex
% !BIB TS-program = biber
% !TeX encoding = UTF-8
% !TeX spellcheck = it_IT
\includeonly{%
simmetrie,
equasecondo,
scompsecgrado,
numdecperiodici,
numirrazionali,
problemi,
fibonacci,
fibonaccigeneralizzato,
fibonaccirettangoli,
Elenco_fibonacci,
sislineari,
brafibo,
Triangolo,
poligoni,
Progressioni,
NumComplFormGonio,
%parabola,
contro,
geometriaanalitica,
coniche,
%ParabolaFormSdopp,
goniometria,
trigonometria,
radicalidoppi,
statistica,
pitagora,
scomposizioni,
Angolivalorinumerici
}


%patch allieamento lista teoremi
\usepackage{regexpatch}
\makeatletter
%\xpatchcmd*{\thm@@thmline}{2.3em}{5em}{}{} % not really needed
\xpatchcmd*{\thm@@thmline@name}{2.3em}{5em}{}{} 
\xpatchcmd*{\thm@@thmline@noname}{2.3em}{5em}{}{}
\makeatother
%fine patch allieamento lista teoremi
\usepackage{CDloghi}
\listfiles
\begin{document}
\begin{titlepage}
\begin{center}	
	\Lgrandedue\\[1cm]    
	\textsc{\LARGE Claudio Duchi}\\[1.4cm]
	\HRule \\[0.4cm]
{ \huge \bfseries Zibaldone}\\[0.4cm]
{ \large \bfseries di}\\[0.4cm]
{ \huge \bfseries Pensieri}\\[0.4cm]
\HRule \\
\vfill
	% Bottom of the page
	\polylogo[5.5]{6}		
		{\large $-$\DTMnow$-$}	
\end{center}
{\centering
Release:\gitReln\ (\gitAbbrevHash)\ Autore:\gitAuthorName\ 
\gitCommitterDate \\
}
\end{titlepage}	
	\hypersetup{pageanchor=true}
		\CDcopyright
		\tableofcontents
		\chapter*{Lista dei teoremi}
		\theoremlisttype{allname}
		\listtheorems{thm,defn,cor,comm,lem,prob}
	\addcontentsline{toc}{chapter}{\listfigurename}%
		\listoffigures
%	\addcontentsline{toc}{chapter}{\listtablename}%
%			\listoftables
			\mainmatter

% !TeX root = Asparsi.tex
% !BIB TS-program = biber
% !TeX encoding = UTF-8
% !TeX spellcheck = it_IT

% !TeX root = Asparsi.tex
% !BIB TS-program = biber
% !TeX encoding = UTF-8
% !TeX spellcheck = it_IT
\chapter{Simmetrie}
\section{Simmetria assiale}
\begin{defn}[Simmetria assiale]\label{defn:Sassiale1}
Abbiamo una simmetria assiale\index{Simmetria!assiale} di asse $r$ quando, presi due punti $P$ e $Q$:
\begin{enumerate}
	\item Il punto medio $M$ del segmento $PQ\in r$
	\item Il segmento $PQ$ è perpendicolare ad $r$
\end{enumerate} 
\end{defn}
\begin{thm}[Simmetria assiale]\label{thm:Sassiale1}
In una simmetria assiale di asse $r:y=mx+q$ la trasformazione che lega i punti della simmetria ha equazione:
\[\begin{cases}
x_0=\frac{2m(y_1-q)+(1-m^2)x_1}{1+m^2}\\
\\
y_0=\frac{2(mx_1+q)+(m^2-1)y_1}{1+m^2}
\end{cases}\]
\end{thm}
\begin{proof}
	Supponiamo di avere due punti $P(x_1,y_1)$ e $Q(x_0,y_0)$ con $x_1\neq x_0$
	
	 Dalla~\cref{defn:Sassiale1} se $P$ e $Q$ sono simmetrici rispetto alla retta $r$ 
	  \begin{equation}
	 y=mx+q
	 \end{equation}\label[equation]{eq:ass1}
	 allora il punto \[M\left(\frac{x_1+x_0}{2},\frac{y_1+y_0}{2}\right)\] appartiene a $r$ quindi avremo 
	 \begin{equation}
	 \dfrac{y_1+y_0}{2}=m\dfrac{x_1+x_0}{2}+q
	 \end{equation}\label[equation]{eq:ass2}
	 
	 Dalla~\cref{defn:Sassiale1} la retta $r$ è perpendicolare a $PQ$. Quindi se $m$ è il coefficiente angolare della retta di~\cref{eq:ass1} avremo:\begin{equation}
	 \dfrac{y_1-y_0}{x_1-x_0}=-\dfrac{1}{m}
	 \end{equation}\label[equation]{eq:ass3}
	 Utilizzando \crefrange{eq:ass2}{eq:ass3} otteniamo il sistema
	 \begin{equation}
	 \begin{cases}
	 \dfrac{y_1+y_0}{2}=m\dfrac{x_1+x_0}{2}+q\\[0.4cm]
	 \dfrac{y_1-y_0}{x_1-x_0}=-\dfrac{1}{m}
	 \end{cases}
	 \end{equation}\label[equation]{eq:ass4}
	 che diventa
	  \begin{align*}
	 & \begin{cases}
	 y_1+y_0=mx_1+mx_0+2q\\
	 y_1-y_0=-\dfrac{1}{m}x_1+\dfrac{1}{m}x_0
	 \end{cases}\\
%	  & \begin{cases}
%	 y_0=mx_1+mx_0+2q-y_1\\
%	 y_1-mx_1-mx_0-2q+y_1=-\dfrac{1}{m}x_1+\dfrac{1}{m}x_0
%	 \end{cases}\\
	 & \begin{cases}
	 y_0=mx_1+mx_0+2q-y_1\\
	 \dfrac{1}{m}x_0+mx_0= y_1-mx_1-2q+y_1+\dfrac{1}{m}x_1
	 \end{cases}\\
	 & \begin{cases}
	 y_0=mx_1+mx_0+2q-y_1\\
	 x_0+m^2x_0= my_1-m^2x_1-2mq+my_1+x_1
	 \end{cases}\\
%	 & \begin{cases}
%	 y_0=mx_1+mx_0+2q-y_1\\
%	 x_0=\dfrac{2my_1-m^2x_1-2mq+x_1}{1+m^2}
%	 \end{cases}\\
	 & \begin{cases}
	 y_0=mx_1+\dfrac{2m^2y_1-m^3x_1-2m^2q+mx_1}{1+m^2}+2q-y_1\\
	 x_0=\dfrac{2my_1-m^2x_1-2mq+x_1}{1+m^2}
	 \end{cases}\\
%	  & \begin{cases}
%	 y_0=\dfrac{(1+m^2)mx_1+2m^2y_1-m^3x_1-2m^2q+mx_1+2(1+m^2)q-(1+m^2)y_1}{1+m?"}\\
%	 x_0=\dfrac{2my_1-m^2x_1-2mq+x_1}{1+m^2}
%	 \end{cases}\\
	 & \begin{cases}
	 	y_0=\dfrac{mx_1+m^3x_1+2m^2y_1-m^3x_1-2m^2q+mx_1+2q+2m^2q-y_1-m^2y_1}{1+m?"}\\
	 	x_0=\dfrac{2my_1-m^2x_1-2mq+x_1}{1+m^2}
	 \end{cases}\\
	 \intertext{quindi}
	 &\begin{cases}
	 x_0=\frac{2m(y_1-q)+(1-m^2)x_1}{1+m^2}\\[0.4cm]
	 y_0=\frac{2(mx_1+q)+(m^2-1)y_1}{1+m^2}
	 \end{cases}
 \end{align*}
 \end{proof}
Discutendo il risultato del~\cref{thm:Sassiale1} otteniamo i seguenti corollari:
\begin{cor}[Bisettrice primo quadrante]
	Per $m=1$ il sistema diviene 
	\begin{equation}
	\begin{cases}
	x_0=y_1-q\\
	y_1=x_1+q
	\end{cases}
	\end{equation}
\end{cor}
\begin{cor}[Bisettrice secondo quadrante]
	Per $m=-1$ il sistema diviene 
	\begin{equation}
	\begin{cases}
	x_0=-y_1-q\\
	y_1=-x_1+q
	\end{cases}
	\end{equation}
\end{cor}
\begin{cor}[Asse x]
	Per $m=0$ il sistema diviene 
	\begin{equation}
	\begin{cases}
	x_0=x_1\\
	y_1=2q-y_1
	\end{cases}
	\end{equation}
\end{cor}
% !TeX root = Asparsi.tex
% !BIB TS-program = biber
% !TeX encoding = UTF-8
% !TeX spellcheck = it_IT
\chapter{Equazioni secondo grado}\label{ch:equazioni-secondo-grado}
\section{Risoluzione}\label{sec:risoluzione}
\begin{thm}[Formula risolutiva]\label{thm:Equasec1}
	Una equazione di secondo grado $ax^2+bx+c=0$ con $a\neq 0$ ha soluzione \begin{equation*}\label[equation]{eq:equasec1}
	x_{1,2}=\dfrac{-b\pm\sqrt{b^2-4ac}}{2a}
	\end{equation*}\index{Equazione!secondo grado!soluzioni}\index{Triniomio!secondo!grado}
\end{thm}
\begin{proof}
	\begin{align*}
	ax^2+bx+c=&0
	\intertext{$a\neq 0$}
	ax^2+bx+\frac{b^2}{4a}+c=&\frac{b^2}{4a}\\
	ax^2+bx+\frac{b^2}{4a}=&\frac{b^2}{4a}-c\\
	\frac{4a^2x^2+4abx+b^2}{4a}=&\frac{b^2-4ac}{4a}\\
	\left(2ax+b\right)^2=&b^2-4ac
	\intertext{Otteniamo un'equazione di primo grado che risolviamo separando le variabili}
	2ax+b=&\pm\sqrt{b^2-4ac}\\
	2ax=&-b\pm\sqrt{b^2-4ac}\\
	x_{1,2}=&\frac{-b\pm\sqrt{b^2-4ac}}{2a}
	\end{align*}
	Da cui la tesi.
\end{proof}
\section{Casi particolari}
\begin{thm}[Formula ridotta]\label{thm:Equasec2}
	Se l'equazione di secondo grado è del tipo\[ax^2+2\beta x+c=0 \] allora la formula risolutiva è \begin{equation*}\label[equation]{eq:equasec2}
	x_{1,2}=\dfrac{-\beta\pm\sqrt{\beta^2- ac}}{a}
	\end{equation*}
\end{thm}
\begin{proof}
Consideriamo l'equazione $ax^2+2\beta x+c=0$, dal~\cref{thm:Equasec1} otteniamo:
\begin{align*}
x_{1,2}=&\dfrac{-2\beta\pm\sqrt{4\beta^2-4ac}}{2a}\\
=&\dfrac{-2\beta\pm\sqrt{4(\beta^2-ac)}}{2a}\\
=&\dfrac{-2\beta\pm 2\sqrt{(\beta^2-ac)}}{2a}\\
=&\dfrac{-\beta\pm \sqrt{(\beta^2-ac)}}{a}
\end{align*}
Come si voleva dimostrare.
\end{proof}
\begin{thm}[somma di due numeri]\label{thm:Equasec3}
	\[px^2+(p+q)x+q=0 \] allora l'equazione ha soluzione \[\begin{cases}
	x_1=-1\\ x_2=-\dfrac{q}{p}
	\end{cases}\]
\end{thm}
\begin{proof}
	Consideriamo l'equazione $px^2+(p+q) x+q=0$, dal~\cref{thm:Equasec1} otteniamo:
	\begin{align*}
	x_{1,2}=&\dfrac{-(p+q)\pm\sqrt{(p+q)^2-4pq}}{2p}\\
	=&\dfrac{-(p+q)\pm\sqrt{p^2+q^2-2pq}}{2p}\\
	=&\dfrac{-(p+q)\pm\sqrt{(p-q)^2}}{2p}\\
	=&\dfrac{-(p+q)\pm(p-q)}{2p}\\
	&\begin{cases}
	x_1=\dfrac{-p-q+p-q}{2p}=-\dfrac{2q}{2p}=-\dfrac{q}{p}\\ \\
	x_2=\dfrac{-p-q-p+q}{2p}=-\dfrac{2p}{2p}=-1\\
	\end{cases}\\
	\end{align*}
	Come si voleva dimostrare.
\end{proof}
% !TeX root = Asparsi.tex
% !BIB TS-program = biber
% !TeX encoding = UTF-8
% !TeX spellcheck = it_IT
\chapter{Scomposizioni trinomio}
\begin{thm}\label{thm:scompsecgrad1}
Data 	una equazione di secondo grado $ax^2+bx+c=0$ con $a\neq 0$ avremo:\begin{equation}\label[equation]{eq:scompsecgrad1}
ax^2+bx+c=a\left[\left(\dfrac{2ax+b}{2a}\right)^2-\dfrac{\Delta}{4a^2}\right]\qquad\Delta=b^2-4ac
\end{equation}
\end{thm}
\begin{proof}
	\begin{align*}
	ax^2+bx+c=&0\\
	\intertext{raccolgo $a$}
	=&a\left(x^2+\frac{b}{a}x+\frac{c}{a}\right)\\
	=&a\left(x^2+\frac{b}{a}x+\frac{b^2}{4a^2}+\frac{c}{a}-\frac{b^2}{4a^2}\right)\\
	=&a\left[\left(\frac{4a^2x^2+4abx+b^2}{4a^2}\right)+\left(\frac{4ac-b^2}{4a^2}\right)\right]\\
	=&a\left[\left(\frac{2ax+b}{2a}\right)^2-\left(\frac{b^2-4ac}{4a^2}\right)\right]\\
	\intertext{$\Delta=b^2-4ac$}
	=&a\left[\left(\frac{2ax+b}{2a}\right)^2-\left(\frac{\Delta}{4a^2}\right)\right]
	\end{align*}
\end{proof}
% !TeX root = Asparsi.tex
% !BIB TS-program = biber
% !TeX encoding = UTF-8
% !TeX spellcheck = it_IT
\chapter{Sistemi lineari}\label{ch:sistemi-lineari}
\section{Metodo di Cramer}\label{sec:metodo-di-cramer}\index{Sistemi!lineari!Cramer}\index{Cramer}
\begin{thm}[Metodo di Cramer]\label[thm]{thm:teoCramer1a}
	Dato un sistema lineare del tipo
\begin{equation}
\begin{cases}
a_1x+b_1y=c_1\\
a_2x+b_2y=c_2
\end{cases}\quad a_1\neq 0\;b_1\neq 0\quad a_2\neq 0\;b_2\neq 0
\end{equation}\label[sistema]{sistema:teoCramer1}
allora 
\begin{align*}
x=&\frac{c_{1} b_{2} - c_{2} b_{1}}{a_{1} b_{2} - a_{2} b_{1}}=\frac{\begin{vmatrix}
	c_{1} & b_{1}  \\
	c_{2} & b_{2}
	\end{vmatrix} }{\begin{vmatrix}
	a_{1} & b_{1}  \\
	a_{2} & b_{2}  \\
	\end{vmatrix}}\\
y=&\frac{c_{2} a_{1} - c_{1} a_{2}}{a_{1} b_{2} - a_{2} b_{1}}=\frac{\begin{vmatrix}
	a_{1} & c_{1}  \\
	a_{2} & c_{2}
	\end{vmatrix} }{\begin{vmatrix}
	a_{1} & b_{1}  \\
	a_{2} & b_{2}  \\
	\end{vmatrix}}\\
&&a_{1} b_{2} - a_{2} b_{1}\neq& 0\\
\end{align*}
\end{thm}
\begin{proof}
	Consideriamo il~\cref{sistema:teoCramer1} moltiplicando la prima riga per $b_2$ e la seconda riga per $b_1$ otteniamo:
\begin{equation}\label[sistema]{si:teoCramer2}
\begin{cases}
a_1b_2x+b_1b_2y=c_1b_2\\
a_2b_1x+b_2b_1y=c_2b_1
\end{cases}
\end{equation}
sottraendo otteniamo
\begin{align*}
a_1b_2x-a_2b_1x+0=&b_2c_1-b_1c_2\\
x=&\frac{b_2c_1-b_1c_2}{a_1b_2-a_2b_1}
\end{align*}
Ripartiamo dal~\cref{sistema:teoCramer1} moltiplicando la prima riga per $a_2$ e la seconda riga per $a_1$ otteniamo:
\begin{equation}\label[sistema]{si:teoCramer3}
\begin{cases}
a_1a_2x+b_1a_2y=c_1a_2\\
a_2a_1x+b_2a_1y=c_2a_1
\end{cases}
\end{equation}
sottraendo otteniamo
\begin{align*}
0+a_1b_2y-a_2b_1y=&a_2c_1-a_1c_2\\
y=&\frac{c_2a_1-c_1a_2}{a_1b_2-a_2b_1}
\end{align*}
\end{proof}
\section{Interpretazione geometrica}
\begin{thm}[Posizioni di due rette]
	Date due rette~\cite{Zwirner1988}c $a_1x+b_1y=c_1$ e $a_2x+b_2y=c_2$ o in forma esplicita $y=m_1x+q_1$ e $y=m_2x+q_2$ allora le due rette sono:
\begin{align*}
&\text{Incidenti}&&\dfrac{a_1}{a_2}\neq\dfrac{b_1}{b_2}&m_1\neq& m_2\\
&\text{Parallele e distinte}&&\dfrac{a_1}{a_2}=\dfrac{b_1}{b_2}\neq\dfrac{c_1}{c_2}&m_1=& m_2\quad q_1\neq q_2\\
&\text{Coincidenti}&&\dfrac{a_1}{a_2}=\dfrac{b_1}{b_2}=\dfrac{c_1}{c_2}&m_1=& m_2\quad q_1= q_2\\
\end{align*}
\end{thm}\index{Retta!parallela distinta}\index{Retta!incidente}\index{Retta!coincidenti}
\begin{proof}
Se $\dfrac{a_1}{a_2}\neq\dfrac{b_1}{b_2}$ allora $a_1b_2-a_2b_1\neq 0$ quindi per il~\cref{thm:teoCramer1a}, il~\cref{sistema:teoCramer1} ha soluzione per cui le due rette hanno un punto in comune.

Se $\dfrac{a_1}{a_2}=\dfrac{b_1}{b_2}\neq\dfrac{c_1}{c_2}$ allora $a_1b_2-a_2b_1= 0$ quindi per il~\cref{thm:teoCramer1a}, il~\cref{sistema:teoCramer1} non ha soluzione per cui le due rette non hanno punti in comune. Inoltre $\dfrac{a_1}{a_2}=\dfrac{b_1}{b_2}$ implica che $a_1=ka_2$ e $b_1=kb_2$. Di conseguenza le due rette diventano $ka_2x+kb_2y=c_1$ e $a_2x+b_2y=c_2$ cioè riscrivendole otteniamo $a_2x+b_2y=\dfrac{c_1}{k}$ $a_2x+b_2y=c_2$ ma per ipotesi $c_2\neq\dfrac{c_1}{k}$, quindi le due rette non coincidono.

Se $\dfrac{a_1}{a_2}=\dfrac{b_1}{b_2}=\dfrac{c_1}{c_2}$ allora il sistema non ha soluzione e $c_1=kc_2$ quindi le  due rette coincidono.

Se le due rette sono in forma esplicita allora 
\begin{align*}
m_1=&\dfrac{a_1}{b_1}\\
m_2=&\dfrac{a_2}{b_2}\\
q_1=&\dfrac{c_1}{b_1}\\
q_2=&\dfrac{c_2}{b_2}
\end{align*} 
Quindi il teorema.
\end{proof}
% !TeX root = Asparsi.tex
% !BIB TS-program = biber
% !TeX encoding = UTF-8
% !TeX spellcheck = it_IT
\chapter{Numeri}\label{ch:numeri-primi}
\section{I numeri primi sono infiniti}\label{sec:i-numeri-primi}
\begin{thm}[Numeri primi]\label{thm:numeriprimiinfiniti1}
I numeri primi sono infiniti
\end{thm}\index{Numero!primo!infiniti}
\chapter{Numeri pari}
\section{Numeri pari per differenza}\label{sec:quadratodifferenza}
\begin{thm}[Quadrato meno differenza]\label{thm:quadratodifferenza}
	Se $x$ è un numero naturale allora $x^2-x$ è pari. 
\end{thm}\index{Numero!pari}
\begin{proof}
		\begin{align*}
		\intertext{Se $x$ è pari allora:}
		x=&2n\\
		x^2-x=&4n^2-2n\\
		=&2n(2n-1)\\
		\intertext{pari}
		\intertext{Se $x$ è dispari allora:}
		x=&2n+1\\
		x^2-x=&4n^2-4n+1-(2n+1)\\
		=&4n^2-4n+1-2n-1\\
		=&4n^2-4n-2n\\
		=&4n^2-6n\\
		=&2n(2n-3)
		\intertext{pari}
	\end{align*}
come si voleva dimostrare
\end{proof}
\chapter{Numeri decimali periodici}\label{ch:numeri-decimali-periodici}
\section{Numero decimale periodico puro}\label{sec:numerodecimaleperiodico-puro}
\begin{thm}[Numero decimale periodico puro]
	La frazione generatrice di un numero decimale periodico puro ha per numeratore la differenza fra in numero formato dalla parte intere e il periodo e il numero formato dalla parte intera. Per denominatore un numero formato da tanti nove per quanto è lungo il periodo.  
\end{thm}\index{Numero!decimale!infinito}
\begin{proof}
	Sia $x=a_1\dots a_m,\overline{c_1\dots c_n}$ dove $a_1\dots a_m$ è la parte intera\index{Numero!parte intera} e $c_1\dots c_n$ il periodo\index{Numero!periodo}. 
	\begin{align*}
	10^nx=&a_1\dots a_m c_1\dots c_n,\overline{c_1\dots c_n}\\
	10^n1x-x=&a_1\dots a_m b_1\dots b_p c_1\dots c_n-a_1\dots a_m b_1\dots b_p\\
	(10^n-1)x=&a_1\dots a_m c_1\dots c_n-a_1\dots a_m\\
	x=&\frac{a_1\dots a_m c_1\dots c_n-a_1\dots a_m}{10^n-1}
	\end{align*}
\end{proof}
\section{Numero decimale periodico misto}\label{sec:numero-decimale-periodico-misto}
\begin{thm}[Numero decimale periodico misto]\label{thm:Numero-decimale-periodico-misto}
La frazione generatrice di un numero decimale periodico misto ha per numeratore la differenza fra in numero formato dalla parte intere, l'antiperiodo e il periodo e il numero formato dalla parte intera e l'antiperiodo. Per denominatore un numero formato da tanti nove per quanto è lungo il periodo e tanti zero per quanto è lungo l'antiperiodo.  
\end{thm}
\begin{proof}
	Sia $x=a_1\dots a_m,b_1\dots b_p \overline{c_1\dots c_n}$ dove $a_1\dots a_m$ è la parte intera\index{Numero!parte intera}, $b_1\dots b_p$ l'antiperiodo\index{Numero!antiperiodo} e $c_1\dots c_n$ il periodo\index{Numero!periodo}. 
	\begin{align*}
	10^px=&a_1\dots a_m b_1\dots b_p,\overline{c_1\dots c_n}\\
	10^n10^px=&a_1\dots a_m b_1\dots b_p c_1\dots c_n,\overline{c_1\dots c_n}\\
	10^n10^px-10^px=&a_1\dots a_m b_1\dots b_p c_1\dots c_n-a_1\dots a_m b_1\dots b_p\\
	10^px(10^n-1)=&a_1\dots a_m b_1\dots b_p c_1\dots c_n-a_1\dots a_m b_1\dots b_p\\
	x=&\frac{a_1\dots a_m b_1\dots b_p c_1\dots c_n-a_1\dots a_m b_1\dots b_p}{10^p(10^n-1)}
	\end{align*}
\end{proof}
% !TeX root = Asparsi.tex
% !BIB TS-program = biber
% !TeX encoding = UTF-8
% !TeX spellcheck = it_IT
\chapter{Numeri irrazionali}\label{ch:numeri-irrazionali}
\section{Preliminari}\label{sec:preliminari}
\begin{lem}[Quadrato di pari]\label[lem]{lem:numirra1}
	Se il quadrato di un numero è pari allora quale numero è pari\index{Numero!pari}\index{Numero!quadrato}
	\begin{equation}
	a^2=2k \Rightarrow\qquad a=2m\quad a\neq 0
	\end{equation}
\end{lem}
\begin{proof}
P.A. Se $a$ non è pari allora $a$ è dispari quindi $a=2m+1$ segue che 
\begin{align*}
a^2=&(2m+1)^2\\
=&4m^2+4m+1\\
=&4m(m+1)+1
\end{align*}
Abbiamo dimostrato che $a^2$ è dispari, assurdo per ipotesi $a^2$ era pari. Segue che  $a$ è pari.
\end{proof}
\begin{lem}[Quadrato coprimi]\label[lem]{lem:numirra2}
	S $a$ e $b$ sono coprimi allora $a^2$ e $b^2$ sono coprimi
\end{lem}
\section{Radice quadrata di due è irrazionale}\label{sec:radice-quadrata-didue}
\begin{thm}[La radice quadrata di due è irrazionale]
	Radice quadrata di due è irrazionale
\end{thm}
\begin{proof}
	Se radice di due non è irrazionale allora è razionale. Quindi esistono due numeri $a$ e$b$, primi fra loro, tali che:
	\begin{align*}
	\frac{a}{b}=&\sqrt{2}\\
	\frac{a^2}{b^2}=&2\\
	a^2=&2b^2
	\intertext{Quindi per il~\vref{lem:numirra1} $a$ è pari, segue}
	a=&2k\\
	4k^2=&2b^2\\
	2k^2=&b^2\\
	\intertext{Per quanto detto prima anche $b$ è pari ma $a$ e $b$ erano primi fra di loro, assurdo}
	\end{align*}
\end{proof}
\section{Radice quadrata di un numero primo è irrazionale}\label{sec:radice-quadrata-diprimo}
\begin{thm}[La radice quadrata di un numero primo è irrazionale]
	La radice quadrata di un numero primo è irrazionale.
\end{thm}

% !TeX root = Asparsi.tex
% !BIB TS-program = biber
% !TeX encoding = UTF-8
% !TeX spellcheck = it_IT
\chapter{Brahmagupta Fibonacci}\label{ch:brahmagupta-fibonacci}
\section{Identità Brahmagupta Fibonacci}\label{sec:identita-brahmagupta-fibonacci}
\begin{thm}[Brahmagupta Fibonacci]\label{thm:Brahmagupta-Fibonacci}
Un numero che è prodotto di due numeri ciascuno somma di due quadrati  può essere scritto come somma di due quadrati.
	\begin{align*}
	(a^2+b^2)(c^2+d^2)=&\\
	&=(ac-bd)^2+(ad+bc)^2\\
	&=(ac+bd)^2+(ad-bc)^2\\
	\end{align*}
\end{thm}
\begin{proof}
\begin{align*}
	(a^2+b^2)(c^2+d^2)=&\\
	=&a^2c^2+a^2b^2+b^2c^2+b^2d^2\\
	=&a^2c^2+b^2d^2-2abcd+a^2d^2+b^2c^2+2abcd\\
	=&(ac-bd)^2+(ad+bc)^2\\
	=&(ac+bd)^2+(ad-bc)^2\\
\end{align*}
\end{proof}

% !TeX root = Asparsi.tex
% !BIB TS-program = biber
% !TeX encoding = UTF-8
% !TeX spellcheck = it_IT
\chapter{Triangolo}\label{ch:triangolo}
\begin{thm}[Posizione punto circonferenza]\label{thm:Posizione-punto-circonferenza}
Data una circonferenza $C$ di diametro$AB$, valgono le seguenti affermazioni
\begin{itemize}
	\item Un punto $P$ appartiene a $C$ se e solo se $A\hat{P}B$ è retto.
	\item Un punto $P$ è interno a $C$ se e solo se $A\hat{P}B$ è ottuso.
	\item Un punto $P$ è esterno a $C$ se e solo se $A\hat{P}B$ è acuto
\end{itemize}
\end{thm}\index{Circonferenza}
\begin{proof} Consideriamo i vari casi\newline
	\begin{itemize}
		\item Un punto $P$ appartiene a $C$ se e solo se $A\hat{P}B$ è retto.\newline
		Se il punto è sulla circonferenza come nella~\cref{fig:circonferenza1} l'angolo in $\hat{P}$ è la metà di un angolo piatto quindi è retto. 
		\item Un punto $P$ è interno a $C$ se e solo se $A\hat{P}B$ è ottuso.\newline Consideriamo la~\cref{fig:circonferenza2}. Se il punto $P$ è interno prolunghiamo $AP$ in modo che incontri in $Q$ la circonferenza. Il Triangolo $AQB$ è retto in$Q$. L'angolo $A\hat{P}B$ è somma di $P\hat{Q}B$ e $P\hat{B}Q$, quindi è ottuso.
		\item Un punto $P$ è esterno a $C$  se e solo se $A\hat{P}B$ è acuto.\newline
		Prendiamo un punto $P$ punto esterno alla circonferenza $C$ come nella~\cref{fig:circonferenza3}. Se $AP$ è secante a $C$ in  $Q$ l'angolo $A\hat{Q}B$ è retto ed è la somma di $Q\hat{P}B$ e $Q\hat{B}P$. Quindi $A\hat{P}B$ è acuto.
	\end{itemize}
\end{proof}
\begin{figure}
	\centering
	\includestandalone{geometria/circonferenza1}
	\caption{Posizione punto circonferenza, angolo retto}
	\label{fig:circonferenza1}
\end{figure}
\begin{figure}
	\centering
	\includestandalone{geometria/circonferenza2}
	\caption{Posizione punto circonferenza, angolo ottuso}
	\label{fig:circonferenza2}
\end{figure}
\begin{figure}
	\centering
	\includestandalone{geometria/circonferenza3}
	\caption{Posizione punto circonferenza, angolo acuto}
	\label{fig:circonferenza3}
\end{figure}
\begin{defn}[Ceviana]\label{defn:ceviana1}
Una Ceviana è un segmento che congiunge un vertice di un triangolo con il suo lato opposto.
\end{defn}\index{Triangolo!ceviana}\index{Ceviana}
\section{Teorema di Stewart}\label{sec:teorema-di-stewart}
\begin{thm}[Teorema di Stewart]\label{thm:Stewart}
Dato un triangolo $ABC$ colleghiamo un vertice con il lato opposto come nella figura~\cref{fig:stewart1} allora vale la relazione 
\begin{align*}
c^2y+b^2x=&a(t^2+xy)\\     
a^2y+c^2x=&b(l^2+xy)\\        
b^2y+a^2x=&c(m^2+xy)
\end{align*}
\end{thm}\index{Triangolo!teorema di Stewart}
\begin{figure}
	\centering
	\includestandalone{geometria/Stewart1}
	\caption{Teorema di Stewart}
	\label{fig:stewart1}
\end{figure}
\begin{proof}
Poniamo
\begin{align*}
AB=&c&&AC=b&&BC=a\\
AP=&t&&BP=x&&PC=y
\end{align*}
Indichiamo con $\alpha$ l'angolo $A\hat{P}B$ Applichiamo il teorema di Carnot\index{Triangolo!Carnot} ai triangoli $APB$ e $PCA$ avremo 
\begin{align*}
c^2=&t^2+x^2-2xt\cos\alpha\\
b^2=&t^2+y^2-2ty\cos(\pi-\alpha)\\
c^2=&t^2+x^2-2xt\cos\alpha\\
b^2=&t^2+y^2+2ty\cos\alpha\\
\intertext{Moltiplichiamo la prima per $y$ e la seconda per $x$ ottengo}
c^2y=&t^2y+x^2y-2xyt\cos\alpha\\
b^2x=&t^2x+y^2x+2xyt\cos\alpha\\
\intertext{sommando otteniamo}
c^2y+b^2x=&t^2y+x^2y+t^2x+y^2x\\
c^2y+b^2x=&t^2(x+y)+xy(x+y)\\
c^2y+b^2x=&(x+y)(t^2+xy)\\
c^2y+b^2x=&a(t^2+xy)\\
\end{align*}
come si volevasi dimostrare
\end{proof}\index{Triangolo!Carnot}\index{Triangolo!ceviana}
\begin{cor}[Lunghezza ceviana]\label{cor:ceviana1}
In un trangolo le ceviane hanno lunhezza:
\begin{align*}
l=&\sqrt{\dfrac{a^2y-x(by-c^2)}{b}}\\
m=&\sqrt{\dfrac{b^2y-x(cy-a^2)}{c}}\\
t=&\sqrt{\dfrac{c^2y-x(ay-b^2)}{a}}\\
\end{align*}
\end{cor}
\section{Circocentro}\label{sec:circocentro}
\begin{defn}[Circonferenza circoscritta]\label{defn:CircCirc1}
Dato un triangolo, una circonferenza circoscritta è una circonferenza che passa per i vertici del triangolo\index{Triangolo!circonferenza!circoscritta}\index{Circonferenza!circoscritta}
\end{defn}
\begin{figure}
	\centering
\includestandalone{geometria/circumcerchio}
	\caption{Circonferenza circonscritta}
	\label{fig:circumcerchio}
\end{figure}
\begin{defn}[Circocentro]\label{defn:Circocentro1}
Il circocentro è il centro della circonferenza circonscritta. \index{Triangolo!circocentro}
\end{defn}
\begin{thm}[Assi dei lati e circocentro]\label{thm:CircoAsse1}
Il circocentro è il punto di incontro degli assi del segmento.\index{Triangolo!circocentro}\index{Triangolo!asse!lato}\index{Circocentro!triangolo}
\end{thm}
\begin{proof}
	Consideriamo il triangolo $ABC$ della figura~\cref{fig:circumcerchio2}, per comodità poniamo il vertice $A$ nell'origine degli assi e il vertice $B$ sull'asse delle $x$. Determiniamo la circonferenza che passa per i tre vertici.
	\begin{align*}
	\intertext{Considero la circonferenza generica}
	x^2+y^2+ax+by+c=&0\\
	\intertext{Passaggio per $A(0,0)$ }
	c=&0\\
	\intertext{Passaggio per $B(s,0)$}
	s^2+as+c=0\\
	\intertext{Passaggio per $C(t,r)$}
	t^2+r^2+at+br+c=&0\\
	\intertext{Risolvendo il sistema otteniamo}
	\begin{cases}	
		a=-s\\
		b=-\dfrac{t^2+r^2-ts}{r}\\
		c=0
	\end{cases}&\\
	\intertext{L'equazione cercata è}
	x^2+y^2-sx-\dfrac{t^2+r^2-ts}{r}y=&0
	\intertext{Il centro $D$ ha coordinate:}
	\begin{cases}
	x=\dfrac{s}{2}\\ \\
	y=\dfrac{t^2+r^2-ts}{2r}\\
	\end{cases}&\\
		\end{align*}
	Troviamo l'intersezione degli assi dei lati
	\begin{align*}
	\intertext{L'asse del lato $AB$ ha equazione}
	x=&\dfrac{s}{2}\\
	\intertext{Per l'asse del lato $BC$ procediamo come segue:}
	(x-s)^2+y^2=&(x-t)^2+(x-r)^2\\
	x(2t-2s)+2ry-t^2-r^2+s^2=&0\\
	y=&\dfrac{x(s-t)}{r}+\dfrac{t^{2}+r^{2}-s^{2}}{2r}
	\intertext{Risolvendo il sistema otteniamo}
	&\begin{cases}
	x=\dfrac{s}{2}\\ \\
	y=\dfrac{t^2+r^2-ts}{2r}\\
	\end{cases}
	\end{align*}
	Da cui la tesi.
\end{proof}
\begin{proof}
	Consideriamo il triangolo $ABC$ della figura~\cref{fig:circumcerchio2}, per comodità poniamo il vertice $A$ nell'origine degli assi e il vertice $B$ sull'asse delle $x$. Determiniamo la circonferenza che passa per i tre vertici.
	\begin{align*}
	\intertext{Considero la circonferenza generica}
	x^2+y^2+ax+by+c=&0\\
	\intertext{Passaggio per $A(0,0)$ }
	c=&0\\
	\intertext{Passaggio per $B(s,0)$}
	s^2+as+c=0\\
	\intertext{Passaggio per $C(t,r)$}
	t^2+r^2+at+br+c=&0\\
	\intertext{Risolvendo il sistema con il metodo di Cramer}
\Delta=&\begin{vmatrix}
0&0&1\\
s&0&1\\
t&r&1\\
\end{vmatrix}=rs\\
\Delta_a=&\begin{vmatrix}
0&0&1\\
-s^2&0&1\\
-t^2-r^2&r&1\\
\end{vmatrix}=-rs^2\\
\Delta_b=&\begin{vmatrix}
0&0&1\\
s&-s^2&1\\
t&-t^2-r^2&1\\
\end{vmatrix}=-s(t^2-st+r^2)\\
\Delta_c=&\begin{vmatrix}
0&0&0\\
s&0&-s^2\\
t&r&-t^2-r^2\\
\end{vmatrix}=0\\
a=&\dfrac{\Delta_a}{\Delta}=-\dfrac{rs^2}{rs}=-s\\
b=&\dfrac{\Delta_b}{\Delta}=-\dfrac{s(t^2-st+r^2)}{rs}=-\dfrac{t^2-st+r^2}{r}\\
c=&\dfrac{\Delta_c}{\Delta}=-\dfrac{0}{rs}=0
	\intertext{L'equazione cercata è}
&x^2+y^2-sx-\dfrac{t^2+r^2-ts}{r}y=0\\
\intertext{Il centro $D$ ha coordinate:}
&	\begin{cases}
x=\dfrac{s}{2}\\ \\
y=\dfrac{t^2+r^2-ts}{2r}\\
\end{cases}\\
	\end{align*}
	Troviamo l'intersezione degli assi dei lati
	\begin{align*}
	\intertext{L'asse del lato $AB$ ha equazione}
	x=&\dfrac{s}{2}\\
	\intertext{Per l'asse del lato $BC$ procediamo come segue:}
	(x-s)^2+y^2=&(x-t)^2+(x-r)^2\\
	2x(s-t)-2ry=&s^2-r^2-t^2
	\intertext{Risolvendo il sistema otteniamo}
	\Delta=&\begin{vmatrix}
	2(s-t)&-2r\\
	1&0\\
	\end{vmatrix}=2r\\
	\Delta_x=&\begin{vmatrix}
	s^2-r^2-t^2&-2r\\
	\dfrac{s}{2}&0\\
	\end{vmatrix}=rs\\
	\Delta_y=&\begin{vmatrix}
	2(s-t)&s^2-r^2-t^2\\
	1&0\\
	\end{vmatrix}=t^2+r^2-st\\
	x=&\dfrac{\Delta_x}{\Delta}=\dfrac{rs}{2r}=\dfrac{s}{2}\\
	y=&\dfrac{\Delta_y}{\Delta}=\dfrac{t^2+r^2-st}{2r}\\
	\end{align*}
	Da cui la tesi.
\end{proof}
\begin{figure}
	\centering
	\includestandalone{geometria/circumcerchio2}
	\caption{Circocentro ed assi segmento}
	\label{fig:circumcerchio2}
\end{figure}
\begin{cor}[Triangolo rettangolo]\label{cor:CircoAsse1}
	In un triangolo rettangolo il circocentro è il punto medio dell'ipotenusa.\index{Triangolo!rettangolo!circocentro}\index{Circocentro!triangolo!rettangolo}
\end{cor}
\begin{proof}
	Dal \cref{thm:CircoAsse1} sappiamo che il circocentro ha coordinate
	\[	\begin{cases}
	x=\dfrac{s}{2}\\ \\
	y=\dfrac{t^2+r^2-ts}{2r}\\
	\end{cases}\]
	Se come nella figura~\cref{fig:circumcerchio3} il triangolo è retto in $B$ quindi le coordinate del punto $C$ diventano $C(s,r)$ e di conseguenza quelle del punto $D$ si trasformano in 	\[\begin{cases}
	x=\dfrac{s}{2}\\ \\
	y=\dfrac{s^2+r^2-s^2}{2r}=\dfrac{r}{2}\\
	\end{cases}\]Per verificare che il punto $D$ è sull'ipotenusa\index{Triangolo!rettangolo!ipotenusa} basta trovare l'equazione di questa. Banalmente l'ipotenusa ha equazione:
	\begin{align*}
	y=&\dfrac{r}{s}x\\
	\intertext{sostiuendo le coordinate del centro $D$}
	\dfrac{r}{2}=&\dfrac{r}{2}
	\end{align*}
	Da cui la tesi.
\end{proof}
\begin{figure}
	\centering
	\includestandalone{geometria/circumcerchio3}
	\caption{Circocentro e triangolo rettangolo}
	\label{fig:circumcerchio3}
\end{figure}
\begin{cor}[Triangolo ottusangolo]\label{cor:CircoAsse2}
	In un triangolo ottusangolo il circocentro è esterno al triangolo.\index{Triangolo!ottusangolo!circocentro}\index{Circocentro!triangolo!ottusangolo}
\end{cor}
\begin{figure}
	\centering
	\includestandalone[width=0.9\linewidth]{geometria/incentro1}
	\caption{Incentro di un triangolo}
	\label{fig:incentro1}
\end{figure}
\section{Incentro}\label{sec:incentro}
\begin{defn}[Incentro]\label{defn:incentro1}
L'incentro è il punto di intersezione delle bisettrici di un triangolo
\end{defn}\index{Triangolo!incentro}\index{Incentro!triangolo}\index{Triangolo!bisettrice}
\begin{thm}[Incentro]\label{thm:incentro1}
	Le tre bisettrici si incontrano in un punto detto incentro.
\end{thm}
%\begin{proof}
%	Consideriamo la~\cref{fig:incentro1}. Poniamo $A(0,0)$, $B(s,0)$ e $C(t,r)$
%	\begin{align*}
%	\intertext{La retta $AC$ ha equazione:}
%	xr+y(r-t)=&0
%	\intertext{La retta $AB$ ha equazione:}	
%	y=&0
%	\intertext{La retta $BC$ ha equazione:}
%	xr+y(s-t)-sr=&0
%	\intertext{La bisettrice tra $AC$ e $AB$ è:}
%	\dfrac{xr+y(r-t)}{\sqrt{r^2+(r-t)^2}}=&\pm y
%	\intertext{La bisettrice tra $AB$ e $BC$ è:}
%	\dfrac{xr+y(s-t)-sr}{\sqrt{r^2+(s-t)^2}}=&\pm y
%	\intertext{Poniamo}
%	a=&\sqrt{r^2+(r-t)^2}\\
%	b=&\sqrt{r^2+(s-t)^2}
%	\intertext{otteniamo quattro sistemi}
%	&\begin{cases}
%\dfrac{xr+y(r-t)}{a}=\pm y\\
%\dfrac{xr+y(s-t)-sr}{b}=\pm y
%	\end{cases}\\
%	\intertext{il primo}
%	&\begin{cases}
%	\dfrac{xr+y(r-t)}{a}=y\\
%	\dfrac{xr+y(s-t)-sr}{b}=y
%	\end{cases}\\
%	\intertext{che ha per soluzione}
%	&\begin{cases}
%	x=\dfrac{s(a-r+t)}{a-b-r+s}\\
%	y=\dfrac{rs}{a-b-r+s}
%	\end{cases}\\
%		\intertext{il secondo}
%	&\begin{cases}
%	\dfrac{xr+y(r-t)}{a}=y\\
%	\dfrac{xr+y(s-t)-sr}{b}=-y
%	\end{cases}\\
%	\intertext{che ha per soluzione}
%	&\begin{cases}
%	x=\dfrac{s(a-r+t)}{a+b-r+s}\\
%	y=\dfrac{rs}{a+b-r+s}
%	\end{cases}\\
%			\intertext{il terzo}
%	&\begin{cases}
%	\dfrac{xr+y(r-t)}{a}=-y\\
%	\dfrac{xr+y(s-t)-sr}{b}=y
%	\end{cases}\\
%	\intertext{che ha per soluzione}
%	&\begin{cases}
%	x=\dfrac{s(a+r-t)}{a+b+r-s}\\
%	y=-\dfrac{rs}{a+b+r-s}
%	\end{cases}\\
%		\intertext{il quarto}
%	&\begin{cases}
%	\dfrac{xr+y(r-t)}{a}=-y\\
%	\dfrac{xr+y(s-t)-sr}{b}=-y
%	\end{cases}\\
%	\intertext{che ha per soluzione}
%	&\begin{cases}
%	x=\dfrac{s(a+r-t)}{a-b+r-s}\\
%	y=-\dfrac{rs}{a-b+r-s}
%	\end{cases}\\
%	\intertext{La bisettrice tra $AC$ e $AB$ è:}
%	\dfrac{xr+y(r-t)}{\sqrt{r^2+(r-t)^2}}=&\pm y
%	\intertext{La bisettrice tra $AC$ e $CB$ è:}
%	\dfrac{xr+y(r-t)}{\sqrt{r^2+(r-t)^2}}=&\pm\dfrac{xr+y(s-t)-sr}{\sqrt{r^2+(s-t)^2}}
%	\intertext{Poniamo}
%	a=&\sqrt{r^2+(r-t)^2}\\
%	b=&\sqrt{r^2+(s-t)^2}
%	\intertext{otteniamo quattro sistemi}
%	&\begin{cases}
%	\dfrac{xr+y(r-t)}{a}=\pm y\\
%	\dfrac{xr+y(r-t)}{a}=\pm\dfrac{xr+y(s-t)-sr}{b}
%	\end{cases}\\
%	\intertext{il primo}
%	&\begin{cases}
%	\dfrac{xr+y(r-t)}{a}=y\\
%	\dfrac{xr+y(r-t)}{a}=\dfrac{xr+y(s-t)-sr}{b}
%	\end{cases}\\
%	\intertext{che ha per soluzione}
%	&\begin{cases}
%	x=\dfrac{s(a-r+t)}{a-b-r+s}\\
%	y=\dfrac{rs}{a-b-r+s}
%	\end{cases}\\
%		\intertext{il secondo}
%	&\begin{cases}
%	\dfrac{xr+y(r-t)}{a}=y\\
%	\dfrac{xr+y(r-t)}{a}=-\dfrac{xr+y(s-t)-sr}{b}
%	\end{cases}\\
%	\intertext{che ha per soluzione}
%	&\begin{cases}
%	x=\dfrac{s(a-r+t)}{a+b-r+s}\\
%	y=\dfrac{rs}{a+b-r+s}
%	\end{cases}\\
%	\end{align*}
%\end{proof}
\section{Ortocentro}\label{sec:ortocentro}
\begin{defn}[Ortocentro]\label{defn:ortocentro1}
	L'ortocentro è il punto di intersezione delle altezze di un triangolo \index{Triangolo!ortocentro}\index{Ortocentro!triangolo}
\end{defn}
\begin{thm}[Ortocentro]\label{thm:ortocentro1}
	Le altezze di un triangolo passano tutte per lo stesso punto
\end{thm}
\begin{figure}
	\centering
	\includestandalone{geometria/ortocentro1}
	\caption{Ortocentro triangolo}
	\label{fig:ortocentro1}
\end{figure}
\begin{proof}
Consideriamo un triangolo come nella figura~\cref{fig:ortocentro1}. Poniamo che $A(0,0)$, $B(0,s)$ e $C(t,r)$ otteniamo che  le altezze relative ai lati sono:
\begin{align*}
y=&-\dfrac{t}{r}(x-s)\\
x=&t\\
y=&\dfrac{s-t}{r}x\\
\intertext{mettendo a sistema le prime due otteniamo}
&\begin{cases}
y=-\dfrac{t}{r}(x-s)\\
x=t
\end{cases}\\
\intertext{otteniamo:}
&\begin{cases}
y=\dfrac{t}{r}(s-t)\\
x=t
\end{cases}\\
\intertext{mettendo a sistema le rimanenti otteniamo}
&\begin{cases}
y=\dfrac{s-t}{r}x\\
x=t
\end{cases}\\
\intertext{otteniamo:}
&\begin{cases}
y=\dfrac{t}{r}(s-t)\\
x=t
\end{cases}\\
\end{align*}
\end{proof}
\begin{cor}[Triangolo rettangolo]
Nel triangolo rettangolo l'ortocentro coincide con il vertice dell'angolo retto.
\end{cor}
\begin{proof}
Dal~\cref{thm:ortocentro1} abbiamo 
\begin{align*}
&\begin{cases}
y=\dfrac{t}{r}(s-t)\\
x=t
\end{cases}
\intertext{basta porre $s=t$ per ottenere}
&\begin{cases}
y=0\\
x=s
\end{cases}
\end{align*}
\end{proof}
\section{Baricentro}\label{sec:baricentro}
\begin{defn}[Mediana]\label{defn:mediana1}
	In un triangolo è il segmento che unisce un vertice con il punto medio del lato opposto.\index{Triangolo!mediana}\index{Mediana!triangolo}
\end{defn}
\begin{thm}[Lunghezza mediana]\label{thm:mediana1}
Consideriamo il triangolo in~\cref{fig:mediana1} le mediane hanno lunghezza
\begin{align*}
m_a=&\dfrac{\sqrt{2(b^2+c^2)-a^2}}{2}\\
m_b=&\dfrac{\sqrt{2(a^2+c^2)-b^2}}{2}\\
m_c=&\dfrac{\sqrt{2(a^2+b^2)-c^2}}{2}\\
\end{align*}
\end{thm}
\begin{proof}
Diretta conseguenza del~\cref{cor:ceviana1}
\end{proof}
\begin{figure}
	\centering
	\includestandalone{geometria/mediana1}
	\caption{Lunghezza mediane}
	\label{fig:mediana1}
\end{figure}
 \begin{defn}[Baricentro]\label{defn:baricentro1}
	Il Baricentro è il punto di intersezione delle mediane di un triangolo \index{Triangolo!baricentro}\index{Baricentro!triangolo}
\end{defn}
\begin{thm}[Coordinate baricentro]\label{thm:baricentro1}
	In un triangolo di vertici $A(a,b)$, $B(c,d)$ e $C(e,f)$ il baricentro $M$ ha coordinate\[M\left(\frac{a+c+e}{3},\frac{b+d+f}{3}\right)\]
\end{thm}\index{Triangolo!baricentro!coordinate}\index{Baricentro!coordinate}	
\begin{proof}
Consideriamo il triangolo della~\cref{fig:baricentro1} e determiniamo le mediane
\begin{align*}
\intertext{La mediana $AF$ ha equazione}
y =& x\dfrac{2b-d-f}{2a-c-e}+\dfrac{a(d+f)-b(c+e)}{2a-c-e}\\
\intertext{La mediana $BD$ ha equazione}
y =& x\dfrac{b - 2d + f}{a - 2c + e} +\dfrac{d(a+e)-c(b+f)}{a - 2c + e}\\
\intertext{Per trovare il punto $M$ di intersezione le metto a sistema}
&\begin{cases}
y = x\dfrac{2b-d-f}{2a-c- e}+\dfrac{a(d+f)-b(c+e)}{2a-c-e}\\ \\
y = x\dfrac{b - 2d + f}{a - 2c + e}+\dfrac{d(a+e)-c(b+f)}{a-2c+e}
\end{cases}
\intertext{che risolto mi da}
&\begin{cases}
x=\dfrac{a+c+e}{3}\\ \\
y=\dfrac{b+d+f}{3}
\end{cases}
\end{align*}
Da cui la tesi
\end{proof}
\begin{figure}
	\centering
	\includestandalone{geometria/baricentro1}
	\caption{Baricentro di un triangolo}
	\label{fig:baricentro1}
\end{figure}
\begin{thm}[Teorema di Eulero]\label{thm:eulero1}
In un triangolo Ortocentro, Circocentro e Baricentro sono allineati
\end{thm}\index{Triangolo!Eulero}\index{Triangolo!ortocentro}\index{Triangolo!circocentro}\index{Circocentro!triangolo}\index{Ortocentro!triangolo}\index{Baricentro!triangolo}\index{Triangolo!baricentro}
\begin{proof}
	Consideriamo la retta che passa per il circocentro e per l'ortocentro. Utilizzando il~\cref{thm:CircoAsse1} e il~\cref{thm:baricentro1} otteniamo
	\[y=x\dfrac{[r^2-3t(s-t)]}{r(s-2t)}+\dfrac{t[r^2-(s+t)(s-t)]}{r(s-2t)}\]
Adattando il~\cref{thm:baricentro1} il Baricentro ha coordinate: \[M\left(\frac{s+t}{3},\frac{t}{3}\right)\] che verificano l'equazione.
\end{proof}
\begin{defn}[Retta di Eulero]\label{defn:rettaEulero1}
La retta che passa per Ortocentro, Circocentro e Baricentro  chiamata retta di Eulero\end{defn}\index{Retta!Eulero}
\section{Triangolo equilatero}\label{sec:triangolo-equilatero}
\begin{thm}[Ortocentro, circocentro e baricentro]\label{thm:triangoloequilatero1}
In un triangolo equilatero Ortocentro, Circocentro e Baricentro coincidono.
\end{thm}\index{Triangolo!equilatero}\index{Triangolo!equilatero!ortocentro}\index{Triangolo!equilatero!circocentro}\index{Circocentro!triangolo!equilatero}\index{Ortocentro!triangolo!equilatero}\index{Triangolo!baricentro}\index{Triangolo!equilatero!baricentro}
\begin{figure}
	\centering
	\includestandalone{geometria/triangoloEquilatero1}
	\caption{Triangolo equilatero}
	\label{fig:triangoloequilatero1}
\end{figure}
\begin{proof}
	Costruiamo la~\cref{fig:triangoloequilatero1}, un triangolo equilatero di lato $s$. Le coordinate del suoi vertici sono $A(0,0)$, $B(s,0)$ e $C(\frac{s}{2},\frac{s}{2}\sqrt{3})$. 
	Per~\cref{thm:CircoAsse1} le coordinate del circocentro\index{Triangolo!circocentro}\index{Circocentro!triangolo!equilatero} sono
	\[\begin{cases}
	x=\dfrac{s}{2}\\ \\
	y=\dfrac{t^2+r^2-ts}{2r}\\
	\end{cases}\]
	Per il~\cref{thm:ortocentro1} le coordinate dell'ortocentro\index{Triangolo!ortocentro}\index{Ortocentro!triangolo!equilatero} sono \[\begin{cases}
	y=\dfrac{t}{r}(s-t)\\
	x=t
	\end{cases}\] 
	Per il~\cref{thm:baricentro1} le coordinate del baricentro\index{Triangolo!baricentro}\index{Baricentro!triangolo!equilatero} sono \[\begin{cases}
	x=\dfrac{a+c+e}{3}\\ \\
	y=\dfrac{b+d+f}{3}
	\end{cases} \]
	Adattando le tre formule al triangolo equilatero otteniamo:
	\[\begin{cases}
	x=\dfrac{s}{2}\\ \\
	y=\dfrac{s}{6}\sqrt{3}
	\end{cases}\]
\end{proof}
\begin{thm}[Raggio della circonferenza circoscritta]\label{thm:raggiocirconferenzacircoscritta}
	In un triangolo equilatero di lato $l$, il raggio della circonferenza circoscritta è \[R=\dfrac{\sqrt{3}}{3}l\]
\end{thm}\index{Triangolo!baricentro}\index{Circonferenza!circoscritta}
\begin{figure}
	\centering
	\includestandalone[width=0.9\linewidth]{geometria/triangolo_equi_raggiocirconferenza_circoscritta}
	\caption{Raggio della circonferenza circoscritta}
	\label{fig:raggiocirconferenzacircoscritta}
\end{figure}
\begin{proof}
		Costruiamo la~\cref{fig:raggiocirconferenzacircoscritta}
		\begin{align*}
		\intertext{Consideriamo il triangolo $ABC$}
			h^2=&l^2-\dfrac{l^2}{4}\\
			=&\dfrac{4l^2-l^2}{4}\\
			=&\dfrac{3}{4}l^2\\
\intertext{Quindi}
h=&\dfrac{l}{2}\sqrt{3}\\
\intertext{Consideriamo il triangolo $OBC$}
x^2=&r^2-\dfrac{l^2}{4}\\
x=&h-R\\
(h-R)^2=&r^2-\dfrac{l^2}{4}\\
2hR-h^2=&\dfrac{l^2}{4}\\
2(\dfrac{l}{2}\sqrt{3})R-(\dfrac{l}{2}\sqrt{3})^2=&\dfrac{l^2}{4}\\
\dfrac{3}{4}l^2-l\sqrt{3}R=&-\dfrac{l^2}{4}\\
\dfrac{3}{4}l^2+\dfrac{l^2}{4}-l\sqrt{3}R=&0\\
l^2-l\sqrt{3}R=&0\\
l=&0\\
l-\sqrt{3}R=&0\\
l=&\sqrt{3}R\\
		\end{align*}
\end{proof}
\section{Excentro}\label{sec:excentro}
\begin{defn}[Exentro]\label{defn:excentro}
Si dice excentro  di un triangolo il punto di incontro delle bisettrici di due angoli esterni e della bisettrice dell'angolo interno non adiacente ad essi.
\end{defn}\index{Triangolo!excentro}
\begin{figure}
	\centering
	\includestandalone[width=0.9\linewidth]{geometria/excentro1}
	\caption{Triangolo e circonferenze tangenti}
	\label{fig:excentro1}
\end{figure}
\section{Formula di Erone}\label{sec:Formula_Erone}
\begin{thm}[Formula di Erone]
Dato un triangolo~\cite{Dodero1999b}, se $2p=a+b+c$ allora l'area del triangolo è \[A=\sqrt{p(p-a)(p-b)(p-c)}\]
\end{thm}\index{Triangolo!Erone}\index{Teorema!Erone}
\begin{proof}
Consideriamo la~\cref{fig:erone1} se indichiamo con $h$ l'altezza del triangolo relativa al lato di lunghezza $a$ otteniamo \begin{align}
h^2=&b^2-x^2\label{eqn:erone1}\\
h^2=&c^2-(a-x)^2\nonumber
\intertext{uguagliando}
b^2-x^2=&c^2-(a-x)^2\nonumber\\
b^2-x^2=&c^2-a^2-x^2+2ax\nonumber\\
2ax=&b^2+a^2-c^2\nonumber\\
x=&\dfrac{b^2+a^2-c^2}{2a}\nonumber\\
\intertext{Per esprire l'altezza in funzione dei lati sostituiamo il risultato precedente nell'~\cref{eqn:erone1}}
h^2=&b^2-\dfrac{\left(b^2+a^2-c^2\right)^2}{4a^2}\nonumber\\
=&\dfrac{4a^2b^2-(b^2+a^2-c^2)^2}{4a^2}\nonumber\\
=&\dfrac{\left[2ab-\left(a^2+b^2-c^2\right)\right]\left[2ab+\left(a^2+b^2-c^2\right)\right]}{4a^2}\nonumber\\
=&\dfrac{\left(2ab-b^2-a^2+c^2\right)\left(2ab+b^2+a^2-c^2\right)}{4a^2}\nonumber\\
=&\dfrac{\left[c^2-\left(b-a\right)^2\right]\left[\left(a+b\right)^2-c^2\right]}{4a^2}\nonumber\\
=&\dfrac{(c-b+a)(c+b-a)(a+b-c)(a+b+c)}{4a^2}\label{eqn:erone2}
\intertext{Ponendo}
a+b+c=&2p\nonumber\\
c+b-a=&a+b+c-2a=2p-2a=2(p-a)\nonumber\\
a+c-b=&a+b+c-2b=2p-2b=2(p-b)\nonumber\\
a+b-c=&a+b+c-2c=2p-2c=2(p-c)\nonumber
\intertext{l'~\cref{eqn:erone2} diventa}
h^2=&\dfrac{2p2(p-a)2p(p-b)2(p-c)}{4a^2}\nonumber\\
=&\dfrac{4p(p-a)(p-b)(p-c)}{a^2}\nonumber\\
h=&\dfrac{2\sqrt{p(p-a)(p-b)(p-c)}}{a}\nonumber\\
\intertext{Quindi l'area del triangolo è}
A=&\dfrac{1}{2}ah\nonumber\\
=&\dfrac{1}{2}a\dfrac{2\sqrt{p(p-a)(p-b)(p-c)}}{a}\nonumber\\
=&\sqrt{p(p-a)(p-b)(p-c)}\nonumber
\end{align}
Come si voleva dimostrare.
\end{proof}
\begin{figure}
	\centering
	\includestandalone{geometria/erone1}
	\caption{Formula di Erone}
	\label{fig:erone1}
\end{figure}

% !TeX root = Asparsi.tex
% !BIB TS-program = biber
% !TeX encoding = UTF-8
% !TeX spellcheck = it_IT
\chapter{Progressioni}\label{ch:progressioni}
\section{Progressione aritmetica}\label{sec:progressione-aritmetica}
\begin{defn}[Progressione aritmetica]\label{defn:ProgAritm1}
	Successione di numeri $a_1,a_2,a_3,\dots,a_n$ in cui la differenza tra un termine e il suo precedente è costante. Tale termine $d=a_r-a_{r-1}$ è chiamato ragione della successione.
\end{defn}\index{Progressione!aritmetica}\index{Progressione!aritmetica!ragione}
\begin{prop}
	Data una successione aritmetica $a_1,a_2,a_3,\dots,a_n$ di ragione $d$ allora\[a_r=a_1+(r-1)d\]
\end{prop}
\begin{thm}[Somma progressione aritmetica]\label{thm:SommaProgAritm1}
Data una progressione aritmetica $a_1,a_2,a_3,\dots,a_n$ allora
\begin{align*}
	S_n=&a_1+a_2+a_3+\dots+a_n\\
	=&\dfrac{a_1+a_n}{2}n\\
\end{align*}
\end{thm}\index{Progressione!aritmetica!somma}
\begin{proof}
\begin{align*}
\intertext{Indichiamo con}
S_n=&a_1+a_2+\dots+a_{n-1}+a_n\\
S_n=&a_n+a_{n-1}+\dots+a_2+a_1\\
S_n+S_n=&(a_1+a_n)+(a_+a_{n-1})+\dots+(a_{n-1}+a_2)+(a_n+a_1)\\
\intertext{I termini $(a_1+a_n)$ per definizione sono uguali quindi}
2S_n=&n(a_1+a_n)\\
S_n=&\dfrac{a_1+a_n}{2}n
\end{align*}
\end{proof}
\section{Progressione geometrica}\label{sec:progressione-geometrica}
\begin{defn}[Progressione geometrica]\label{defn:ProgGeom1}
		Successione di numeri $a_1,a_2,a_3,\dots,a_n$ in cui il rapporto tra un termine e il suo precedente è costante. Tale termine $d=\dfrac{a_r}{a{r-1}}$ è chiamato ragione della successione.
\end{defn}\index{Progressione!geometrica}\index{Progressione!geometrica!ragione}
\begin{prop}
	Data una successione geometrica $a_1,a_2,a_3,\dots,a_n$ di ragione $d$ allora\[a_r=a_1d^{r-1}\]
\end{prop}
\begin{thm}[Somma progressione geometrica]\label{thm:SommaProgGeo}
		Data una successione geometrica $a_1,a_2,a_3,\dots,a_n$ di ragione $d$ allora 
		\begin{align*}
			S_n=&a_1+a_2+a_3+\dots+a_n\\
			=&a_1\dfrac{d^n-1}{d-1}
		\end{align*}
\end{thm}\index{Progressione!geometrica!somma}
\begin{proof}
\begin{align*}
\intertext{Indichiamo con}
S_n=&a_1+a_2+\dots+a_{n-1}+a_n\\
\intertext{moltiplico ogni termine per la ragione $d$}
S_nd=&a_1d+a_2d+\dots+a_{n-1}d+a_nd\\
\intertext{Sottraiamo i termini}
S_nd-S_n=&-a_1+a_nd\\
a_n=&a_1d^{n-1}\\
S_nd-S_n=&a_1d^{n-1}d-a_1\\
S_nd-S_n=&a_1d^{n}-a_1\\
S_n(d-1)=&a_1(d^{n}-1)\\
S_n=&\dfrac{d^{n}-1}{d-1}a_1\\
\end{align*}
\end{proof}
\chapter{Forma goniometrica dei numeri complessi}
\section{Definizioni}
\begin{defn}[Forma goniometrica]
	Un numero complesso $z\in\Co$ è in forma goniometrica se
	\[z=r\left[\cos\theta+\uimm\sin\theta \right] \] dove
	\begin{itemize}
		\item $\uimm$ unità immaginaria
		\item $r\in\R\; r\geq 0$ detto modulo
		\item $-\pi <\theta\leq\pi\; \vee\; 0\leq\theta<2\pi$ detto anomalia.
	\end{itemize}
\end{defn}\index{Numero!complesso}\index{Numero!complesso!goniometrica}
\begin{thm}[Prodotto]\label{thm:Compl_Prodotto}
Se $z_1=a_1(cos\theta_1+\uimm\sin\theta_1)$ e $z_2=a_2(cos\theta_2+\uimm\sin\theta_2)$ allora il prodotto \[z_1\cdot z_2=a_1\cdot a_2\left[cos(\theta_1+\theta_2)+\uimm\sin(\theta_1+\theta_2)\right] \] 
\end{thm}
\begin{proof} del teorema
	\begin{align*}
	z_1=&a_1(cos\theta_1+\uimm\sin\theta_1)\\
	z_2=&a_2(cos\theta_2+\uimm\sin\theta_2)\\
	z_1\cdot z_2=&a_1(cos\theta_1+\uimm\sin\theta_1)\cdot a_2(cos\theta_2+\uimm\sin\theta_2)\\
	=&a_1\cdot a_2\left(\cos\theta_1\cos\theta_2+\uimm\sin\theta_2\cos\theta_1+\uimm\sin\theta_1\cos\theta_2-\sin\theta_1\sin\theta_2\right)\\
	=&a_1\cdot a_2\left[\cos\theta_1\cos\theta_2-\sin\theta_1\sin\theta_2+\uimm(sin\theta_1\cos\theta_2+\cos\theta_1\sin\theta_2)\right]\\
	=&a_1\cdot a_2\left[\cos\left(\theta_1+\theta_2\right)+\uimm\sin\left(\theta_1+\theta_2\right)\right]
	\end{align*}
	Da cui la tesi.
\end{proof}\index{Numero!complesso!prodotto}
Dal teorema precedente segue il seguente corollario:
\begin{cor}[Reciproco]\label{cor:Complex:reciproco}
	Dato in numero complesso $z=a(cos\theta+\uimm\sin\theta)$ allora il suo reciproco è \begin{equation*}
	\dfrac{1}{z}=\dfrac{1}{a}\left[\cos (-\theta)+\uimm\sin(-\theta)\right]
	\end{equation*}\label{equa:Compl_reciproco}
\end{cor}
\begin{proof}\index{Numero!complesso!reciproco}
	Per il prodotto di due numeri reciproci è l'unità quindi verifichiamo 
\begin{align*}
z\cdot\dfrac{1}{z}=&\\
=&a(cos\theta+\uimm\sin\theta)\cdot\dfrac{1}{a}\left[\cos (-\theta)+\uimm\sin(-\theta)\right]\\
=&a\cdot\dfrac{1}{a}\left[\cos(\theta-\theta)+\uimm\sin(\theta-\theta)\right]\\
=&1\cdot\left[cos(0)-\uimm\sin(0)\right]\\
=&1\cdot\left[1-\uimm 0\right]\\
=&1
\end{align*}
Come si voleva dimostrare.
\end{proof}
\begin{thm}[Quoziente]\index{Numero!complesso!quoziente}
Se $z_1=a_1(cos\theta_1+\uimm\sin\theta_1)$ e $z_2=a_2(cos\theta_2+\uimm\sin\theta_2)$ allora il quoziente \[\dfrac{z_1}{z_2}=\dfrac{a_1}{a_2}\left[cos(\theta_1-\theta_2)+\uimm\sin(\theta_1-\theta_2)\right] \] 
\end{thm}
\begin{proof}
	Dal~\cref{cor:Complex:reciproco} abbiamo:
	\begin{align*}
	\dfrac{z_1}{z_2}=&\dfrac{a_1(cos\theta_1+\uimm\sin\theta_1)}{a_2(cos\theta_2+\uimm\sin\theta_2}\\
	=&a_1(cos\theta_1+\uimm\sin\theta_1)\cdot\dfrac{1}{a_2}\left[cos(-\theta_2)+\uimm\sin(-\theta_2) \right]\\
	=&\dfrac{a_1}{a_2}\left(\cos\theta_1\cos(-\theta_2)+\uimm\sin(-\theta_2)\cos\theta_1+\uimm\sin\theta_1\cos(-\theta_2)-\sin\theta_1\sin(-\theta_2)\right)\\
	=&\dfrac{a_1}{a_2}\left[\cos\theta_1\cos(-\theta_2)-\sin\theta_1\sin(-\theta_2)+\uimm(sin\theta_1\cos(-\theta_2)+\cos\theta_1\sin(-\theta_2))\right]\\
	=&\dfrac{a_1}{a_2}\left[\cos\left(\theta_1-\theta_2\right)+\uimm\sin\left(\theta_1-\theta_2\right)\right]
	\end{align*}
	Come si voleva dimostrare.
\end{proof}
\chapter{Geometria analitica}
\section{Retta}
\subsection{Rette perpendicolari}
\begin{thm}[Rette perpendicolari]\label{thm:rette_perpendicolari}
	Due rette $y=m_1x+q_1$ e $y=m_2x+q_2$ incidenti sono perpendicolari se solo se \[m_1\cdot m_2=-1\]
\end{thm}
\begin{figure}
	\centering
	\includestandalone{geometria/retteperp1}
	\caption{Rette perpendicolari}
	\label{fig:retteperp1}
\end{figure}
\begin{proof}
	Consideriamo due rette perpendicolari~\cite{Dodero1999b}, trasportiamo il centro del sistema di riferimento in modo che coincida co il punto di inteserzione trala due rette come nella~\cref{fig:retteperp1}. Le rette $y=m_1x+q_1$ e $y=m_2x+q_2$ diventano $y=m_1x$ e $y=m_2x$. Consideriamo due punti $C$ e $D$ di ascissa uno.  Il triangolo $ABC$ è retto in $\hat{A}$ di conseguenza vale il teorema di Pitagora\index{Teorema!Pitagora}
	\begin{align*}
	\overline{BC}^2=&\overline{AB}^2+\overline{AC}^2\\
	\intertext{Dato che}
	\overline{BC}^2=&\abs{m_2-m_1}^2\\
	\overline{AB}^2=&1+m_1^2\\
	\overline{AC}^2=&1+m_2^2
	\intertext{Quindi}
	\abs{m_2-m_1}^2=&1+m_1^2+1+m_2^2\\
	m_1^2+m_2^2-2m_1m_2=&1+m_1^2+1+m_2^2\\
	-2m_1m_2=2\\
	m_1m_2=&-1\\
	\end{align*}
	Dimostriamo il viceversa. Consideriamo la~\cref{fig:retteperp1}, le rette $y=m_1x$ e $y=m_2x$ sono tali che \[m_1m_2=-1\] Poniamo che $C(1,m_1)$, $H(1,0)$ e $B(1,m_2)$ 
	
Quindi 
\begin{align*}
\abs{m_1}\cdot\abs{m_2}=&\abs{m_1\cdot m_2}=1
\intertext{$ABH$ è retto in $\hat{H}$ quindi}
\overline{AB}^2=&\overline{HB}^2+\overline{HA}^2
\intertext{$ACH$ è retto in $\hat{H}$ quindi}
\overline{AC}^2=&\overline{HC}^2+\overline{HA}^2
\intertext{Per cui}
\overline{AB}^2+\overline{AC}^2=&\overline{HC}^2+\overline{HA}^2+\overline{HB}^2+\overline{HA}^2\\
=&\overline{HC}^2+\overline{HB}^2+2\overline{HA}^2\\
\intertext{Ma $\overline{HA}^2=\overline{HC}\cdot\overline{HB}$ quindi}
=&\overline{HC}^2+\overline{HB}^2+2\overline{HC}\cdot\overline{HB}\\
=&(\overline{HC}+\overline{HB})^2\\
=&\overline{CB}^2
\intertext{Segue}
\overline{AB}^2+\overline{AC}^2=&\overline{CB}^2
\end{align*}
come si voleva dimostrare.
\end{proof}
Altra dimostrazione della prima parte del~\cref{thm:rette_perpendicolari}
\begin{thm}[Rette perpendicolari]\label{thm:rette_perpendicolari1}
	Due rette $y=m_1x+q_1$ e $y=m_2x+q_2$ incidenti sono perpendicolari se \[m_1\cdot m_2=-1\]
\end{thm}
\begin{proof}
Procediamo come il~\cref{thm:rette_perpendicolari} e consideriamo la~\cref{fig:retteperp1}. Dato che il triangolo $ABC$ è retto vale il secondo teorema di Euclide\index{Teorema!Euclide}
\begin{align*}
\overline{HC}\cdot\overline{HB}=&\overline{HA}^2
\intertext{Dato che}
\overline{HC}=&m_1\\
\overline{HB}=&-m_2\\
\overline{HA}=&1\\
\intertext{otteniamo}
-m_2\cdot m_1=1\\
m_2\cdot m_1=-1\\ 
\end{align*}
Come si voleva dimostrare
\end{proof}
\section{Parabola}
\begin{defn}[Definizione parabola]
	Definisco parabola l'equazione $y=ax^2+bx+c$ $a\neq0$
\end{defn}\index{Parabola}
\subsection{Proprietà}
\begin{thm}[Complemento al quadrato]\label{thm:Parabola_complemento}
	La parabola $y=ax^2+bx+c$ $a\neq0$ è equivalente a \begin{equation*}
	y+\dfrac{\Delta}{4a}=a\left(x+\dfrac{b}{2a}\right)^2\quad\Delta=b^2-4ac
	\end{equation*}\label{equa:Parabola_scomposizione}
\end{thm}\index{Parabola!complemento!quadrato}
\begin{proof}
	\begin{align*}
	y=&ax^2+bx+c\\
	y-c=&ax^2+bx\\
	y+\dfrac{b^2}{4a}-c=&ax^2+bx+\dfrac{b^2}{4a}\\
	y+\dfrac{b^2-4ac}{4a}=&a\dfrac{4a^2x^2+4abx+b^2}{4a^2}\\
	y+\dfrac{\Delta}{4a}=&a\left(\dfrac{2ax+b}{2a}\right)^2\\
	y+\dfrac{\Delta}{4a}=&a\left(x+\dfrac{b}{2a}\right)^2
	\end{align*}
	Da cui la tesi.
\end{proof}
\subsection{Concavità}
\begin{lem}
	Se $m>0$ $\forall n$ $m-n>-n$ se $m<0$ $\forall n$ $m-n<-n$
\end{lem}
\begin{thm}[Concavità]\label{thm:concavitaparabola}
	Data una parabola~\cite{Zwirner1988} $y=ax^2+bx+c$ $a\neq0$ allora se $a>0$ la parabola ha la concavità rivolata verso l'alto e il vertice è il punto di minima ordinata, se $a<0$ la parabola ha la concavità rivolata verso il basso e il vertice è il punto di massima ordinata
\end{thm}
\begin{proof}
	Dal~\cref{thm:Parabola_complemento} abbiamo
	\begin{align*}
	y=&a\left(x+\dfrac{b}{2a}\right)^2-\dfrac{b^2-4ac}{4a}& x\neq&\dfrac{b}{2a}\\
	\intertext{Il vertice della parabola ha ordinata}
	y_v=&-\dfrac{b^2-4ac}{4a}
	\intertext{Se $a>0$ }
	a\left(x+\dfrac{b}{2a}\right)^2>&0\\
	\intertext{Quindi}
	a\left(x+\dfrac{b}{2a}\right)^2-\dfrac{b^2-4ac}{4a}>&-\dfrac{b^2-4ac}{4a}\\
	\intertext{Analogamente}
	\intertext{Se $a<0$ }
	a\left(x+\dfrac{b}{2a}\right)^2<&0\\
	\intertext{Quindi}
	a\left(x+\dfrac{b}{2a}\right)^2-\dfrac{b^2-4ac}{4a}<&-\dfrac{b^2-4ac}{4a}\\
	\end{align*}
	Da cui la tesi
\end{proof}
%\subsection{Tangente formula di sdoppiamento}
\begin{thm}[Formula di sdoppiamento]\label{thm:Formulasdoppiamento_parab}
	Data una parabola~\cite{ReFraschini2008} $y=ax^2+bx+c$ e $P(x_0,y_0)$ un suo punto. Allora la retta tangente alla parabola per $P$ ha equazione \[ \dfrac{y+y_0}{2}=ax_0x+b\left(\dfrac{x+x_0}{2}\right)+c \] 
\end{thm}\index{Parabola!tangente}
\begin{proof}
	Poniamo a sistema la parabola con il fascio di rette di centro $P$ 
	\begin{align*}
	&\begin{cases}
	y=ax^2+bx+c\\y-y_0=m(x-x_0)
	\end{cases}
	\intertext{Otteniamo}
	ax^2+bx+c-y_0-mx+mx_0=&0\\
	ax^2+(b-m)x+c-y_0+mx_0=&0\\
	\intertext{Dato che $P$ è un punto di tangenza, le soluzioni sono coincidenti, quindi la somma delle soluzioni è}
	-\dfrac{b-m}{a}=&2x_0\\
	m=&2ax_0+b\\
	\intertext{Sostituendo nell'equazione del fascio di rette }
	y-y_0=&(2ax_0+b)(x-x_0)\\
	y=&2ax_0x-2ax_0^2+bx-bx_0+y_0\\
	\intertext{Il punto $P$ appartiene alla parabola quindi}
		y=&2ax_0x-2ax_0^2+bx-bx_0+ax_0^2+bx_0+c\\
\intertext{Dividendo per due ambi i membri}
\dfrac{y}{2}=&ax_0x-ax_0^2+\dfrac{b}{2}x+\dfrac{a}{2}x_0^2+\dfrac{c}{2}\\
\intertext{Aggiungendo a sinistra e a destra $\dfrac{x_0+y_0}{2}$ otteniamo}
\dfrac{y}{2}+\dfrac{x_0+y_0}{2}=&ax_0x-ax_0^2+\dfrac{b}{2}x+\dfrac{a}{2}x_0^2+\dfrac{c}{2}+\dfrac{x_0+y_0}{2}\\
\dfrac{y}{2}+\dfrac{y_0}{2}=&-\dfrac{x_0}{2}+ax_0x-ax_0^2+\dfrac{b}{2}x+\dfrac{a}{2}x_0^2+\dfrac{c}{2}+\dfrac{x_0}{2}+\dfrac{y_0}{2}\\
\intertext{Semplificando}
\dfrac{y+y_0}{2}=&ax_0x-ax_0^2+\dfrac{b}{2}x+\dfrac{a}{2}x_0^2+\dfrac{c}{2}+\dfrac{y_0}{2}\\
\intertext{Il punto $P$ appartiene alla parabola quindi}
\dfrac{y+y_0}{2}=&ax_0x-ax_0^2+\dfrac{b}{2}x+\dfrac{a}{2}x_0^2+\dfrac{c}{2}+\dfrac{a}{2}x_0^2+\dfrac{b}{2}x_0+\dfrac{c}{2}\\
\intertext{Semplificando}
\dfrac{y+y_0}{2}=&ax_0x+\dfrac{b}{2}x+\dfrac{b}{2}x_0+c\\
\dfrac{y+y_0}{2}=&ax_0x+b\dfrac{x+x_0}{2}+c\\
	\end{align*}
	Da cui la tesi.
\end{proof}
% !TeX root = Asparsi.tex
% !BIB TS-program = biber
% !TeX encoding = UTF-8
% !TeX spellcheck = it_IT
\chapter{Goniometria}
\section{Somma sottrazione di angoli}
\begin{figure}
	\centering
\includestandalone[scale=0.68]{geometria/cosenosommadifferenza1}
	\caption{Coseno della differenza}
	\label{fig:cosenosommadifferenza1}
\end{figure}
\begin{thm}[Coseno della differenza di due angoli]\label{thm:Cosenodelladifferenza}
Dati due angoli $\alpha$ e $\beta$ allora
\[\cos(\alpha-\beta)=\cos\alpha\cos\beta+\sin\alpha\sin\beta  \]
\end{thm}\index{Coseno!differenza}
\begin{proof}
Consideriamo due angoli $\alpha$ e $\beta$ con $\alpha>\beta$. Otteniamo la~\vref{fig:cosenosommadifferenza1}. Costruiamo una angolo di ampiezza $\alpha-\beta$ con vertice in $O$ e un lato $AD$. Otteniamo quattro punti $A(\cos\alpha;\sin\alpha)$, $B (\cos\beta;\sin\beta)$, $C(\cos\left(\alpha-\beta\right);\sin\left(\alpha-\beta\right))$ e $D(1;0)$. I triangoli $AOB$ e $COD$ sono congruenti in particolare abbiamo: \begin{align*}
\widebar{CD}=&\widebar{AB}\\
\left[\cos(\alpha-\beta)-1\right]^2+\sin^2(\alpha-\beta)=&\left[\cos\alpha-\cos\beta\right]^2+\left[\sin\alpha-\sin\beta\right]^2\\
\cos^2(\alpha-\beta)-2\cos(\alpha-\beta)+1+\sin^2(\alpha-\beta)=&\cos^2\alpha+\cos^2\beta-2\cos\alpha\cos\beta+\sin^2\alpha+\sin^2\beta-2\sin\alpha\sin\beta\\
\intertext{ma}
\cos^2(\alpha-\beta)+\sin^2(\alpha-\beta)=&1\\
\cos^2\alpha+\sin^2\alpha=&1\\
\cos^2\beta+\sin^2\beta=&1\\
\intertext{quindi}
1+1-2\cos(\alpha-\beta)=&1+1-2\cos\alpha\cos\beta-2\sin\alpha\sin\beta\\
2\cos(\alpha-\beta)=&-2\cos\alpha\cos\beta-2\sin\alpha\sin\beta\\
\intertext{semplifichiamo}
\cos(\alpha-\beta)=&\cos\alpha\cos\beta+\sin\alpha\sin\beta 
\end{align*}
Da cui la tesi.
\end{proof}
\begin{cor}[Coseno della somma di due angoli]\label{cor:Cosenodellasomma}
Dati due angoli $\alpha$ e $\beta$ allora
\[\cos(\alpha+\beta)=\cos\alpha\cos\beta-\sin\alpha\sin\beta  \]
\end{cor}\index{Coseno!somma}
\begin{proof}
	Dal~\vref{thm:Cosenodelladifferenza}
	\begin{align*}
	\cos(\alpha+\beta)=&\cos\left[\alpha-(-\beta)\right]\\
	\cos\left[\alpha-(-\beta)\right]=&\cos\alpha\cos(-\beta)+\sin\alpha\sin(-\beta)\\
	\intertext{ma}
	\cos(-\beta)=&\cos\beta\\
	\sin(-\beta)=&-\sin\beta
	\intertext{quindi}
	\cos(\alpha+\beta)=&\cos\alpha\cos\beta-\sin\alpha\sin\beta
	\end{align*}
	Da cui la tesi.
\end{proof}
\begin{cor}[Seno della differenza di due angoli]\label{cor:Senodelladifferenza}
Dati due angoli $\alpha$ e $\beta$ allora
\[\sin(\alpha-\beta)=\sin\alpha\cos\beta-\cos\alpha\sin\beta  \]
\end{cor}\index{Seno!differenza}
\begin{proof}
		Dal~\vref{thm:Cosenodelladifferenza}
		\begin{align*}
		\sin(\alpha-\beta)=&\cos\left[\ang{90}-\left(\alpha-\beta\right)\right]\\
		=&\cos\left[\left(\ang{90}-\alpha\right)+\beta\right]\\
		=&\cos\left(\ang{90}-\alpha\right)\cos\beta-\sin\left(\ang{90}-\alpha\right)\sin\beta\\
		\intertext{ma}
		\cos\left(\ang{90}-\alpha\right)=&\sin\alpha\\
		\sin\left(\ang{90}-\alpha\right)=&\cos\alpha\\
		\intertext{quindi}
		\sin(\alpha-\beta)=&\sin\alpha\cos\beta-\cos\alpha\sin\beta\\
		\end{align*}
		Da cui la tesi.
\end{proof}
\begin{cor}[Seno della somma di due angoli]
	Dati due angoli $\alpha$ e $\beta$ allora
	\[\sin(\alpha+\beta)=\sin\alpha\cos\beta+\cos\alpha\sin\beta  \]
\end{cor}\index{Seno!somma}
\begin{proof}
	Dal~\vref{cor:Cosenodellasomma}
	\begin{align*}
	\sin(\alpha+\beta)=&\cos\left[\ang{90}-\left(\alpha+\beta\right)\right]\\
	=&\cos\left[\left(\ang{90}-\alpha\right)-\beta\right]\\
	=&\cos\left(\ang{90}-\alpha\right)\cos\beta+\sin\left(\ang{90}-\alpha\right)\sin\beta\\
	\intertext{ma}
	\cos\left(\ang{90}-\alpha\right)=&\sin\alpha\\
	\sin\left(\ang{90}-\alpha\right)=&\cos\alpha\\
	\intertext{quindi}
	\sin(\alpha\beta)=&\sin\alpha\cos\beta+\cos\alpha\sin\beta\\
	\end{align*}
	Da cui la tesi.
\end{proof}
\begin{thm}[Tangente differenza di angoli]\label{thm:tangentedifferenza}
Dati due angoli $\alpha$ e $\beta$ allora\[\tan\left(\alpha-\beta\right)=\dfrac{\tan\alpha-\tan\beta}{1+\tan\alpha\tan\beta}\qquad\alpha,\beta,(\alpha-\beta)\neq\ang{90}+k\ang{180}\]
\end{thm}\index{Tangente!differenza}
\begin{proof}
	Dalla definizione dal~\vref{thm:Cosenodelladifferenza} e dal~\vref{cor:Senodelladifferenza}
	\begin{align*}
	\tan\left(\alpha-\beta\right)=&\dfrac{\sin\left(\alpha-\beta\right)}{\cos\left(\alpha-\beta\right)}\\
	=&\dfrac{\sin\alpha\cos\beta-\cos\alpha\sin\beta}{\cos\alpha\cos\beta+\sin\alpha\sin\beta}\\
	\intertext{Dato che $\cos\alpha\cos\beta\neq 0$}
	=&\dfrac{\dfrac{\sin\alpha\cos\beta}{\cos\alpha\cos\beta}-\dfrac{\cos\alpha\sin\beta}{\cos\alpha\cos\beta}}{\dfrac{\cos\alpha\cos\beta}{\cos\alpha\cos\beta}+\dfrac{\sin\alpha\sin\beta}{\cos\alpha\cos\beta}}\\
	=&\dfrac{\tan\alpha-\tan\beta}{1+\tan\alpha\tan\beta}\\
	\end{align*}
	Da cui la tesi.
\end{proof}
\begin{cor}[Tangente somma di angoli]\label{cor:Tangentesommadiangoli}
	Dati due angoli $\alpha$ e $\beta$ allora\[\tan\left(\alpha+\beta\right)=\dfrac{\tan\alpha+\tan\beta}{1-\tan\alpha\tan\beta}\qquad\alpha,\beta,(\alpha-\beta)\neq\ang{90}+k\ang{180}\]
\end{cor}\index{Tangente!somma}
\begin{proof}
	Dal~\vref{thm:tangentedifferenza} avremo:
	\begin{align*}
	\tan\left(\alpha+\beta\right)=&\tan\left[\alpha-\left(-\beta\right)\right]\\
	=&\dfrac{\tan\alpha-\tan\left(-\beta\right)}{1+\tan\alpha\tan\left(-\beta\right)}
	\intertext{ma}
	\tan\left(-\beta\right)=&-\tan\beta\\
	=&\dfrac{\tan\alpha+\tan\beta}{1-\tan\alpha\tan\beta}\\
	\end{align*}
	Da cui la tesi.
\end{proof}
\section{Formule parametriche}
\begin{thm}[Formule parametriche]\label{thm:formuleparametriche1}
	Se $\alpha$ è un angolo allora
	\begin{align*}
\sin\alpha=&\frac{2t}{1+t^2}\\
\cos\alpha=&\frac{1-t^2}{1+t^2}\\
t=&\tan\frac{\alpha}{2}&\alpha\neq\pi+2k\pi
	\end{align*}\index{Seno}\index{Coseno}\index{Tangente}\index{Formule!parametriche}
\end{thm}
\begin{proof} Dalla somma di due angoli otteniamo le seguenti relazioni
	\begin{align*}
	\intertext{Prima relazione}
	\sin2\alpha=&2\sin\alpha\cos\alpha\\
	=&\frac{2\sin\alpha\cos\alpha}{1}\\
	=&\frac{2\sin\alpha\cos\alpha}{\cos^2\alpha+\sin^2\alpha}\\
	=&\dfrac{\dfrac{2\sin\alpha\cos\alpha}{\cos^2\alpha}}{\dfrac{\cos^2\alpha+\sin^2\alpha}{cos^2\alpha}}&\alpha\neq\frac{\pi}{2}+k\pi\\
	=&\dfrac{\dfrac{2\sin\alpha}{\cos\alpha}}{\dfrac{\cos^2\alpha}{\cos^2\alpha}+\dfrac{\sin^2\alpha}{cos^2\alpha}}\\
	=&\dfrac{2\tan\alpha}{1+\tan^2\alpha}\\
	\alpha&\longmapsto\dfrac{\alpha}{2}\\
	\sin\alpha=&\dfrac{2\tan\dfrac{\alpha}{2}}{1+\tan^2\dfrac{\alpha}{2}}\\
	t=&\tan\frac{\alpha}{2}\\
	\sin\alpha=&\frac{2t}{1+t^2}\\
	\end{align*}
	\begin{align*}
	\intertext{Seconda relazione}
	\cos 2\alpha=&\cos^2\alpha-\sin^2\alpha\\
	=&\dfrac{\cos^2\alpha-\sin^2\alpha}{1}\\
	=&\frac{\cos^2\alpha-\sin^2\alpha}{\cos^2\alpha+\sin^2\alpha}\\
	=&\frac{\dfrac{\cos^2\alpha-\sin^2\alpha}{\cos^2\alpha}}{\dfrac{\cos^2\alpha+\sin^2\alpha}{\cos^2\alpha}}&\alpha\neq\frac{\pi}{2}+k\pi\\ 
	=&\frac{1-\dfrac{\sin^2\alpha}{\cos^2\alpha}}{1+\dfrac{\sin^2\alpha}{\cos^2\alpha}}\\
	=&\frac{1-\tan^2\alpha}{1+\tan^2\alpha}\\
	\alpha&\longmapsto\dfrac{\alpha}{2}\\
	\cos\alpha=&\frac{1-\tan^2\dfrac{\alpha}{2}}{1+\tan^2\dfrac{\alpha}{2}}\\
	t=&\tan\frac{\alpha}{2}\\
	\cos\alpha=&\frac{1-t^2}{1+t^2}
	\end{align*}
	da cui la tesi.
\end{proof}
\chapter{Trigonometria}
\section{Teorema di Carnot}\index{Carnot!Teorema}
\begin{thm}[Teorema di Carnot]\label{thm:TeoremadiCarnot}
	In un triangolo, il quadrato della lunghezza di un lato è uguale alla somma dei quadrati delle lunghezze dei rimanenti diminuito dal doppio del prodotto delle lunghezze di questi lati per il coseno dell'angolo fra essi compreso. 
	\begin{align*}
	a^2=&b^2+c^2-2bc\cos\alpha\\
	b^2=&a^2+c^2-2ac\cos\beta\\
	c^2=&a^2+b^2-2ab\cos\gamma\\
	\end{align*}
\end{thm}
\begin{figure}
	\centering
	\includestandalone{geometria/triangolopitagorico1}
	\caption{Teorema di Carnot, angolo acuto}
	\label{fig:triangolopitagorico1}
\end{figure}
\begin{proof}
	Consideriamo il triangolo~\vref{fig:triangolopitagorico1} cioè supponiamo che l'angolo $\beta$ sia un angolo acuto. Prendiamo il triangolo rettangolo $ADC$ allora:
	\begin{align*}
	\intertext{Per il teorema di Pitagora}
\widebar{AC}^2=&\widebar{AD}^2+\widebar{DC}^2\\
\widebar{AC}=&b\\
\widebar{AD}=&c\sin\beta\\
\widebar{BD}=&c\cos\beta\\
\widebar{DC}=&a-c\cos\beta\\
\intertext{Otteniamo}
b^2=&c^2\sin^2\beta+\left(a-c\cos\beta\right)^2\\
=&c^2\sin^2\beta+a^2+c^2\cos^2\beta-2ac\cos\beta\\
=&c^2\left(\sin^2\beta+\cos^2\beta\right)+a^2-2ac\cos\beta\\
=&c^2+a^2-2ac\cos\beta\\
	\end{align*}
	Consideriamo il triangolo come in~\vref{fig:triangolopitagorico2} cioè supponiamo che l'angolo $\beta$ sia un angolo ottuso. Prendiamo il triangolo rettangolo $ADC$ allora:
		\begin{align*}
	\intertext{Per il teorema di Pitagora}
	\widebar{AC}^2=&\widebar{AD}^2+\widebar{DC}^2\\
	\widebar{AC}=&b\\
	\widebar{AD}=&c\sin\left(\ang{180}-\beta\right)\\
	=&c\sin\beta\\
	\widebar{BD}=&c\cos\left(\ang{180}-\beta\right)\\
	=&-c\cos\beta\\
	\widebar{DC}=&a+c\cos\left(\ang{180}-\beta\right)\\
	=&a-c\cos\beta\\
	\intertext{Otteniamo}
	b^2=&c^2\sin^2\beta+\left(a-c\cos\beta\right)^2\\
	=&c^2\sin^2\beta+a^2+c^2\cos^2\beta-2ac\cos\beta\\
	=&c^2\left(\sin^2\beta+\cos^2\beta\right)+a^2-2ac\cos\beta\\
	=&c^2+a^2-2ac\cos\beta\\
	\end{align*}
	Da cui la tesi.
\end{proof}
\begin{cor}[Teorema di Carnot]\label{cor:TeoremaCarnot1}
	In un triangolo avremo che
	\begin{align*}
	\cos\alpha=&\dfrac{b^2+c^2-a^2}{2bc}\\
	\cos\beta=&\dfrac{a^2+c^2-b^2}{2ac}\\
	\cos\gamma=&\dfrac{a^2+b^2-c^2}{2bc}\\
	\end{align*}
\end{cor}
\begin{cor}[Teorema di Carnot]
	In un triangolo un angolo $\alpha\in[0,\ang{180}]$ è acuto, retto o ottuso se $b^2+c^2-a^2$ è positivo, nullo o negativo.
\end{cor}
\begin{proof}
	Dal~\cref{cor:TeoremaCarnot1} \begin{align*}
	\cos\alpha=&\dfrac{b^2+c^2-a^2}{2bc}\\
	\intertext{Il segno della frazione è il segno del numeratore, quindi}
	\intertext{Se:}
	b^2+c^2-a^2>&0\\
	\intertext{Il coseno è positivo, quindi l'angolo è acuto}
	\intertext{Se:}
	b^2+c^2-a^2=&0\\
	\intertext{Il coseno è nullo, quindi l'angolo è retto}
		\intertext{Se:}
	b^2+c^2-a^2<&0\\
	\intertext{Il coseno è negativo, quindi l'angolo è ottuso}
	\end{align*}
	Da cui la tesi.
\end{proof}
\begin{figure}
	\centering
	\includestandalone{geometria/triangolopitagorico2}
	\caption{Teorema di Carnot, angolo ottuso}
	\label{fig:triangolopitagorico2}
\end{figure}
%\chapter{Parabola}
\section{Definizione}
\begin{defn}[Definizione parabola]
Definisco parabola l'equazione $y=ax^2+bx+c$ $a\neq0$
\end{defn}
\section{Proprietà}
\begin{thm}[Complemento al quadrato]
	La parabola $y=ax^2+bx+c$ $a\neq0$ è equivalente a \begin{equation*}
	y+\dfrac{\Delta}{4a}=a\left(x+\dfrac{b}{2a}\right)^2\quad\Delta=b^2-4ac
	\end{equation*}\label{equa:Parabola_scomposizione}
\end{thm}\index{Parabola!complemento!quadrato}
\begin{proof}
	\begin{align*}
	y=&ax^2+bx+c\\
	y-c=&ax^2+bx\\
	y+\dfrac{b^2}{4a}-c=&ax^2+bx+\dfrac{b^2}{4a}\\
	y+\dfrac{b^2-4ac}{4a}=&a\dfrac{4a^2x^2+4abx+b^2}{4a^2}\\
	y+\dfrac{\Delta}{4a}=&a\left(\dfrac{2ax+b}{2a}\right)^2\\
	y+\dfrac{\Delta}{4a}=&a\left(x+\dfrac{b}{2a}\right)^2
	\end{align*}
\end{proof}
\chapter{Contro esempi}
\section{Potenze}
\begin{cexmp}[Valori assoluti]
\[-1=1\]
\end{cexmp}
\begin{proof}
		\begin{align*}
	1=&1\\
	41-40=&61-60\\
	16+25-40=&36+25-60\\
	4^2+5^2-2(5\cdot 4)=&6^2+5^2-2(6\cdot 5)\\
	(4-5)^2=&(6-5)^2\\
	(4-5)=&(6-5)\\
	-1=&1\\
	\end{align*}
	non sempre è vero  che $A^2=B^2$ $A=B$ es. $(-2)^2=(2)^2$ ma $-2\neq 2$
\end{proof}

\section{Radicali}
\subsection{Somma}
\begin{cexmp}[Somma radicali]
	\[\sqrt[n]{a}+\sqrt[n]{b}\neq\sqrt[n]{a+b}\\\]
\end{cexmp}
\begin{proof}
\begin{align*}
\sqrt{36}+\sqrt{64}\neq&\sqrt{100}\\
\sqrt{36}+\sqrt{64}=&6+8=14\\
\sqrt{100}=10\\
\end{align*}
Quindi la somma di radicali non il radicale della somma.
\end{proof}
\subsection{Prodotto}
\begin{cexmp}[Esistenza]
	Se $\sqrt[n]{a}$ e $\sqrt[n]{b}$ esistono allora esiste $\sqrt[n]{ab}$ Il viceversa non sempre è vero infatti mentre esiste $\sqrt{(-4)(-2)}$ non esistono   $\sqrt{-4}$ $\sqrt{-2}$ non esistono.
\end{cexmp}
\section{Goniometria}
\subsection{Coseno}
\begin{cexmp}[Non periodico]
	$f(x)=cosx^2$ non è periodica
\end{cexmp}\index{Coseno}
\begin{proof}
	Supponiamo che sia periodica di periodo $T$ allora
	\begin{align*}
	\cos(x+T)^2=&\cos x^2\\
	x^2+2xT+T^2\pm x^2=&2k\pi&z\in&\Z\\
	\end{align*}
	Assurdo, a sinistra abbiamo un'espressione che dipende dalla variabile continua $x$ a destra l'espressione può assumere solo valori interi.
\end{proof}
% !TeX root = Asparsi.tex
% !BIB TS-program = biber
% !TeX encoding = UTF-8
% !TeX spellcheck = it_IT
\chapter{Radicali doppi}\label{ch:radicalidoppi}
\section{Radicali doppi}
\begin{thm}[Radicali doppi]\label{thm:Radicalidoppi1}
Se  $a$ e $b$ sono numeri interi allora
\begin{align*}
		\sqrt{a+\sqrt{b}}=&\sqrt{\dfrac{a+\sqrt{a^2-b}}{2}}+\sqrt{\dfrac{a-\sqrt{a^2-b}}{2}}\\
		\sqrt{a-\sqrt{b}}=&\sqrt{\dfrac{a+\sqrt{a^2-b}}{2}}-\sqrt{\dfrac{a-\sqrt{a^2-b}}{2}}
		\\
\end{align*}
\end{thm}\index{Radicali!doppi}
\begin{proof}
	\begin{align*}
	\intertext{Dimostriamo la prima relazione}
	\sqrt{a+\sqrt{b}}=&\sqrt{x}+\sqrt{y}\\
	\intertext{Elevando al quadrato}
	a+\sqrt{b}=&x+y+2\sqrt{xy}\\
	a+\sqrt{b}=&x+y+\sqrt{4xy}\\
	\intertext{Otteniamo il sistema simmetrico}
	&\begin{cases}
	x+y=a\\
	xy=\dfrac{b}{4}
	\end{cases}\\
	t^2-at+\dfrac{b}{4}=&0\\
	\intertext{Che ha per soluzioni}
	t_{1,2}=&\dfrac{4a\pm\sqrt{16a^2-16b}}{8}\\
	=&\dfrac{4a\pm 4\sqrt{a^2-b}}{8}\\
	=&\dfrac{a\pm\sqrt{a^2-b}}{2}\\
	&\begin{cases}
	x=\dfrac{a+\sqrt{a^2-b}}{2}\\
	\\
	y=\dfrac{a-\sqrt{a^2-b}}{2}\\
	\end{cases}
	\intertext{Quindi}
	\sqrt{a+\sqrt{b}}=&\sqrt{\dfrac{a+\sqrt{a^2-b}}{2}}+\sqrt{\dfrac{a-\sqrt{a^2-b}}{2}}\\
	\end{align*}
		\begin{align*}
	\intertext{Dimostriamo la seconda relazione}
	\sqrt{a-\sqrt{b}}=&\sqrt{x}-\sqrt{y}\\
	\intertext{Elevando al quadrato}
	a+\sqrt{b}=&x+y-2\sqrt{xy}\\
	a+\sqrt{b}=&x+y-\sqrt{4xy}\\
	\intertext{Otteniamo il sistema simmetrico}
	&\begin{cases}
	x+y=a\\
	xy=\dfrac{b}{4}
	\end{cases}\\
	t^2-at+\dfrac{b}{4}=&0\\
	\intertext{Che ha per soluzioni}
	t_{1,2}=&\dfrac{4a\pm\sqrt{16a^2-16b}}{8}\\
	=&\dfrac{4a\pm 4\sqrt{a^2-b}}{8}\\
	=&\dfrac{a\pm\sqrt{a^2-b}}{2}\\
	&\begin{cases}
	x=\dfrac{a+\sqrt{a^2-b}}{2}\\
	\\
	y=\dfrac{a-\sqrt{a^2-b}}{2}\\
	\end{cases}
	\intertext{Quindi}
	\sqrt{a-\sqrt{b}}=&\sqrt{\dfrac{a+\sqrt{a^2-b}}{2}}-\sqrt{\dfrac{a-\sqrt{a^2-b}}{2}}\\
	\end{align*}
\end{proof}  
\begin{cor}[Quadrato perfetto]\label{cor:quadratoperfetto}
	Dato $\sqrt{a+\sqrt{b}}$ e $\sqrt{a-\sqrt{b}}$ se $a^2-b=x^2$ è un quadrato perfetto allora
	\begin{align*}
	\sqrt{a+\sqrt{b}}=&\sqrt{\dfrac{a+x}{2}}+\sqrt{\dfrac{a-x}{2}}\\
	\sqrt{a-\sqrt{b}}=&\sqrt{\dfrac{a+x}{2}}-\sqrt{\dfrac{a-x}{2}}\\
	\end{align*}
\end{cor}\index{Quadrato!perfetto}
% !TeX root = Asparsi.tex
% !BIB TS-program = biber
% !TeX encoding = UTF-8
% !TeX spellcheck = it_IT
\chapter{Statistica}
\section{Varianza}
\begin{thm}[Varianza]\index{Varianza}
	Se abbiamo una distribuzione di frequenze di n dati  $x_{i}$ di frequenza $n_{i}$, la varianza è: \[\sigma^{2}=\dfrac{\sum_{i=1}^{n}x_{i}^{2}\cdot n_{i}}{\sum_{i=1}^{n} n_{i}}-M^2\] 
\end{thm}
\begin{proof}
	\begin{align*}
	\sigma^{2}=&\dfrac{\sum_{i=1}^{n}(x_{i}-M)^{2}\cdot n_{i}}{\sum_{i=1}^{n} n_{i}}\\
	=&\dfrac{\sum_{i=1}^{n}(x_{i}^{2} -2Mx_{i}+M^{2})\cdot n_{i}}{\sum_{i=1}^{n} n_{i}}\\
	=&\dfrac{\sum_{i=1}^{n}x_{i}^{2}\cdot n_{i}}{\sum_{i=1}^{n} n_{i}}-2 \dfrac{\sum_{i=1}^{n}Mx_{i}\cdot n_{i}}{\sum_{i=1}^{n} n_{i}} +\dfrac{\sum_{i=1}^{n}M^{2}\cdot n_{i}}{\sum_{i=1}^{n} n_{i}}\\
	=&\dfrac{\sum_{i=1}^{n}x_{i}^{2}\cdot n_{i}}{\sum_{i=1}^{n} n_{i}}-2M \dfrac{\sum_{i=1}^{n}x_{i}\cdot n_{i}}{\sum_{i=1}^{n} n_{i}} +M^{2}\dfrac{\sum_{i=1}^{n}\cdot n_{i}}{\sum_{i=1}^{n} n_{i}}\\
	=&\dfrac{\sum_{i=1}^{n}x_{i}^{2}\cdot n_{i}}{\sum_{i=1}^{n} n_{i}}-2MM +M^{2}\\
	=&\dfrac{\sum_{i=1}^{n}x_{i}^{2}\cdot n_{i}}{\sum_{i=1}^{n} n_{i}}-2M^{2} +M^{2}\\
	=&\dfrac{\sum_{i=1}^{n}x_{i}^{2}\cdot n_{i}}{\sum_{i=1}^{n} n_{i}}-M^{2}\\
	\end{align*}
	Come si voleva dimostrare
\end{proof}
\section{Media}
\begin{thm}[Scarto dalla media]\index{Media!scarto}
	Lo scarto dalla media vale zero.\[\sum_{i=1}^{n}(x_{i}-M)=0\] 
\end{thm}
\begin{proof}
	\begin{align*}
	\sum_{i=1}^{n}(x_{i}-M)=&\\
	=&\sum_{i=1}^{n}x_{i}-\sum_{i=1}^{n}M\\
	=&nM-nM=0\\
	\end{align*}
	Come volevasi dimostrare
\end{proof}
			

\nocite{*}
 \addcontentsline{toc}{chapter}{\bibname}
\printbibliography
 \addcontentsline{toc}{chapter}{\indexname}
 \printindex
 \appendix
 \chapter{Mezzi usati}
 \CDMezziUsati
\end{document}
