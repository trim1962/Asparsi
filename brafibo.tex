% !TeX root = Asparsi.tex
% !BIB TS-program = biber
% !TeX encoding = UTF-8
% !TeX spellcheck = it_IT
\chapter{Brahmagupta Fibonacci}
\section{Identità Brahmagupta Fibonacci}
\begin{thm}[Brahmagupta Fibonacci]
Un numero che è somma di quattro quadrati può essere scritto come somma di due quadrati.
	\begin{align*}
	(a^2+b^2)(c^2+d^2)=&\\
	&=(ac-bd)^2+(ad+bc)^2\\
	&=(ac+bd)^2+(ad-bc)^2\\
	\end{align*}
\end{thm}
\begin{proof}
\begin{align*}
	(a^2+b^2)(c^2+d^2)=&\\
	=&a^2c^2+a^2b^2+b^2c^2+b^2d^2\\
	=&a^2c^2+b^2d^2-2abcd+a^2d^2+b^2c^2+2abcd\\
	=&(ac-bd)^2+(ad+bc)^2\\
	=&(ac+bd)^2+(ad-bc)^2\\
\end{align*}
\end{proof}