% !TeX root = Asparsi.tex
% !BIB TS-program = biber
% !TeX encoding = UTF-8
% !TeX spellcheck = it_IT
\chapter{Equazioni secondo grado}\label{ch:equazioni-secondo-grado}
\section{Risoluzione}\label{sec:risoluzione}
\begin{thm}[Formula risolutiva]\label{thm:Equasec1}
	Un'equazione di secondo grado $ax^2+bx+c=0$ con $a\neq 0$ ha per soluzioni \begin{equation*}\label[equation]{eq:equasec1}
	x_{1,2}=\dfrac{-b\pm\sqrt{b^2-4ac}}{2a}
	\end{equation*}\index{Equazione!secondo grado!soluzioni}\index{Triniomio!secondo!grado}
\end{thm}
\begin{proof}
	\begin{align*}
	ax^2+bx+c=&0
	\intertext{$a\neq 0$}
	ax^2+bx+\frac{b^2}{4a}+c=&\frac{b^2}{4a}\\
	ax^2+bx+\frac{b^2}{4a}=&\frac{b^2}{4a}-c\\
	\frac{4a^2x^2+4abx+b^2}{4a}=&\frac{b^2-4ac}{4a}\\
	\left(2ax+b\right)^2=&b^2-4ac
	\intertext{Otteniamo un'equazione di primo grado che risolviamo separando le variabili}
	2ax+b=&\pm\sqrt{b^2-4ac}\\
	2ax=&-b\pm\sqrt{b^2-4ac}\\
	x_{1,2}=&\frac{-b\pm\sqrt{b^2-4ac}}{2a}
	\end{align*}
	Da cui la tesi.
\end{proof}
\begin{cor}[Proprietà delle soluzioni]\label{cor:secondogradopropsoluzioni}
Data un'equazione di secondo grado $ax^2+bx+c=0$ con $a\neq 0$, se $x_1$ e $x_2$ sono le sue soluzioni allora\[\begin{cases}
	x_1+x_2=-\dfrac{b}{a}\\
	x_1\cdot x_2=\dfrac{c}{a}\\
\end{cases}\]
\end{cor}
\begin{proof}Dal~\vref{thm:Equasec1} abbiamo:
	\begin{align*}
	&\begin{cases}
	x_1=\frac{-b+\sqrt{b^2-4ac}}{2a}\\
	\\
	x_2=\frac{-b-\sqrt{b^2-4ac}}{2a}\\
	\end{cases}\\
	x_1+x_2=&\frac{-b+\sqrt{b^2-4ac}}{2a}+\frac{-b-\sqrt{b^2-4ac}}{2a}\\
	=&\frac{-b+\sqrt{b^2-4ac}-b-\sqrt{b^2-4ac}}{2a}\\
	=&-\frac{2b}{2a}\\
	=&-\dfrac{b}{a}\\
	x_1\cdot x_2=&\frac{-b+\sqrt{b^2-4ac}}{2a}\cdot\frac{-b-\sqrt{b^2-4ac}}{2a}\\
	=&\frac{b^2-b^2+4ac}{4a^2}\\
	=&\frac{4ac}{4a^2}\\
	=&\frac{c}{a}
	\end{align*}
	Da cui la tesi.
\end{proof}
\begin{cor}[Scomposizione del trinomio di secondo grado]\label{cor:Scomposizionetrisecgrad}
Data un'equazione di secondo grado $ax^2+bx+c=0$ con $a\neq 0$, se $x_1$ e $x_2$ sono le sue soluzioni allora:\[ax^2+bx+c=a\left(x-x_1\right)\left(x-x_2\right) \]
\end{cor}\index{Trinomio!scomposizione}
\begin{proof}Dal~\vref{cor:secondogradopropsoluzioni} otteniamo
	\begin{align*}
ax^2+bx+c=&a\left(x^2+\dfrac{b}{a}+\dfrac{c}{a}\right)\\
=&a\left[x^2-\left(x_1+x_2\right)x+x_1\cdot x_2\right]\\
=&a\left(x^2-x_1x-x_2x+x_1\cdot x_2\right)\\
=&a\left[x\left(x-x_1\right)-x_2\left(x-x_1\right)\right]\\
=&a\left(x-x_1\right)\left(x-x_2\right)\\
	\end{align*}
\end{proof}
\section{Casi particolari}
\begin{thm}[Formula ridotta]\label{thm:Equasec2}
	Se l'equazione di secondo grado è del tipo\[ax^2+2\beta x+c=0 \] allora la formula risolutiva è \begin{equation*}\label[equation]{eq:equasec2}
	x_{1,2}=\dfrac{-\beta\pm\sqrt{\beta^2- ac}}{a}
	\end{equation*}
\end{thm}
\begin{proof}
Consideriamo l'equazione $ax^2+2\beta x+c=0$, dal~\cref{thm:Equasec1} otteniamo:
\begin{align*}
x_{1,2}=&\dfrac{-2\beta\pm\sqrt{4\beta^2-4ac}}{2a}\\
=&\dfrac{-2\beta\pm\sqrt{4(\beta^2-ac)}}{2a}\\
=&\dfrac{-2\beta\pm 2\sqrt{(\beta^2-ac)}}{2a}\\
=&\dfrac{-\beta\pm \sqrt{(\beta^2-ac)}}{a}
\end{align*}
Come si voleva dimostrare.
\end{proof}
\begin{thm}[somma di due numeri]\label{thm:Equasec3}
	\[px^2+(p+q)x+q=0 \] allora l'equazione ha soluzione \[\begin{cases}
	x_1=-1\\ x_2=-\dfrac{q}{p}
	\end{cases}\]
\end{thm}
\begin{proof}
	Consideriamo l'equazione $px^2+(p+q) x+q=0$, dal~\cref{thm:Equasec1} otteniamo:
	\begin{align*}
	x_{1,2}=&\dfrac{-(p+q)\pm\sqrt{(p+q)^2-4pq}}{2p}\\
	=&\dfrac{-(p+q)\pm\sqrt{p^2+q^2-2pq}}{2p}\\
	=&\dfrac{-(p+q)\pm\sqrt{(p-q)^2}}{2p}\\
	=&\dfrac{-(p+q)\pm(p-q)}{2p}\\
	&\begin{cases}
	x_1=\dfrac{-p-q+p-q}{2p}=-\dfrac{2q}{2p}=-\dfrac{q}{p}\\ \\
	x_2=\dfrac{-p-q-p+q}{2p}=-\dfrac{2p}{2p}=-1\\
	\end{cases}\\
	\end{align*}
	Come si voleva dimostrare.
\end{proof}